\documentclass[10pt,a4paper]{report}
\usepackage[utf8]{inputenc}
\usepackage{amsmath}
\usepackage{amsthm}
\usepackage{amsfonts}
\usepackage{amssymb}
\usepackage{tikz-cd}
\usepackage{calc}
\usepackage{setspace}
\usepackage[german]{babel}
\usetikzlibrary{babel}
\usepackage{cleveref}

\newcommand{\comment}[1]{}
\renewcommand{\baselinestretch}{1.1}

\newcommand{\ModulsOfDifferenzials}{Kommutativ Algebra with a view Torwards Algebraic Geometrie [David Eisenbud 1994]}

\newcounter{Aussage}[chapter]

\newtheorem{satz}[Aussage]{Satz}
\newtheorem{theorem}[Aussage]{Theorem}
\newtheorem{prop}[Aussage]{Proposition}
\newtheorem{korrolar}[Aussage]{Korrolar}
\newtheorem{lemma}[Aussage]{Lemma}
\newtheorem{bem}[Aussage]{Bemerkung}
\newtheorem{definition}[Aussage]{Definition}

\newcommand{\functionfront}[3]{\nolinebreak{#1:#2 \longrightarrow #3}}
\newcommand{\functionback}[3]{\nolinebreak{#1:#2 \longmapsto #3}}
\newcommand{\function}[5]{\nolinebreak{#1:#2 \longrightarrow #3 \, , \, #4 \longmapsto #5}}
\newcommand{\divR}[2]{\Omega_{#1/#2}}
\newcommand{\Tensor}[3]{#1 \otimes_{#2} #3}
\newcommand{\tensor}[3]{#1 \otimes #3}
\newcommand{\lok}[2]{#1 [#2^{-1}]}
\newcommand{\loke}[3]{(\frac{#1}{#2})_{_{#3}}}
\comment{\newcommand{\loke}[3]{(#1,#2)_{mod\sim_{#3}}}}

\newcommand{\colimes}[0]{\lim\limits_{ \longrightarrow }}
\newcommand{\infunctionfront}[3]{\nolinebreak{#1:#2 \hookrightarrow #3}}
\newcommand*{\defeq}{\mathrel{\vcenter{\baselineskip0.5ex \lineskiplimit0pt
                     \hbox{\scriptsize.}\hbox{\scriptsize.}}}%
                     =}
\newcommand*{\defeqr}{= \mathrel{\vcenter{\baselineskip0.5ex \lineskiplimit0pt
                     \hbox{\scriptsize.}\hbox{\scriptsize.}}}}
                     
\newcommand*{\defshow}{\stackrel{!}{=}}
\newcommand{\kernel}[1]{kern(#1)}
\newcommand{\immage}[1]{im(#1)}

\begin{document}
\chapter{Grundlegende Sätze}
\textit{Proposition 16.3 aus \ModulsOfDifferenzials}
\begin{satz} \label{prop16.3}
\raggedright
Sei $\functionfront{\pi}{S}{T}$ ein R-Algebrenephimorphismus mit Kern($\pi$):= I \\
Dann ist folgende Sequenz rechtsexakt: \\
\begin{center}
\begin{tikzcd}
I/I^2 \arrow[r, "f"] & \Tensor{T}{S}{\divR{S}{R}} \arrow[r, "g"] & \divR{T}{R} \arrow[r] & 0
\end{tikzcd}
\end{center}
\begin{spacing}{1.5}
mit: $\function{f}{I/I^2}{{\Tensor{T}{S}{\divR{S}{R}}}}{[a]_{I^2}}{\tensor{1}{S}{d_S(a)}}$\\
\textsc{\leftskip2.3em} $\function{g}{\Tensor{T}{S}{\divR{S}{R}}}{\divR{T}{R}}{\tensor{b}{S}{d_S(c)}}{b \cdot (d \circ \pi)(c)}$
\end{spacing}
\end{satz}

\begin{proof} \ \\
\underline{$f$ ist wohldefiniert:} Seien $a,b\in I^2$. Zeige $f(a \cdot b)=0$ :
$$ f(a \cdot b) =
\tensor{1}{S}{( d_S \circ \pi )(a \cdot b)} =
\tensor{1}{S}{\pi(a) \cdot (d_S \circ \pi )(b) + \pi(b) \cdot ( d_S \circ \pi )(a)} =0$$
\underline{$D\pi$ ist surjektiv:}

\begin{center}
\begin{tikzcd}
\divR{S}{R} \arrow[r, " D\pi "]                & \divR{T}{R}                \\
S \arrow[r, " \pi "] \arrow[u, " d_S "] & T \arrow[u, " d_T "]
\end{tikzcd}
\end{center}
Da $\divR{S}{R}$ und $\divR{T}{S}$ jeweils von $d_S$ und $d_T$ erzeugt werden, vererbt sich die Surjektivität von $\pi$ auf $D\pi$. Somit ist auch $\Tensor{1}{S}{D\pi}$ surjektiv.\\
\underline{$\immage{f}=\kernel{g}$:}\\ Dies folgt direkt aus  der Isomorphie $(\Tensor{T}{S}{\divR{S}{R}})/Im(f) \simeq \divR{T}{R}$:
\begin{align*}
(\Tensor{T}{S}{\divR{S}{R}})/Im(f) \\
= (\Tensor{T}{S}{\divR{S}{R}})/(\Tensor{T}{S}{d_S(I)}) \\
= \Tensor{T}{S}{(\divR{S}{R}/d_S(I))} \\ 
= \Tensor{T}{S}{(d_S(S)/d_S(I))} \\
\simeq \Tensor{T}{S}{d_S(S/I)} \\
\simeq \Tensor{T}{S}{d_T(T)} \\
\end{align*}
\end{proof}

\chapter{Kolimes}
In diesem Kapitel werden wir das Konzept des Kolimes als Konstrukt der Kathegorientheorie kennen lernen. Am Ende des Kapitels werden wir sehen, dass der Kolimes von R-Algebren mit der Bildung des Kähler-Differenzials harmoniert. Dies wird es uns im folgenden vereinfachen, bestimmte Eigenschaften des Kähler-Differenzials nachzuweisen.\\
\textit{Vgl. Anhang A6 aus \ModulsOfDifferenzials .}
\begin{definition}
Sei $\mathcal{A}$ eine Kategorie und $C \in \mathcal{A}$ ein Objekt
\begin{itemize}
\item Ein \underline{Diagramm} über $\mathcal{A}$ ist eine Kategorie $\mathcal{B}$ zusammen mit einem Funktor $\functionfront{\mathcal{F}}{\mathcal{B}}{\mathcal{A}}$.
\item Ein \underline{Morphismus} $\functionfront{\psi}{\mathcal{F}}{C}$ ist eine Menge von Funktionen 
$\nolinebreak{\lbrace \psi_B \in Hom(F(B),C) \vert B \in \mathcal{B} \rbrace}$, wobei für alle $B_1,B_2 \in \mathcal{B}$ und $\varphi \in Hom(B_1,B_2)$ folgendes Diagramm kommutiert:
\begin{center}
\begin{tikzcd}
\mathcal{F}(B_1) \arrow[rrd, "\psi_{B_1}"] \arrow[dd, "\mathcal{F}(\varphi )"] &  &   \\
                                   &  & C \\
\mathcal{F}(B_2) \arrow[rru, "\psi_{B_2}"]                 &  &  
\end{tikzcd}
\end{center}
\item Der \underline{Kolimes} $\colimes \mathcal{F}$ eines Diagramms $\functionfront{\mathcal{F}}{\mathcal{B}}{\mathcal{A}}$ ist ein Objekt $A \in \mathcal{A}$ zusammen mit einem Morphismus $\functionfront{\psi}{\mathcal{F}}{A}$ und folgender universellen Eigenschaft:
\begin{center}
\comment{$\forall Morphismen \functionfront{\psi '}{\mathcal{F}}{\mathcal{A}'}\exists ! \varphi \in Hom_{\mathcal{A}}(A,A') \forall B \in \mathcal{B}: \varphi \circ \psi_B = \psi'_B $}

für alle Morphismen $\functionfront{\psi '}{\mathcal{F}}{A'}$ existiert genau eine Funktion $\varphi \in Hom(A,A')$, sodass folgendes Diagramm kommutiert:
\begin{tikzcd}
  & \mathcal{F} \arrow[rd, "\psi"] \arrow[ld, "\psi '"'] &                            \\
A' &                                    & A \arrow[ll, "\exists ! \varphi "', dashed]
\end{tikzcd}
\end{center}

\end{itemize}
\end{definition}
Bevor der Kolimes weiter charakterisiert wird, zeigen wir zunächst, dass er durch die obige Definition eindeutig bestimmt ist.
\begin{lemma}
Seien $\mathcal{B},\mathcal{A}$ zwei Kategorien und $\functionfront{\mathcal{F}}{\mathcal{B}}{\mathcal{A}}$ ein Funktor, so git:\\ 
Im Falle der Existenz sind $\colimes \mathcal{F}$ und der dazugehörige Morphismus $\functionfront{\psi}{\mathcal{F}}{A}$ bis auf eine eindeutige Isomorphie eindeutig bestimmt.
\end{lemma}
\begin{proof}
Seien $A_1 \in \mathcal{A}, (\functionfront{\psi_1}{\mathcal{F}}{A_1}) $ und $A_2 \in \mathcal{A} , (\functionfront{\psi_2}{\mathcal{F}}{A_2}) $ beide gleich $\colimes \mathcal{F}$.\\
Erhalte durch die universelle Eigenschaft des Kolimes die eindeutig bestimmten Funktionen $\varphi_1 \in Hom_{\mathcal{A}}(A_1,A_2)$ und $\varphi_2 \in Hom_{\mathcal{A}}(A_2,A_1)$, für welche die folgende Diagramme kommutieren:

\comment{$\functionfront{\varphi_1}{\mathcal{A}_1}{\mathcal{A}_2}$ und $\functionfront{\varphi_2}{\mathcal{A}_2}{\mathcal{A}_1}$}
\begin{center}
\begin{tikzcd}
  & \mathcal{F} \arrow[rd, "\psi_1"] \arrow[ld, "\psi_2"'] &                            &  &   & \mathcal{F} \arrow[rd, "\psi_2"] \arrow[ld, "\psi_1"'] &                            \\
A_2 &                                    & A_1 \arrow[ll, "\exists ! \varphi_1"', dashed] &  & A_1 &                                    & A_2 \arrow[ll, "\exists ! \varphi_2"', dashed]
\end{tikzcd}
\end{center}
\begin{flushleft}
Wende nun die Universelle Eigenschaft von $\psi_1$ auf $\psi_1$ selbst an und erhalte $id_{A_1} = \varphi_2 \circ \varphi_1$. Analog erhalte auch $id_{A_2} = \varphi_1 \circ \varphi_2$.
\end{flushleft}
\begin{center}
\begin{tikzcd}
  & \mathcal{F} \arrow[rd, "\psi_1"] \arrow[ld, "\psi_1"'] &                            \\
A_1 &                                    & A_1 \arrow[ll, "\exists ! id_{A_1} = \varphi_2 \circ \varphi_1"', dashed]
\end{tikzcd}
\end{center}
Somit existiert genau eine Isomorphie $\functionfront{\varphi_1}{A_1}{A_2}$.
\end{proof}

Im folgenden beschäftigen wir uns mit dem Fall des $\colimes \infunctionfront{\mathcal{F}}{\mathcal{B}}{\mathcal{A}}$, bei welchem $\mathcal{B}$ eine Unterkategorie von $\mathcal{A}$ ist. Zur Vereinfachung unterschlagen dabei die triviale Existenz des Funktors $\colimes \infunctionfront{\mathcal{F}}{\mathcal{B}}{\mathcal{A}}$. Wir werden also im folgenden von dem Diagramm $\mathcal{B}$ und dem entsprechenden Kolimes $\colimes \mathcal{B}$, sowie dem Morphismus $\functionfront{\phi}{\mathcal{B}}{A} $ sprechen.\\ 
Zunächst untersuchen wir bei einer gegebenen Kategorie $\mathcal{A}$ das Koprodukt einer Menge von Objekten $B_i \in \mathcal{A}$, sowie den Differenzkokern zweier Morphismen $f,g \in Hom_{\mathcal{A}}(C_1,C_2)$.

\begin{definition} \label{DifferenzkokernUndKoproduktDef}
Sei $\mathcal{A}$ eine Kategorie.
\begin{itemize}
\item Das Koprodukt von $ \lbrace B_i \rbrace \subseteq \mathcal{A}$ wird durch $\prod_i \lbrace B_i \rbrace := \colimes\mathcal{B}$ definiert, 
wobei $\mathcal{B}$ $\lbrace B_i \rbrace$ als Objekte und die Identitätsabbildungen $\functionfront{id_{B_i}}{B_i}{B_i}$ als Morphismen enthält.
\item Der Differenzkokern (oder auch Coequilizer) von $f,g \in Hom_{\mathcal{A}}(C_1,C_2)$ wird durch $\colimes \mathcal{C}$ definiert,
wobei $\mathcal{C}$ $\lbrace C_1,C_2 \rbrace$ als Objekte und $ \lbrace f,g \rbrace \defeq Hom_{\mathcal{C}}(C_1,C_2)$ als Morphismen enthält.
\end{itemize}
\end{definition}

In der Einführung des Differenzkokern`s in \cref{DifferenzkokernUndKoproduktDef} ist deutliche zu sehen, inwiefern dieser ein Kolimes ist. Um aber mit dem Differenzkokern besser zu arbeiten wird er meist anders eingeführt. Daher betrachten auch wir ab nun eine andere, aber äquivalente Definition des Differenzkokern`s.

\begin{lemma}\label{NeuDifferenzenkokerndef} Sei $\mathcal{A}$ eine Kategorie mit $C_1,C_2 \in Hom_{\mathcal{A}}(C_1,C_2)$, so sind folgende Formulierungen äquivalent zur Definition des Differenzkokern`s $T \defeq \colimes \mathcal{C}$
\begin{itemize}
\item[1.] Es existiert ein Morphismus $\functionfront{\psi}{\mathcal{C}}{T}$, mit der Eigenschaft, dass für alle Morphismen $\functionfront{\psi '}{\mathcal{C}}{T'}$ genau ein $\varphi \in Hom_{\mathcal{A}}(T,T')$ mit $\varphi \circ \psi = \psi '$ existiert.
\item[2.] Es existiert ein $q \in Hom_{\mathcal{A}}(C_2,T)$ mit $q \circ f = q \circ g$ und der Eigenschaft, dass für alle Morphismen $q' \in Hom_{\mathcal{A}}(C_2,Z)$ mit $q' \circ f = q' \circ g$ genau ein $\varphi \in Hom_{\mathcal{A}}(T,T')$ mit $\varphi \circ q = q'$ existiert.
\begin{center}
\begin{tikzcd}
C_1 \arrow[r, "{f,g}"] \arrow[r] & C_2 \arrow[r, "q"] \arrow[rd, "q'"] & T \arrow[d, "\exists !\varphi", dashed] \\
                                 &                                     & T'                                     
\end{tikzcd}
\end{center}
\end{itemize}
\end{lemma}
\begin{proof}
\textit{1.} ist offensichtlich eine Ausformulierung der Einführung des Kolimes aus \cref{DifferenzkokernUndCoproduktDef}, zeige also im folgenden noch die Äquivalenz von \textit{1.} und \textit{2.}
\begin{itemize}
\item \underline{1 $\Rightarrow$ 2:}
\begin{itemize}
\item[] Da $\functionfront{\psi}{\mathcal{C}}{T}$ ein Morphismus ist, gilt für $\lbrace f,g \rbrace = Hom_{\mathcal{C}}(C_1,C_2)$:\\ $\psi_{C_1} = \psi_{C_2} \circ f = \psi_{C_1} \circ \psi_{C_2}$, setze also 
 $q  \defeq \psi_{C_2}$.
\item[] Sei nun $q' \in Hom_{\mathcal{A}}(C_2,T)$ mit der Eigenschaft $q' \circ f = q' \circ g$ gegeben:\\
 Definiere den Morphismus $\functionfront{\psi '}{\mathcal{C}}{T}$ als $\lbrace \psi_1 = q' \circ f , \psi_2 = q' \rbrace$,  somit folgt direkt aus der Universellen Eigenschaft von $\psi$, dass genau ein $\varphi \in Hom_{A}(C_2,T)$ existiert, mit $ \varphi \circ q = q '$.
\end{itemize}
\item \underline{2 $\Rightarrow$ 1:}
\begin{itemize}
\item[] Definiere $\functionfront{\psi }{\mathcal{C}}{T}$ als $\lbrace \psi_1 = q \circ f , \psi_2 = q \rbrace$.
Durch die Eigenschaft von $q$ gilt $\psi_{C_1} = \psi_{C_2} \circ f = \psi_{C_2} \circ g$.
\item[] Sei nun $\functionfront{\psi '}{\mathcal{C}}{\mathcal{A}}$ ein beliebiger Morphismus.\\
Definiere $d' \defeq \psi '$, somit existiert durch die Eigenschaft von $d$ genau ein $\varphi \in Hom_{\mathcal{A}}(C_2,T)$ mit $\varphi \circ q = q'$.
\begin{gather*}
\Rightarrow \varphi \circ \psi_2 = \psi '_2 \\
\textit{und }\varphi \circ \psi_1 = \varphi \circ \psi_2 \circ f = \varphi \circ \psi '_2 \circ f = \varphi \circ \psi '_1
\end{gather*}
\end{itemize}
\end{itemize}
\end{proof}
Wenn im weiteren Verlauf von dem Differenzkokern zweier Homomorphismen $\functionfront{f,g}{C_1}{C_2}$ gesprochen wird, meinen wir damit den Homomorphismus $\functionfront{q}{C_2}{T}$ aus \cref{NeuDifferenzenkokerndef}.


\begin{bem}
Seien $f,g \in Hom_{\mathcal{A}}(S_1,S_2)$ R-Algebra-Homomorphismen, so können wir für den Differenzenkokern $\functionfront{q}{S_2}{T}$ für ein beliebiges $S_1$-Modul das Tensorprodukt $\Tensor{T}{C_1}{M}$ definieren. 
\begin{gather*}
\textit{für } s_1 \in S_1 \textit{ und } \tensor{t}{S_1}{m}) \in \Tensor{T}{C_1}{M} \textit{ gilt: }\\
s_1 \cdot (\tensor{t}{S_1}{m}) = \tensor{((q \circ f)(s_1)) \cdot t}{S_1}{m} = \tensor{((q \circ g)) \cdot (s_1)t}{S_1}{m}
\end{gather*}
\end{bem}


Da wir es hauptsächlich mit R-Algebren zu tun haben, wollen wir natürlich auch wissen, wie sich Koprodukte und Differenzkokerne von R-Algebren verhalten. Daher betrachten wir in der folgenden Proposition genauer welche Form diese haben.
\begin{prop} \label{R-Algebra-Kolimiten}
in der Kategorie der R-Algebren existieren Koprodukte und Differenzkokerne, wobei:
\begin{itemize}
\item[\textbf{1.}] Das Koprodukt einer endlichen Familie von $R-Algebren$ $\lbrace S_i \rbrace_{i \in \Lambda}$ entspricht deren Tesorprodukt $\bigotimes_{i \in \Lambda} S_i$. 
\item[\textbf{2.}] Der Differenzkokern zweier R-Algebra-Homomorphismen $\functionfront{f,g}{S_1}{S_2}$ einspricht dem Homomorphismus $\function{q}{S_2}{S_2/Q}{y}{[y]}$, wobei $Q \defeq \lbrace f(x) - g(x)\mid x \in S_2 \rbrace$ das Bild der Differenz von $f$ und $g$ ist.
\end{itemize}
\end{prop}

\begin{proof}
Zu \textit{\textbf{1.}}:\\
Sei $\mathcal{B}$ die Unterkategorie der R-Algebren, welche $\lbrace S_i \rbrace_{i \in \Lambda}$ zusammen mit den Identitätsabbildungen enthält. Wir wollen die universellen Eigenschaften des Tensorproduktes und des Kähler-Differenzials nutzen, um einen Isomorphismus zwischen $\colimes \mathcal{F}$ und $\bigotimes_{i \in \Lambda} B_i$ zu finden.\\ 
Es sind der Morphismus $\functionfront{\psi}{\mathcal{B}}{\colimes \mathcal{B}}$ und die bilineare Abbildung $\functionfront{g}{\oplus_i S_i}{\otimes_i S_i}$ gegeben.\\
Konstruiere den Morphismus $\functionfront{\psi'}{\mathcal{B}}{\otimes_i S_i}$ durch $\function{\psi'_i}{S_i}{\otimes_i S_i}{s_i}{g(1,..,1,s_i,1,..,1)}$ für $i \in \lambda$ und die bilineare Abbildung $\function{f}{\oplus_i S_i}{\colimes \mathcal{B}}{s}{\prod_i \psi_i(s_i)}$.\\
\ \\
Somit liefern uns die universellen Eigenschaften folgende zwei R-Algebra-Homomorphismen:
\begin{gather*}
\functionfront{\varphi}{\colimes \mathcal{B}}{\bigotimes_i S_i} \\
\functionfront{\phi}{\bigotimes_i S_i}{\colimes \mathcal{B}}.
\end{gather*}
\begin{center}
\begin{tikzcd}
  & \mathcal{B} \arrow[rd, "\psi"] \arrow[ld, "\psi'"'] &                                            &  &   & \oplus_i S_i \arrow[rd, "g"] \arrow[ld, "f"'] &                                         \\
\otimes_i S_i &                                           & \colimes \mathcal{B} \arrow[ll, "\exists ! \varphi"', dashed] &  & \colimes \mathcal{B} &                                    & \otimes_i S_i \arrow[ll, "\exists ! \phi"', dashed]
\end{tikzcd}
\end{center}
Die Eindeutigkeit der universellen Eigenschaften liefert uns, das $\varphi$ und $\phi$ zueinander Inverse sind und somit haben wir unsere gesuchten Isomorphismen zwischen $\colimes \mathcal{B}$ und $\bigotimes_i S_i$ gefunden.
\begin{center}
\begin{tikzcd}
         & \mathcal{B} \arrow[rd, "\psi"] \arrow[ld, "\psi"'] &                                                                              &  &                  & \bigoplus_i S_i \arrow[rd, "g"] \arrow[ld, "g"'] &                                                                                     \\
\colimes \mathcal{B} &                                                    & \colimes \mathcal{F} \arrow[ll, "\exists ! id_{\colimes \mathcal{B}} = \phi \circ \varphi "', dashed] &  & \bigotimes_i S_i &                                                  & \bigotimes_i S_i \arrow[ll, "\exists ! id_{\bigotimes_i S_i} = \varphi \circ \phi "', dashed]
\end{tikzcd}
\end{center}
\ \\
Zu \textit{\textbf{2.}}:\\
Zeige, dass $\functionfront{q}{S_2}{S_2/Q}$ die in \cref{NeuDifferenzenkokerndef} eingeführten Eigenschaften des Differenzkokern`s  besitzt.
\begin{gather*}
q \circ f = q \circ g \text{ gilt, da } \kernel{q} = Q = \lbrace f(x) - g(x)\mid x \in C_2 \rbrace.
\end{gather*}
Sei nun eine Funktion $q' \in Hom_{\mathcal{A}}(S_2,T')$ mit $q' \circ f = q' \circ$ gegeben.\\
Somit gilt $q' \circ (f - g) = 0$, wodurch $Q$ ein Untermodul von $Q' \defeq \kernel{q'}$ ist.\\ Mit dem Isomorphiesatz \comment{HOMOMORPHIESATZ [kommutative Algebra 2.10]} für R-Algebren erhalten wir:
\begin{gather*}
 \nolinebreak{S_2/Q' \simeq (S_2/Q)/(Q'/Q)}.
\end{gather*}
Somit ist $\function{q'}{S_2}{(S_2/Q)/(Q'/Q)}{y}{[y]'}$ eine isomorphe Darstellung von $\functionfront{q'}{S_2}{T'}$.
\begin{gather*}
\Rightarrow \exists ! \function{\varphi}{S_2/Q}{(S_2/Q)/(Q'/Q)}{[y]}{[y]'}\textit{ mit }(\varphi \circ q) = q'.
\end{gather*}
Also ist $\functionfront{q}{S_2}{S_2/Q}$ der bis auf Isomorphie eindeutig bestimmte Differenzkokern von $f$ und $g$.
\end{proof}


Um im weiteren auch Koprodukte bzw. Differenzenkokerne des Kählerdifferenzials betrachten zu können, wollen wir wissen, wie sich diese in der Kategorie der R-Module verhalten. Daher betrachten wir in der folgenden Proposition genauer welche Form diese haben.
\begin{prop}\label{R-Modul-Kolimiten}
In Der Kategorie der R-Module  existieren Koprodukte und Differenzkokerne, wobei:
\begin{itemize}
\item[\textbf{1.}] das Koprodukt $\colimes \mathcal{B}$ von R-Modulen $M_i \in (R-Module)$entspricht der direkten Summe $\sum_i M_i$.
\item[\textbf{2.}] der Differenzenkokern zweier Homomorphismen $\functionfront{f,g}{M_1}{M_2}$ entspricht dem Kokern $M_2/\immage{f-g}$ der Differenzenabbildung.
\end{itemize}
\end{prop}
\begin{proof}
\begin{itemize}

für \textit{\textbf{1.}} Sei $\functionfront{\phi}{\lbrace M_i \rbrace}{\mathcal{B}}$ ein beliebiger Morphismus. Zeige: \\
\begin{center}
\begin{tikzcd}
  & \mathcal{B} \arrow[rd, "\psi_i"] \arrow[ld, "\psi_i"'] &                                            \\
M' &                                              & \bigoplus_i M_i \arrow[ll, "\exists ! \varphi"', dashed]
\end{tikzcd}\\
\end{center}
Für ein beliebiges $i$ existiert genau ein $\function{\varphi_i}{M_i \oplus 0}{M'}{(0,...,0,m_i,0,...,0}{\psi_i '(m_i)}$
 mit $\psi_i ' = \psi_i \circ \varphi_i$\\
$\Rightarrow  \exists ! \function{\varphi}{\bigoplus_i M_i}{M'}{(m_1,...,m_n)}{\sum_i \psi_i(m_i)}$\\
\ \\
\textit{\textbf{2.}} ist Analog zu \cref{R-Algebra-Kolimiten}
\end{itemize}
\end{proof}
Die in \cref{R-Modul-Kolimiten} gezeigten Darstellungen gelten mit kurzen Überlegungen auch für S-Module, wobei S eine R-Algebra ist.\\
\ \\
Eine Anwendung des Kolimes ist die Lokalisation. Hier können wir den Kolimes nutzen, um die multiplikativ abgeschlossene Menge $U \subseteq S$ ohne Einschränkung auf den Fall $U = \lbrace t^n \vert n \in \mathbb{N}_0 \rbrace$ für ein $t \in S$ herunter zu brechen. Betrachte dazu die folgenden zwei Aussagen.\\
\textit{Aufgabe A6.7 aus \ModulsOfDifferenzials.}
\begin{lemma}\label{Lokalisierung von Algebren als Kolimes}
Sei $S$ eine $R-Algebra$ und $U \subseteq S$ multiplikativ abgeschlossen.
Dann gilt:
\begin{gather*}
 S[U^{-1}] = \colimes \mathcal{B}
\end{gather*}
Wobei $\mathcal{B}$ aus den Objekten $\lbrace \lok{S}{t} \vert t \in U \rbrace$ und den Morphismen\\
$\lok{S}{t} \longrightarrow \lok{S}{tt'}, \loke{s}{t^n}{t} \longmapsto \loke{st'^n}{(tt')^n}{(tt')} \,
\forall t,t' \in U$ besteht.\\
\end{lemma}
\begin{proof}
Sei $\functionfront{\psi}{\mathcal{B}}{A}$ der Kolimes von $\mathcal{B}$. Zeige $\lok{S}{U} \simeq A$, definiere dazu:
\begin{gather*}
\functionfront{\psi'}{\mathcal{B}}{\lok{S}{U}}\\
\function{\psi'_{\lok{S}{t}}}{\lok{S}{t}}{\lok{S}{t}}{\loke{s}{t^n}{t}}{\loke{s}{t^n}{U}}
\end{gather*}
$\psi'$ ist ein Morphismus, da für beliebige $t,t' \in U$ und $s \in S$ gilt:
$$\loke{s}{t^n}{U} = \loke{st'^n}{(tt')^n}{U}$$
Durch die Universelle Eigenschaft des Kolimes erhalten wir den eindeutigen Homomorphismus $\functionfront{\varphi}{A}{\lok{S}{U}}$.
\begin{center}
\begin{tikzcd}
            & \mathcal{B} \arrow[rd, "\psi"] \arrow[ld, "\psi'"'] &                                            \\
{S[U^{-1}]} &                                                     & A \arrow[ll, "\exists ! \varphi"', dashed]
\end{tikzcd}
\end{center}
Für $\functionfront{\phi}{S[U^{-1}]}{A}$ benötigen wir kleinere Vorüberlegungen.\\
Zunächst können wir jedes Element $\loke{s}{u}{U} \in \lok{S}{U}$ als $\psi_{\lok{S}{t}}(\loke{s}{t}{t})$ schreiben.\\
\comment{\label{wobei u = t}}
Weiter gilt für alle $s_1,s_2 \in S , \, t_1,t_2 \in U$: 
\begin{align*}
\textit{Sei }\psi'_{\lok{S}{t}}(\loke{s_1}{t_1}{t}) = \psi'_{\lok{S}{t}}(\loke{s_2}{t_2}{t})\\
\Rightarrow  \exists u \in U: (s_1t_1 - s_2t_2) \cdot u = 0\\
\Rightarrow  \loke{s_1u}{t_1u}{tu} = \loke{s_2u}{t_2u}{tu}\\
\Rightarrow  \psi_{\lok{S}{t}}(\loke{s_1}{t_1}{t}) = \psi_{\lok{S}{t}}(\loke{s_2}{t_2}{t})
\end{align*}
Mit diesem Wissen können wir den R-Algebra-Homomorphismus $\functionfront{\phi}{\lok{S}{U}}{A}$ definieren:
\begin{gather*}
\function{\phi}{\lok{S}{U}}{A}{\psi'_{\lok{S}{t}}(\loke{s}{t}{t})}{\psi_{\lok{S}{t}}(\loke{s}{t}{t})}
\end{gather*}
$\phi \circ \varphi = id_A$ ergibt sich direkt aus der Universellen Eigenschaft des Kolimes:
\begin{center}
\begin{tikzcd}
  & \mathcal{B} \arrow[rd, "\psi"] \arrow[ld, "\psi"'] &                                                              \\
A &                                                    & A \arrow[ll, "\exists ! id_A = \phi \circ \varphi"', dashed]
\end{tikzcd}
\end{center}
Für $\varphi \circ \phi \defshow id_{\lok{S}{U}}$ wähle beliebige $s \in S , t \in U$, für diese gilt:
\begin{gather*}
(\varphi \circ \phi)(\psi'(\loke{s}{t}{t})) =
 \varphi (\psi(\loke{s}{t}{t}) =
  \psi'(\loke{s}{t}{t})
\end{gather*}
Damit haben wir gezeigt, dass $\varphi,\phi$ Isomorphismen sind und somit $A \simeq \lok{S}{U}$ gilt.\\
Da der Kolimes bis auf Isomorphie eindeutig ist, definiere ab sofort $\lok{S}{U}$ als den eindeutigen Kolimes von $\mathcal{B}$.
 \end{proof}


\begin{korrolar}\comment{\label{Lokalisierung von Moduln als Kolimes}}
Sei M ein S-Modul, wobei S eine R-Algebra ist. Sei weiter $U \subseteq S$ multiplikativ abgeschlossen. Dann gilt:
\begin{gather*}
\lok{M}{U} = \colimes \mathcal{C}
\end{gather*}
Wobei $\mathcal{C}$ aus den Objekten $\lbrace \Tensor{\lok{S}{U}}{\lok{S}{t}}{\lok{M}{t}} \vert t \in U \rbrace$ und folgenden Morphismen besteht:
\begin{gather*}
\tensor{\lok{S}{U}}{\lok{S}{t}}{\lok{M}{t}} \longrightarrow
\tensor{\lok{S}{U}}{\lok{S}{(tt')}}{\lok{M}{(tt')}} ,\\
\tensor{\loke{s}{u}{U}}{\lok{S}{t}}{\loke{m}{t^n}{t}} \longmapsto
\tensor{\loke{s}{u}{U}}{\lok{S}{t}}{\loke{t'^nm}{(tt')^n}{t}} 
\end{gather*}
\end{korrolar}
\textit{Auch wenn sich \cref{Lokalisierung von Algebren als Kolimes} hier nicht direkt anwenden lässt, so können wir doch im Beweis gleich vorgehen.}
\begin{proof}
Schließe zunächst den trivialen Fall $0 \in U$ aus.\\
Sei $\functionfront{\psi}{\mathcal{C}}{A}$ der Colimes von $\mathcal{C}$. Zeige $\lok{S}{U} \simeq A$, definiere dazu folgenden Morphismus \comment{\label{das phi ein Mophismus ist überlasse ich dem Leser}}:
\begin{gather*}
\functionfront{\psi}{\mathcal{C}}{\lok{M}{U}} \\
\comment{
\function{\psi_{\Tensor{\lok{S}{U}}{\lok{S}{t}}{\lok{M}{t}}}}{\Tensor{\lok{S}{U}}{\lok{S}{t}}{\lok{M}{t}}}{\lok{M}{U}}{\tensor{\loke{s}{u}{U}}{\lok{S}{t}}{\loke{m}{t^n}{t}}}{\loke{sm}{ut^n}{U}} \\
}
\function{\psi_{t}}{\Tensor{\lok{S}{U}}{\lok{S}{t}}{\lok{M}{t}}}{\lok{M}{U}}{\tensor{\loke{s}{u}{U}}{\lok{S}{t}}{\loke{m}{t^n}{t}}}{\loke{sm}{ut^n}{U}}
\end{gather*}

Die Wohldefiniertheit von $\psi'_t$ für ein beliebiges $t \in U$ folgt direkt aus der Universellen Eigenschaft des Tensorprodukt`s. Denn für die bilineare Abbildung
 $\function{f}{\lok{S}{U} \oplus \lok{M}{t}}{\lok{M}{t}}{(\loke{s}{u}{U}, \loke{m}{t^n}{t})}{\loke{sm}{ut^n}{U}}$  gilt:
\begin{center}
\begin{tikzcd}
\lok{S}{U} \oplus \lok{M}{t} \arrow[r, "g"] \arrow[rd, "f"'] & \Tensor{\lok{S}{U}}{\lok{S}{t}}{\lok{M}{t}} \arrow[d, "\exists ! \psi'_t", dashed] \\
                                      & \lok{M}{U}                               
\end{tikzcd}
\end{center} 
 
Durch die Universelle Eigenschaft des Kolimes erhalten wir nun den eindeutigen Homomorphismus $\functionfront{\varphi}{A}{\lok{M}{U}}$.
\begin{center}
\begin{tikzcd}
  & \mathcal{C} \arrow[rd, "\psi"] \arrow[ld, "\psi'"'] &                                            \\
\lok{M}{U} &                                                     & A \arrow[ll, "\exists ! \varphi"', dashed]
\end{tikzcd}
\end{center}
Für $\functionfront{\phi}{\lok{M}{U}}{A}$ benötigen wir kleinere Vorüberlegungen.\\
Zunächst können wir jedes Element $\loke{m}{u} \in \lok{M}{U}$ als $\nolinebreak{\psi(\tensor{\loke{1}{u}{U}}{\lok{M}{t}}{\loke{m}{1}{t}}})$ schreiben.
Wobei mit $\psi$ gemeint ist, dass wir ein beliebiges $t \in U$ wählen und dann $\psi_t$ betrachten. Diese Verallgemeinerung ist möglich, da für beliebige $t_1,t_2,u \in U$ und $m \in M$ gilt:
\begin{gather*}
\psi_{t_1}({\tensor{\loke{1}{u}{U}}{\lok{M}{t_1}}{\loke{m}{1}{t_1}}}) =
\loke{m}{u}{U} = 
\psi_{t_2}({\tensor{\loke{1}{u}{U}}{\lok{M}{t_2}}{\loke{m}{1}{t_2}}})
\end{gather*}
Definiere nun mit diesem Wissen folgenden Homomorphismus:
\begin{gather*}
\function{\phi}{\lok{M}{U}}{A}{\psi(\tensor{\loke{1}{u}{U}}{\lok{\loke{m}{1}{}}{M}}{t})}{\psi'(\tensor{\loke{1}{u}{U}}{\lok{\loke{m}{1}{t}}{M}}{t})}
\end{gather*}
$\phi \circ \varphi = id_A$ ergibt sich direkt aus der Universellen Eigenschaft des Kolimes.\\
Für $\varphi \circ \phi \defshow id_{\lok{M}{U}}$ wähle $\loke{m}{u}{U} \in \lok{M}{U}$ beliebig, für dieses gilt:
\begin{gather*}
(\varphi \circ \phi) (\psi'(\tensor{\loke{1}{u}{U}}{\lok{M}{t}}{\loke{m}{1}{t}})) \\
 =\varphi(\psi(\tensor{\loke{1}{u}{U}}{\lok{M}{t}}{\loke{m}{1}{t}})) \\
  =\psi'(\tensor{\loke{1}{u}{U}}{\lok{M}{t}}{\loke{m}{1}{t}})
\end{gather*}
Damit haben wir $A \simeq \lok{M}{U}$ gezeigt, definiere also ab sofort $\lok{M}{U}$ als den eindeutigen Kolimes von $\mathcal{C}$.
\end{proof}
\ \\

Wie schon angekündigt, bringen wir nun dieses Kapitel zu einem Ende, indem wir die gerade kennen gelernte Theorie mit dem Kähler-Differenzial in Verbindung setzen.\\
Folgende Proposition zeigt uns, dass Differenzkokerne und Koprodukte unter der Bildung des Kähler-Differenzial`s erhalten bleiben.\\
\textit{vlg.Korrolar 16.7 aus \ModulsOfDifferenzials.}
\comment{Beide Beweise sind sehr kurz gefasst}
\begin{prop} \comment{\label{Kählerdifferenzial des Kolimes}}
\ \\
\begin{itemize}
\item[\textbf{1.}]
Sei $T = \otimes_{i \in \Lambda} S_i$ das Koprodukt der R-Algebren $S_i$.\\
Dann gilt:
\begin{gather*}
\divR{T}{R} \simeq \bigoplus_{i\in \Lambda} ( \Tensor{T}{S_i}{\divR{S_i}{R}} )
\end{gather*}
\item[\textbf{2.}]
Seien $S_1,S_2$ R-Algebren und $\functionfront{\varphi,\varphi'}{S_1}{S_2}$ R-Algebra-Homomorphismen. Sei weiter $\functionfront{q}{S_2}{T}$ der Differenzkokern von $\varphi$,$\varphi '$.
Dann ist folgende Sequenz rechtsexakt:
\begin{center}
\begin{tikzcd}
\Tensor{T}{S_1}{\divR{S_1}{R}} \arrow[r, "f"] & \Tensor{T}{S_2}{\divR{S_2}{R}} \arrow[r, "g"] & \divR{T}{R} \arrow[r] & 0
\end{tikzcd}
\begin{gather*}
\textit{mit: } \function{f}{\tensor{T}{S_1}{\divR{S_1}{R}}}{\Tensor{T}{S_2}{\divR{S_2}{R}}}{\tensor{t}{S_2}{d_{S_1}(x_1)}}{\tensor{t}{S_2}{d_{S_2}(\varphi(x_1) - \varphi(x_2))}}\\
\function{g}{\Tensor{T}{S_2}{\divR{S_2}{R}}}{\divR{T}{R}}{\tensor{t}{S_2}{d_{S_2}(x_2)}}{(d_{S_2}\circ q)(x_2)}
\end{gather*}
\end{center}
\end{itemize}
\end{prop}
\begin{proof}\ \\
Für \textit{\textbf{1.}} finde durch die Universelle Eigenschaft des Kähler-Differenzials Isomorphismen $ \divR{T}{R} \longleftrightarrow \bigoplus_{i \in \Lambda} ( \Tensor{T}{S_i}{\divR{S_i}{R}} )$.\\
Definiere das Differenzial $\function{e}{T}{\sum_{i \in \Lambda} \Tensor{T}{S_i}{\divR{S_i}{R}}}{(\tensor{s_i}{R}{...})}{(\tensor{1}{S_i}{d_{S_1},...)}}$ und erhalte dadurch
\begin{center}
\begin{tikzcd}
T \arrow[rd, "e"'] \arrow[r, "d_T"] & \divR{T}{R} \arrow[d, "\exists ! \varphi", dashed] \\
                                    & \sum_{i\in \Lambda} \Tensor{T}{S_i}{\divR{S_i}{R}}                                       
\end{tikzcd}
$\functionfront{\varphi}{\divR{T}{R}}{\bigoplus_{i\in \Lambda} ( \Tensor{T}{S_i}{\divR{S_i}{R}} )}$.
\end{center}
Definiere nun das Differenzial $k: S_i \hookrightarrow T \longrightarrow \divR{T}{R}$ und erhalte dadurch
\begin{center}
\begin{tikzcd}
S_i \arrow[rd, "k"'] \arrow[r, "d_{S_i}"] & \divR{S_i}{R} \arrow[d, "\exists ! k'", dashed] \arrow[r, "a"] & \Tensor{T}{S_i}{\divR{S_i}{R}} \arrow[ld, "\phi_i"] \\
                                          & \divR{T}{R}                                                    &                     
\end{tikzcd}
$\functionfront{\phi_i}{\bigoplus_{i\in \Lambda} ( \Tensor{T}{S_i}{\divR{S_i}{R}} )}{\divR{T}{R}}$\\
\begin{gather*}
\function{\phi}{\sum_{i\in \Lambda} ( \Tensor{T}{S_i}{\divR{S_i}{R}})}{\divR{T}{R}}{(\tensor{t_1}{S_1}{d_{S_i}s_1},...)}{\prod_{i\in \Lambda} t_i \cdot \phi_i(d_{S_i}(s_i))}. 
\end{gather*}
\end{center}
Damit haben wir zwei zueinander inverse Funktionen $\varphi ,\phi$ gefunden.\\
$\Rightarrow \divR{T}{R} \simeq \bigoplus_{i\in \Lambda} ( \Tensor{T}{S_i}{\divR{S_i}{R}} )$\\
\ \\
Für \textit{\textbf{2.}} Wende \cref{prop16.3} auf den Differenzkokern $\functionfront{q}{S_2}{S_2/Q}$ \textit{(vlg. \cref{R-Algebra-Kolimiten})} an und erhalte dadurch eine exakte Sequenz, welche ähnlich zu der gesuchten ist:
\begin{center}
\begin{tikzcd}
Q/Q^2 \arrow[r, "f'"] & \tensor{T}{S_2}{\divR{S_2}{R}} \arrow[r, "g"] & \divR{T}{R} \arrow[r] & 0
\end{tikzcd}
\end{center}
mit $\function{f'}{Q/Q^2}{{\Tensor{T}{S_2}{\divR{S}{R}}}}{[s_2]_{Q^2}}{\tensor{1}{S_2}{d_{S_2}(s_2)}}$.\\
Somit gilt $\immage{f} = \Tensor{T}{S_2}{d_{S_2}(Q)} = \immage{f'}$.\\
$\Rightarrow$ die gesuchte Sequenz ist exakt.
\end{proof}

\chapter{Aufgaben}
\begin{itemize}
\item Aufgabe A6.7 aus \ModulsOfDifferenzials ist \cref{Lokalisierung als Kolimes}.
\end{itemize}
\end{document}
