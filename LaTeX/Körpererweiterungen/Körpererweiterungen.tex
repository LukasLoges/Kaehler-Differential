\documentclass[10pt,a4paper]{report}
\usepackage[utf8]{inputenc}
\usepackage{amsmath}
\usepackage{amsthm}
\usepackage{amsfonts}
\usepackage{amssymb}
\usepackage{color}
\usepackage{tikz-cd}
\usepackage{calc}
\usepackage{setspace}
\usepackage[german]{babel}
\usetikzlibrary{babel}
\usepackage{cleveref}

\newcommand{\comment}[1]{}
\renewcommand{\baselinestretch}{1.1}

\newcommand{\ModulsOfDifferenzials}{David Eisenbud 1994}
\newcommand{\Algebra}{Christian Karpfinger, Kurt Meyberg 2009}

\newcounter{Aussage}[chapter]

\newtheorem{satz}[Aussage]{Satz}
\newtheorem{theorem}[Aussage]{Theorem}
\newtheorem{prop}[Aussage]{Proposition}
\newtheorem{korrolar}[Aussage]{Korrolar}
\newtheorem{lemma}[Aussage]{Lemma}
\newtheorem{bem}[Aussage]{Bemerkung}
\newtheorem{definition}[Aussage]{Definition}
\newtheorem{bsp}[Aussage]{Beispiel}

\newcommand{\functionfront}[3]{\nolinebreak{#1:#2 \longrightarrow #3}}
\newcommand{\functionback}[3]{\nolinebreak{#1:#2 \longmapsto #3}}
\newcommand{\function}[5]{\nolinebreak{#1:#2 \longrightarrow #3 \, , \, #4 \longmapsto #5}}
\newcommand{\infunctionfront}[3]{\nolinebreak{#1:#2 \hookrightarrow #3}}
\newcommand{\divR}[2]{\Omega_{#1/#2}}
\newcommand{\divf}[1]{d_{#1}}
\comment{\newcommand{\divf}[2][]{d_{#1}}}
\newcommand{\Tensor}[3]{#1 \otimes_{#2} #3}
\newcommand{\tensor}[3]{#1 \otimes #3}
\newcommand{\lok}[2]{#1 [#2^{-1}]}
\newcommand{\loke}[3]{(\frac{#1}{#2})_{_{#3}}}
\comment{\newcommand{\loke}[3]{(#1,#2)_{mod\sim_{#3}}}}

\newcommand{\colimes}[0]{\lim\limits_{ \longrightarrow }}
\newcommand*{\defeq}{\mathrel{\vcenter{\baselineskip0.5ex \lineskiplimit0pt
                     \hbox{\scriptsize.}\hbox{\scriptsize.}}}%
                     =}
\newcommand*{\defeqr}{= \mathrel{\vcenter{\baselineskip0.5ex \lineskiplimit0pt
                     \hbox{\scriptsize.}\hbox{\scriptsize.}}}}

\newcommand*{\defshow}{\stackrel{!}{=}}
\newcommand{\kernel}[1]{kern(#1)}
\newcommand{\immage}[1]{im(#1)}
\newcommand{\Verz}[1]{\langle #1 \rangle}

\begin{document}
\chapter{Körpererweiterungen}
\section{Einführung in die Körpererweiterungen}
\textcolor{blue}{\textbf{Definition Transzenddenzbasis} \textit{[vlg. Anhang A1 \ModulsOfDifferenzials]}}
\begin{def}\comment{\label{Definition Transzenddenzbasis}}
Sei $L \supset k$ eine Körpererweiterung. Dann definieren wir:
\begin{itemize}
\item[•] Eine endliche Teilmenge $\lbrace l_1, \dots ,l_n \rbrace \subseteq L$ heißt \underline{algebraisch abhängig} über $k$, falls gilt:
\begin{gather*}
\exists P(x_1, \dots , x_n) \in k[x_1,\dots,x_n] : \, P(l_1, \dots , l_n) = 0
\end{gather*}
\item[•] Eine endliche Teilmengen $\lbrace l_1, \dots ,l_n \rbrace \subseteq L$ heißt \underline{algebraisch unabhängig} über $k$, falls gilt:
\begin{gather*}
\forall P(x_1, \dots , x_n) \in k[x_1,\dots,x_n] : \, P(l_1, \dots , l_n) \neq 0
\end{gather*}
\item[•] Eine Teilmenge $B \subseteq L$ heißt \underline{transzendent} über $k$, falls jede ihrer endlichen Teilmengen $\lbrace b_1, \dots , b_n \rbrace$ algebraisch unabhängig über $k$ ist.
\item[•] Eine Teilmenge $B \subseteq L$ ist eine \underline{Transzendenzbasis} von $L$ über $k$, falls sie transzendent über $k$ und die Körpererweiterung $L \supset k(B)$ algebraisch ist. 
\end{itemize}
\end{def}


\ \\
\textcolor{blue}{\textbf{Transzendenzbasis ist maximale transzendente Menge} \textit{[Lemma 22.1 \Algebra]}}
\begin{lemma}\label{Transzendenzbasis ist maximale transzendente Menge}
Sei $L \supset k$ ein Körpererweiterung und $B \subseteq L$ eine über $k$ transzendente Teilmenge. Dann gilt:\\
B ist genau dann eine Transzendenzbasis von $L$ über $k$, wenn $B$ bezüglich der Inklusion ein maximales Element der Menge aller über $k$ transzendenten Elemente aus $L$ ist.
\end{lemma}


\ \\
\begin{bem}
Für jede Körpererweiterung $L \subseteq k$ existiert eine Transzendenzbasis $B \subseteq L$ von $L$ über k.
\end{bem}


\ \\
\textcolor{blue}{\textbf{De} \textit{[]}}
\textcolor{red}{
\begin{itemize}
\item[\underline{\textbf{Erinnerung:}}] Eine Algebraische Körpererweiterung $L \supset k$ heißt \underline{seperabel}, falls für alle $\alpha \in L$ das Minimalpolynom $f(x) \in k[x]$ von $\alpha$ über $L[x]$ in Linearfaktoren zerfällt.
\end{itemize}
}
\begin{definition}\label{Definition Seperabel}
Sei $L \supset k$ eine Körpererweiterung. Dann definieren wir:
\begin{itemize}
\item $L$ ist \underline{seperabel generiert} über $k$, falls eine Transzendenzbasis $B$ von $L$ über $k$ existiert, sodass $L/k(B)$ eine seperable Körpererweiterung ist.
\item $k$ ist \underline{seperabel} über $k$, falls jeder über $k$ endlich genierte Teilkörper von $L$ über $k$ seperabel generiert ist.
\end{itemize}
\end{definition}


\ \\
\begin{definition}
Sei $k$ ein Körper mit charakteristik p und sei weiter $L/k$ eine Körpererweiterung. Dann definieren wir:
\begin{itemize}
\item Eine endliche Teilmenge $B \subseteq L$ heißt p-Basis von $L$ über $k$, falls $W \defeq \lbrace \prod_{b \in B} b^i \vert i < p \rbrace$ eine Vektorraumbasis von K über $k * K^p$ bildet.
\end{itemize}
\end{definition}

\section{Differential von Körpererweiterungen}

\textcolor{blue}{\textbf{Definition der Differenzialbasis} \textit{[vlg. Chapter 16.5 \ModulsOfDifferenzials]}}
\begin{definition}\comment{\label{Definition der Differenzialbasis}}
Sei $L \supset k$ eine Körpererweiterung. Dann nennen wir eine Teilmenge $\lbrace b_i \rbrace_{i \in \Lambda} \subseteq L$ eine \underline{Differenzialbasis} von $L$ über $k$, falls $\lbrace \divf{K}(b_i)\rbrace_{i \in \Lambda}$ eine Vektorraumbasis von $\divR{L}{R}$ über $L$ ist.
\end{definition}


\ \\
\textcolor{blue}{\textbf{Differential von rationalen Funktionen 1} \textit{[vlg. Chapter 16.5 \ModulsOfDifferenzials]}}
\begin{bsp}\label{Differential von rationalen Funktionen 1}
Sei $k$ ein Körper und $L = k(\lbrace x_i \rbrace_{i \in \lbrace 1,\dots,n \rbrace})$ der Körper der rationalen Funktionen in $n$ Varablen über $k$.\\
Dann gilt:
\begin{gather*}
\divR{L}{k} \simeq L \Verz{\divf{k[x_1,\dots x_n]}(x_i)}
\end{gather*}
Insbesondere ist $\lbrace x_i \rbrace_{i \in \lbrace 1,\dots,n \rbrace}$ eine Differenzialbasis von $\divR{L}{k}$.
\end{bsp}


\ \\
\textcolor{blue}{\textbf{Differential von rationalen Funktionen 2} \textit{[Aufgabe 16.6 \ModulsOfDifferenzials]}}
\begin{korrolar}\label{Differential von rationalen Funktionen 2}
Sei $k$ ein Körper und $L \supset k$ eine Körpererweiterung und $T = L(\lbrace x_i \rbrace_{i \in \lbrace 1,\dots,n \rbrace})$ der Körper der rationalen Funktionen in $n$ Varablen über $L$. Dann gilt:
\begin{gather*}
\divR{T}{k} \simeq (\Tensor{T}{L}{\divR{L}{R}}) \oplus \bigoplus_{i \in \lbrace 1,\dots,n \rbrace} T \Verz{\divf{T}(x_i)}
\end{gather*}
\end{korrolar}


\ \\
\textcolor{blue}{\textbf{Cotangent Sequenz von Koerpern 1} \textit{[Aufgabe 16.6 \ModulsOfDifferenzials]}}
\begin{bem}\label{Cotangent Sequenz von Koerpern 1}
Sei $L \supset k$ eine Körpererweiterung und $T = L(x_1, \dots ,x_n)$ der Körper der rationalen Funktionen in $n$ Variablen über $L$. Dann ist die COTANGENT SEQUENZ \textit{(\cref{Cotangent Sequenz})} von $k \hookrightarrow L \hookrightarrow T$ eine kurze Exakte Sequenz:
\begin{center}
\begin{tikzcd}
0 \arrow[r] & \Tensor{T}{L}{\divR{L}{k}} \arrow[r] & \divR{T}{k} \arrow[r] & \divR{T}{L} \arrow[r] & 0
\end{tikzcd}
\end{center}
Im Genauen ist $\function{\varphi}{\Tensor{T}{L}{\divR{L}{k}}}{\divR{T}{k}}{\tensor{t}{L}{\divf{L}(l)}}{t \cdot \divf{T}(l)}$ injektiv.
\end{bem}


\ \\
\textcolor{blue}{\textbf{Aufbaulemma Koerperdifferenzial} \textit{[vlg. Lemma 16.15 \ModulsOfDifferenzials]}}
\begin{lemma}\label{Aufbaulemma Koerperdifferenzial}
Sei $L \subset T$ eine seperable und algebraische Körpererweiterung und $R \longrightarrow L$ ein Ringhomomorphismus. Dann gilt:
\begin{gather*}
\divR{T}{R} = \Tensor{T}{L}{\divR{L}{R}}
\end{gather*}
Insbesondere ist in diesem Fall die COTANGENT SEQUENZ \textit{(\cref{Cotangent Sequenz})} von $R \rightarrow L \hookrightarrow T$ eine kurze Exakte Sequenz:
\begin{center}
\begin{tikzcd}
0 \arrow[r] & \Tensor{T}{L}{\divR{L}{R}} \arrow[r] & \divR{T}{R} \arrow[r] & \divR{T}{L} \arrow[r] & 0
\end{tikzcd}
\end{center}
\end{lemma}


\ \\
\textcolor{blue}{\textbf{Transzendenzbasis ist Differenzialbasis} \textit{[vlg. Theorem 16.4 \ModulsOfDifferenzials]}}
\begin{theorem}\comment{\label{Transzendenzbasis ist Differenzialbasis}}
Sei $T \supset k$ eine seperabel generierte Körpererweiterung und $B = \lbrace b_i \rbrace_{i \in \Lambda} \subseteq T$. Dann ist $B$ genau dann eine Differenzialbasis von $T$ über $k$, falls eine der folgedenen Bedingungen erfüllt ist:
\begin{itemize}
\item[\textbf{1.}] char(k) = 0 und $B$ ist eine Transzendenzbasis von $T$ über $k$.
\item[\textbf{2.}] char(k) = p und $B$ ist eine p-Basis von $T$ über $k$.
\end{itemize}
\end{theorem}

\end{document}