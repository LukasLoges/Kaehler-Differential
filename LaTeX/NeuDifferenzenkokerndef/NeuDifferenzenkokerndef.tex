\documentclass[10pt,a4paper]{report}
\usepackage[utf8]{inputenc}
\usepackage{amsmath}
\usepackage{amsthm}
\usepackage{amsfonts}
\usepackage{amssymb}
\usepackage{color}
\usepackage{tikz-cd}
\usepackage{calc}
\usepackage{setspace}
\usepackage[german]{babel}
\usetikzlibrary{babel}
\usepackage{cleveref}

\newcommand{\comment}[1]{}
\renewcommand{\baselinestretch}{1.1}

\newcommand{\ModulsOfDifferenzials}{David Eisenbud 1994}
\newcommand{\Algebra}{Christian Karpfinger, Kurt Meyberg 2009}

\newcounter{Aussage}[chapter]

\newtheorem{satz}[Aussage]{Satz}
\newtheorem{theorem}[Aussage]{Theorem}
\newtheorem{prop}[Aussage]{Proposition}
\newtheorem{korrolar}[Aussage]{Korrolar}
\newtheorem{lemma}[Aussage]{Lemma}
\newtheorem{bem}[Aussage]{Bemerkung}
\newtheorem{definition}[Aussage]{Definition}
\newtheorem{bsp}[Aussage]{Beispiel}

\newcommand{\functionfront}[3]{\nolinebreak{#1:#2 \longrightarrow #3}}
\newcommand{\functionback}[3]{\nolinebreak{#1:#2 \longmapsto #3}}
\newcommand{\function}[5]{\nolinebreak{#1:#2 \longrightarrow #3 \, , \, #4 \longmapsto #5}}
\newcommand{\infunctionfront}[3]{\nolinebreak{#1:#2 \hookrightarrow #3}}
\newcommand{\divR}[2]{\Omega_{#1/#2}}
\newcommand{\divf}[1]{d_{#1}}
\comment{\newcommand{\divf}[2][]{d_{#1}}}
\newcommand{\Tensor}[3]{#1 \otimes_{#2} #3}
\newcommand{\tensor}[3]{#1 \otimes #3}
\newcommand{\lok}[2]{#1 [#2^{-1}]}
\newcommand{\loke}[3]{(\frac{#1}{#2})_{_{#3}}}
\comment{\newcommand{\loke}[3]{(#1,#2)_{mod\sim_{#3}}}}

\newcommand{\colimes}[0]{\lim\limits_{ \longrightarrow }}
\newcommand*{\defeq}{\mathrel{\vcenter{\baselineskip0.5ex \lineskiplimit0pt
                     \hbox{\scriptsize.}\hbox{\scriptsize.}}}%
                     =}
\newcommand*{\defeqr}{= \mathrel{\vcenter{\baselineskip0.5ex \lineskiplimit0pt
                     \hbox{\scriptsize.}\hbox{\scriptsize.}}}}

\newcommand*{\defshow}{\stackrel{!}{=}}
\newcommand{\kernel}[1]{kern(#1)}
\newcommand{\immage}[1]{im(#1)}
\newcommand{\Verz}[1]{\langle #1 \rangle}


\begin{document}
\textcolor{blue}{\textbf{DifferenzkokernUndKoproduktDef}}
\begin{definition}\label{DifferenzkokernUndKoproduktDef}
\end{definition}


\ \\
\textcolor{blue}{\textbf{NeuDifferenzenkokerndef''}}
\begin{bem}\label{NeuDifferenzenkokerndef''} \textit{[Wikipedia]}
Sei $\mathcal{A}$ eine Kategorie. Sei weiter $C_1,C_2 \in Obj_{\mathcal{A}}$ und $f,g \in Hom_{\mathcal{A}}(C_1,C_2)$.\\
Im Falle der Existenz ist der Differnenzenkokern von $f,g$ nach \cref{DifferenzkokernUndKoproduktDef} durch ein Objekt $C \in Obj_{\mathcal{A}}$ und einen Morphismus $\psi = \lbrace \psi_{C_1}, \psi_{C_2}\rbrace$ gegeben, wobei gilt:
\begin{gather*}
\psi_{C_2} = f \circ \psi_1 = g \circ \psi_2
\end{gather*}
Wir sehen, dass $\psi$ eindeutig durch $q \defeq \psi_2 \in Hom_{\mathcal{A}}(C_1,C_2)$ gegeben ist. Der Differnzenkokern ist also eindeutig durch $(C \in obj_\mathcal{A},q \in Hom_{\mathcal{A}}(C_1,C_2))$ gegeben, wobei $q$ folgenden Eigenschaften besitzt:
\begin{center}
Es gilt $f \circ q = g \circ g$ und\\
für alle $C \in Obj_{A}$ und $q' \in Hom_{\mathcal{A}}$ mit $f \circ q' = g \circ q'$ existiert genau ein $\varphi \in Hom_{\mathcal{A}}$, mit $q \circ \varphi = q'$:\\
\ \\
\begin{tikzcd}
C_1 \arrow[r, "{f,g}"] \arrow[r] & C_2 \arrow[r, "q"] \arrow[rd, "q'"] & C \arrow[d, "\exists !\varphi", dashed] \\
                                 &                                     & C'                                     
\end{tikzcd}
\end{center}
Wenn wir fortan vom Differenzkokern sprechen meinen wir damit das Paar $(C,q)$.
\end{bem}


\ \\
\textcolor{blue}{\textbf{NeuDifferenzenkokerndef'}}
\begin{lemma}\label{NeuDifferenzenkokerndef'} \textit{[Wikipedia]}
\comment{
\item Der Differenzkokern von $f,g \in Hom_{\mathcal{A}}(C_1,C_2)$ wird durch $\colimes \mathcal{C}$ definiert,
wobei $\lbrace C_1,C_2 \rbrace$ die Objekte und $ \lbrace f,g \rbrace$ zusammen mit den Identitätsabbildungen die Morphismen von $\mathcal{C}$ sind.
}


\begin{itemize}
\item[(1.)]
Sei weiter $C \in Obj_{\mathcal{C}}$ und $\functionfront{q}{C_2}{C}$ mit $f \circ q = g \circ q$ und der folgenden universellen Eigenschaft gegeben:\\
\begin{center}
Für alle $C' \in Obf_{\mathcal{C}}$ und $q' \in Hom_{\mathcal{A}}(C_2,C')$, mit $f \circ q = g \circ q$ erfüllen existiert genau ein $\varphi \in Hom_{\mathcal{A}}$, mit $q \circ \varphi = q'$:\\
\ \\
\begin{tikzcd}
C_1 \arrow[r, "{f,g}"] \arrow[r] & C_2 \arrow[r, "q"] \arrow[rd, "q'"] & C \arrow[d, "\exists !\varphi", dashed] \\
                                 &                                     & C'                                     
\end{tikzcd}
\end{center}
Dann existiert den Differenzkokern von $f,g$ und ist durch $C$ und den Morphismus $\functionfront{\psi}{\mathcal{C}}{C}$ mit $\psi_{C_1} = f\circ q$ und $\psi_{C_2} = q$ gegeben.
\item[(2.)] Falls der Differenzkokern $(C,\psi)$ von $f,g$ existiert, so existiert auch ein eindeutiges $q \in Hom_{\mathcal{A}}(C_2,C)$, welches die in (1.) beschriebenen Eigenschaften erfüllt.
\end{itemize}
\end{lemma}
\begin{proof}
\end{proof}


\ \\
\textcolor{blue}{\textbf{NeuDifferenzenkokerndef} \textit{[vlg. Wikipedia aber eigener Beweis]}}
\begin{lemma}\label{NeuDifferenzenkokerndef} Sei $\mathcal{A}$ eine Kategorie mit $C_1,C_2 \in Hom_{\mathcal{A}}(C_1,C_2)$, so sind folgende Formulierungen äquivalent zur Definition des Differenzkokern`s $T \defeq \colimes \mathcal{C}$
\begin{itemize}
\item[1.] Es existiert ein Morphismus $\functionfront{\psi}{\mathcal{C}}{T}$, mit der Eigenschaft, dass für alle Morphismen $\functionfront{\psi '}{\mathcal{C}}{T'}$ genau ein $\varphi \in Hom_{\mathcal{A}}(T,T')$ mit $\varphi \circ \psi = \psi '$ existiert.
\item[2.] Es existiert ein $q \in Hom_{\mathcal{A}}(C_2,T)$ mit $q \circ f = q \circ g$ und der Eigenschaft, dass für alle Morphismen $q' \in Hom_{\mathcal{A}}(C_2,Z)$ mit $q' \circ f = q' \circ g$ genau ein $\varphi \in Hom_{\mathcal{A}}(T,T')$ mit $\varphi \circ q = q'$ existiert.
\begin{center}
\begin{tikzcd}
C_1 \arrow[r, "{f,g}"] \arrow[r] & C_2 \arrow[r, "q"] \arrow[rd, "q'"] & T \arrow[d, "\exists !\varphi", dashed] \\
                                 &                                     & T'                                     
\end{tikzcd}
\end{center}
\end{itemize}
\end{lemma}
\begin{proof}
\textit{1.} ist offensichtlich eine Ausformulierung der Einführung des Kolimes aus \cref{DifferenzkokernUndCoproduktDef}, zeige also im folgenden noch die Äquivalenz von \textit{1.} und \textit{2.}
\begin{itemize}
\item \underline{1 $\Rightarrow$ 2:}
\begin{itemize}
\item[] Da $\functionfront{\psi}{\mathcal{C}}{T}$ ein Morphismus ist, gilt für $\lbrace f,g \rbrace = Hom_{\mathcal{C}}(C_1,C_2)$:\\ $\psi_{C_1} = \psi_{C_2} \circ f = \psi_{C_1} \circ \psi_{C_2}$, setze also 
 $q  \defeq \psi_{C_2}$.
\item[] Sei nun $q' \in Hom_{\mathcal{A}}(C_2,T)$ mit der Eigenschaft $q' \circ f = q' \circ g$ gegeben:\\
 Definiere den Morphismus $\functionfront{\psi '}{\mathcal{C}}{T}$ als $\lbrace \psi_1 = q' \circ f , \psi_2 = q' \rbrace$,  somit folgt direkt aus der Universellen Eigenschaft von $\psi$, dass genau ein $\varphi \in Hom_{A}(C_2,T)$ existiert, mit $ \varphi \circ q = q '$.
\end{itemize}
\item \underline{2 $\Rightarrow$ 1:}
\begin{itemize}
\item[] Definiere $\functionfront{\psi }{\mathcal{C}}{T}$ als $\lbrace \psi_1 = q \circ f , \psi_2 = q \rbrace$.
Durch die Eigenschaft von $q$ gilt $\psi_{C_1} = \psi_{C_2} \circ f = \psi_{C_2} \circ g$.
\item[] Sei nun $\functionfront{\psi '}{\mathcal{C}}{\mathcal{A}}$ ein beliebiger Morphismus.\\
Definiere $d' \defeq \psi '$, somit existiert durch die Eigenschaft von $d$ genau ein $\varphi \in Hom_{\mathcal{A}}(C_2,T)$ mit $\varphi \circ q = q'$.
\begin{gather*}
\Rightarrow \varphi \circ \psi_2 = \psi '_2 \\
\textit{und }\varphi \circ \psi_1 = \varphi \circ \psi_2 \circ f = \varphi \circ \psi '_2 \circ f = \varphi \circ \psi '_1
\end{gather*}
\end{itemize}
\end{itemize}
\end{proof}
Wenn im weiteren Verlauf von dem Differenzkokern zweier Homomorphismen $\functionfront{f,g}{C_1}{C_2}$ gesprochen wird, meinen wir damit den Homomorphismus $\functionfront{q}{C_2}{T}$ aus \cref{NeuDifferenzenkokerndef}.
\end{document}