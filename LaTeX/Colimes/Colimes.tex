\documentclass[10pt,a4paper]{report}
\usepackage[utf8]{inputenc}
\usepackage{amsmath}
\usepackage{amsthm}
\usepackage{amsfonts}
\usepackage{amssymb}
\usepackage{tikz-cd}
\usepackage{calc}
\usepackage{setspace}
\usepackage[german]{babel}
\usetikzlibrary{babel}
\usepackage{cleveref}

\newcommand{\comment}[1]{}
\renewcommand{\baselinestretch}{1.1}


\newcounter{Aussage}[chapter]

\newtheorem{satz}[Aussage]{Satz}
\newtheorem{theorem}[Aussage]{Theorem}
\newtheorem{prop}[Aussage]{Proposition}
\newtheorem{korrolar}[Aussage]{Korrolar}
\newtheorem{lemma}[Aussage]{Lemma}
\newtheorem{bem}[Aussage]{Bemerkung}
\newtheorem{definition}[Aussage]{Definition}

\newcommand{\functionfront}[3]{\nolinebreak{#1:#2 \longrightarrow #3}}
\newcommand{\functionback}[3]{\nolinebreak{#1:#2 \longmapsto #3}}
\newcommand{\function}[5]{\nolinebreak{#1:#2 \longrightarrow #3 \, , \, #4 \longmapsto #5}}
\newcommand{\divR}[2]{\Omega_{#1/#2}}
\newcommand{\Tensor}[3]{#1 \otimes_{#2} #3}
\newcommand{\tensor}[3]{#1 \otimes #3}
\newcommand{\lok}[2]{#1 [#2^{-1}]}
\newcommand{\loke}[3]{(#1,#2)_{mod\sim_{#3}}}

\newcommand{\colimes}[0]{\lim\limits_{ \longrightarrow }}
\newcommand{\infunctionfront}[3]{\nolinebreak{#1:#2 \hookrightarrow #3}}
\newcommand*{\defeq}{\mathrel{\vcenter{\baselineskip0.5ex \lineskiplimit0pt
                     \hbox{\scriptsize.}\hbox{\scriptsize.}}}%
                     =}
\newcommand*{\defshow}{\stackrel{!}{=}}
\newcommand{\kernel}[1]{kern(#1)}
\newcommand{\immage}[1]{im(#1)}

\begin{document}
Zu \textit{\textbf{2.}}:\\
Zeige, dass $\functionfront{q}{S_2}{S_2/Q}$ die in \cref{NeuDifferenzenkokerndef} eingeführten Eigenschaften des Differenzkokern`s  besitzt.
\begin{gather*}
q \circ f = q \circ g \text{ gilt, da } \kernel{q} = Q = \lbrace f(x) - g(x)\mid x \in C_2 \rbrace.
\end{gather*}
Sei nun eine Funktion $q' \in Hom_{\mathcal{A}}(S_2,T')$ mit $q' \circ f = q' \circ$ gegeben.\\
Somit gilt $q' \circ (f - g) = 0$, wodurch $Q$ ein Untermodul von $Q' \defeq \kernel{q'}$ ist.\\ Mit dem Isomorphiesatz \comment{HOMOMORPHIESATZ [kommutative Algebra 2.10]} für R-Algebren erhalten wir:
\begin{gather*}
 \nolinebreak{S_2/Q' \simeq (S_2/Q)/(Q'/Q))}.\\
\end{gather*}
Somit ist $\function{q'}{S_2}{(S_2/Q)/(Q'/Q))}{y}{[y]'}$ eine isomorphe Darstellung von $\functionfront{q'}{S_2}{T'}$\\
\begin{gather*}
\Rightarrow \exists ! \function{\varphi}{S_2/Q}{(S_2/Q)/(Q'/Q)}{[y]}{[y]'}\textit{ mit }(\varphi \circ q) = q'.
\end{gather*}
Also ist $\functionfront{q}{S_2}{S_2/Q}$ der bis auf Isomorphie eindeutig bestimmte Differenzkokern von $f$ und $g$.
\end{document}
