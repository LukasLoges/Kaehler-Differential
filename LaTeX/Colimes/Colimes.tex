\documentclass[10pt,a4paper]{report}
\usepackage[utf8]{inputenc}
\usepackage{amsmath}
\usepackage{amsthm}
\usepackage{amsfonts}
\usepackage{amssymb}
\usepackage{tikz-cd}
\usepackage{calc}
\usepackage{setspace}
\usepackage[german]{babel}
\usetikzlibrary{babel}
\usepackage{cleveref}

\newcommand{\comment}[1]{}
\renewcommand{\baselinestretch}{1.1}


\newcounter{Aussage}[chapter]

\newtheorem{satz}[Aussage]{Satz}
\newtheorem{theorem}[Aussage]{Theorem}
\newtheorem{prop}[Aussage]{Proposition}
\newtheorem{korrolar}[Aussage]{Korrolar}
\newtheorem{lemma}[Aussage]{Lemma}
\newtheorem{bem}[Aussage]{Bemerkung}
\newtheorem{definition}[Aussage]{Definition}

\newcommand{\functionfront}[3]{\nolinebreak{#1:#2 \longrightarrow #3}}
\newcommand{\functionback}[3]{\nolinebreak{#1:#2 \longmapsto #3}}
\newcommand{\function}[5]{\nolinebreak{#1:#2 \longrightarrow #3 \, , \, #4 \longmapsto #5}}
\newcommand{\divR}[2]{\Omega_{#1/#2}}
\newcommand{\Tensor}[3]{#1 \otimes_{#2} #3}
\newcommand{\tensor}[3]{#1 \otimes #3}
\newcommand{\lok}[2]{#1 [#2^{-1}]}
\newcommand{\loke}[3]{(#1,#2)_{mod\sim_{#3}}}

\newcommand{\colimes}[0]{\lim\limits_{ \longrightarrow }}
\newcommand{\infunctionfront}[3]{\nolinebreak{#1:#2 \hookrightarrow #3}}
\newcommand*{\defeq}{\mathrel{\vcenter{\baselineskip0.5ex \lineskiplimit0pt
                     \hbox{\scriptsize.}\hbox{\scriptsize.}}}%
                     =}
\newcommand*{\defshow}{\stackrel{!}{=}}
\newcommand{\kernel}[1]{kern(#1)}
\newcommand{\immage}[1]{im(#1)}

\begin{document}
Exersize A6.7
\begin{lemma}\comment{\label{Lokalisierung als Colimes}}
Seit $S$ eine $R-Algebra$ und $U \subseteq S$ multiplikativ abgeschlossen.
Dann gilt:
\begin{gather*}
 S[U^{-1}] = \colimes (\infunctionfront{\mathcal{F}}{\mathcal{B}}{(R-Algebren)})
\end{gather*}
Wobei $\mathcal{B}$ aus den Objekten $\lbrace \lok{S}{t} \vert t \in U \rbrace$ und den Morphismen
$\lok{S}{t} \longrightarrow \lok{S}{tt'}, \loke{s}{t^n}{t} \longmapsto \loke{st'^n}{t^nt'^n}{(tt')} \,
\forall t,t' \in U$ besteht.
\end{lemma}
\begin{proof}
Sei $\functionfront{\psi}{\mathcal{F}}{A}$ der Colimes von $\mathcal{F}$. Zeige $\lok{S}{U} \simeq A$, definiere dazu:
\begin{gather*}
\functionfront{\psi'}{\mathcal{F}}{\lok{S}{U}}\\
\function{\psi'_{\lok{S}{t}}}{\lok{S}{t}}{\lok{S}{t}}{\loke{s}{t^n}{t}}{\loke{s}{t^n}{U}}
\end{gather*}
$\psi'$ ist ein Morphismus, da für beliebige $t,t' \in U$ und $s \in S$ gilt:
$$\loke{s}{t^n}{U} = \loke{st'^n}{t^nt'^n}{U}$$
Durch die Universelle Eigenschaft des Colimes, erhalten wir den Homomorphismus $\functionfront{\varphi}{A}{\lok{S}{U}}$.
\begin{center}
\begin{tikzcd}
            & \mathcal{F} \arrow[rd, "\psi"] \arrow[ld, "\psi'"'] &                                            \\
{S[U^{-1}]} &                                                     & A \arrow[ll, "\exists ! \varphi"', dashed]
\end{tikzcd}
\end{center}
Für $\functionfront{\phi}{S[U^{-1}]}{A}$ benötigen wir kleinere Vorüberlegungen.\\
Zunächst können wir jedes Element $(s,u)_{mod\sim_{U}} \in \lok{S}{U}$ als $\psi_{\lok{S}{u}}(\loke{s}{u}{U})$ schreiben.\\
Weiter gilt für alle $s_1,s_2 \in S , \, t_1,t_2 \in U$: 
\begin{align*}
\psi'_{\lok{S}{t}}(\loke{s_1}{t_1}{t}) = \psi'_{\lok{S}{t}}(\loke{s_2}{t_2}{t})\\
\Rightarrow  \exists u \in U: (s_1t_1 - s_2t_2) \cdot u = 0\\
\Rightarrow  \loke{s_1u}{t_1u}{tu} = \loke{s_2u}{t_2u}{tu}\\
\Rightarrow  \psi_{\lok{S}{t}}(\loke{s_1}{t_1}{t}) = \psi_{\lok{S}{t}}(\loke{s_2}{t_2}{t})
\end{align*}
Mit diesem Wissen können wir den R-Algebra-Homomorphismus $\functionfront{\phi}{\lok{S}{U}}{A}$ definieren:
\begin{gather*}
\function{\phi}{\lok{S}{U}}{A}{\psi'_{\lok{S}{t}}(\loke{s}{t}{t})}{\psi_{\lok{S}{t}}(\loke{s}{t}{t})}
\end{gather*}
$\phi \circ \varphi = id_A$ ergibt sich direkt aus der Universellen Eigenschaft des Colimes:
\begin{center}
\begin{tikzcd}
  & \mathcal{F} \arrow[rd, "\psi"] \arrow[ld, "\psi"'] &                                                              \\
A &                                                    & A \arrow[ll, "\exists ! id_A = \phi \circ \varphi"', dashed]
\end{tikzcd}
\end{center}
Für $\varphi \circ \phi = id_{\lok{S}{U}}$ wähle beliebige $s \in S , t \in U$, für diese gilt:
\begin{gather*}
(\varphi \circ \phi)(\psi(\loke{s}{t}{t})) =
 \varphi (\psi'(\loke{s}{t}{t}) =
  \psi(\loke{s}{t}{t})
\end{gather*}
Damit haben wir gezeigt, dass $\varphi,\phi$ Isomorphismen sind und somit $A \simeq \lok{S}{U}$ gilt.\\
Da der Colimes bis auf Isomorphie eindeutig ist, definiere ab sofort $\lok{S}{U}$ als den eindeutigen Colimes von
 $\functionfront{\mathcal{F}}{\mathcal{B}}{(R-Algebren)}$.
\end{proof}
\end{document}
