\documentclass[10pt,a4paper]{report}
\usepackage[utf8]{inputenc}
\usepackage{amsmath}
\usepackage{amsthm}
\usepackage{amsfonts}
\usepackage{amssymb}
\usepackage{color}
\usepackage{tikz-cd}
\usepackage{calc}
\usepackage{setspace}
\usepackage[german]{babel}
\usetikzlibrary{babel}
\usepackage{cleveref}

\newcommand{\comment}[1]{}
\renewcommand{\baselinestretch}{1.1}

\newcommand{\ModulsOfDifferenzials}{David Eisenbud 1994}

\newcounter{Aussage}[chapter]

\newtheorem{satz}[Aussage]{Satz}
\newtheorem{theorem}[Aussage]{Theorem}
\newtheorem{prop}[Aussage]{Proposition}
\newtheorem{korrolar}[Aussage]{Korrolar}
\newtheorem{lemma}[Aussage]{Lemma}
\newtheorem{bem}[Aussage]{Bemerkung}
\newtheorem{definition}[Aussage]{Definition}
\newtheorem{bsp}[Aussage]{Beispiel}

\newcommand{\functionfront}[3]{\nolinebreak{#1:#2 \longrightarrow #3}}
\newcommand{\functionback}[3]{\nolinebreak{#1:#2 \longmapsto #3}}
\newcommand{\function}[5]{\nolinebreak{#1:#2 \longrightarrow #3 \, , \, #4 \longmapsto #5}}
\newcommand{\infunctionfront}[3]{\nolinebreak{#1:#2 \hookrightarrow #3}}
\newcommand{\divR}[2]{\Omega_{#1/#2}}
\newcommand{\divf}[1]{d_{#1}}
\comment{\newcommand{\divf}[2][]{d_{#1}}}
\newcommand{\Tensor}[3]{#1 \otimes_{#2} #3}
\newcommand{\tensor}[3]{#1 \otimes #3}
\newcommand{\lok}[2]{#1 [#2^{-1}]}
\newcommand{\loke}[3]{(\frac{#1}{#2})_{_{#3}}}
\comment{\newcommand{\loke}[3]{(#1,#2)_{mod\sim_{#3}}}}

\newcommand{\colimes}[0]{\lim\limits_{ \longrightarrow }}
\newcommand*{\defeq}{\mathrel{\vcenter{\baselineskip0.5ex \lineskiplimit0pt
                     \hbox{\scriptsize.}\hbox{\scriptsize.}}}%
                     =}
\newcommand*{\defeqr}{= \mathrel{\vcenter{\baselineskip0.5ex \lineskiplimit0pt
                     \hbox{\scriptsize.}\hbox{\scriptsize.}}}}

\newcommand*{\defshow}{\stackrel{!}{=}}
\newcommand{\kernel}[1]{kern(#1)}
\newcommand{\immage}[1]{im(#1)}
\newcommand{\Verz}[1]{\langle #1 \rangle}


\begin{document}
\begin{satz}\label{Differenzial von Polynomalgebren 1}
\end{satz}
\textcolor{blue}{\textbf{Differenzial algebraischer Algebren ist Null} \textit{[Aufgabe 16.11 \ModulsOfDifferenzials]}}
\begin{bsp}\comment{\label{Differenzial algebraischer Algebren ist Null}}
Sei $K$ ein Körper mit $char$ und $T$ eine noethersche K-Algebra. Dann gilt:
\begin{gather*}
\divR{T}{K} = 0 \\
\Leftrightarrow \\
T = \oplus_{i \in \lbrace 1, \dots, n \rbrace} K(\alpha_i) \textit{Ist ein endliches Produkt algebraischer Körpererweiterungen} 
\end{gather*}
\end{bsp}
\begin{proof}
\ \\
\begin{itemize}
\item[\underline{{\glqq $\Rightarrow$ \grqq :}}]
Da $T$ noethersch ist, ist $T$ als Algebra über $K$ endlich erzeugt, also:
\begin{gather*}
\comment{
T = \bigoplus_{i \in \lbrace 1,\dots,n \rbrace} K[\alpha_i] \\
\text{Mit: }(\alpha)_{i \in \lbrace 1,\dots,n \rbrace} \text{ sind die Erzeuger von T} \\
\text{und } I = \bigoplus \subseteq \bigoplus_{i \in \lbrace 1,\dots,n \rbrace} K[\alpha_i\rbrace] \defeq T' \text{ ist ein Ideal.}
}
T = T'/I \\
Wobei: S \defeq \bigoplus_{i \in \lbrace 1,\dots,n \rbrace} K[\alpha_i] \\
\text{und } I \defeq \bigoplus_{i \in \lbrace 1,\dots,n \rbrace} I_i \subseteq T' \text{ ist ein Ideal, mit } \forall i \in \lbrace 1,\dots,n \rbrace: I_i \subseteq K[\alpha_i].\text{ ist ein Ideal.} \\
\text{Also gilt } T = \bigoplus_{i \in \lbrace 1,\dots,n \rbrace} K[\alpha_i]/I_i
\end{gather*}
Betrachte also jeweils für $i \in \lbrace 1,\dots,n \rbrace$ den Ring $K[\alpha_i]/I_i$.\\ 
\comment{dies geht dank \label{*Proposition 16.10}}
Setze dabei $I_i \neq K[\alpha_i]$ voraus, da sonst T isomorph einer Algebra T' wäre, welche
$(\alpha_j)_j \in \lbrace 1,\dots,n \rbrace \setminus \lbrace i \rbrace$ als Erzeuger hat.\\
Unterteile nun in die zwei möglichen Fälle $I = 0$ und $I \neq 0$:
\begin{itemize}
\item[\underline{\textbf{1.} $I_i = 0$:}] Da $\divR{T}{K} = 0$ gilt, muss  $K[\alpha_i] = 0$ gelten.\\
$\mathbb{A}$nnhame: $\alpha_i$ ist transzendent über K.
\begin{gather*}
\text{Dies bedeutet } K[\alpha_i] \simeq K[x] \\
\Rightarrow \divR{K[\alpha]}{K} \simeq K[x]\langle \divf{K[x]}(x) \rangle \neq 0 \textit{ (\cref{Differenzial von Polynomalgebren 1})}
\end{gather*}
Dies steht allerdings im Widerspruch zu $K[\alpha_i] = 0$. Folglich war unsere $\mathbb{A}$nnahme falsch und $\alpha$ ist algebraisch über K. Somit ist $K[\alpha] = K(\alpha)$ \comment{Nutze hier \label{*fuer a algebraisch gilt K[a] = K(a)}} eine algebraische Körpererweiterung.
\item[\underline{\textbf{2.} $I_i \neq 0$:}]
Zunächst sehen wir, dass $\alpha_i$ transzendent sein muss, da sonst $K[\alpha_i] = K(\alpha_i)$ ein Körper und somit $I_i = K(\alpha_i)$ gelten würde. Also ist $\alpha_i$ transzendent und es gilt:
\begin{gather*}
K[\alpha_i] \simeq K[x] \text{ und } I \simeq (f(x)) \text{ mit } f(x) \in K[x] \\
\Rightarrow K[\alpha_i] \simeq K[\beta_1, \dots \beta_n] = K(\beta_1, \dots \beta_n) \text{, wobei $\beta_1, \dots \beta_n$ die Nullstellen von f sind.}
\end{gather*}
Somit haben wir gezeigt, dass auch in diesem Fall $K[\alpha_i]/I_i$ eine Algebraische Körpererweiterung ist. 
\end{itemize}
\end{itemize}
\end{proof}
\end{document}
