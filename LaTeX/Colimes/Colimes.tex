\documentclass[10pt,a4paper]{report}
\usepackage[utf8]{inputenc}
\usepackage{amsmath}
\usepackage{amsthm}
\usepackage{amsfonts}
\usepackage{amssymb}
\usepackage{tikz-cd}
\usepackage{calc}
\usepackage{setspace}
\usepackage[german]{babel}
\usetikzlibrary{babel}
\usepackage{cleveref}

\newcommand{\comment}[1]{}
\renewcommand{\baselinestretch}{1.1}


\newcounter{Aussage}[chapter]

\newtheorem{satz}[Aussage]{Satz}
\newtheorem{theorem}[Aussage]{Theorem}
\newtheorem{prop}[Aussage]{Proposition}
\newtheorem{korrolar}[Aussage]{Korrolar}
\newtheorem{lemma}[Aussage]{Lemma}
\newtheorem{bem}[Aussage]{Bemerkung}
\newtheorem{definition}[Aussage]{Definition}

\newcommand{\functionfront}[3]{\nolinebreak{#1:#2 \longrightarrow #3}}
\newcommand{\functionback}[3]{\nolinebreak{#1:#2 \longmapsto #3}}
\newcommand{\function}[5]{\nolinebreak{#1:#2 \longrightarrow #3 \, , \, #4 \longmapsto #5}}
\newcommand{\divR}[2]{\Omega_{#1/#2}}
\newcommand{\Tensor}[3]{#1 \otimes_{#2} #3}
\newcommand{\tensor}[3]{#1 \otimes #3}
\newcommand{\lok}[2]{#1 [#2^{-1}]}
\newcommand{\loke}[3]{(#1,#2)_{mod\sim_{#3}}}

\newcommand{\colimes}[0]{\lim\limits_{ \longrightarrow }}
\newcommand{\infunctionfront}[3]{\nolinebreak{#1:#2 \hookrightarrow #3}}
\newcommand*{\defeq}{\mathrel{\vcenter{\baselineskip0.5ex \lineskiplimit0pt
                     \hbox{\scriptsize.}\hbox{\scriptsize.}}}%
                     =}
\newcommand*{\defshow}{\stackrel{!}{=}}
\newcommand{\kernel}[1]{kern(#1)}
\newcommand{\immage}[1]{im(#1)}

\begin{document}
\begin{theorem}\comment{\label{Lokalisierung des Kähler-Differenzials}}
Sei $S$ eine $R-Algebra$ und $U \subseteq S$ multiplikativ abgeschlossen.
Dann gilt:
\begin{gather*}
\divR{\lok{S}{U}}{R} \simeq \Tensor{\lok{S}{U}}{S}{\divR{S}{R}} \\
\text{Wobei } d_{\lok{S}{U}}(\loke{1}{u}{U}) \longmapsto -\tensor{\loke{1}{u^2}{U}}{S}{d_S(u)}
\end{gather*}
\end{theorem}
\begin{proof}
Wir wollen THEOREM16.8 \comment{\label{THEOREM16.8}} auf $\mathcal{B} = \lbrace \lok{S}{t} \vert t \in U \rbrace$ aus \cref{Lokalisierung von Algebren als Kolimes} anwenden.\\
Sei zunächst $t \in U$ beliebig. Zeige $\divR{\lok{S}{t}}{R} \simeq Tensor{\lok{S}{t}}{S}{\divR{S}{R}}$:
\begin{itemize}
\item[]
Verwende die Existenz der Isomorphismen $\functionfront{\alpha}{\lok{S}{t}}{S[x]/(tx -1)}$ und $\functionfront{\beta}{S[x]/(tx -1)}{\lok{S}{t}}$. Weiter gilt nach PROPOSITION16.6 \comment{\label{PROPOSITION16.6}}$\divR{S[x]}{R} \simeq \Tensor{S[x]}{S}{\divR{S}{R}} \oplus S[x]d_{S[x]}(x)$. Somit folgt:
\begin{gather*}
\divR{\lok{S}{t}}{R} \\
 \simeq (\Tensor{S[x]}{S}{\divR{S}{R}} \oplus S[x] dx) / ((tx - 1) \cdot d_{S[x]}(tx - 1)) \\
  \simeq (\Tensor{S[x]/(tx -1)}{S}{\divR{S}{R}} \oplus (S[x]/(tx - 1)) dx) / (td_{S[x]}(x) + xd_{S[x]}(t)) \\
   \simeq (\Tensor{\lok{S}{t}}{S}{\divR{S}{R}} \oplus (\lok{S}{t}) d_{S[x]}(x) / (td_{S[x]}(x) + xd_{S[x]}(t))
\end{gather*}
Zeige, dass sich jedes Element aus $(\Tensor{\lok{S}{t}}{S}{\divR{S}{R}} \oplus (\lok{S}{t}) d_{S[x]}(x) / (td_{S[x]}(x) + xd_{S[x]}(t))$ eindeutig durch ein Element aus $\Tensor{\lok{S}{t}}{S}{\divR{S}{R}} \oplus 0 $ darstellen lässt.\\
Sei dazu $[(\loke{s}{t^n}{t}d_{S}(s'') , \loke{s'}{t^{n'}}{t}d_{S[x]}(x) )]$ ein beliebiger Erzeuger von $(\Tensor{\lok{S}{t}}{S}{\divR{S}{R}} \oplus (\lok{S}{t}) dx) / (td_{S[x]}(x) + xd_{S[x]}(t))$. Somit gilt:
\begin{gather*}
\beta(x) = \loke{1}{t}{t} \textit{ und } sd_{S}(x) + xd_{S}(t) = 0 \\
 \Rightarrow d_{S}(x) = -\loke{1}{t^2}{t} \\
  \Rightarrow [(\loke{s}{t^n}{t}d_{S}(s'') , \loke{s'}{t^{n'}}{t}d_{S[x]}(x) )] \\
   = [( \loke{s}{t^n}{t}d_{S}(s'') - \loke{s'}{t^{n' +2}}{t} d_{s}(t), 0 )]
\end{gather*}
Damit habe wir ein Repräsentantensystem \comment{für $\Tensor{\lok{S}{t}}{S}{\divR{S}{R}} \oplus (\lok{S}{t}) d_{S[x]}x) / (td_{S[x]}(x) + xd_{S[x]}(t))$} gefunden und es folgt:
\begin{gather*} 
\divR{\lok{S}{t}}{R} \comment{\simeq \Tensor{S[x]/(tx -1)}{S}{\divR{S}{R}}} \simeq \Tensor{\lok{S}{U}}{S}{\divR{S}{R}}
\end{gather*}
\end{itemize}
\comment{
Zeige nun den Allgemeinen Fall $\divR{\lok{S}{U}}{R} \simeq \Tensor{\lok{S}{U}}{S}{\divR{S}{R}}$:
\begin{itemize}
\item[]
Wähle wie schon erwähnt $\mathcal{B} = \lbrace \lok{S}{t} \vert t \in U \rbrace$ wie in \cref{Lokalisierung von Algebren als Kolimes}, sodass $\colimes \mathcal{B} = \lok{S}{U}$ gilt.\\
Mit THEOREM16.8 \comment{\label{THEOREM16.8}} folgt somit:
\begin{gather*}
\divR{\lok{S}{U}}{T} = \colimes \mathcal{F}, für:\\
\function{\mathcal{F}}{\mathcal{B}}{(\lok{S}{U}-Module)}{\lok{S}{t}}{\tensor{\lok{S}{U}}{\lok{S}{t}}{\divR{\lok{S}{t}}{R}}}
(\functiofront{\varphi}{S}{S'}) \longmapsto (\fuctionfront{\tensor{1}{S}{D\varphi}}{}{})
\end{gather*}
\end{itemize}
}
\end{proof}
\end{document}
