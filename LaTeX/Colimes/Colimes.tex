\documentclass[10pt,a4paper]{report}
\usepackage[utf8]{inputenc}
\usepackage{amsmath}
\usepackage{amsthm}
\usepackage{amsfonts}
\usepackage{amssymb}
\usepackage{color}
\usepackage{tikz-cd}
\usepackage{calc}
\usepackage{setspace}
\usepackage[german]{babel}
\usetikzlibrary{babel}
\usepackage{cleveref}

\newcommand{\comment}[1]{}
\renewcommand{\baselinestretch}{1.1}

\newcommand{\ModulsOfDifferenzials}{David Eisenbud 1994}
\newcommand{\Algebra}{Christian Karpfinger, Kurt Meyberg 2009}

\newcounter{Aussage}[chapter]

\newtheorem{satz}[Aussage]{Satz}
\newtheorem{theorem}[Aussage]{Theorem}
\newtheorem{prop}[Aussage]{Proposition}
\newtheorem{korrolar}[Aussage]{Korrolar}
\newtheorem{lemma}[Aussage]{Lemma}
\newtheorem{bem}[Aussage]{Bemerkung}
\newtheorem{definition}[Aussage]{Definition}
\newtheorem{bsp}[Aussage]{Beispiel}

\newcommand{\functionfront}[3]{\nolinebreak{#1:#2 \longrightarrow #3}}
\newcommand{\functionback}[3]{\nolinebreak{#1:#2 \longmapsto #3}}
\newcommand{\function}[5]{\nolinebreak{#1:#2 \longrightarrow #3 \, , \, #4 \longmapsto #5}}
\newcommand{\infunctionfront}[3]{\nolinebreak{#1:#2 \hookrightarrow #3}}
\newcommand{\divR}[2]{\Omega_{#1/#2}}
\newcommand{\divf}[1]{d_{#1}}
\comment{\newcommand{\divf}[2][]{d_{#1}}}
\newcommand{\Tensor}[3]{#1 \otimes_{#2} #3}
\newcommand{\tensor}[3]{#1 \otimes #3}
\newcommand{\lok}[2]{#1 [#2^{-1}]}
\newcommand{\loke}[3]{(\frac{#1}{#2})_{_{#3}}}
\comment{\newcommand{\loke}[3]{(#1,#2)_{mod\sim_{#3}}}}

\newcommand{\colimes}[0]{\lim\limits_{ \longrightarrow }}
\newcommand*{\defeq}{\mathrel{\vcenter{\baselineskip0.5ex \lineskiplimit0pt
                     \hbox{\scriptsize.}\hbox{\scriptsize.}}}%
                     =}
\newcommand*{\defeqr}{= \mathrel{\vcenter{\baselineskip0.5ex \lineskiplimit0pt
                     \hbox{\scriptsize.}\hbox{\scriptsize.}}}}

\newcommand*{\defshow}{\stackrel{!}{=}}
\newcommand{\kernel}[1]{kern(#1)}
\newcommand{\immage}[1]{im(#1)}
\newcommand{\Verz}[1]{\langle #1 \rangle}


\begin{document}
\textcolor{blue}{\textbf{Definition Transzenddenzbasis} \textit{[vlg. Anhang A1 \ModulsOfDifferenzials]}}
\begin{def}\comment{\label{Definition Transzenddenzbasis}}
Sei $L \supset k$ eine Körpererweiterung. Dann definieren wir:
\begin{itemize}
\item[•] Eine endliche Teilmenge $\lbrace l_1, \dots ,l_n \rbrace \subseteq L$ heißt \underline{algebraisch abhängig} über $k$, falls gilt:
\begin{gather*}
\exists P(x_1, \dots , x_n) \in k[x_1,\dots,x_n] : \, P(l_1, \dots , l_n) = 0
\end{gather*}
\item[•] Eine endliche Teilmengen $\lbrace l_1, \dots ,l_n \rbrace \subseteq L$ heißt \underline{algebraisch unabhängig} über $k$, falls gilt:
\begin{gather*}
\forall P(x_1, \dots , x_n) \in k[x_1,\dots,x_n] : \, P(l_1, \dots , l_n) \neq 0
\end{gather*}
\item[•] Eine Teilmenge $B \subseteq L$ heißt \underline{transzendent} über $k$, falls jede ihrer endlichen Teilmengen $\lbrace b_1, \dots , b_n \rbrace$ algebraisch unabhängig über $k$ ist.
\item[•] Eine Teilmenge $B \subseteq L$ ist eine \underline{Transzendenzbasis} von $L$ über $k$, falls sie transzendent über $k$ und die Körpererweiterung $L \supset k(B)$ algebraisch ist. 
\end{itemize}
\end{def}


\ \\
\textcolor{blue}{\textbf{Transzendenzbasis ist maximale transzendente Menge} \textit{[Lemma 22.1 \Algebra]}}
\begin{lemma}\label{Transzendenzbasis ist maximale transzendente Menge}
Sei $L \supset k$ ein Körpererweiterung und $B \subseteq L$ eine über $k$ transzendente Teilmenge. Dann gilt:\\
B ist genau dann eine Transzendenzbasis von $L$ über $k$, wenn $B$ bezüglich der Inklusion ein maximales Element der Menge aller über $k$ transzendenten Elemente aus $L$ ist.
\end{lemma}
\begin{proof} \ \\
\begin{itemize}
\item[\underline{\glqq $\Rightarrow$:\grqq}] Sei B eine Transzendenzbasis über $k$. Zeige, dass für ein beliebiges Element $a \in L \setminus B$ die Menge $B \cup \lbrace a \rbrace \subseteq L$ nicht transzendent über $k$ ist:
\begin{gather*}
\text{Da die Körpererweiterung $L \supset k(B)$ algebraisch ist existiert }\\
P(x) \in k(\lbrace b_1, \dots ,n \rbrace) \text{ mit } P(a) = 0.
\end{gather*}
Wir können ohne weitere Einschränkung annehmen, dass $P(x) \in k[\lbrace b_1, \dots ,n \rbrace]$ gilt, denn falls dies nicht der Fall sein sollte, wähle $m \in \mathbb{N}$ groß genug, sodass $P(x) \cdot \left( {\prod_i}^n b_i \right)^m \in k[\lbrace b_1, \dots ,n \rbrace]$ gilt.
\begin{gather*}
\text{Wähle nun } P'(x_1, \dots , x_n , x) \in  k[x_1, \dots , x_n , x] \text{ mit } P(b_1, \dots , b_n, x) = P(x)
\end{gather*}
Für dieses gilt $P'(b_1, \dots , b_n , a) = 0$. Somit ist $\lbrace b_1, \dots, b_n \rbrace$ nicht algebraisch unabhängig und insbesondere $B \cup \lbrace a \rbrace$ nicht transzendent.
\item[\underline{\glqq $\Leftarrow$:\grqq}] Sei $B$ bezüglich der Inklusion ein maximales Element der Menge aller über $k$ transzendenten Elemente aus $L$. Zeige für ein beliebiges Element $a \in L \setminus k(B)$, dass dieses algebraisch über $k(B)$ ist:\\
\ \\
Nach Voraussetzung existiert eine endliche Teilmenge von $B \cup \lbrace a \rbrace$, welche algebraisch abhängig über $k$ ist. Da $B$ transzendent über $k$ ist, muss diese $a$ enthalten. Somit gilt:
\begin{gather*}
\exists \lbrace b_1, \dots, b_n \rbrace \subseteq B : \lbrace b_1, \dots, b_n ,a \rbrace \textit{ ist algebraisch abhängig über k}\\
\Rightarrow \exists P(x_1, \dots, x_{n+1}) \in k[x] : P(b_1, \dots , b_n , a) = 0 \\
\Rightarrow \text{Für } P'(x) \defeq P(b_1, \dots , b_n , x) \in k(B)[x] \text{ gilt } P'(a) = 0
\end{gather*}
Es existiert also ein Polynom $P'(x) \defeq P(b_1, \dots , b_n , x) \in k(B)[x]$ mit $P'(a) = 0$ gefunden. Somit ist $a$ algebraisch über $k(B)$.
\end{itemize}
\end{proof}
\ \\
\textcolor{blue}{\textbf{Transzendenzbasen sind immer gleich lang} \textit{[Theorem A1.1 \ModulsOfDifferenzials]}}
\begin{prop}\comment{\label{Transzendenzbasen sind immer gleich lang}}
Sei $L \supset k$ eine Körpererweiterung. Seinen weiter $A$, $B$ zwei Transzendenzbasen von $L$ über $k$. Dann gilt:
\begin{gather*}
\vert A \vert = \vert B \vert
\end{gather*}
Wir nennen $\vert B \vert$ den \underline{Transzendenzgrad} von $L$ über $k$.
\end{prop}
\begin{proof}
Im Fall von $\vert A \vert = \vert B \vert = \infty$ sind wir schon fertig, sei also ohne Einschränkung $A = \lbrace a_1, \dots , a_m \rbrace$ und $B = \lbrace b_1, \dots , b_n \rbrace$ mit $min(m,n) = n < \infty$.\\
Wir wollen zunächst in $n$ Schritten die Elemente aus $B$ durch Elemente aus $A$ ersetzten und damit zeigen, dass $\lbrace a_1, \dots , a_n \rbrace$ eine Transzendenzbasis von $L$ über $k$ ist:\\
\ \\
Für den $i$-ten Schritt definiere $A_i \defeq \lbrace a_1,\dots,a_{i-1} \rbrace \subseteq A$, $B_i \defeq \lbrace b_i,\dots,b_n \rbrace \subseteq B$ und gehe davon aus, dass $A_i \cup B_i$ eine Transzendenzbasis ist:\\
Nach \cref{Transzendenzbasis ist maximale transzendente Menge} ist $\lbrace a_i \rbrace \cup A_{i} \cup B_{i} = A_{i+1} \cup B_{i}$ nicht transzendent und somit algebraisch abhängig.
\begin{gather*}
\text{Also existiert } P \in k[x,x_1,\dots,x_n] \text{ mit } P(a_i,a_1,\dots,a_{i-1},b_i,\dots,b_n) = 0. \\
\text{Definiere } P'(x) \defeq P(a_i,a_1,\dots,a_{i-1},x,b_{i+1},\dots,b_n) \in k(A_{i+1} \cup B_{i+1})[x].\\
\text{Dieses erfüllt } P'(b_i) = 0.
\end{gather*}
Da $A_i \subseteq A$ algebraisch unabhängig ist, gilt $P(a_1,\dots,a_{i-1},x_i,\dots,x_n) \neq 0$. Nummeriere also gegebenenfalls $B$ vor der Bildung von $P'(x)$ so um, dass auch $P'(x) \neq 0$ gilt.\\
Die Existenz eines solchen $P'(x)$ zeigt uns, dass die Körpererweiterungen \\$L \subset k(A_{i+1} \cup B_i) = k(A_{i+1} \cup B_{i+1})(\lbrace b_i \rbrace) \subset k(A_{i+1} \cup B_{i+1})$ algebraisch sind und legt nahe, dass $A_{i+1} \cup B_{i+1}$ wieder eine Transzendenzbasis ist.\\
Um dies zu zeigen nehme zunächst an $A_{i+1} \cup B_{i+1}$ wäre algebraisch abhängig.
\begin{gather*}
\text{Also existiert } Q \in k[x_1, \dots ,x_n] \text{ mit } Q(a_1, \dots ,a_i,b_{i+1},\dots,b_n) = 0. \\
\text{Definiere } Q'(x) \defeq Q(a_1,\dots,a_{i-1},x,b_{i+1},b_n) \in k(a_1,\dots,a_{i-1},b_{i+1},b_n)[x]. \\
\text{Dieses erfüllt } Q'(a_i) = 0.
\end{gather*}
Da $(A_{i+1} \cup B_{i+1}) \setminus \lbrace a_i \rbrace \subseteq A_i \cup B_i$ algebraisch unabhängig ist gilt $Q'(x) \neq 0$.\\
Die Existenz eines solchen $Q'(x)$ zeigt uns, dass die Körpererweiterung \\$L \subset k(A_{i+1} \cup B_{i+1}) \subset k((A_{i+1} \cup B_{i+1})\setminus\lbrace a_i \rbrace) = k((A_i \cup B_i)\setminus\lbrace b_i \rbrace)$ algebraisch ist. Damit ist $(A_i\cup B_i)\setminus\lbrace b_i \rbrace$ eine Transzendenzbasis, was nach \cref{Transzendenzbasis ist maximale transzendente Menge} im Widerspruch dazu steht, dass $A_i \cup B_i$ eine Transzendenzbasis ist.\\
Folglich ist $A_{i+1} \cup B_{i+1}$ transzendent und somit eine Transzendenzbasis von $L$ über $k$.\ \\
\ \\
Dieses Verfahren zeigt uns, dass $\lbrace a_1, \dots , a_n \rbrace \subseteq A$ eine Transzendenbasis von $L$ über $k$ ist. Nach \cref{Transzendenzbasis ist maximale transzendente Menge} muss somit $A = \lbrace a_1, \dots , a_n \rbrace$ und $m = n$ gelten.
\end{proof}
\ \\
\begin{bem}
Für jede Körpererweiterung $L \subseteq k$ existiert eine Transzendenzbasis $B \subseteq L$ von $L$ über k.
\end{bem}


\ \\
\textcolor{blue}{\textbf{De} \textit{[]}}
\begin{definition}\label{Definition Seperabel}

\end{definition}
\end{document}
