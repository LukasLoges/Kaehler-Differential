\documentclass[10pt,a4paper]{report}
\usepackage[utf8]{inputenc}
\usepackage{amsmath}
\usepackage{amsthm}
\usepackage{amsfonts}
\usepackage{amssymb}
\usepackage{color}
\usepackage{tikz-cd}
\usepackage{calc}
\usepackage{setspace}
\usepackage[german]{babel}
\usetikzlibrary{babel}
\usepackage{cleveref}

\newcommand{\comment}[1]{}
\renewcommand{\baselinestretch}{1.1}

\newcommand{\ModulsOfDifferenzials}{David Eisenbud 1994}

\newcounter{Aussage}[chapter]

\newtheorem{satz}[Aussage]{Satz}
\newtheorem{theorem}[Aussage]{Theorem}
\newtheorem{prop}[Aussage]{Proposition}
\newtheorem{korrolar}[Aussage]{Korrolar}
\newtheorem{lemma}[Aussage]{Lemma}
\newtheorem{bem}[Aussage]{Bemerkung}
\newtheorem{definition}[Aussage]{Definition}
\newtheorem{bsp}[Aussage]{Beispiel}

\newcommand{\functionfront}[3]{\nolinebreak{#1:#2 \longrightarrow #3}}
\newcommand{\functionback}[3]{\nolinebreak{#1:#2 \longmapsto #3}}
\newcommand{\function}[5]{\nolinebreak{#1:#2 \longrightarrow #3 \, , \, #4 \longmapsto #5}}
\newcommand{\infunctionfront}[3]{\nolinebreak{#1:#2 \hookrightarrow #3}}
\newcommand{\divR}[2]{\Omega_{#1/#2}}
\newcommand{\divf}[1]{d_{#1}}
\comment{\newcommand{\divf}[2][]{d_{#1}}}
\newcommand{\Tensor}[3]{#1 \otimes_{#2} #3}
\newcommand{\tensor}[3]{#1 \otimes #3}
\newcommand{\lok}[2]{#1 [#2^{-1}]}
\newcommand{\loke}[3]{(\frac{#1}{#2})_{_{#3}}}
\comment{\newcommand{\loke}[3]{(#1,#2)_{mod\sim_{#3}}}}

\newcommand{\colimes}[0]{\lim\limits_{ \longrightarrow }}
\newcommand*{\defeq}{\mathrel{\vcenter{\baselineskip0.5ex \lineskiplimit0pt
                     \hbox{\scriptsize.}\hbox{\scriptsize.}}}%
                     =}
\newcommand*{\defeqr}{= \mathrel{\vcenter{\baselineskip0.5ex \lineskiplimit0pt
                     \hbox{\scriptsize.}\hbox{\scriptsize.}}}}

\newcommand*{\defshow}{\stackrel{!}{=}}
\newcommand{\kernel}[1]{kern(#1)}
\newcommand{\immage}[1]{im(#1)}
\newcommand{\Verz}[1]{\langle #1 \rangle}


\begin{document}
\begin{satz}\label{Cotangent Sequenz}
Cotangent Sequenz
\end{satz}

\begin{satz}\label{Differenzial von Polynomalgebren 2}
Differenzial von Polynomalgebren 2
\end{satz}

\begin{satz}\label{Differenzial der Lokalisierung}
Differenzial der Lokalisierung
\end{satz}

\begin{satz}\label{Differential von rationalen Funktionen 1}
Differential von rationalen Funktionen 1
\end{satz}


\ \\
\textcolor{blue}{\textbf{Differential von rationalen Funktionen 2} \textit{[Aufgabe 16.6 \ModulsOfDifferenzials]}}
\begin{korrolar}\label{Differential von rationalen Funktionen 2}
Sei $k$ ein Körper und $L \supset k$ eine Körpererweiterung und $T = L(\lbrace x_i \rbrace_{i \in \lbrace 1,\dots,n \rbrace})$ der Körper der rationalen Funktionen in $n$ Varablen über $L$. Dann gilt:
\begin{gather*}
\divR{T}{k} \simeq (\Tensor{T}{L}{\divR{L}{R}}) \oplus \bigoplus_{i \in \lbrace 1,\dots,n \rbrace} T \Verz{\divf{T}(x_i)}
\end{gather*}
\end{korrolar}
\begin{proof}
Betrachten $T$ als Lokalisierung von $L[x_1,\dots,x_n]$ und gehen dann analog zu \cref{Differential von rationalen Funktionen 1} vor:
\begin{gather*}
\divR{T}{k} \simeq \Tensor{T}{L[x_1,\dots,x_n]}{\divR{L[x_1,\dots,x_n]}{k}} \textit{ (\cref{Differenzial der Lokalisierung})} \\
\divR{L[x_1,\dots,x_n]}{R} \simeq (\Tensor{L[x_1,\dots,x_n]}{L}{\divR{L}{R}}) \oplus_{i \in \lbrace 1,...,n \rbrace} L[x_1,\dots,x_n] \Verz{\divf{L[x_1,\dots,x_n]}(x_i)} \textit{ (\cref{Differenzial von Polynomalgebren 2})} \\
\Rightarrow \divR{T}{k} \simeq (\Tensor{T}{L}{\divR{L}{R}}) \oplus_{i \in \lbrace 1,\dots,n \rbrace} T \Verz{\divf{T}(x_i)}
\end{gather*}
\end{proof}


\ \\
\textcolor{blue}{\textbf{Cotangent Sequenz von Körpern 1} \textit{[Aufgabe 16.6 \ModulsOfDifferenzials]}}
\begin{bem} \comment{\label{Cotangent Sequenz von Körpern 1}}
Sei $L \supset k$ eine Körpererweiterung und $T = L(x_1, \dots ,x_n)$ der Körper der rationalen Funktionen in $n$ Variablen über $L$. Dann ist die COTANGENT SEQUENZ \textit{(\cref{Cotangent Sequenz})} von $k \hookrightarrow L \hookrightarrow T$ eine kurze Exakte Sequenz:
\begin{center}
\begin{tikzcd}
0 \arrow[r] & \Tensor{T}{L}{\divR{L}{k}} \arrow[r] & \divR{T}{k} \arrow[r] & \divR{T}{L} \arrow[r] & 0
\end{tikzcd}
\end{center}
Im Genauen ist $\function{\varphi}{\Tensor{T}{L}{\divR{L}{k}}}{\divR{T}{k}}{\tensor{t}{L}{\divf{L}(l)}}{t \cdot \divf{T}(l)}$ injektiv.
\end{bem}
\begin{proof}
Die Injektivität von $\varphi$ folgt direkt aus der isomorphen Darstellung von $\divR{T}{k}$, die wir uns in \cref{Differential von rationalen Funktionen 2} erarbeitet haben.
\begin{gather*}
\divR{T}{k} \simeq (\Tensor{T}{L}{\divR{L}{R}}) \oplus \bigoplus_{i \in \lbrace 1,\dots,n \rbrace} T \Verz{\divf{T}(x_i)}
\end{gather*}
Um sicher zu gehen definiere $\varphi' \simeq \varphi$ und durchlaufe die in \cref{Differential von rationalen Funktionen 2} genutzten Isomorphismen noch einmal Schritt für Schritt:
\begin{center}
$\functionfront{\varphi'}{\Tensor{T}{L}{\divR{L}{k}}}{\Tensor{T}{L}{\divR{L}{R}} \oplus \bigoplus_{i \in \lbrace 1,\dots,n \rbrace} T \Verz{\divf{T}(x_i)}}$\\
\ \\
\begin{tikzcd}
\Tensor{T}{L}{\divR{L}{k}} \arrow[d, hook]                 &  &  & \tensor{t}{L}{\divf{L}(l)} \arrow[d, maps to] \\
\divR{T}{k} \arrow[d, "\cref{Differenzial der Lokalisierung}", two heads, hook] &  &  & t\divf{T}(l) \arrow[d, maps to] \\
\Tensor{T}{S}{\divR{L[x_1,\dots ,x_n]}{k}} \arrow[d, "\cref{Differenzial von Polynomalgebren 2}", two heads, hook] &  &  & \tensor{t}{S}{\divf{S}(l)} \arrow[d, maps to] \\
\Tensor{T}{S}{((\Tensor{S}{L}{\divR{L}{k}}) \oplus \bigoplus_{i \in \lbrace 1,\dots,n \rbrace} S \Verz{\divf{S}(x_i)}}) \arrow[d, "", two heads, hook] &  &  & \tensor{t}{S}{(\divf{L}(l),0)} \arrow[d, maps to] \\
(\Tensor{T}{L}{\divR{L}{R}}) \oplus \bigoplus_{i \in \lbrace 1,\dots,n \rbrace} T \Verz{\divf{T}(x_i)}                                 &  &  & (\tensor{t}{L}{\divf{L}(l)},0)                   
\end{tikzcd} 
\end{center}
Damit ist $\varphi$ eine injektive Einbettung von $\Tensor{T}{L}{\divR{L}{k}}$ in $\divR{T}{k}$.
\end{proof}
\end{document}
