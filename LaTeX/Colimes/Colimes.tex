\documentclass[10pt,a4paper]{report}
\usepackage[utf8]{inputenc}
\usepackage{amsmath}
\usepackage{amsthm}
\usepackage{amsfonts}
\usepackage{amssymb}
\usepackage{color}
\usepackage{tikz-cd}
\usepackage{calc}
\usepackage{setspace}
\usepackage[german]{babel}
\usetikzlibrary{babel}
\usepackage{cleveref}

\newcommand{\comment}[1]{}
\renewcommand{\baselinestretch}{1.1}

\newcommand{\ModulsOfDifferenzials}{David Eisenbud 1994}
\newcommand{\Algebra}{Christian Karpfinger, Kurt Meyberg 2009}

\newcounter{Aussage}[chapter]

\newtheorem{satz}[Aussage]{Satz}
\newtheorem{theorem}[Aussage]{Theorem}
\newtheorem{prop}[Aussage]{Proposition}
\newtheorem{korrolar}[Aussage]{Korrolar}
\newtheorem{lemma}[Aussage]{Lemma}
\newtheorem{bem}[Aussage]{Bemerkung}
\newtheorem{definition}[Aussage]{Definition}
\newtheorem{bsp}[Aussage]{Beispiel}

\newcommand{\functionfront}[3]{\nolinebreak{#1:#2 \longrightarrow #3}}
\newcommand{\functionback}[3]{\nolinebreak{#1:#2 \longmapsto #3}}
\newcommand{\function}[5]{\nolinebreak{#1:#2 \longrightarrow #3 \, , \, #4 \longmapsto #5}}
\newcommand{\infunctionfront}[3]{\nolinebreak{#1:#2 \hookrightarrow #3}}
\newcommand{\divR}[2]{\Omega_{#1/#2}}
\newcommand{\divf}[1]{d_{#1}}
\comment{\newcommand{\divf}[2][]{d_{#1}}}
\newcommand{\Tensor}[3]{#1 \otimes_{#2} #3}
\newcommand{\tensor}[3]{#1 \otimes #3}
\newcommand{\lok}[2]{#1 [#2^{-1}]}
\newcommand{\loke}[3]{(\frac{#1}{#2})_{_{#3}}}
\comment{\newcommand{\loke}[3]{(#1,#2)_{mod\sim_{#3}}}}

\newcommand{\colimes}[0]{\lim\limits_{ \longrightarrow }}
\newcommand*{\defeq}{\mathrel{\vcenter{\baselineskip0.5ex \lineskiplimit0pt
                     \hbox{\scriptsize.}\hbox{\scriptsize.}}}%
                     =}
\newcommand*{\defeqr}{= \mathrel{\vcenter{\baselineskip0.5ex \lineskiplimit0pt
                     \hbox{\scriptsize.}\hbox{\scriptsize.}}}}

\newcommand*{\defshow}{\stackrel{!}{=}}
\newcommand{\kernel}[1]{kern(#1)}
\newcommand{\immage}[1]{im(#1)}
\newcommand{\Verz}[1]{\langle #1 \rangle}


\begin{document}
\chapter{Kolimes}
\section{Einführung in den Kolimes}
\textcolor{blue}{\textbf{Definition des Kolimes}}
\begin{definition}\label{Definition des Kolimes} \textit{[vgl. Anhang A6 \ModulsOfDifferenzials]}
Sei $\mathcal{A}$ eine Kategorie.
\begin{itemize}
\item Ein \underline{Diagramm} über $\mathcal{A}$ ist eine Kategorie $\mathcal{B}$ zusammen mit einem Funktor $\functionfront{\mathcal{F}}{\mathcal{B}}{\mathcal{A}}$.
\item Sei $\functionfront{\mathcal{F}}{\mathcal{B}}{\mathcal{A}}$ ein Diagramm und $A \in \mathcal{A}$ ein Objekt. Dann definieren wir einen \underline{Morphismus} $\functionfront{\psi}{\mathcal{F}}{A}$ als eine Menge von Funktionen 
$\nolinebreak{\lbrace \psi_B \in Hom(F(B),A) \vert B \in \mathcal{B} \rbrace}$, wobei für alle $B_1,B_2 \in \mathcal{B}$ und $\varphi \in Hom(B_1,B_2)$ folgendes Diagramm kommutiert:
\begin{center}
\begin{tikzcd}
\mathcal{F}(B_1) \arrow[rrd, "\psi_{B_1}"] \arrow[dd, "\mathcal{F}(\varphi )"] &  &   \\
                                   &  & C \\
\mathcal{F}(B_2) \arrow[rru, "\psi_{B_2}"']                 &  &  
\end{tikzcd}
\end{center}
\item Der \underline{Kolimes} $\colimes \mathcal{F}$ eines Diagramms $\functionfront{\mathcal{F}}{\mathcal{B}}{\mathcal{A}}$ ist ein Paar aus einem Objekt $A \in \mathcal{A}$ zusammen mit einem Morphismus $\functionfront{\psi}{\mathcal{F}}{A}$, welche folgende universelle Eingenschaft erfüllen:
\begin{center}
Für Objekte $A' \in \mathcal{A}$ und alle Morphismen $\functionfront{\psi '}{\mathcal{F}}{A'}$ existiert genau eine Funktion $\varphi \in Hom(A,A')$, sodass folgendes Diagramm kommutiert:
\begin{tikzcd}
  & \mathcal{F} \arrow[rd, "\psi"] \arrow[ld, "\psi '"'] &                            \\
A' &                                    & A \arrow[ll, "\exists ! \varphi "', dashed]
\end{tikzcd}
\end{center}
\end{itemize}
\end{definition}


\ \\
\textcolor{blue}{\textbf{Eindeutigkeit des Kolimes} \textit{[vgl. A6 \ModulsOfDifferenzials]}}
\begin{lemma}\label{Eindeutigkeit des Kolimes}
Seien $\mathcal{B},\mathcal{A}$ zwei Kategorien und $\functionfront{\mathcal{F}}{\mathcal{B}}{\mathcal{A}}$ ein Funktor. Dann ist im Falle der Existenz $\colimes \mathcal{F}$ eindeutig bestimmt.
\end{lemma}
\begin{proof}
Seien $A_1 \in \mathcal{A}, (\functionfront{\psi_1}{\mathcal{F}}{A_1}) $ und $A_2 \in \mathcal{A} , (\functionfront{\psi_2}{\mathcal{F}}{A_2}) $ beide gleich $\colimes \mathcal{F}$.\\
Erhalte durch die universelle Eigenschaft des Kolimes die eindeutig bestimmten Funktionen $\varphi_1 \in Hom_{\mathcal{A}}(A_1,A_2)$ und $\varphi_2 \in Hom_{\mathcal{A}}(A_2,A_1)$, für welche die folgende Diagramme kommutieren:
\begin{center}
\begin{tikzcd}
  & \mathcal{F} \arrow[rd, "\psi_1"] \arrow[ld, "\psi_2"'] &                            &  &   & \mathcal{F} \arrow[rd, "\psi_2"] \arrow[ld, "\psi_1"'] &                            \\
A_2 &                                    & A_1 \arrow[ll, "\exists ! \varphi_1"', dashed] &  & A_1 &                                    & A_2 \arrow[ll, "\exists ! \varphi_2"', dashed]
\end{tikzcd}
\end{center}
\begin{flushleft}
Wende nun die Universelle Eigenschaft von $\psi_1$ auf $\psi_1$ selbst an und erhalte $id_{A_1} = \varphi_2 \circ \varphi_1$. Analog erhalte auch $id_{A_2} = \varphi_1 \circ \varphi_2$.
\end{flushleft}
\begin{center}
\begin{tikzcd}
  & \mathcal{F} \arrow[rd, "\psi_1"] \arrow[ld, "\psi_1"'] &                            \\
A_1 &                                    & A_1 \arrow[ll, "\exists ! id_{A_1} = \varphi_2 \circ \varphi_1"', dashed]
\end{tikzcd}
\end{center}
Somit existiert genau eine Isomorphie $\functionfront{\varphi_1}{A_1}{A_2}$.
\end{proof}


\ \\
\textcolor{blue}{\textbf{Vereinfachung des Kolimes}}
\begin{korrolar}\label{Vereinfachung des Kolimes} \textit{[Eigene Überlegung ]}\\
Sei $\mathcal{A}$ eine Kategorie und $(\mathcal{B}, \functionfront{\mathcal{F}}{\mathcal{B}}{\mathcal{A}})$ ein Diagramm. Betrachte die Unterkategorie $\mathcal{F}(B) \subseteq \mathcal{A}$ zusammen mit dem Inklusionsfunktor $\mathcal{F}(B)\hookrightarrow \mathcal{A}$ ebenfalls als Diagramm. Dann gilt:
\begin{center}
$\colimes \mathcal{F}$ existiert genau dann, wenn $\colimes (\mathcal{F}(\mathcal{B}) \hookrightarrow \mathcal{A})$ existiert.\\
Mit $\colimes \mathcal{F} = \colimes (\mathcal{F}(\mathcal{B}) \hookrightarrow \mathcal{A})$.
\end{center}
\end{korrolar}
\begin{proof}
Dies folgt direkt aus unserer Definition von Morphismen:\\
In \cref{Definition des Kolimes} haben wir einen Morphismus $\functionfront{\psi}{\mathcal{F}}{A}$ als eine Menge von Funktionen $\mathcal{\psi_B} \in Hom_{\mathcal{A}}(\mathcal{F}(B),A)$ definiert. Dies zeigt, dass es keinen Unterschied macht, ob wir von Morphismen $\functionfront{\psi}{\mathcal{F}}{A}$ oder von Morphismen $\functionfront{\psi}{(\mathcal{F}(B)\hookrightarrow \mathcal{A})}{A}$ reden.\\
Wenn wir nun die universelle Eigenschaft des Kolimes genauer betrachten, sehen wir, dass diese sich nur auf Morphismen $\mathcal{F} \longrightarrow A$ bzw. $(\mathcal{F}(\mathcal{B}) \hookrightarrow \mathcal{A}) \longrightarrow A$ und auf die Kategorie $\mathcal{A}$ bezieht. Es macht also keinen Unterschied, ob wir vom Kolimes des Diagramms $(\mathcal{B}, \functionfront{\mathcal{F}}{\mathcal{B}}{\mathcal{A}})$ oder vom Kolimes des Diagramms $(\mathcal{F}(B),\mathcal{F}(B)\hookrightarrow \mathcal{A})$ sprechen.
\end{proof}
Es genügt also im Fall von Kolimtenn Diagramme $(\mathcal{B},\mathcal{B}\hookrightarrow\mathcal{A})$ mit $\mathcal{B} \subseteq \mathcal{A}$ zu betrachten. Zur Vereinfachung schreibe für $\mathcal{B} \subseteq \mathcal{A}$ in Zukunft $\colimes \mathcal{B}$ anstatt von $\colimes (\mathcal{B} \hookrightarrow \mathcal{A})$.


\ \\
\textcolor{blue}{\textbf{DifferenzkokernUndKoproduktDef}}
\begin{definition}\label{DifferenzkokernUndKoproduktDef} \textit{[vlg. A6 \ModulsOfDifferenzials]}\\
Sei $\mathcal{A}$ eine Kategorie.
\begin{itemize}
\item Das Koprodukt von $ \lbrace B_i \rbrace_{i \in \Lambda} \subseteq \mathcal{A}$ wird durch $\coprod_{i \in \Lambda} \lbrace B_i \rbrace := \colimes\mathcal{B}$ definiert, wobei $\lbrace B_i \rbrace_{i \in \Lambda}$ die Objekte und die Identitätsabbildungen $\lbrace \functionfront{id_{B_i}}{B_i}{B_i} \rbrace_{i \in \Lambda}$ die einzigen Morphismen von $\mathcal{B}$ sind.
\item Der Differenzkokern von $f,g \in Hom_{\mathcal{A}}(C_1,C_2)$ wird durch $\colimes \mathcal{C}$ definiert,
wobei $\lbrace C_1,C_2 \rbrace$ die Objekte und $ \lbrace f,g \rbrace$ zusammen mit den Identitätsabbildungen die Morphismen von $\mathcal{C}$ sind.
\end{itemize}
\end{definition}


\ \\
\textcolor{blue}{\textbf{NeuDifferenzenkokerndef}}
\begin{bem}\label{NeuDifferenzenkokerndef} \textit{[Wikipedia]}\\
Sei $\mathcal{A}$ eine Kategorie. Sei weiter $C_1,C_2 \in Obj_{\mathcal{A}}$ und $f,g \in Hom_{\mathcal{A}}(C_1,C_2)$.\\
Im Falle der Existenz ist der Differnenzenkokern von $f,g$ nach \cref{DifferenzkokernUndKoproduktDef} durch ein Objekt $C \in Obj_{\mathcal{A}}$ und einen Morphismus $\psi = \lbrace \psi_{C_1}, \psi_{C_2}\rbrace$ gegeben, wobei gilt:
\begin{gather*}
\psi_{C_2} = f \circ \psi_1 = g \circ \psi_2
\end{gather*}
Wir sehen, dass $\psi$ eindeutig durch $q \defeq \psi_2 \in Hom_{\mathcal{A}}(C_1,C_2)$ gegeben ist. Der Differnzenkokern ist also eindeutig durch $(C \in obj_\mathcal{A},q \in Hom_{\mathcal{A}}(C_1,C_2))$ gegeben, wobei $q$ folgenden Eigenschaften besitzt:
\begin{center}
Es gilt $f \circ q = g \circ g$ und\\
für alle $C \in Obj_{A}$ und $q' \in Hom_{\mathcal{A}}$ mit $f \circ q' = g \circ q'$ existiert genau ein $\varphi \in Hom_{\mathcal{A}}$, mit $q \circ \varphi = q'$:\\
\ \\
\begin{tikzcd}
C_1 \arrow[r, "{f,g}"] \arrow[r] & C_2 \arrow[r, "q"] \arrow[rd, "q'"] & C \arrow[d, "\exists !\varphi", dashed] \\
                                 &                                     & C'                                     
\end{tikzcd}
\end{center}
\end{bem}
Wenn wir fortan vom Differenzkokern sprechen meinen wir damit das Paar $(C,q)$.


\ \\
\textcolor{blue}{\textbf{Kolimes durch Koprodukt und Differenzkokern}}
\begin{theorem}\label{Kolimes durch Koprodukt und Differenzkokern} \textit{[Proposition A6.1 \ModulsOfDifferenzials]}\\
Sei $\mathcal{A}$ eine Kategorie, in der Koprodukte beliebiger Mengen von Objekten und Differenzkokerne von je zwei Morphismen existieren. Dann existiert für jedes Diagramm $\functionfront{\mathcal{F}}{\mathcal{B}}{\mathcal{A}}$ dessen Kolimes $\colimes \mathcal{F}$.
\end{theorem}
\begin{proof}
In \cref{Vereinfachung des Kolimes} haben wir gesehen, dass es genügt den Fall $\mathcal{B} \subseteq \mathcal{A}$ zu betrachten. Konstruiere also für eine beliebige Unterkategorie $\mathcal{B} \subseteq \mathcal{A}$ deren Kolimes $\colimes\mathcal{B}$:\\
Bezeichne für jeden Morphismus $\gamma \in Morph_{\mathcal{C}}$ dessen Definitionsbreich mit $B_{\gamma} \in \mathcal{B}$. Weiter, wenn wir einen Morphismus $\psi$ gegeben haben und $\psi_{\gamma(B_{\gamma})}$ betrachten, ist damit $\psi_{B}$ gemeint, wobei $B$ die Zielmenge von $\gamma$ ist. Definiere nun:
\begin{itemize}
\item $C_1 \defeq \coprod_{\gamma \in Morph_{\mathcal{B}}} B_{\gamma}$ ist das Koprodukt aller Objekte von $\mathcal{B}$, in dem jedes Objekt so oft vorkommt, wie es Definitionsbereich eines $\gamma \in Morph_{\mathcal{B}}$ ist.\\
Sei $\functionfront{\psi^1}{\lbrace B_{\gamma} \vert \gamma \in Morph_{\mathcal{B}}\rbrace}{C_1}$ der dazugehörige Morphismus.
\item $C_2 \defeq \coprod_{B \in Obj_\mathcal{B}}$ ist das Koprodukt aller Objekte von $\mathcal{B}$.\\
Sei $\functionfront{\psi^2}{\lbrace B \vert B \in Obj_\mathcal{B} \rbrace}{C_2}$ der dazugehörige Morphismus.
\end{itemize}
Konstruiere nun $f,g \in Hom_{\mathcal{A}}(C_1,C_2)$ so, dass der Differenzkokern von $f$ und $g$ dem Kolimes von $\mathcal{B}$ entspricht. Nutze dazu die universelle Eigenschaft von $(C_1,\psi^1) = \colimes \lbrace B_{\gamma} \vert \gamma \in Morph_{\mathcal{B}}\rbrace$:
\begin{itemize}
\item[]
Für $f$ betrachte den Morphismus $\functionfront{\zeta}{\lbrace B_{\gamma} \vert \gamma \in Morph_{\mathcal{B}}\rbrace}{C_2}$,\\
mit $\zeta_{B_{\gamma}} \defeq \psi^2_{\gamma(B_{\gamma})}$ für $B_{\gamma} \in \lbrace B_{\gamma} \vert \gamma \in Morph_{\mathcal{B}}\rbrace$.\\
Wähle $f \in Hom_{\mathcal{B}}(C_1,C_2)$ als die eindeutige Funktion, mit $\zeta = f \circ \psi^1$.
\item[]
Für $g$ betrachte den Morphismus $\functionfront{\zeta'}{\lbrace B_{\gamma} \vert \gamma \in Morph_{\mathcal{B}}\rbrace}{C_2}$,\\
mit $\zeta'_{B_{\gamma}} \defeq \psi^2_{\gamma(B_{\gamma})} \circ \gamma$ für $B_{\gamma} \in \lbrace B_{\gamma} \vert \gamma \in Morph_{\mathcal{B}}\rbrace$.\\
Wähle $g \in Hom_{\mathcal{B}}(C_1,C_2)$ als die eindeutige Funktion, mit $\zeta' = g \circ \psi^1$.
\end{itemize}
\begin{center}
\begin{tikzcd}
    & \lbrace B_{\gamma} \vert \gamma \in Morph_{\mathcal{B}}\rbrace \arrow[rd, "\psi^1"] \arrow[ld, "\zeta"'] &                                        &     & \lbrace B_{\gamma} \vert \gamma \in Morph_{\mathcal{B}}\rbrace \arrow[rd, "\psi^1"] \arrow[ld, "\zeta'"'] &                                       \\
C_2 &                                                                                                          & C_1 \arrow[ll, "\exists ! f"', dashed] & C_2 &                                                                                                           & C_1 \arrow[ll, "\exists! g"', dashed] \\
    & \zeta_{B_{\gamma}} \defeq \psi^2_{\gamma(B_{\gamma})}                                                                                                  &                                        &     & \zeta'_{B_{\gamma}} \defeq \psi^2_{\gamma(B_{\gamma})} \circ \gamma                                                                                                   &                                      
\end{tikzcd}
\end{center}
Sei $C \in Obj_{\mathcal{B}}$ zusammen mit $q \in Hom_{\mathcal{A}}(C_2,C)$ der Differenzkokern von $f$,$g$.\\
Betrachte abschließend $\functionfront{\psi}{\mathcal{B}}{C}$, mit $\psi_{B} = q \circ \psi^2_B$ für $B \in Obj_{\mathcal{B}}$.\\
Um zu sehen, dass $\psi$ ein Morphismus ist, wähle $B_1,B_2 \in Obj_{\mathcal{B}}$ beliebig und betrachte folgendes kommutatives Diagramm:
\begin{center}
\begin{tikzcd}
B_1 \arrow[rd, "\psi^1_{B_1}"] \arrow[rr, "\psi^2_{B_1} = \zeta_{B_1}"] \arrow[dd, "\gamma"'] \arrow[rrdd, "\zeta'_{B_2}"', bend right] &                                      & C_2 \arrow[rd, "q"]  &   \\
                                                                                                                        & C_1 \arrow[ru, "f"'] \arrow[rd, "g"] &                      & C \\
B_2 \arrow[rr, "\psi^2_{B_2}"']                                                                                         &                                      & C_2 \arrow[ru, "q"'] &  
\end{tikzcd}
\end{center}
Zeige nun, dass $(C,\psi)$ die Universelle Eigenschaft des Kolimes besitzt.
Nutze dazu nacheinander die universellen Eigenschaften von $(C_2,\psi^2)$ und $(q,C)$:
\begin{itemize}
\item[]
Da $\psi'$ ein Morphismus von $\mathcal{B}$ nach $C'$ ist, ist $\psi'$ insbesondere auch ein Morphismus von $\lbrace B \vert B \in Obj_{\mathcal{B}} \rbrace$ nach $C$. Somit existiert genau ein $q' \in Hom_{\mathcal{B}}(C_2,C')$ mit $\psi^2 \circ q' = \psi'$.
\comment{
\begin{center}
\begin{tikzcd}
   & \lbrace B \vert B \in Obj_{\mathcal{B}} \rbrace \arrow[rd, "\psi^2"] \arrow[ld, "\psi'"'] &                                         \\
C' &                                                                                           & C_2 \arrow[ll, "\exists ! q'"', dashed]
\end{tikzcd}
\end{center}
}
\item[]
Zeige nun $q' \circ f \defshow q' \circ g$. Sei dazu $c \in C_1$ beliebig und $\gamma \in Morph_{\mathcal{B}}, \, b \in B_{\gamma}$ mit $\psi^1_{B_{\gamma}}(b) = c$, dann gilt:
\begin{gather*}
(q' \circ f)(c)
= (q' \circ f \circ \psi^1_{B_{\gamma}})(b)
= (q' \circ \zeta_{B_{\gamma}})(b)
= (q' \circ \psi^2_{B_{\gamma}})(b)
= \psi'_{B_{\gamma}}(b) \\
(q' \circ g)(c)
= (q' \circ g \circ \psi^1_{B_{\gamma}})(b)
= (q' \circ \zeta'_{B_{\gamma}})(b)\\
= (q' \circ \psi^2_{\gamma({B_{\gamma}})} \circ \gamma)(b)
= (\psi'_{\gamma(B_{\gamma})} \circ \gamma)(b)
= \psi'_{B_{\gamma}}(b)
\end{gather*}
Somit können wir die universelle Eigenschaft von $q$ auf $q'$ anwenden und erhalten ein eindeutiges $\varphi \in Hom(C,C')$ mit $q' = q \circ \varphi$.
\begin{center}
\begin{tikzcd}
   &  & \mathcal{B} \arrow[rd, "\psi^2"] \arrow[lldd, "\psi'"'] \arrow[rrdd, "\psi", bend left=49] &                                                                &                                               \\
   &  &                                                                                            & C_2 \arrow[rd, "q"] \arrow[llld, "\exists ! q'"', dashed] &                                               \\
C' &  &                                                                                            &                                                                & C \arrow[llll, "\exists ! \varphi"', dashed]
\end{tikzcd}
\end{center}
\end{itemize}
Dieses $\varphi \in Hom(C,C')$ erfüllt auch $\psi \circ \varphi = \psi^2 \circ q \circ \varphi = \psi^2 \circ q' = \psi'$ und ist nach Konstruktion eindeutig. Damit gilt $\colimes \mathcal{B} = (C,\psi)$.
\end{proof}


\ \\
\begin{bem}\label{Unendliche Indexmengen}(Unendliche Indexmengen)\\
Wir wollen uns hier nochmal kurz in Erinnerung rufen, was es bedeutet, wenn wir eine unendlich große Indexmenge $\Lambda$ vor uns haben:
\begin{itemize}
\item[1.] Sei $\mathcal{A}$ eine Kategorie und $\lbrace B_i \rbrace_{i \in \Lambda} \subseteq Obj_{\mathcal{A}}$, dann gilt:
\begin{gather*}
\bigoplus_{i \in \Lambda} B_i 
= \bigcup_{\lbrace i_1, \dots, i_n \rbrace \subseteq \Lambda} \bigoplus_{k = 1}^{n} B_{i_k} 
= \left\lbrace (b_{i_1}, \dots , b_{i_n}) \vert n \in \mathbb{N} \wedge \lbrace i_1, \dots ,i_n \rbrace \subseteq \Lambda \right\rbrace
\end{gather*}
\item[2.] Sei $\lbrace M_i \rbrace_{i \in \Lambda}$ eine Menge von $R$-Moduln (oder $R$-Algebren), dann gilt:
\begin{gather*}
\bigotimes_{i \in \Lambda} M_i 
= \bigcup_{\lbrace i_1, \dots, i_n \rbrace \subseteq \Lambda} \bigotimes_{k = 1}^{n} M_{i_k} 
= \left\lbrace (m_{i_1} \otimes \dots \otimes m_{i_n}) \vert n \in \mathbb{N} \wedge \lbrace i_1, \dots ,i_n \rbrace \subseteq \Lambda \right\rbrace
\end{gather*}
\item[3.] Für den Polynomring über $R$ in unendlich vielen Variablen $\lbrace x_i \rbrace_{i \in \Lambda}$ gilt:
\begin{gather*}
P[\lbrace x_i \rbrace_{i \in \Lambda}] 
= \bigcup_{\lbrace i_1, \dots, i_n \rbrace \subseteq \Lambda} P[x_{i_1} , \dots , x_{i_n}] 
= \left\lbrace P(x_{i_1}, \dots , x_{i_n}) \vert n \in \mathbb{N} \wedge \lbrace i_1, \dots ,i_n \rbrace \subseteq \Lambda \right\rbrace
\end{gather*}
\end{itemize}
Dies zeigt, dass sich diesen drei Fällen eine unendliche Indexmenge $\Lambda$ immer auf endliche Indexmengen $\lbrace 1, \dots , n\rbrace$ zurückführen lässt.
\end{bem}


\ \\
\textcolor{blue}{\textbf{Darstellung der Polynomalgebra als Tensorprodukt}}
\begin{bem}\label{Darstellung der Polynomalgebra als Tensorprodukt}\textit{[Eigene Überlegung]}\\
Die Polynomalgebra $R[\lbrace x_i \rbrace_{i \in \Lambda}]$ über R lässt sich wie folgt als Tensorprodukt darstellen:
\begin{gather*}
R[\lbrace x_i \rbrace_{i \in  \Lambda}] = \bigotimes_{i \in \Lambda} R[x_i]
\end{gather*}
\end{bem}
\begin{proof}
Im Falle einer endlichen Indexmenge $\Lambda$ wollen wir induktiv vorgehen. Seien für den Induktionsschritt $S_x \defeq R[x_1, \dots x_n]$ und $S_y \defeq R[y_1, \dots , y_m]$ zwei Polynomalgebren über R, zeige:
\begin{gather*}
S_{xy} \defeq R[x_1, \dots , x_n, y_1 , \dots , y_m] \simeq \Tensor{S_x}{R}{S_y}
\end{gather*}
Dazu betrachten wir folgende bilineare Funktion:
\begin{gather*}
\function{g'}{S_x \oplus S_y}{S}{(P,Q)}{P \cdot Q}
\end{gather*}
Erhalte nun eine Funktion $\functionfront{\varphi}{\Tensor{S_x}{R}{S_y}}{S_{xy}}$ aus der universellen Eigenschaft des Tensorproduktes:
\begin{center}
\begin{tikzcd}
S_x \oplus S_y \arrow[r, "g"] \arrow[rd, "g'"'] & \Tensor{S_x}{R}{S_y} \arrow[d, "\exists ! \varphi", dashed] \\
                                                & S_{xy}                                       
\end{tikzcd}
\ \\
$\function{\varphi}{\Tensor{S_x}{R}{S_y}}{S_{xy}}{\tensor{P}{R}{Q}}{P \cdot Q}$
\end{center}
Der Homomorphismus $\varphi$ ist surjektiv und bildet die Erzeuger $\lbrace \tensor{x_i}{R}{1} \rbrace \cup \lbrace \tensor{1}{R}{y_j} \rbrace$ von $\Tensor{S_x}{R}{S_y}$ eindeutig auf die Erzeuger $\lbrace x_i \rbrace \cup \lbrace y_j \rbrace$ von $S_{xy}$ ab. Folglich ist $\varphi$ ein Isomorphismus.\\
Indunktiv erhalten wir daraus für den Fall $\vert \Lambda \vert < \infty$ folgenden Isomorphismus:
\begin{gather*}
\function{\Phi}{\bigotimes_{i \in \Lambda} R[x_i]}{R[\lbrace x_i \rbrace_{i \in \Lambda} ]}{(P_1(x_1), \dots P_n(x_n))}{\prod_{i=1}^{n} P_i(x_i)}
\end{gather*}
Dies ist auch im Fall $\Lambda = \infty$ ein Isomorphismus, da wir auch in diesem Fall nur Tensorprodukte endlich vieler Polynome bzw. Polynome in endlich vielen Variablen betrachten \textit{(siehe \cref{Unendliche Indexmengen})}.\\
Da das Tensorprodukt $\bigotimes_{i \in \Lambda} R[x_i]$ bis auf eine Eindeutige Isomorphie eindeutig bestimmt ist, definiere dies ab jetzt als $R[\lbrace x_i \rbrace_{i \in \Lambda}]$.
\end{proof}


\comment{
\textcolor{blue}{\textbf{Tensorprodukt des Differenzenkokerns} \textit{[Eigene Bemerkung]}}
\begin{bem} \comment{\label{Tensorprodukt des Differenzenkokerns}}
Seien $f,g \in Hom_{\mathcal{A}}(S_1,S_2)$ R-Algebra-Homomorphismen, so können wir für den Differenzenkokern $\functionfront{q}{S_2}{T}$ für ein beliebiges $S_1$-Modul das Tensorprodukt $\Tensor{T}{C_1}{M}$ definieren. 
\begin{gather*}
\textit{für } s_1 \in S_1 \textit{ und } \tensor{t}{S_1}{m}) \in \Tensor{T}{C_1}{M} \textit{ gilt: }\\
s_1 \cdot (\tensor{t}{S_1}{m}) = \tensor{((q \circ f)(s_1)) \cdot t}{S_1}{m} = \tensor{((q \circ g)) \cdot (s_1)t}{S_1}{m}
\end{gather*}
\end{bem}
}


\ \\
\textcolor{blue}{\textbf{R-Algebra-Kolimiten}}
\begin{prop} \label{R-Algebra-Kolimiten} \textit{[vlg. Proposition A6.7 \ModulsOfDifferenzials]}\\
In der Kategorie der R-Algebren existieren Kolimiten beliebiger Diagramme, wobei gilt:
\begin{itemize}
\item[\textbf{1.}] Das Koprodukt einer Familie von $R-Algebren$ $\lbrace S_i \rbrace_{i \in \Lambda}$ entspricht deren Tesorprodukt $\bigotimes_{i \in \Lambda} S_i$. 
\item[\textbf{2.}] Der Differenzkokern zweier R-Algebrenhomomorphismen $\functionfront{f,g}{S_1}{S_2}$ einspricht dem Homomorphismus $\function{q}{S_2}{S_2/Q}{y}{[y]}$,\\
wobei $Q \defeq \lbrace f(x) - g(x)\mid x \in S_1 \rbrace$ das Bild der Differenz von $f$ und $g$ ist.
\end{itemize}
\end{prop}
\begin{proof} \ \\
\underline{Zu \textit{\textbf{1.}}:} Sei $\mathcal{B} = \lbrace S_i \rbrace_{i \in \Lambda}$ die Unterkategorie der R-Algebren, welche $\lbrace S_i \rbrace_{i \in \Lambda}$ zusammen mit den Identitätsabbildungen enthält. Somit gilt nach \cref{DifferenzkokernUndKoproduktDef} $\coprod_{i \in \Lambda} S_i = \colimes \mathcal{B}$. Seien weiter:
\begin{itemize}
\item[]$\functionfront{\psi}{\mathcal{B}}{\coprod_{i \in \Lambda} S_i}$ der Morphismus des Koprodukts und
\item[]$\functionfront{g}{\bigoplus_{i \in \Lambda} S_i}{\bigotimes_{i \in \Lambda} S_i}$ die multilineare Abbildung des Tensorprodukts.
\end{itemize}
Konstruiere daraus einen Morphismus $\psi'$ und eine multilineare Abbildung $g'$:
\begin{gather*}
\functionfront{\psi'}{\mathcal{B}}{\bigotimes_{i \in \Lambda} S_i} \text{, mit } \function{\psi'_{S_i}}{S_i}{\bigotimes_{i \in \Lambda} S_i}{s_i}{g(1,..,1,s_i,1,..,1)} \text{ für } i \in \Lambda\\
\function{g'}{\bigoplus_{i \in \Lambda} S_i}{\coprod_{i \in \Lambda} S_1}{s}{\prod_{i \in \lbrace i \in \Lambda \vert s_i \neq 0 \rbrace} \psi_i(s_i)}
\end{gather*}
\ \\
Somit liefern uns die universellen Eigenschaften folgende zwei R-Algebra-Homomorphismen:
\begin{center}
\begin{tikzcd}
  & \mathcal{B} \arrow[rd, "\psi"] \arrow[ld, "\psi'"'] &                                            &   & \bigoplus_i S_i \arrow[ld, "g'"'] \arrow[rd, "g"] &                                         \\
\bigotimes_{i \in \Lambda} S_i &                                           & \coprod_{i \in \Lambda} S_1 \arrow[ll, "\exists ! \varphi"', dashed] & \coprod_{i \in \Lambda} S_i &                                     & \bigotimes_{i \in \Lambda} S_i \arrow[ll, "\exists ! \phi"', dashed]
\end{tikzcd}
\ \\
\ \\
$\functionfront{\varphi}{\coprod_{i \in \Lambda} S_i}{\bigotimes_{i \in \Lambda} S_i} \hspace{9 em} \functionfront{\phi}{\bigotimes_{i \in \Lambda} S_i}{\coprod_{i \in \Lambda} S_i}$
\end{center}
Wende nun die Universelle Eigenschaft von $\psi$ auf $\psi$ selbst an und erhalte $id_{\coprod_{i \in \Lambda} S_i} = \phi \circ \varphi$. Analog erhalte auch durch die universelle Eigenschschaft des Tensorpruduktes $id_{\bigotimes_i S_i} = \varphi \circ \phi$.
\begin{center}
\begin{tikzcd}
         & \mathcal{B} \arrow[rd, "\psi"] \arrow[ld, "\psi"'] &                                                                              &  &                  & \bigoplus_i S_i \arrow[rd, "g"] \arrow[ld, "g"'] &                                                                                     \\
\coprod_{i \in \Lambda} S_i &                                                    & \coprod_{i \in \Lambda} S_i \arrow[ll, "\exists ! id_{\coprod_{i \in \Lambda} S_i} = \phi \circ \varphi "', dashed] &  & \bigotimes_{i \in \Lambda} S_i &                                                  & \bigotimes_i S_i \arrow[ll, "\exists ! id_{\bigotimes_i S_i} = \varphi \circ \phi "', dashed]
\end{tikzcd}
\end{center}
Damit haben wir Isomorphismen zwischen $\coprod_{i \in \Lambda} S_i$ und $\bigotimes_i S_i$ gefunden.\\
Da das Koprodukt $\coprod_{i \in \Lambda} S_i = \colimes \mathcal{B}$ bis auf eine eindeutige Isomorphie eindeutig bestimmt ist \textit{(\cref{Eindeutigkeit des Kolimes})}, definiere dies ab jetzt als $\bigotimes_{i \in \Lambda} S_i$.\\
\ \\
\underline{Zu \textit{\textbf{2.}}:} Zeige, dass $\functionfront{q}{S_2}{S_2/Q}$ die in \cref{NeuDifferenzenkokerndef} eingeführten Eigenschaften des Differenzkokern`s  besitzt:
\begin{gather*}
q \circ f = q \circ g \text{ gilt, da } \kernel{q} = Q = \lbrace f(x) - g(x)\mid x \in C_2 \rbrace.
\end{gather*}
Sei nun ein R-Algabrahomomorphismus $\functionfront{q'}{S_2}{T'}$ mit $q' \circ f = q' \circ g$ gegeben.\\
Somit gilt $q' \circ (f - g) = 0$, wodurch $Q$ ein Untermodul von $Q' \defeq \kernel{q'}$ ist.\\ Mit dem Isomorphiesatz für R-Algebren erhalten wir:
\begin{gather*}
 \nolinebreak{S_2/Q' \simeq (S_2/Q)/(Q'/Q)}.
\end{gather*}
Somit ist $\function{q'}{S_2}{(S_2/Q)/(Q'/Q)}{y}{[y]'}$ eine isomorphe Darstellung von $\functionfront{q'}{S_2}{T'}$.
\begin{gather*}
\Rightarrow \exists ! \function{\varphi}{S_2/Q}{(S_2/Q)/(Q'/Q)}{[y]}{[y]'}\textit{ mit }(\varphi \circ q) = q'.
\end{gather*}
Also ist $S_2/Q$ zusammen mit $\functionfront{q}{S_2}{S_2/Q}$ der bis auf eine eindeutige Isomorphie eindeutig bestimmte Differenzkokern von $f$ und $g$.\\
\ \\
Damit haben wir gezeigt, dass Koprodukte beliebiger Mengen von R-Algebren und Differenzkokerne von je zwei R-Algebrenhomomorphismus existieren. Nach \cref{Kolimes durch Koprodukt und Differenzkokern} existieren somit in der Kategorie der R-Algebren Kolimiten beliebiger Diagramme.
\end{proof}


\ \\
\textcolor{blue}{\textbf{R-Modul-Kolimiten}}
\begin{prop}\label{R-Modul-Kolimiten} \textit{[Proposition A6.2 \ModulsOfDifferenzials]}\\
In der Kategorie der R-Moduln existieren Kolimiten beliebiger Diagramme, wobei gilt:
\begin{itemize}
\item[\textbf{1.}] Das Koprodukt einer Familie von $R-Moduln$ $\lbrace M_i \rbrace_{i \in \Lambda}$ entspricht deren direkter Summe $\bigoplus_{i \in \Lambda} M_i$.
\item[\textbf{2.}] Der Differenzenkokern zweier R-Modulhomomorphismen $\functionfront{f,g}{M_1}{M_2}$ entspricht dem Homomorphismus $\function{q}{M_2}{M_2/Q}{y}{[y]}$,\\
wobei $Q \defeq \lbrace f(x) - g(x)\mid x \in M_1 \rbrace$ das Bild der Differenz von $f$ und $g$ ist.
\end{itemize}
\end{prop}
\begin{proof} \ \\
\underline{Zu \textit{\textbf{1.}}:}
Sei $\mathcal{B} = \lbrace M_i \rbrace_{i \in \Lambda}$ die Unterkategorie der R-Moduln, welche $\lbrace M_i \rbrace_{i \in \Lambda}$ zusammen mit den Identitätsabbildungen enthält. Betrachte als Morphismus $\psi$ die jeweilige Einbettung von $M_i$ in $\bigoplus_{i \in \Lambda} M_i$:
\begin{gather*}
\functionfront{\psi}{\mathcal{B}}{\bigoplus_{i \in \Lambda} M_i} \text{ mit } \function{\psi_{M_i}}{M_i}{\bigoplus_{i \in \Lambda} M_i}{m_i}{(0, ...\cdots ,0,m_i,0, \cdots ,0)} \text{ für } i \in \Lambda
\end{gather*}
Somit lässt sich jedes $(m_1, \cdots m_n) \in \bigoplus_{i \in \Lambda} M_i$ \textit{(im Fall von $\vert \lambda \vert = \infty$ siehe \cref{Unendliche Indexmengen})} eindeutig durch die Elemente $m_i \in M_i$ (für $i \in \lbrace i , \cdots , n \rbrace$) darstellen:
\begin{gather*}
(m_1, \cdots ,m_n) = \sum_{i = 1}^n \psi_{M_i}(m_i)
\end{gather*}
Damit erfüllt $\psi$ die universelle Eigenschaft von $\colimes \mathcal{B}$, denn sei $\functionfront{\psi'}{\mathcal{B}}{M'}$ ein bieliebiger Morphismus, so existiert genau ein R-Modulhomomorphismus:
\begin{center}
$\function{\varphi}{\bigoplus_{i \in \Lambda} M_i }{M'}{(m_1, \cdots , m_n)}{\sum_{i = 1}^n \psi'_{M_i}(m_i)}$
\begin{tikzcd}
  & \mathcal{B} \arrow[rd, "\psi"] \arrow[ld, "\psi'"'] &                                            \\
M' &                                              & \bigoplus_i M_i \arrow[ll, "\exists ! \varphi"', dashed]
\end{tikzcd}\\
\end{center}
\comment{
Für ein beliebiges $i$ existiert genau ein $\function{\varphi_i}{M_i \oplus 0}{M'}{(0,...,0,m_i,0,...,0}{\psi_i '(m_i)}$ mit $\psi_i ' = \psi_i \circ \varphi_i$\\
$\Rightarrow  \exists ! \function{\varphi}{\bigoplus_i M_i}{M'}{(m_1,...,m_n)}{\sum_i \psi_i(m_i)}$\\
}
Also ist $\oplus_{i \in \Lambda} M_i$ zusammen mit den Einbettungen $\infunctionfront{\psi_{M_i}}{M_i}{\bigoplus_{i \in \Lambda} M_i}$ das bis auf eine eindeutige Isomorphie eindeutig bestimmte Koprodukt von $\lbrace M_i \rbrace_{i \in \Lambda}$.
\ \\
\textit{\textbf{2.}} Gehe hier vor wie bei \cref{R-Algebra-Kolimiten}. Dort haben wir schon gezeigt, dass der Differenzkokern von zwei R-Algebra-Homomorphismen dem Kokern, von deren Differenz entspricht.\\
\ \\
Damit haben wir gezeigt, dass Koprodukte beliebiger Mengen von R-Moduln und Differenzkokerne von je zwei R-Modulhomomorphismen existieren. Nach \cref{Kolimes durch Koprodukt und Differenzkokern} existieren somit in der Kategorie der R-Moduln Kolimiten beliebiger Diagramme.
\end{proof}


\section{Darstellung von Lokalisierung als Kolimes}
\ \\
\textcolor{blue}{\textbf{Lokalisierung von Algebren als Kolimes}}
\begin{prop}\label{Lokalisierung von Algebren als Kolimes} \textit{[vlg. Aufgabe A6.7 \ModulsOfDifferenzials]} \\
Sei $S$ eine $R-Algebra$ und $U \subseteq S$ multiplikativ abgeschlossen.
Dann gilt:
\begin{gather*}
 S[U^{-1}] = \colimes \mathcal{B}
\end{gather*}
Wobei $\mathcal{B}$ aus den Objekten $\lbrace \lok{S}{t} \vert t \in U \rbrace$ und den Morphismen\\
$\lok{S}{t} \longrightarrow \lok{S}{tt'}, \loke{s}{t^n}{t} \longmapsto \loke{st'^n}{(tt')^n}{(tt')}$ (für $t,t' \in U$) besteht.\\
\end{prop}
\begin{proof}
Sei $\functionfront{\psi}{\mathcal{B}}{T}$ der Kolimes von $\mathcal{B}$. Zeige $\lok{S}{U} \simeq T$, definiere dazu:
\begin{gather*}
\functionfront{\psi'}{\mathcal{B}}{\lok{S}{U}}\\
\function{\psi'_{\lok{S}{t}}}{\lok{S}{t}}{\lok{S}{U}}{\loke{s}{t^n}{t}}{\loke{s}{t^n}{U}}
\end{gather*}
$\psi'$ ist ein Morphismus, da für beliebige $t,t' \in U$ und $s \in S$ gilt:
\begin{gather*}
\loke{s}{t^n}{U} = \loke{st'^n}{(tt')^n}{U}
\end{gather*}
Durch die Universelle Eigenschaft des Kolimes erhalten wir einen eindeutigen Homomorphismus $\varphi$ mit:
\begin{gather*}
\varphi \circ \psi_{\lok{S}{t}} = \psi'_{\lok{S}{t}} \text{ für alle } \lok{S}{t} \in \mathcal{B}.
\end{gather*}
\comment{
\begin{center}
\begin{tikzcd}
            & \mathcal{B} \arrow[rd, "\psi"] \arrow[ld, "\psi'"'] &                                            \\
{S[U^{-1}]} &                                                     & A \arrow[ll, "\exists ! \varphi"', dashed]
\end{tikzcd}
\end{center}
}
Für die Umkehrabbildung $\functionfront{\phi}{S[U^{-1}]}{T}$ benötigen wir kleinere Vorüberlegungen:\\
Zunächst stellen wir fest, dass $\psi'$ ganz $\lok{S}{U}$ abdeckt, also:
\begin{gather*}
\text{Jedes } \loke{s}{u}{U} \in \lok{S}{U} \text{ lässt sich in der Form } \loke{s}{u}{U} = \psi_{\lok{S}{t}}(\loke{s}{t}{t}) \text{ schreiben }\textit{(für t = u).}
\end{gather*}
Allerdings ist diese Darstellung nicht eindeutig. Zeige also noch, dass $\phi$ unabhängig von der Wahl von eines Repräsentanten ist. Seien dazu $s_1,s_2 \in S , \, t_1,t_2 \in U$ beliebig, somit gilt:
\begin{align*}
\textit{Sei }\psi'_{\lok{S}{t}}(\loke{s_1}{t_1}{t}) = \psi'_{\lok{S}{t}}(\loke{s_2}{t_2}{t})\\
\Rightarrow  \exists u \in U: (s_1t_1 - s_2t_2) \cdot u = 0\\
\Rightarrow  \loke{s_1u}{t_1u}{tu} = \loke{s_2u}{t_2u}{tu}\\
\Rightarrow  \psi_{\lok{S}{t}}(\loke{s_1}{t_1}{t}) = \psi_{\lok{S}{t}}(\loke{s_2}{t_2}{t})
\end{align*}
Mit diesem Wissen können wir den R-Algebra-Homomorphismus $\functionfront{\phi}{\lok{S}{U}}{T}$ definieren:
\begin{gather*}
\function{\phi}{\lok{S}{U}}{T}{\psi'_{\lok{S}{t}}(\loke{s}{t}{t})}{\psi_{\lok{S}{t}}(\loke{s}{t}{t})}
\end{gather*}
$\phi \circ \varphi = id_T$ ergibt sich direkt aus der universellen Eigenschaft des Kolimes:
\begin{center}
\begin{tikzcd}
  & \mathcal{B} \arrow[rd, "\psi"] \arrow[ld, "\psi"'] &                                                              \\
T &                                                    & T \arrow[ll, "\exists ! id_T = \phi \circ \varphi"', dashed]
\end{tikzcd}
\end{center}
Für $\varphi \circ \phi \defshow id_{\lok{S}{U}}$ wähle $s \in S , t \in U$ beliebig. Für diese gilt:
\begin{gather*}
(\varphi \circ \phi)(\psi'(\loke{s}{t}{t})) =
 \varphi (\psi(\loke{s}{t}{t}) =
  \psi'(\loke{s}{t}{t})
\end{gather*}
Damit haben wir gezeigt, dass $\varphi,\phi$ Isomorphismen sind und somit $T \simeq \lok{S}{U}$ gilt. Da der Kolimes bis auf eine eindeutige Isomorphie eindeutig ist \textit{(siehe \cref{Eindeutigkeit des Kolimes})}, definiere ab sofort $\colimes \mathcal{B}$ als $\lok{S}{U}$.\end{proof}


\ \\
\textcolor{blue}{\textbf{Lokalisierung von Moduln als Kolimes} \textit{[Beweis von Proposition 16.9 \ModulsOfDifferenzials]}}
\begin{korrolar}\comment{\label{Lokalisierung von Moduln als Kolimes}}
Sei M ein S-Modul, wobei S eine R-Algebra ist. Sei weiter $U \subseteq S$ multiplikativ abgeschlossen. Dann gilt:
\begin{gather*}
\lok{M}{U} = \colimes \mathcal{C}
\end{gather*}
Wobei $\mathcal{C}$ aus den Objekten $\lbrace \Tensor{\lok{S}{U}}{\lok{S}{t}}{\lok{M}{t}} \vert t \in U \rbrace$ und folgenden Morphismen besteht:
\begin{gather*}
\Tensor{\lok{S}{U}}{\lok{S}{t}}{\lok{M}{t}} \longrightarrow
\Tensor{\lok{S}{U}}{\lok{S}{(tt')}}{\lok{M}{(tt')}} ,\\
\tensor{\loke{s}{u}{U}}{\lok{S}{t}}{\loke{m}{t^n}{t}} \longmapsto
\tensor{\loke{s}{u}{U}}{\lok{S}{t}}{\loke{t'^nm}{(tt')^n}{t}} 
\end{gather*}
\end{korrolar}
Auch wenn sich \cref{Lokalisierung von Algebren als Kolimes} hier nicht direkt anwenden lässt, so können wir doch im Beweis gleich vorgehen.
\begin{proof}
Sei $\functionfront{\psi}{\mathcal{C}}{T}$ der Colimes von $\mathcal{C}$. Zeige $\lok{M}{U} \simeq T$, definiere dazu folgenden Morphismus:
\begin{gather*}
\functionfront{\psi'}{\mathcal{C}}{\lok{M}{U}} \\
\function{\psi'_{t}}{\Tensor{\lok{S}{U}}{\lok{S}{t}}{\lok{M}{t}}}{\lok{M}{U}}{\tensor{\loke{s}{u}{U}}{\lok{S}{t}}{\loke{m}{t^n}{t}}}{\loke{sm}{ut^n}{U}}
\end{gather*}
Die Wohldefiniertheit von $\psi'_t$ für ein beliebiges $t \in U$ folgt direkt aus der Universellen Eigenschaft des Tensorprodukt`s. Denn für die bilineare Abbildung
 $\function{f}{\lok{S}{U} \oplus \lok{M}{t}}{\lok{M}{t}}{(\loke{s}{u}{U}, \loke{m}{t^n}{t})}{\loke{sm}{ut^n}{U}}$  gilt:
\begin{center}
\begin{tikzcd}
\lok{S}{U} \oplus \lok{M}{t} \arrow[r, "g"] \arrow[rd, "f"'] & \Tensor{\lok{S}{U}}{\lok{S}{t}}{\lok{M}{t}} \arrow[d, "\exists ! \psi'_t", dashed] \\
                                      & \lok{M}{U}                               
\end{tikzcd}
\end{center} 
Durch die Universelle Eigenschaft des Kolimes erhalten wir nun einen eindeutigen Homomorphismus $\functionfront{\varphi}{T}{\lok{M}{U}}$ mit:
\begin{gather*}
\varphi \circ \psi_{t} = \psi'_{t} \text{ für alle } t \in U.
\end{gather*}
\comment{
\begin{center}
\begin{tikzcd}
  & \mathcal{C} \arrow[rd, "\psi"] \arrow[ld, "\psi'"'] &                                            \\
\lok{M}{U} &                                                     & T \arrow[ll, "\exists ! \varphi"', dashed]
\end{tikzcd}
\end{center}
}
Für die Umkehrabbildung $\functionfront{\phi}{\lok{M}{U}}{T}$ benötigen wir kleinere Vorüberlegungen:\\
Wir stellen fest, dass für jedes $t \in U$ gilt:
\begin{gather*}
\text{Jedes } \loke{m}{u}{U} \in \lok{M}{U} \text{ lässt sich in der Form } \loke{m}{u}{U} = {\psi_t(\tensor{\loke{1}{u}{U}}{\lok{M}{t}}{\loke{m}{1}{t}}}) \text{ schreiben.}
\end{gather*}
Diese Darstellung ist unabhängig von den Wahl von $t \in U$, denn für beliebige $t_1,t_2,u \in U$ und $m \in M$ gilt:
\begin{gather*}
\psi'_{t_1}({\tensor{\loke{1}{u}{U}}{\lok{M}{t_1}}{\loke{m}{1}{t_1}}}) 
= \loke{m}{u}{U} 
= \psi'_{t_2}({\tensor{\loke{1}{u}{U}}{\lok{M}{t_2}}{\loke{m}{1}{t_2}}})\\
\text{Für $\psi$ gilt in diesem Fall: }\\
\psi_{t_1}({\tensor{\loke{1}{u}{U}}{\lok{M}{t_1}}{\loke{m}{1}{t_1}}})
= \psi_{t_1t_2}({\tensor{\loke{1}{u}{U}}{\lok{M}{t_1t_2}}{\loke{m}{1}{t_1t_2}}})
= \psi_{t_2}({\tensor{\loke{1}{u}{U}}{\lok{M}{t_2}}{\loke{m}{1}{t_2}}})
\end{gather*}
Definiere nun mit diesem Wissen folgenden Homomorphismus:
\begin{gather*}
\function{\phi}{\lok{M}{U}}{T}{\psi_t(\tensor{\loke{1}{u}{U}}{\lok{M}{t}}{\loke{m}{1}{t}})}{\psi'_t(\tensor{\loke{1}{u}{U}}{\lok{M}{t}}{\loke{m}{1}{t}})}
\end{gather*}
$\phi \circ \varphi = id_A$ ergibt sich direkt aus der Universellen Eigenschaft des Kolimes.\\
Für $\varphi \circ \phi \defshow id_{\lok{M}{U}}$ wähle $\loke{m}{u}{U} \in \lok{M}{U}$ beliebig, für dieses gilt:
\begin{gather*}
(\varphi \circ \phi) (\psi'_t(\tensor{\loke{1}{u}{U}}{\lok{M}{t}}{\loke{m}{1}{t}}))
 =\varphi(\psi_t(\tensor{\loke{1}{u}{U}}{\lok{M}{t}}{\loke{m}{1}{t}}))
  =\psi'_t(\tensor{\loke{1}{u}{U}}{\lok{M}{t}}{\loke{m}{1}{t}})
\end{gather*}
Damit haben wir $T \simeq \lok{M}{U}$ gezeigt, definiere also ab sofort $\lok{M}{U}$ als den Kolimes von $\mathcal{C}$.
\end{proof}

\section{Kähler-Differenzial von Kolimiten}
\textcolor{blue}{\textbf{Differenzial des Kolimes von R-Algebren} \textit{[vlg. Korolar 16.7 \ModulsOfDifferenzials]}}
\comment{Beide Beweise sind sehr kurz gefasst}
\begin{prop} \label{Differenzial des Kolimes von R-Algebren}
\ \\
\begin{itemize}
\item[\textbf{1.}]
Sei $T = \otimes_{i \in \Lambda} S_i$ das Koprodukt der R-Algebren $S_i$.\\
Dann gilt:
\begin{gather*}
\divR{T}{R} \simeq \bigoplus_{i\in \Lambda} ( \Tensor{T}{S_i}{\divR{S_i}{R}} )
\end{gather*}
\item[\textbf{2.}]
Seien $S_1,S_2$ R-Algebren und $\functionfront{\varphi,\varphi'}{S_1}{S_2}$ R-Algebra-Homomorphismen. Sei weiter $\functionfront{q}{S_2}{T}$ der Differenzkokern von $\varphi$,$\varphi '$.
Dann ist folgende Sequenz rechtsexakt:
\begin{center}
\begin{tikzcd}
\Tensor{T}{S_1}{\divR{S_1}{R}} \arrow[r, "f"] & \Tensor{T}{S_2}{\divR{S_2}{R}} \arrow[r, "g"] & \divR{T}{R} \arrow[r] & 0
\end{tikzcd}
\begin{gather*}
\textit{mit: } \function{f}{\tensor{T}{S_1}{\divR{S_1}{R}}}{\Tensor{T}{S_2}{\divR{S_2}{R}}}{\tensor{t}{S_2}{\divf{S_1}(x_1)}}{\tensor{t}{S_2}{\divf{S_2}(\varphi(x_1) - \varphi(x_2))}}\\
\function{g}{\Tensor{T}{S_2}{\divR{S_2}{R}}}{\divR{T}{R}}{\tensor{t}{S_2}{\divf{S_2}(x_2)}}{(\divf{T}\circ q)(x_2)}
\end{gather*}
\end{center}
\end{itemize}
\end{prop}
\begin{proof}\ \\
Für \textit{\textbf{1.}} finde durch die Universelle Eigenschaft des Kähler-Differenzials Isomorphismen $ \divR{T}{R} \longleftrightarrow \bigoplus_{i \in \Lambda} ( \Tensor{T}{S_i}{\divR{S_i}{R}} )$.\\
Definiere das Differenzial $\function{e}{T}{\bigoplus_{i \in \Lambda} \Tensor{T}{S_i}{\divR{S_i}{R}}}{(\tensor{s_i}{R}{...})}{(\tensor{1}{S_i}{\divf{S_1},...)}}$ und erhalte dadurch
\begin{center}
\begin{tikzcd}
T \arrow[rd, "e"'] \arrow[r, "\divf{T}"] & \divR{T}{R} \arrow[d, "\exists ! \varphi", dashed] \\
                                    & \bigoplus_{i\in \Lambda} \Tensor{T}{S_i}{\divR{S_i}{R}}                                       
\end{tikzcd}
$\functionfront{\varphi}{\divR{T}{R}}{\bigoplus_{i\in \Lambda} ( \Tensor{T}{S_i}{\divR{S_i}{R}} )}$.
\end{center}
Definiere nun das Differenzial $k: S_i \hookrightarrow T \longrightarrow \divR{T}{R}$ und erhalte dadurch:
\begin{center}
\begin{tikzcd}
S_i \arrow[rd, "k"'] \arrow[r, "\divf{S_i}"] & \divR{S_i}{R} \arrow[d, "\exists ! k'", dashed] \arrow[r, "a"] & \Tensor{T}{S_i}{\divR{S_i}{R}} \arrow[ld, "\phi_i"] \\
                                          & \divR{T}{R}                                                    &                     
\end{tikzcd}
$\functionfront{\phi_i}{\bigoplus_{i\in \Lambda} ( \Tensor{T}{S_i}{\divR{S_i}{R}} )}{\divR{T}{R}}$\\
\begin{gather*}
\function{\phi}{\bigoplus_{i\in \Lambda} ( \Tensor{T}{S_i}{\divR{S_i}{R}})}{\divR{T}{R}}{(...,\tensor{t_i}{S_i}{\divf{S_i}(s_i)},...)}{\prod_{i\in \Lambda} t_i \cdot \phi_i(\divf{S_i}(s_i))}
\end{gather*}
\end{center}
Damit haben wir zwei zueinander inverse Funktionen $\varphi ,\phi$ gefunden.\\
$\Rightarrow \divR{T}{R} \simeq \bigoplus_{i\in \Lambda} ( \Tensor{T}{S_i}{\divR{S_i}{R}} )$\\
\ \\
Für \textit{\textbf{2.}} Wende \cref{Konormale Sequenz} auf den Differenzkokern $\functionfront{q}{S_2}{S_2/Q}$ \textit{(vlg. \cref{R-Algebra-Kolimiten})} an und erhalte dadurch eine exakte Sequenz, welche ähnlich zu der gesuchten ist:
\begin{center}
\begin{tikzcd}
Q/Q^2 \arrow[r, "f'"] & \tensor{T}{S_2}{\divR{S_2}{R}} \arrow[r, "g"] & \divR{T}{R} \arrow[r] & 0
\end{tikzcd}
\end{center}
mit $\function{f'}{Q/Q^2}{{\Tensor{T}{S_2}{\divR{S}{R}}}}{[s_2]_{Q^2}}{\tensor{1}{S_2}{\divf{S_2}(s_2)}}$.\\
Somit gilt $\immage{f} = \Tensor{T}{S_2}{\divf{S_2}(Q)} = \immage{f'}$.\\
$\Rightarrow$ die gesuchte Sequenz ist exakt.
\end{proof}s


\ \\
\textcolor{blue}{\textbf{Differenzial von Polynomalgebren 1} \textit{[vlg. Proposition 16.1 \ModulsOfDifferenzials]}}
\begin{korrolar}\label{Differenzial von Polynomalgebren 1}
Sei $S = R[x_1,...,x_n]$ eine Polynomalgebra über R. Dann gilt:
\begin{gather*}
\divR{S}{R} = \bigoplus_{i \in \lbrace 1,...,n \rbrace} S \langle \divf{S}(x_i) \rangle 
\end{gather*}
Wobei $S\langle \divf{S}(x_i)\rangle$ das von $\divf{S}(x_i)$ erzeugt Modul über S ist.
\end{korrolar}
\begin{proof}
Wie in \cref{Darstellung der Polynomalgebra als Tensorprodukt} gezeigt, können wir $S$ als $\bigotimes_{i \in \lbrace 1,...,n \rbrace} R[x_i]$ schreiben. In \cref{Differenzial des Kolimes von R-Algebren} haben wir gezeigt, wie das Differenzial eines solchen Tensorproduktes aussieht:
\begin{gather*}
\divR{S}{R} = \bigoplus_{i \in \lbrace 1,...,n \rbrace} (\Tensor{S}{R[x_i]}{\divR{R[x_i]}{R}})
\end{gather*}
Da $R[x_i]$ die aus dem Element $x_i$ erzeugte Algebra über $R$ ist, folgt \textit{[vlg. BEMERKUNG ZU ENDLICH ERZEUGTEN ALGEBREN]}: 
\begin{gather*}
\divR{S}{R} = \bigoplus_{i \in \lbrace 1,...,n \rbrace} (\Tensor{S}{R[x_i]}{R[x_i]\langle \divf{S[x_i]}(x_i) \rangle})
\simeq \bigoplus_{i \in \lbrace 1,...,n \rbrace} S \langle \divf{S}(x_i) \rangle 
\end{gather*}
Für die letzte Isomorphie nutze, dass wegen $R[x_i] \subseteq S$ zum Einen $\divf{R[x_i]}$ als Einschränkung von $\divf{S}$ gesehen werden kann und zum Anderen $\Tensor{S}{R[x_i]}{R[x_i]} \simeq S$ gilt.
\end{proof}


\ \\
\textcolor{blue}{\textbf{Differenzial von Polynomalgebren 2} \textit{[vgl. Korrolar 16.6 \ModulsOfDifferenzials]}}
\begin{korrolar}\label{Differenzial von Polynomalgebren 2}
Sei S eine R-Algebra und $T \defeq S[x_1,...,x_n]$ eine Polynomalgebra über S. Dann gilt:
\begin{gather*}
\divR{T}{R} \simeq (\Tensor{T}{S}{\divR{S}{R}}) \oplus \bigoplus_{i \in \lbrace 1,...,n \rbrace} T \Verz{\divf{T}(x_i)}
\end{gather*}
\end{korrolar}
\begin{proof}
Betrachte $T$ als Tensorprodukt über R-Algebren und wende anschließend \cref{Differenzial des Kolimes von R-Algebren} an:
\begin{gather*}
T \simeq \Tensor{S}{R}{R[x_1,...,x_n]} \\
\Rightarrow \divR{T}{R} \simeq (\Tensor{T}{S}{\divR{S}{R}}) \oplus (\Tensor{T}{R[x_1,...,x_n]}{\divR{R[x_1,...,x_n]}{R}})
\end{gather*}
Zuletzt wende den soeben gezeigten \cref{Differenzial von Polynomalgebren 1} an und nutze schließlich $R[x_1,...,x_n] \subseteq T$ um das Tensorprodukt zu vereinfachen:
\begin{gather*}
\Tensor{T}{R[x_1,...,x_n]}{\divR{R[x_1,...,x_n]}{R}}\\
\simeq \Tensor{T}{R[x_1,...,x_n]}{\bigoplus_{i \in \lbrace 1,...,n \rbrace} R[x_1,...,x_n]\langle \divf{R[x_i]}(x_i) \rangle } \\
\simeq \bigoplus_{i \in \lbrace 1,...,n \rbrace} T \langle \divf{R}(x_i) \rangle
\end{gather*}
\end{proof}


\ \\
\textcolor{blue}{\textbf{Differenzial der Lokalisierung} \textit{[vlg. Proposition 16.9 \ModulsOfDifferenzials]}}
\begin{theorem}\label{Differenzial der Lokalisierung}
Sei $S$ eine $R-Algebra$ und $U \subseteq S$ multiplikativ abgeschlossen.
Dann gilt:
\begin{gather*}
\divR{\lok{S}{U}}{R} \simeq \Tensor{\lok{S}{U}}{S}{\divR{S}{R}} \text{, Wobei:}\\
 \divf{\lok{S}{U}}(\loke{1}{u}{U}) \longmapsto -\tensor{\loke{1}{u^2}{U}}{S}{\divf{S}(u)}
\end{gather*}
\end{theorem}
\begin{proof}
Wir wollen THEOREM16.8 \comment{\label{THEOREM16.8}} auf $\mathcal{B} = \lbrace \lok{S}{t} \vert t \in U \rbrace$ aus \cref{Lokalisierung von Algebren als Kolimes} anwenden.\\
Zeige also zunächsten den einfacheren Fall $\divR{\lok{S}{t}}{R} \simeq \Tensor{\lok{S}{t}}{S}{\divR{S}{R}}$ für ein beliebiges $t \in U$:
\begin{itemize}
\item[]
Nutze hierfür die Isomorphe Darstellung $\lok{S}{t} \simeq S[x]/(tx -1)$, sowie die Isomorphie $\divR{S[x]}{R} \simeq \Tensor{S[x]}{S}{\divR{S}{R}} \oplus S[x]\divf{S[x]}(x)$. aus \cref{Differenzial von Polynomalgebren 2}\\
Daraus erhalten wir folgende Isomorphismen:
\begin{gather*}
\functionfront{\alpha}{\lok{S}{t}}{S[x]/(tx -1)} \\
\functionfront{\beta}{S[x]/(tx -1)}{\lok{S}{t}} \\
\functionfront{\gamma}{\divR{S[x]}{R}}{ \Tensor{S[x]}{S}{\divR{S}{R}} \oplus S[x]\divf{S[x]}(x) }
\end{gather*}
Nutze diese nun, um $\divR{\lok{S}{t}}{R}$ isomorph zu $\Tensor{\lok{S}{t}}{S}{\divR{S}{R}}$ umzuformen:
\begin{center}
\begin{tikzcd}
\divR{\lok{S}{t}}{R} \arrow[d, "D\alpha"] & d_{\lok{S}{t}}(\loke{s}{t}{t}) \arrow[d, "D\alpha", maps to] \\
\divR{S[x]}{R}/\divf{S[x]}(tx - 1) \arrow[d, "\gamma"]  & {[\divf{S[x]}(sx)] = [x\divf{S[x]}(s) + s\divf{S[x]}(x)]} \arrow[d, "\gamma", maps to]  \\
(\Tensor{S[x]}{S}{\divR{S}{R}} \oplus S[x] \divf{S[x]}x) / ((tx - 1) \divf{S[x]}(tx - 1)) \arrow[d, "\beta"]   & {[\tensor{x}{S}{\divf{S}(s)} , s \divf{S[x]}(x)]} \arrow[d, "\beta", maps to]   \\
(\Tensor{\lok{S}{t}}{S}{\divR{S}{R}}) \oplus \lok{S}{t} \divf{S[x]}(x) / \divf{S[x]}(tx - 1) \defeqr M \arrow[d, "f"]       & {[\tensor{\loke{1}{t}{t}}{S}{\divf{S}(s)} , s \divf{S[x]}(x)]} \arrow[d, "f", maps to]       \\
\Tensor{\lok{S}{t}}{S}{\divR{S}{R}}                      & ( \tensor{\loke{1}{t}{t}}{S}{\divf{S}(s)} ) - ( \tensor{\loke{s}{t^2}{t}}{S}{\divf{S}(t)} )                              
\end{tikzcd}
\end{center}
Die ersten drei Schritte ergeben sich aus den oben angegeben Isomorphismen. Für den letzten Schritt definiere:
\begin{gather*}
\function{f}{M}{\Tensor{\lok{S}{t}}{S}{\divR{S}{R}}}{[\tensor{\loke{1}{t}{t}}{S}{\divf{S}(s)} , s \divf{S[x]}(x)]}{( \tensor{\loke{1}{t}{t}}{S}{\divf{S}(s)} ) - ( \tensor{\loke{s}{t^2}{t}}{S}{\divf{S}(t)} )}
\end{gather*}
Damit $f$ ein Isomorphismus ist, genügt es zu zeigen, dass $\Tensor{\lok{S}{t}}{S}{\divR{S}{R}}$ ein eindeutiges Repräsentantensystem von $M$ ist.\\
Sei dazu $[m_1,\loke{s}{t^{n}}{t}\divf{S[x]}(x)]$ ein beliebiger Erzeuger von $M$. Somit gilt:
\begin{gather*}
\divf{S[x]}(tx-1) = t\divf{S[x]}(x) + \beta(x)\divf{S[x]}(s)\\
\Rightarrow [0,\divf{S[x]}(x)] = [-\loke{1}{t^2}{t}\divf{S}(t),0]  \\ \comment{\label{\d_S[x] ~ d_S}}
\Rightarrow [m_1,\loke{s}{t^{n}}{t}\divf{S[x]}(x)] = [m_1 - \loke{s}{t^{n + 2}}{t}\divf{S}(t),0] = [f([m_1,\loke{s}{t^{n}}{t}\divf{S[x]}(x)]),0]
\end{gather*}
f ist also wie vermutet ein Isomorphismus und aus obigen Umformungen folgt
$\divR{\lok{S}{t}}{R} \simeq \Tensor{\lok{S}{t}}{S}{\divR{S}{R}} = \lok{\divR{S}{R}}{t}$. \\
Definiere für beliebige $t \in U$ folgenden Isomorphismus:
\begin{gather*}
f \circ \beta \circ \gamma  \circ D\alpha \defeqr \function{\delta_t}{\divR{\lok{S}{t}}{R}}{\lok{\divR{S}{R}}{t}}{\divf{\lok{S}{t}}(\loke{1}{t}{t})}{- \loke{\divf{S}(t)}{t^2}{t}}
\end{gather*}
\end{itemize}
Zeige nun den Allgemeinen Fall $\divR{\lok{S}{U}}{R} \simeq \Tensor{\lok{S}{U}}{S}{\divR{S}{R}}$:\\
Wähle $\mathcal{B} = \lbrace \lok{S}{t} \vert t \in U \rbrace$ wie in \cref{Lokalisierung von Algebren als Kolimes}, sodass $\colimes \mathcal{B} = \lok{S}{U}$ gilt.\\
Mit THEOREM16.8 \comment{\label{THEOREM16.8}} folgt somit:
\begin{gather*}
\divR{\lok{S}{U}}{R}  = \colimes{\mathcal{F}} \text{ mit:}\\
\function{\mathcal{F}}{\mathcal{B}}{(\lok{S}{U} - Module)}{\lok{S}{t}}{\tensor{\lok{S}{U}}{\lok{S}{t}}{\divR{\lok{S}{t}}{R}}}\\
( \functionfront{\varphi}{\lok{S}{t}}{\lok{S}{tt'}} )\\ \longmapsto 
( \functionfront{\tensor{1}{\lok{S}{t}}{D\varphi}}{ \Tensor{\lok{S}{U}}{\lok{S}{t}}{\divR{\lok{S}{t}}{R}}}{ \Tensor{\lok{S}{U}}{\lok{S}{t}}{( \Tensor{\lok{S}{t}}{\lok{S}{t}}{\divR{\lok{S}{tt'}}{R}})}} )
\end{gather*}
Zur Vereinfachung der Morphismen in $\mathcal{F}(\mathcal{B})$ definiere folgenden Isomorphismus:
\begin{gather*}
\functionfront{g}{ \Tensor{\lok{S}{U}}{\lok{S}{t}}{( \Tensor{\lok{S}{t}}{\lok{S}{t}}{\divR{\lok{S}{tt'}}{R}})}}{\Tensor{\lok{S}{U}}{\lok{S}{tt'}}{\divR{\lok{S}{tt'}}{R}}}\\
\tensor{\loke{s}{u}{U}}{\lok{S}{t}}{( \tensor{\loke{s'}{t}{t}}{\lok{S}{t}}{\divf{\lok{S}{tt'}}(x)} )} 
\longmapsto \tensor{\loke{s}{u}{U}}{\lok{S}{tt'}}{\varphi(\loke{s'}{t}{t})\divf{\lok{S}{tt'}}(x)}
\end{gather*}
Als letzten Schritt wollen wir \cref{Lokalisierung von Moduln als Kolimes} anwenden. Nutze dazu $\functionfront{\delta_t}{\divR{\lok{S}{t}}{R}}{\lok{\divR{S}{R}}{t}}$ um den zu $\mathcal{F}$ isomorphen Funktor $\mathcal{F'} \defeq \delta \circ \mathcal{F}$ zu erhalten. Um ein genaueres Bild von $\mathcal{F'}$ zu erlangen, betrachte folgendes Kommutatives Diagramm:
\begin{center}
\begin{tikzcd}
\lok{S}{t} \arrow[r, "\varphi"] \arrow[dd, "\mathcal{F}"] & \lok{S}{tt'} \arrow[d, "\mathcal{F}"]  \\
                                                 & \Tensor{\lok{S}{U}}{\lok{S}{t}}{( \Tensor{\lok{S}{t}}{\lok{S}{t}}{\divR{\lok{S}{tt'}}{R}})} \arrow[d, "g"]            \\
\Tensor{\lok{S}{U}}{\lok{S}{t}}{\divR{\lok{S}{t}}{R}} \arrow[ru, "\tensor{1}{\lok{S}{t}}{D\varphi}"] \arrow[d, "\delta_t"]         & \Tensor{\lok{S}{U}}{\lok{S}{tt'}}{\divR{\lok{S}{tt'}}{R}} \arrow[d, "\delta_{tt'}"] \\
\Tensor{\lok{S}{U}}{\lok{S}{t}}{\lok{\divR{S}{R}}{t}} \arrow[r, "\tensor{1}{\lok{S}{t}}{\varphi}"]                              & \Tensor{\lok{S}{U}}{\lok{S}{tt'}}{\lok{\divR{S}{R}}{tt'}}                          \\
\ \\
\loke{s}{t}{t} \arrow[r, "\varphi"] \arrow[d, "\divf{\lok{S}{t}}"]  & \loke{st'}{tt'}{tt'} \arrow[d, "\divf{\lok{S}{tt'}}"]           \\
{ \tensor{1}{\lok{S}{t}}{( \loke{1}{t}{t}\divf{\lok{S}{t}}(\loke{s}{1}{t}) + \loke{s}{1}{t}\divf{\lok{S}{t}}(\loke{1}{t}{t}))} \; } \arrow[d, "\delta_t"] \arrow[r, "g \circ (\tensor{1}{\lok{S}{t}}{D\varphi}) "] & {\; \tensor{1}{\lok{S}{tt'}}{( \loke{1}{tt'}{tt'}\divf{\lok{S}{tt'}}(\loke{st'}{1}{tt'}) + \loke{st'}{1}{tt'}\divf{\lok{S}{tt'}}(\loke{1}{tt'}{tt'}))}} \arrow[d, "\delta_{tt'}"] \\
\tensor{1}{\lok{S}{t}}{( \loke{\divf{S}(s)}{t}{t} - \loke{s\divf{S}(t)}{t^2}{t}  )} \arrow[r, "\tensor{1}{\lok{S}{t}}{\varphi}"]                     & \tensor{1}{\lok{S}{tt'}}{( \loke{t'\divf{S}(s)}{tt'}{tt'} - \loke{st'\divf{S}(t)}{(tt')^2}{tt'}  )} \textbf{(*)}                         
\end{tikzcd}
\end{center}
Dass das Diagramm in dieser Form kommutiert, ergibt sich in fast allen Fällen direkt aus dem Einsetzen in die entsprechenden Homomorphismen. Der einzige Fall, welcher nicht direkt klar ist, ist \textbf{(*)}. Rechne diesen also nochmal nach:
\begin{gather*}
\delta_{tt'}( \tensor{1}{\lok{S}{tt'}}{( \loke{1}{tt'}{tt'}\divf{\lok{S}{tt'}}(\loke{st'}{1}{tt'}) + \loke{st'}{1}{tt'}\divf{\lok{S}{tt'}}(\loke{1}{tt'}{tt'}))} )\\
= \tensor{1}{\lok{S}{tt'}}{( \loke{\divf{S}(st')}{tt'}{tt'} - \loke{t's\divf{S}(tt')}{(tt')^2}{tt'} )} \\
= \tensor{1}{\lok{S}{tt'}}{( \loke{t'\divf{S}(s')}{tt'}{tt'} + \loke{s\divf{S}(t')}{tt'}{tt'} 
- \loke{tt'\divf{S}(t')}{(tt')^2}{tt'} - \loke{t'^2s\divf{S}(t)}{(tt')^2}{tt'} )} \\
\comment{ = \tensor{1}{\lok{S}{tt'}}{( \loke{t'\divf{S}(s)}{tt'}{tt'} + \loke{s\divf{S}(t')}{tt'}{tt'} - \loke{s\divf{S}(tt')}{tt'}{tt'} - \loke{t'^2s\divf{S}(t)}{(tt')^2}{tt'} )} \\ }
= \tensor{1}{ \lok{S}{tt'}}{( \loke{t'\divf{S}(s)}{tt'}{tt'} - \loke{t'^2s\divf{S}(t)}{(tt')^2}{tt' } )} \\
= (\tensor{1}{\lok{S}{t}}{\varphi})(\tensor{1}{\lok{S}{t}}{( \loke{\divf{S}(s)}{t}{t} - \loke{s\divf{S}(t)}{t^2}{t}  )})
\end{gather*}
\comment{Damit haben wir einen zu $\mathcal{F}$ isomorphen Funktor gefunden:
\begin{gather*}
\functionfront{\mathcal{F'}}{\mathcal{B}}{(\lok{S}{U} - Module)}\\
\lok{S}{t} \longmapsto \Tensor{\lok{S}{U}}{\lok{S}{t}}{\lok{\divR{S}{R}}{t}}\\
\varphi \longmapsto \tensor{1}{\lok{S}{t}}{\varphi}
\end{gather*}}
Damit ist $\mathcal{F'}$ zu $\mathcal{F}$ isomorph und für $\mathcal{C} \defeq \mathcal{F'}(\mathcal{B})$ gilt $\divR{\lok{S}{U}}{R}  = \colimes{\mathcal{F'}} = \colimes{\mathcal{C}}$ \textit{[vlg. \cref{Vereinfachung des Kolimes}]}.
Wobei die Form von $\mathcal{C}$ genau dem Fall aus \cref{Lokalisierung von Moduln als Kolimes} entspricht:
\begin{gather*}
\comment{\divR{\lok{S}{U}}{R}  = \colimes{\mathcal{C}} \text{, wobei:} \\}
\mathcal{C} = \lbrace \Tensor{\lok{S}{U}}{\lok{S}{t}}{\lok{\divR{S}{R}}{t}} \vert t \in U \rbrace \textit{ mit den Morphismen }\\
\functionfront{\tensor{1}{\lok{S}{t}}{\varphi}}{\Tensor{\lok{S}{U}}{\lok{S}{t}}{\lok{\divR{S}{R}}{t}}}{\Tensor{\lok{S}{U}}{\lok{S}{tt'}}{\lok{\divR{S}{R}}{tt'}}} \\
\tensor{\loke{s}{u}{U}}{\lok{S}{t}}{\loke{\divf{S}(x)}{t^n}{t}} \longmapsto \tensor{\loke{s}{u}{U}}{\lok{S}{tt'}}{\loke{t'^n\divf{S}(x)}{(tt')^n}{tt'}}
\end{gather*}
Somit folgt $\colimes \mathcal{C} = \lok{\divR{S}{R}}{U}$ und wir haben $\divR{\lok{S}{U}}{R} = \lok{\divR{S}{R}}{U}$ gezeigt.
\end{proof}
\end{document}