\documentclass[10pt,a4paper]{report}
\usepackage[utf8]{inputenc}
\usepackage{amsmath}
\usepackage{amsthm}
\usepackage{amsfonts}
\usepackage{amssymb}
\usepackage{color}
\usepackage{tikz-cd}
\usepackage{calc}
\usepackage{setspace}
\usepackage[german]{babel}
\usetikzlibrary{babel}
\usepackage{cleveref}

\newcommand{\comment}[1]{}
\renewcommand{\baselinestretch}{1.1}

\newcommand{\ModulsOfDifferenzials}{David Eisenbud 1994}
\newcommand{\Algebra}{Christian Karpfinger, Kurt Meyberg 2009}

\newcounter{Aussage}[chapter]

\newtheorem{satz}[Aussage]{Satz}
\newtheorem{theorem}[Aussage]{Theorem}
\newtheorem{prop}[Aussage]{Proposition}
\newtheorem{korrolar}[Aussage]{Korrolar}
\newtheorem{lemma}[Aussage]{Lemma}
\newtheorem{bem}[Aussage]{Bemerkung}
\newtheorem{definition}[Aussage]{Definition}
\newtheorem{bsp}[Aussage]{Beispiel}

\newcommand{\functionfront}[3]{\nolinebreak{#1:#2 \longrightarrow #3}}
\newcommand{\functionback}[3]{\nolinebreak{#1:#2 \longmapsto #3}}
\newcommand{\function}[5]{\nolinebreak{#1:#2 \longrightarrow #3 \, , \, #4 \longmapsto #5}}
\newcommand{\infunctionfront}[3]{\nolinebreak{#1:#2 \hookrightarrow #3}}
\newcommand{\divR}[2]{\Omega_{#1/#2}}
\newcommand{\divf}[1]{d_{#1}}
\comment{\newcommand{\divf}[2][]{d_{#1}}}
\newcommand{\Tensor}[3]{#1 \otimes_{#2} #3}
\newcommand{\tensor}[3]{#1 \otimes #3}
\newcommand{\lok}[2]{#1 [#2^{-1}]}
\newcommand{\loke}[3]{(\frac{#1}{#2})_{_{#3}}}
\comment{\newcommand{\loke}[3]{(#1,#2)_{mod\sim_{#3}}}}

\newcommand{\colimes}[0]{\lim\limits_{ \longrightarrow }}
\newcommand*{\defeq}{\mathrel{\vcenter{\baselineskip0.5ex \lineskiplimit0pt
                     \hbox{\scriptsize.}\hbox{\scriptsize.}}}%
                     =}
\newcommand*{\defeqr}{= \mathrel{\vcenter{\baselineskip0.5ex \lineskiplimit0pt
                     \hbox{\scriptsize.}\hbox{\scriptsize.}}}}

\newcommand*{\defshow}{\stackrel{!}{=}}
\newcommand{\kernel}[1]{kern(#1)}
\newcommand{\immage}[1]{im(#1)}
\newcommand{\Verz}[1]{\langle #1 \rangle}


\begin{document}
\chapter{Kolimes}
\section{Einführung in den Kolimes}
\textcolor{blue}{\textbf{Definition des Kolimes}}
\begin{definition}\label{Definition des Kolimes} \textit{[vgl. Anhang A6 \ModulsOfDifferenzials]}
Sei $\mathcal{A}$ eine Kategorie.
\begin{itemize}
\item Ein \underline{Diagramm} über $\mathcal{A}$ ist eine Kategorie $\mathcal{B}$ zusammen mit einem Funktor $\functionfront{\mathcal{F}}{\mathcal{B}}{\mathcal{A}}$.
\item Sei $\functionfront{\mathcal{F}}{\mathcal{B}}{\mathcal{A}}$ ein Diagramm und $A \in \mathcal{A}$ ein Objekt. Dann definieren wir einen \underline{Morphismus} $\functionfront{\psi}{\mathcal{F}}{A}$ als eine Menge von Funktionen 
$\nolinebreak{\lbrace \psi_B \in Hom(F(B),A) \vert B \in \mathcal{B} \rbrace}$, wobei für alle $B_1,B_2 \in \mathcal{B}$ und $\varphi \in Hom(B_1,B_2)$ folgendes Diagramm kommutiert:
\begin{center}
\begin{tikzcd}
\mathcal{F}(B_1) \arrow[rrd, "\psi_{B_1}"] \arrow[dd, "\mathcal{F}(\varphi )"] &  &   \\
                                   &  & C \\
\mathcal{F}(B_2) \arrow[rru, "\psi_{B_2}"']                 &  &  
\end{tikzcd}
\end{center}
\item Der \underline{Kolimes} $\colimes \mathcal{F}$ eines Diagramms $\functionfront{\mathcal{F}}{\mathcal{B}}{\mathcal{A}}$ ist ein Paar aus einem Objekt $A \in \mathcal{A}$ zusammen mit einem Morphismus $\functionfront{\psi}{\mathcal{F}}{A}$, welche folgende universelle Eingenschaft erfüllen:
\begin{center}
Für Objekte $A' \in \mathcal{A}$ und alle Morphismen $\functionfront{\psi '}{\mathcal{F}}{A'}$ existiert genau eine Funktion $\varphi \in Hom(A,A')$, sodass folgendes Diagramm kommutiert:
\begin{tikzcd}
  & \mathcal{F} \arrow[rd, "\psi"] \arrow[ld, "\psi '"'] &                            \\
A' &                                    & A \arrow[ll, "\exists ! \varphi "', dashed]
\end{tikzcd}
\end{center}
\end{itemize}
\end{definition}


\ \\
\textcolor{blue}{\textbf{Eindeutigkeit des Kolimes} \textit{[vgl. A6 \ModulsOfDifferenzials]}}
\begin{lemma}\comment{\label{Eindeutigkeit des Kolimes}}
Seien $\mathcal{B},\mathcal{A}$ zwei Kategorien und $\functionfront{\mathcal{F}}{\mathcal{B}}{\mathcal{A}}$ ein Funktor. Dann ist im Falle der Existenz $\colimes \mathcal{F}$ eindeutig bestimmt.
\end{lemma}
\begin{proof}
Seien $A_1 \in \mathcal{A}, (\functionfront{\psi_1}{\mathcal{F}}{A_1}) $ und $A_2 \in \mathcal{A} , (\functionfront{\psi_2}{\mathcal{F}}{A_2}) $ beide gleich $\colimes \mathcal{F}$.\\
Erhalte durch die universelle Eigenschaft des Kolimes die eindeutig bestimmten Funktionen $\varphi_1 \in Hom_{\mathcal{A}}(A_1,A_2)$ und $\varphi_2 \in Hom_{\mathcal{A}}(A_2,A_1)$, für welche die folgende Diagramme kommutieren:
\begin{center}
\begin{tikzcd}
  & \mathcal{F} \arrow[rd, "\psi_1"] \arrow[ld, "\psi_2"'] &                            &  &   & \mathcal{F} \arrow[rd, "\psi_2"] \arrow[ld, "\psi_1"'] &                            \\
A_2 &                                    & A_1 \arrow[ll, "\exists ! \varphi_1"', dashed] &  & A_1 &                                    & A_2 \arrow[ll, "\exists ! \varphi_2"', dashed]
\end{tikzcd}
\end{center}
\begin{flushleft}
Wende nun die Universelle Eigenschaft von $\psi_1$ auf $\psi_1$ selbst an und erhalte $id_{A_1} = \varphi_2 \circ \varphi_1$. Analog erhalte auch $id_{A_2} = \varphi_1 \circ \varphi_2$.
\end{flushleft}
\begin{center}
\begin{tikzcd}
  & \mathcal{F} \arrow[rd, "\psi_1"] \arrow[ld, "\psi_1"'] &                            \\
A_1 &                                    & A_1 \arrow[ll, "\exists ! id_{A_1} = \varphi_2 \circ \varphi_1"', dashed]
\end{tikzcd}
\end{center}
Somit existiert genau eine Isomorphie $\functionfront{\varphi_1}{A_1}{A_2}$.
\end{proof}


\ \\
\textcolor{blue}{\textbf{Vereinfachung des Kolimes}}
\begin{korrolar}\label{Vereinfachung des Kolimes} \textit{[Eigene Überlegung ]}\\
Sei $\mathcal{A}$ eine Kategorie und $(\mathcal{B}, \functionfront{\mathcal{F}}{\mathcal{B}}{\mathcal{A}})$ ein Diagramm. Betrachte die Unterkategorie $\mathcal{F}(B) \subseteq \mathcal{A}$ zusammen mit dem Inklusionsfunktor $\mathcal{F}(B)\hookrightarrow \mathcal{A}$ ebenfalls als Diagramm. Dann gilt:
\begin{center}
$\colimes \mathcal{F}$ existiert genau dann, wenn $\colimes (\mathcal{F}(\mathcal{B}) \hookrightarrow \mathcal{A})$ existiert.\\
Mit $\colimes \mathcal{F} = \colimes (\mathcal{F}(\mathcal{B}) \hookrightarrow \mathcal{A})$.
\end{center}
\end{korrolar}
\begin{proof}
Dies folgt direkt aus unserer Definition von Morphismen:\\
In \cref{Definition des Kolimes} haben wir einen Morphismus $\functionfront{\psi}{\mathcal{F}}{A}$ als eine Menge von Funktionen $\mathcal{\psi_B} \in Hom_{\mathcal{A}}(\mathcal{F}(B),A)$ definiert. Dies zeigt, dass es keinen Unterschied macht, ob wir von Morphismen $\functionfront{\psi}{\mathcal{F}}{A}$ oder von Morphismen $\functionfront{\psi}{(\mathcal{F}(B)\hookrightarrow \mathcal{A})}{A}$ reden.\\
Wenn wir nun die universelle Eigenschaft des Kolimes genauer betrachten, sehen wir, dass diese sich nur auf Morphismen $\mathcal{F} \longrightarrow A$ bzw. $(\mathcal{F}(\mathcal{B}) \hookrightarrow \mathcal{A}) \longrightarrow A$ und auf die Kategorie $\mathcal{A}$ bezieht. Es macht also keinen Unterschied, ob wir vom Kolimes des Diagramms $(\mathcal{B}, \functionfront{\mathcal{F}}{\mathcal{B}}{\mathcal{A}})$ oder vom Kolimes des Diagramms $(\mathcal{F}(B),\mathcal{F}(B)\hookrightarrow \mathcal{A})$ sprechen.
\end{proof}
Es genügt also im Fall von Kolimtenn Diagramme $(\mathcal{B},\mathcal{B}\hookrightarrow\mathcal{A})$ mit $\mathcal{B} \subseteq \mathcal{A}$ zu betrachten. Zur Vereinfachung schreibe für $\mathcal{B} \subseteq \mathcal{A}$ in Zukunft $\colimes \mathcal{B}$ anstatt von $\colimes (\mathcal{B} \hookrightarrow \mathcal{A})$.


\ \\
\textcolor{blue}{\textbf{DifferenzkokernUndKoproduktDef}}
\begin{definition}\label{DifferenzkokernUndKoproduktDef} \textit{[vlg. A6 \ModulsOfDifferenzials]}\\
Sei $\mathcal{A}$ eine Kategorie.
\begin{itemize}
\item Das Koprodukt von $ \lbrace B_i \rbrace_{i \in \Lambda} \subseteq \mathcal{A}$ wird durch $\coprod_{i \in \Lambda} \lbrace B_i \rbrace := \colimes\mathcal{B}$ definiert, wobei $\lbrace B_i \rbrace_{i \in \Lambda}$ die Objekte und die Identitätsabbildungen $\lbrace \functionfront{id_{B_i}}{B_i}{B_i} \rbrace_{i \in \Lambda}$ die einzigen Morphismen von $\mathcal{B}$ sind.
\item Der Differenzkokern von $f,g \in Hom_{\mathcal{A}}(C_1,C_2)$ wird durch $\colimes \mathcal{C}$ definiert,
wobei $\lbrace C_1,C_2 \rbrace$ die Objekte und $ \lbrace f,g \rbrace$ zusammen mit den Identitätsabbildungen die Morphismen von $\mathcal{C}$ sind.
\end{itemize}
\end{definition}


\ \\
\textcolor{blue}{\textbf{Kolimes durch Koprodukt und Differenzkokern}}
\begin{theorem}\label{Kolimes durch Koprodukt und Differenzkokern} \textit{[Proposition A6.1 \ModulsOfDifferenzials]}\\
Sei $\mathcal{A}$ eine Kategorie, in der Koprodukte beliebiger Mengen von Objekten und Differenzkokerne von je zwei Morphismen existieren. Dann existiert für jedes Diagramm $\functionfront{\mathcal{F}}{\mathcal{B}}{\mathcal{A}}$ dessen Kolimes $\colimes \mathcal{F}$.
\end{theorem}
\begin{proof}
In \cref{Vereinfachung des Kolimes} haben wir gesehen, dass es genügt den Fall $\mathcal{B} \subseteq \mathcal{A}$ zu betrachten. Konstruiere also für eine beliebige Unterkategorie $\mathcal{B} \subseteq \mathcal{A}$ deren Kolimes $\colimes(\mathcal{F})$:\\
Bezeichne für jeden Morphismus $\gamma \in Morph_{\mathcal{C}}$ dessen Definitionsbreich mit $B_{\gamma} \in \mathcal{B}$. Definiere nun:
\begin{itemize}
\item $C_1 \defeq \coprod_{\gamma \in Morph_{\mathcal{B}}} B_{\gamma}$ ist das Koprodukt aller Objekte von $\mathcal{B}$, in dem jedes Objekt so oft vorkommt, wie es Definitionsbereich eines $\gamma \in Morph_{\mathcal{B}}$ ist.\\
Sei $\functionfront{\psi^1}{\lbrace B_{\gamma} \vert \gamma \in Morph_{\mathcal{B}}\rbrace}{C_1}$ der dazugehörige Morphismus.
\item $C_2 \defeq \coprod_{B \in Obj_\mathcal{B}}$ ist das Koprodukt aller Objekte von $\mathcal{B}$. Sei $\functionfront{\psi^2}{\lbrace B \vert B \in Obj_\mathcal{B} \rbrace}{C_2}$ der dazugehörige Morphismus.
\end{itemize}
Konstruiere nun $f,g \in Hom_{\mathcal{B}}{C_1,C_2}$ so, dass der Differenzkokern von $f$ und $g$ gleich dem Kolimes von $\mathcal{B}$ ist:
\begin{center}
Für $f$ betrachte den Morphismus $\functionfront{\Phi}{\lbrace B_{\gamma} \vert \gamma \in Morph_{\mathcal{B}}\rbrace}{C_2}$,\\
mit $\Phi_{B_{\gamma}} = \psi^2_{\gamma(B_{\gamma})} \circ \gamma$ für $B_{\gamma} \in \lbrace B_{\gamma} \vert \gamma \in Morph_{\mathcal{B}}\rbrace$.\\
Wende die universelle Eigenschaft von $C_1 = \colimes \lbrace B_{\gamma} \vert \gamma \in Morph_{\mathcal{B}}\rbrace$ an:\\
\ \\
\begin{tikzcd}
    & \lbrace B_{\gamma} \vert \gamma \in Morph_{\mathcal{B}}\rbrace \arrow[rd, "\psi^1"] \arrow[ld, "\psi^1"'] &                                        \\
C_2 &                                                                                                                      & C_1 \arrow[ll, "\exists ! f"', dashed]
\end{tikzcd}
\ \\
Wähle $f \in Hom_{\mathcal{B}}(C_1,C_2)$ als die eindeutige Funktion, für die $\Phi = \psi^1 \circ f$ gilt.\\
\ \\
Für $g$ betrachte den Morphismus $\functionfront{\Phi'}{\lbrace B_{\gamma} \vert \gamma \in Morph_{\mathcal{B}}\rbrace}{C_2}$,\\
mit $\Phi'_{B_{\gamma}} = \psi^2_{\gamma(B_{\gamma})}$ für $B_{\gamma} \in \lbrace B_{\gamma} \vert \gamma \in Morph_{\mathcal{B}}\rbrace$.\\
Wähle analog zu $f$ mithilfe der universellen Eigenschaft des Kolimes $g \in Hom_{\mathcal{B}}(C_1,C_2)$ als die eindeutige Funktion, für die $\Phi' = \psi^1 \circ g$ gilt.
\end{center}
Sei $C \in Obj_{\mathcal{B}}$ zusammen mit dem Morphismus $\zeta$ der Differenzkokern von $f$,$g$.\\
Zeige Abschließend, dass $(C, \functionfront{\psi}{\mathcal{B}}{\zeta \circ \psi^2}$ Der Kolimes von $\mathcal{B})$ ist:\\
\end{proof}


\ \\
\textcolor{blue}{\textbf{NeuDifferenzenkokerndef} \textit{[vlg. Wikipedia aber eigener Beweis]}}
\begin{lemma}\label{NeuDifferenzenkokerndef} Sei $\mathcal{A}$ eine Kategorie mit $C_1,C_2 \in Hom_{\mathcal{A}}(C_1,C_2)$, so sind folgende Formulierungen äquivalent zur Definition des Differenzkokern`s $T \defeq \colimes \mathcal{C}$
\begin{itemize}
\item[1.] Es existiert ein Morphismus $\functionfront{\psi}{\mathcal{C}}{T}$, mit der Eigenschaft, dass für alle Morphismen $\functionfront{\psi '}{\mathcal{C}}{T'}$ genau ein $\varphi \in Hom_{\mathcal{A}}(T,T')$ mit $\varphi \circ \psi = \psi '$ existiert.
\item[2.] Es existiert ein $q \in Hom_{\mathcal{A}}(C_2,T)$ mit $q \circ f = q \circ g$ und der Eigenschaft, dass für alle Morphismen $q' \in Hom_{\mathcal{A}}(C_2,Z)$ mit $q' \circ f = q' \circ g$ genau ein $\varphi \in Hom_{\mathcal{A}}(T,T')$ mit $\varphi \circ q = q'$ existiert.
\begin{center}
\begin{tikzcd}
C_1 \arrow[r, "{f,g}"] \arrow[r] & C_2 \arrow[r, "q"] \arrow[rd, "q'"] & T \arrow[d, "\exists !\varphi", dashed] \\
                                 &                                     & T'                                     
\end{tikzcd}
\end{center}
\end{itemize}
\end{lemma}
\begin{proof}
\textit{1.} ist offensichtlich eine Ausformulierung der Einführung des Kolimes aus \cref{DifferenzkokernUndCoproduktDef}, zeige also im folgenden noch die Äquivalenz von \textit{1.} und \textit{2.}
\begin{itemize}
\item \underline{1 $\Rightarrow$ 2:}
\begin{itemize}
\item[] Da $\functionfront{\psi}{\mathcal{C}}{T}$ ein Morphismus ist, gilt für $\lbrace f,g \rbrace = Hom_{\mathcal{C}}(C_1,C_2)$:\\ $\psi_{C_1} = \psi_{C_2} \circ f = \psi_{C_1} \circ \psi_{C_2}$, setze also 
 $q  \defeq \psi_{C_2}$.
\item[] Sei nun $q' \in Hom_{\mathcal{A}}(C_2,T)$ mit der Eigenschaft $q' \circ f = q' \circ g$ gegeben:\\
 Definiere den Morphismus $\functionfront{\psi '}{\mathcal{C}}{T}$ als $\lbrace \psi_1 = q' \circ f , \psi_2 = q' \rbrace$,  somit folgt direkt aus der Universellen Eigenschaft von $\psi$, dass genau ein $\varphi \in Hom_{A}(C_2,T)$ existiert, mit $ \varphi \circ q = q '$.
\end{itemize}
\item \underline{2 $\Rightarrow$ 1:}
\begin{itemize}
\item[] Definiere $\functionfront{\psi }{\mathcal{C}}{T}$ als $\lbrace \psi_1 = q \circ f , \psi_2 = q \rbrace$.
Durch die Eigenschaft von $q$ gilt $\psi_{C_1} = \psi_{C_2} \circ f = \psi_{C_2} \circ g$.
\item[] Sei nun $\functionfront{\psi '}{\mathcal{C}}{\mathcal{A}}$ ein beliebiger Morphismus.\\
Definiere $d' \defeq \psi '$, somit existiert durch die Eigenschaft von $d$ genau ein $\varphi \in Hom_{\mathcal{A}}(C_2,T)$ mit $\varphi \circ q = q'$.
\begin{gather*}
\Rightarrow \varphi \circ \psi_2 = \psi '_2 \\
\textit{und }\varphi \circ \psi_1 = \varphi \circ \psi_2 \circ f = \varphi \circ \psi '_2 \circ f = \varphi \circ \psi '_1
\end{gather*}
\end{itemize}
\end{itemize}
\end{proof}
Wenn im weiteren Verlauf von dem Differenzkokern zweier Homomorphismen $\functionfront{f,g}{C_1}{C_2}$ gesprochen wird, meinen wir damit den Homomorphismus $\functionfront{q}{C_2}{T}$ aus \cref{NeuDifferenzenkokerndef}.


\ \\
\textcolor{blue}{\textbf{Tensorprodukt des Differenzenkokerns} \textit{[Eigene Bemerkung]}}
\begin{bem} \comment{\label{Tensorprodukt des Differenzenkokerns}}
Seien $f,g \in Hom_{\mathcal{A}}(S_1,S_2)$ R-Algebra-Homomorphismen, so können wir für den Differenzenkokern $\functionfront{q}{S_2}{T}$ für ein beliebiges $S_1$-Modul das Tensorprodukt $\Tensor{T}{C_1}{M}$ definieren. 
\begin{gather*}
\textit{für } s_1 \in S_1 \textit{ und } \tensor{t}{S_1}{m}) \in \Tensor{T}{C_1}{M} \textit{ gilt: }\\
s_1 \cdot (\tensor{t}{S_1}{m}) = \tensor{((q \circ f)(s_1)) \cdot t}{S_1}{m} = \tensor{((q \circ g)) \cdot (s_1)t}{S_1}{m}
\end{gather*}
\end{bem}


\ \\
\textcolor{blue}{\textbf{R-Algebra-Kolimiten} \textit{[vlg. Proposition A6.7 \ModulsOfDifferenzials]}}
\begin{prop} \label{R-Algebra-Kolimiten}
in der Kategorie der R-Algebren existieren Koprodukte und Differenzkokerne, wobei:
\begin{itemize}
\item[\textbf{1.}] Das Koprodukt einer endlichen Familie von $R-Algebren$ $\lbrace S_i \rbrace_{i \in \Lambda}$ entspricht deren Tesorprodukt $\bigotimes_{i \in \Lambda} S_i$. 
\item[\textbf{2.}] Der Differenzkokern zweier R-Algebra-Homomorphismen $\functionfront{f,g}{S_1}{S_2}$ einspricht dem Homomorphismus $\function{q}{S_2}{S_2/Q}{y}{[y]}$, wobei $Q \defeq \lbrace f(x) - g(x)\mid x \in S_2 \rbrace$ das Bild der Differenz von $f$ und $g$ ist.
\end{itemize}
\end{prop}
\begin{proof}
Zu \textit{\textbf{1.}}:\\
Sei $\mathcal{B}$ die Unterkategorie der R-Algebren, welche $\lbrace S_i \rbrace_{i \in \Lambda}$ zusammen mit den Identitätsabbildungen enthält. Wir wollen die universellen Eigenschaften des Tensorproduktes und des Kähler-Differenzials nutzen, um einen Isomorphismus zwischen $\colimes \mathcal{F}$ und $\bigotimes_{i \in \Lambda} B_i$ zu finden.\\ 
Es sind der Morphismus $\functionfront{\psi}{\mathcal{B}}{\colimes \mathcal{B}}$ und die bilineare Abbildung $\functionfront{g}{\oplus_i S_i}{\otimes_i S_i}$ gegeben.\\
Konstruiere den Morphismus $\functionfront{\psi'}{\mathcal{B}}{\otimes_i S_i}$ durch $\function{\psi'_i}{S_i}{\otimes_i S_i}{s_i}{g(1,..,1,s_i,1,..,1)}$ für $i \in \lambda$ und die bilineare Abbildung $\function{f}{\oplus_i S_i}{\colimes \mathcal{B}}{s}{\prod_i \psi_i(s_i)}$.\\
\ \\
Somit liefern uns die universellen Eigenschaften folgende zwei R-Algebra-Homomorphismen:
\begin{gather*}
\functionfront{\varphi}{\colimes \mathcal{B}}{\bigotimes_i S_i} \\
\functionfront{\phi}{\bigotimes_i S_i}{\colimes \mathcal{B}}.
\end{gather*}
\begin{center}
\begin{tikzcd}
  & \mathcal{B} \arrow[rd, "\psi"] \arrow[ld, "\psi'"'] &                                            &  &   & \oplus_i S_i \arrow[rd, "g"] \arrow[ld, "f"'] &                                         \\
\otimes_i S_i &                                           & \colimes \mathcal{B} \arrow[ll, "\exists ! \varphi"', dashed] &  & \colimes \mathcal{B} &                                    & \otimes_i S_i \arrow[ll, "\exists ! \phi"', dashed]
\end{tikzcd}
\end{center}
Die Eindeutigkeit der universellen Eigenschaften liefert uns, das $\varphi$ und $\phi$ zueinander Inverse sind und somit haben wir unsere gesuchten Isomorphismen zwischen $\colimes \mathcal{B}$ und $\bigotimes_i S_i$ gefunden.
\begin{center}
\begin{tikzcd}
         & \mathcal{B} \arrow[rd, "\psi"] \arrow[ld, "\psi"'] &                                                                              &  &                  & \bigoplus_i S_i \arrow[rd, "g"] \arrow[ld, "g"'] &                                                                                     \\
\colimes \mathcal{B} &                                                    & \colimes \mathcal{F} \arrow[ll, "\exists ! id_{\colimes \mathcal{B}} = \phi \circ \varphi "', dashed] &  & \bigotimes_i S_i &                                                  & \bigotimes_i S_i \arrow[ll, "\exists ! id_{\bigotimes_i S_i} = \varphi \circ \phi "', dashed]
\end{tikzcd}
\end{center}
\ \\
Zu \textit{\textbf{2.}}:\\
Zeige, dass $\functionfront{q}{S_2}{S_2/Q}$ die in \cref{NeuDifferenzenkokerndef} eingeführten Eigenschaften des Differenzkokern`s  besitzt.
\begin{gather*}
q \circ f = q \circ g \text{ gilt, da } \kernel{q} = Q = \lbrace f(x) - g(x)\mid x \in C_2 \rbrace.
\end{gather*}
Sei nun eine Funktion $q' \in Hom_{\mathcal{A}}(S_2,T')$ mit $q' \circ f = q' \circ$ gegeben.\\
Somit gilt $q' \circ (f - g) = 0$, wodurch $Q$ ein Untermodul von $Q' \defeq \kernel{q'}$ ist.\\ Mit dem Isomorphiesatz \comment{HOMOMORPHIESATZ [kommutative Algebra 2.10]} für R-Algebren erhalten wir:
\begin{gather*}
 \nolinebreak{S_2/Q' \simeq (S_2/Q)/(Q'/Q)}.
\end{gather*}
Somit ist $\function{q'}{S_2}{(S_2/Q)/(Q'/Q)}{y}{[y]'}$ eine isomorphe Darstellung von $\functionfront{q'}{S_2}{T'}$.
\begin{gather*}
\Rightarrow \exists ! \function{\varphi}{S_2/Q}{(S_2/Q)/(Q'/Q)}{[y]}{[y]'}\textit{ mit }(\varphi \circ q) = q'.
\end{gather*}
Also ist $\functionfront{q}{S_2}{S_2/Q}$ der bis auf Isomorphie eindeutig bestimmte Differenzkokern von $f$ und $g$.
\end{proof}


\ \\
\textcolor{blue}{\textbf{Darstellung der Polynomalgebra als Tensorprodukt} \textit{[Eigene Überlegung]}}
\begin{bem}\label{Darstellung der Polynomalgebra als Tensorprodukt}
Die Polynomalgebra $R[x_1,...,x_d]$ über R lässt sich wie folgt als Tensorprodukt darstellen:
\begin{gather*}
R[x_1,...,x_n] = \bigotimes_{i \in \lbrace 1,...,n \rbrace} R[x_i]
\end{gather*}
Genauer gilt für zwei Polynomalgebren $A = R[x_1,...,x_{n_A}], \, B = R[y_1,...,y_{n_B}]$ über R:
\begin{gather*}
\Tensor{A}{R}{B}  = R[x_1,...,x_{n_A},y_1,...,y_{n_B}]
\end{gather*}
\end{bem}
Skizziere den Beweis.
\begin{proof}
Zeige, dass für $\function{g}{A \oplus B}{R[x_1,...,x_{n_A},y_1,...,y_{n_B}]}{(a,b)}{a \cdot b}$ die Universelle Eigenschaft des Tensorproduktes gilt:
\begin{center}
\begin{tikzcd}
A \oplus B \arrow[r, "g"] \arrow[rd, "f"] & {R[x_1,...,x_{n_A},y_1,...,y_{n_B}]} \arrow[d, "\exists ! \varphi", dashed] \\
                                          & M                                                                  
\end{tikzcd}
\end{center}
Es ist leicht nachzurechnen, dass es sich bei $\varphi$ um folgende Funktion handeln muss:
\begin{gather*}
\function{\varphi}{R[x_1,...,x_{n_A},y_1,...,y_{n_B}]}{M}{(x_i \cdot y_j)}{f(x_i,1) \cdot f(1,y_i)}
\end{gather*}
\end{proof}


\ \\
\textcolor{blue}{\textbf{R-Modul-Kolimiten} \textit{[Proposition A6.2 \ModulsOfDifferenzials]}}
\begin{prop}\label{R-Modul-Kolimiten}
In Der Kategorie der R-Module  existieren Koprodukte und Differenzkokerne, wobei:
\begin{itemize}
\item[\textbf{1.}] das Koprodukt $\colimes \mathcal{B}$ von R-Modulen $M_i \in (R-Module)$entspricht der direkten Summe $\sum_i M_i$.
\item[\textbf{2.}] der Differenzenkokern zweier Homomorphismen $\functionfront{f,g}{M_1}{M_2}$ entspricht dem Kokern $M_2/\immage{f-g}$ der Differenzenabbildung.
\end{itemize}
\end{prop}
\begin{proof}
\begin{itemize}

für \textit{\textbf{1.}} Sei $\functionfront{\phi}{\lbrace M_i \rbrace}{\mathcal{B}}$ ein beliebiger Morphismus. Zeige: \\
\begin{center}
\begin{tikzcd}
  & \mathcal{B} \arrow[rd, "\psi_i"] \arrow[ld, "\psi_i"'] &                                            \\
M' &                                              & \bigoplus_i M_i \arrow[ll, "\exists ! \varphi"', dashed]
\end{tikzcd}\\
\end{center}
Für ein beliebiges $i$ existiert genau ein $\function{\varphi_i}{M_i \oplus 0}{M'}{(0,...,0,m_i,0,...,0}{\psi_i '(m_i)}$
 mit $\psi_i ' = \psi_i \circ \varphi_i$\\
$\Rightarrow  \exists ! \function{\varphi}{\bigoplus_i M_i}{M'}{(m_1,...,m_n)}{\sum_i \psi_i(m_i)}$\\
\ \\
\textit{\textbf{2.}} ist Analog zu \cref{R-Algebra-Kolimiten}
\end{itemize}
\end{proof}
Die in \cref{R-Modul-Kolimiten} gezeigten Darstellungen gelten mit kurzen Überlegungen auch für S-Module, wobei S eine R-Algebra ist.\\

\section{Darstellung von Lokalisierung als Kolimes}
\ \\
\textcolor{blue}{\textbf{Lokalisierung von Algebren als Kolimes} \textit{[vlg. Aufgabe A6.7 \ModulsOfDifferenzials]}}
\begin{lemma}\label{Lokalisierung von Algebren als Kolimes}
Sei $S$ eine $R-Algebra$ und $U \subseteq S$ multiplikativ abgeschlossen.
Dann gilt:
\begin{gather*}
 S[U^{-1}] = \colimes \mathcal{B}
\end{gather*}
Wobei $\mathcal{B}$ aus den Objekten $\lbrace \lok{S}{t} \vert t \in U \rbrace$ und den Morphismen\\
$\lok{S}{t} \longrightarrow \lok{S}{tt'}, \loke{s}{t^n}{t} \longmapsto \loke{st'^n}{(tt')^n}{(tt')} \,
\forall t,t' \in U$ besteht.\\
\end{lemma}
\begin{proof}
Sei $\functionfront{\psi}{\mathcal{B}}{A}$ der Kolimes von $\mathcal{B}$. Zeige $\lok{S}{U} \simeq A$, definiere dazu:
\begin{gather*}
\functionfront{\psi'}{\mathcal{B}}{\lok{S}{U}}\\
\function{\psi'_{\lok{S}{t}}}{\lok{S}{t}}{\lok{S}{t}}{\loke{s}{t^n}{t}}{\loke{s}{t^n}{U}}
\end{gather*}
$\psi'$ ist ein Morphismus, da für beliebige $t,t' \in U$ und $s \in S$ gilt:
$$\loke{s}{t^n}{U} = \loke{st'^n}{(tt')^n}{U}$$
Durch die Universelle Eigenschaft des Kolimes erhalten wir den eindeutigen Homomorphismus $\functionfront{\varphi}{A}{\lok{S}{U}}$.
\begin{center}
\begin{tikzcd}
            & \mathcal{B} \arrow[rd, "\psi"] \arrow[ld, "\psi'"'] &                                            \\
{S[U^{-1}]} &                                                     & A \arrow[ll, "\exists ! \varphi"', dashed]
\end{tikzcd}
\end{center}
Für $\functionfront{\phi}{S[U^{-1}]}{A}$ benötigen wir kleinere Vorüberlegungen.\\
Zunächst können wir jedes Element $\loke{s}{u}{U} \in \lok{S}{U}$ als $\psi_{\lok{S}{t}}(\loke{s}{t}{t})$ schreiben.\\
\comment{\label{wobei u = t}}
Weiter gilt für alle $s_1,s_2 \in S , \, t_1,t_2 \in U$: 
\begin{align*}
\textit{Sei }\psi'_{\lok{S}{t}}(\loke{s_1}{t_1}{t}) = \psi'_{\lok{S}{t}}(\loke{s_2}{t_2}{t})\\
\Rightarrow  \exists u \in U: (s_1t_1 - s_2t_2) \cdot u = 0\\
\Rightarrow  \loke{s_1u}{t_1u}{tu} = \loke{s_2u}{t_2u}{tu}\\
\Rightarrow  \psi_{\lok{S}{t}}(\loke{s_1}{t_1}{t}) = \psi_{\lok{S}{t}}(\loke{s_2}{t_2}{t})
\end{align*}
Mit diesem Wissen können wir den R-Algebra-Homomorphismus $\functionfront{\phi}{\lok{S}{U}}{A}$ definieren:
\begin{gather*}
\function{\phi}{\lok{S}{U}}{A}{\psi'_{\lok{S}{t}}(\loke{s}{t}{t})}{\psi_{\lok{S}{t}}(\loke{s}{t}{t})}
\end{gather*}
$\phi \circ \varphi = id_A$ ergibt sich direkt aus der Universellen Eigenschaft des Kolimes:
\begin{center}
\begin{tikzcd}
  & \mathcal{B} \arrow[rd, "\psi"] \arrow[ld, "\psi"'] &                                                              \\
A &                                                    & A \arrow[ll, "\exists ! id_A = \phi \circ \varphi"', dashed]
\end{tikzcd}
\end{center}
Für $\varphi \circ \phi \defshow id_{\lok{S}{U}}$ wähle beliebige $s \in S , t \in U$, für diese gilt:
\begin{gather*}
(\varphi \circ \phi)(\psi'(\loke{s}{t}{t})) =
 \varphi (\psi(\loke{s}{t}{t}) =
  \psi'(\loke{s}{t}{t})
\end{gather*}
Damit haben wir gezeigt, dass $\varphi,\phi$ Isomorphismen sind und somit $A \simeq \lok{S}{U}$ gilt.\\
Da der Kolimes bis auf Isomorphie eindeutig ist, definiere ab sofort $\lok{S}{U}$ als den eindeutigen Kolimes von $\mathcal{B}$.
 \end{proof}


\ \\
\textcolor{blue}{\textbf{Lokalisierung von Moduln als Kolimes} \textit{[Beweis von Proposition 16.9 \ModulsOfDifferenzials]}}
\begin{korrolar}\comment{\label{Lokalisierung von Moduln als Kolimes}}
Sei M ein S-Modul, wobei S eine R-Algebra ist. Sei weiter $U \subseteq S$ multiplikativ abgeschlossen. Dann gilt:
\begin{gather*}
\lok{M}{U} = \colimes \mathcal{C}
\end{gather*}
Wobei $\mathcal{C}$ aus den Objekten $\lbrace \Tensor{\lok{S}{U}}{\lok{S}{t}}{\lok{M}{t}} \vert t \in U \rbrace$ und folgenden Morphismen besteht:
\begin{gather*}
\tensor{\lok{S}{U}}{\lok{S}{t}}{\lok{M}{t}} \longrightarrow
\tensor{\lok{S}{U}}{\lok{S}{(tt')}}{\lok{M}{(tt')}} ,\\
\tensor{\loke{s}{u}{U}}{\lok{S}{t}}{\loke{m}{t^n}{t}} \longmapsto
\tensor{\loke{s}{u}{U}}{\lok{S}{t}}{\loke{t'^nm}{(tt')^n}{t}} 
\end{gather*}
\end{korrolar}
\textit{Auch wenn sich \cref{Lokalisierung von Algebren als Kolimes} hier nicht direkt anwenden lässt, so können wir doch im Beweis gleich vorgehen.}
\begin{proof}
Schließe zunächst den trivialen Fall $0 \in U$ aus.\\
Sei $\functionfront{\psi}{\mathcal{C}}{A}$ der Colimes von $\mathcal{C}$. Zeige $\lok{S}{U} \simeq A$, definiere dazu folgenden Morphismus \comment{\label{das phi ein Mophismus ist überlasse ich dem Leser}}:
\begin{gather*}
\functionfront{\psi}{\mathcal{C}}{\lok{M}{U}} \\
\comment{
\function{\psi_{\Tensor{\lok{S}{U}}{\lok{S}{t}}{\lok{M}{t}}}}{\Tensor{\lok{S}{U}}{\lok{S}{t}}{\lok{M}{t}}}{\lok{M}{U}}{\tensor{\loke{s}{u}{U}}{\lok{S}{t}}{\loke{m}{t^n}{t}}}{\loke{sm}{ut^n}{U}} \\
}
\function{\psi_{t}}{\Tensor{\lok{S}{U}}{\lok{S}{t}}{\lok{M}{t}}}{\lok{M}{U}}{\tensor{\loke{s}{u}{U}}{\lok{S}{t}}{\loke{m}{t^n}{t}}}{\loke{sm}{ut^n}{U}}
\end{gather*}

Die Wohldefiniertheit von $\psi'_t$ für ein beliebiges $t \in U$ folgt direkt aus der Universellen Eigenschaft des Tensorprodukt`s. Denn für die bilineare Abbildung
 $\function{f}{\lok{S}{U} \oplus \lok{M}{t}}{\lok{M}{t}}{(\loke{s}{u}{U}, \loke{m}{t^n}{t})}{\loke{sm}{ut^n}{U}}$  gilt:
\begin{center}
\begin{tikzcd}
\lok{S}{U} \oplus \lok{M}{t} \arrow[r, "g"] \arrow[rd, "f"'] & \Tensor{\lok{S}{U}}{\lok{S}{t}}{\lok{M}{t}} \arrow[d, "\exists ! \psi'_t", dashed] \\
                                      & \lok{M}{U}                               
\end{tikzcd}
\end{center} 
Durch die Universelle Eigenschaft des Kolimes erhalten wir nun den eindeutigen Homomorphismus $\functionfront{\varphi}{A}{\lok{M}{U}}$.
\begin{center}
\begin{tikzcd}
  & \mathcal{C} \arrow[rd, "\psi"] \arrow[ld, "\psi'"'] &                                            \\
\lok{M}{U} &                                                     & A \arrow[ll, "\exists ! \varphi"', dashed]
\end{tikzcd}
\end{center}
Für $\functionfront{\phi}{\lok{M}{U}}{A}$ benötigen wir kleinere Vorüberlegungen.\\
Zunächst können wir jedes Element $\loke{m}{u}{U} \in \lok{M}{U}$ als $\nolinebreak{\psi(\tensor{\loke{1}{u}{U}}{\lok{M}{t}}{\loke{m}{1}{t}}})$ schreiben.
Wobei mit $\psi$ gemeint ist, dass wir ein beliebiges $t \in U$ wählen und dann $\psi_t$ betrachten. Diese Verallgemeinerung ist möglich, da für beliebige $t_1,t_2,u \in U$ und $m \in M$ gilt:
\begin{gather*}
\psi_{t_1}({\tensor{\loke{1}{u}{U}}{\lok{M}{t_1}}{\loke{m}{1}{t_1}}}) =
\loke{m}{u}{U} = 
\psi_{t_2}({\tensor{\loke{1}{u}{U}}{\lok{M}{t_2}}{\loke{m}{1}{t_2}}})
\end{gather*}
Definiere nun mit diesem Wissen folgenden Homomorphismus:
\begin{gather*}
\function{\phi}{\lok{M}{U}}{A}{\psi(\tensor{\loke{1}{u}{U}}{\lok{\loke{m}{1}{}}{M}}{t})}{\psi'(\tensor{\loke{1}{u}{U}}{\lok{\loke{m}{1}{t}}{M}}{t})}
\end{gather*}
$\phi \circ \varphi = id_A$ ergibt sich direkt aus der Universellen Eigenschaft des Kolimes.\\
Für $\varphi \circ \phi \defshow id_{\lok{M}{U}}$ wähle $\loke{m}{u}{U} \in \lok{M}{U}$ beliebig, für dieses gilt:
\begin{gather*}
(\varphi \circ \phi) (\psi'(\tensor{\loke{1}{u}{U}}{\lok{M}{t}}{\loke{m}{1}{t}})) \\
 =\varphi(\psi(\tensor{\loke{1}{u}{U}}{\lok{M}{t}}{\loke{m}{1}{t}})) \\
  =\psi'(\tensor{\loke{1}{u}{U}}{\lok{M}{t}}{\loke{m}{1}{t}})
\end{gather*}
Damit haben wir $A \simeq \lok{M}{U}$ gezeigt, definiere also ab sofort $\lok{M}{U}$ als den eindeutigen Kolimes von $\mathcal{C}$.
\end{proof}


\end{document}