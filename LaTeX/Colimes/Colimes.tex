\documentclass[10pt,a4paper]{report}
\usepackage[utf8]{inputenc}
\usepackage{amsmath}
\usepackage{amsthm}
\usepackage{amsfonts}
\usepackage{amssymb}
\usepackage{color}
\usepackage{tikz-cd}
\usepackage{calc}
\usepackage{setspace}
\usepackage[german]{babel}
\usetikzlibrary{babel}
\usepackage{cleveref}

\newcommand{\comment}[1]{}
\renewcommand{\baselinestretch}{1.1}

\newcommand{\ModulsOfDifferenzials}{David Eisenbud 1994}
\newcommand{\Algebra}{Christian Karpfinger, Kurt Meyberg 2009}

\newcounter{Aussage}[chapter]

\newtheorem{satz}[Aussage]{Satz}
\newtheorem{theorem}[Aussage]{Theorem}
\newtheorem{prop}[Aussage]{Proposition}
\newtheorem{korrolar}[Aussage]{Korrolar}
\newtheorem{lemma}[Aussage]{Lemma}
\newtheorem{bem}[Aussage]{Bemerkung}
\newtheorem{definition}[Aussage]{Definition}
\newtheorem{bsp}[Aussage]{Beispiel}

\newcommand{\functionfront}[3]{\nolinebreak{#1:#2 \longrightarrow #3}}
\newcommand{\functionback}[3]{\nolinebreak{#1:#2 \longmapsto #3}}
\newcommand{\function}[5]{\nolinebreak{#1:#2 \longrightarrow #3 \, , \, #4 \longmapsto #5}}
\newcommand{\infunctionfront}[3]{\nolinebreak{#1:#2 \hookrightarrow #3}}
\newcommand{\divR}[2]{\Omega_{#1/#2}}
\newcommand{\divf}[1]{d_{#1}}
\comment{\newcommand{\divf}[2][]{d_{#1}}}
\newcommand{\Tensor}[3]{#1 \otimes_{#2} #3}
\newcommand{\tensor}[3]{#1 \otimes #3}
\newcommand{\lok}[2]{#1 [#2^{-1}]}
\newcommand{\loke}[3]{(\frac{#1}{#2})_{_{#3}}}
\comment{\newcommand{\loke}[3]{(#1,#2)_{mod\sim_{#3}}}}

\newcommand{\colimes}[0]{\lim\limits_{ \longrightarrow }}
\newcommand*{\defeq}{\mathrel{\vcenter{\baselineskip0.5ex \lineskiplimit0pt
                     \hbox{\scriptsize.}\hbox{\scriptsize.}}}%
                     =}
\newcommand*{\defeqr}{= \mathrel{\vcenter{\baselineskip0.5ex \lineskiplimit0pt
                     \hbox{\scriptsize.}\hbox{\scriptsize.}}}}

\newcommand*{\defshow}{\stackrel{!}{=}}
\newcommand{\kernel}[1]{kern(#1)}
\newcommand{\immage}[1]{im(#1)}
\newcommand{\Verz}[1]{\langle #1 \rangle}


\begin{document}
\textcolor{blue}{Definition Leibnizregel}
\begin{definition}\label{Definition Leibnizregel} \textit{[Kapitel 16 \ModulsOfDifferenzials]}\\
Sei S ein Ring und M ein S-Modul
\begin{itemize}
\item[]Ein Homomoprphismus abelscher Gruppen $\functionfront{d}{S}{M}$ ist eine \underline{Ableitung}, falls gilt:
\begin{gather*}
\forall s_1,s_2 \in S :\, d(s_1 \cdot s_2) = s_1d(s_2) + s_2d(s_1) \textbf{ (Leibnitzregel)}
\end{gather*}
\item[]Sei S eine R-Algebra, dann nennen wir eine \underline{Ableitung} $\functionfront{d}{S}{M}$ \underline{$R$-linear}, falls sie zusätzlich ein R-Modulhomomorphismus ist, also falls gilt:
\begin{gather*}
\forall r_1,r_2 \in R \, \forall s_1,s_2 \in S : \, d(r_1 s_1 + r_2 s_2) = r_1 d(s_1) + r_2 d(s_2)
\end{gather*}
\end{itemize}
\end{definition}


\ \\
\textcolor{blue}{\textbf{Differenzial indempotenter Elemente}}
\begin{lemma}\label{Differenzial indempotenter Elemente} \textit{[Aufgabe 16.1 \ModulsOfDifferenzials]} \\
Sei S ein Ring, $M$ ein $S$-Modul und $\functionfront{d}{S}{M}$ eine Ableitung. Sei weiter $a \in S$ ein indempotentes Element ($a^2 = a$).
\begin{center}
Dann gilt $d(a) = 0$. 
\end{center}
Insbesondere gilt somit auch $d(1) = 0$.
\end{lemma}
\begin{proof}
Nutze hierfür allein die Leibnizregel \textit{(cref{Definition Leibnizregel})}
\begin{gather*}
\text{Schritt 1: } \divf{S}(a) = \divf{S}(a^2) = a\divf{S}(a) + a\divf{S}(a) \\
\text{Schritt 2: } a\divf{S}(a) = a\divf{S}(a^2) = a^2\divf{S}(a) + a^2\divf{S}(a) = a\divf{S}(a) + a\divf{S}(a)\\
\Rightarrow \divf{S}(a) = a\divf{S}(a) = 0
\end{gather*}
\end{proof}


\ \\
\begin{definition}
Sei $S$ eine $R$-Algebra.\\
Das $S$-Modul $\divR{S}{R}$ der Kähler-Differenziale von $S$ über $R$ und die dazugehörige universelle $R$-lineare Ableitung $\functionfront{\divf{S}}{S}{\divR{S}{R}}$ mit $\immage{\divf{S}} = \divR{S}{R}$ sind durch die folgende universelle Eigenschaft definiert:

\begin{center}
Für alle R-linearen Ableitungen $\functionfront{e}{S}{M}$ von S in ein $S$-Modul $M$ existiert genau ein $S$-Modulhomomorphismus $\functionfront{e'}{\divR{S}{R}}{M}$, sodass folgendes Diagramm kommutiert:\\
\begin{tikzcd}
S \arrow[r, "\divf{S}"] \arrow[rd, "e"'] & \divR{S}{R} \arrow[d, "\exists ! e'", dashed] \\
                                  & M                                  
\end{tikzcd}
\end{center}
\end{definition}


\ \\
\textcolor{blue}{\textbf{Eindeutigkeit des Kaehler-Differentials}}
\begin{lemma}\label{Eindeutigkeit des Kaehler-Differentials}(Das Kähler-Differentials ist eindeutig)\\
Sei S eine $R$-Algebra.\\
Dann ist das $S$-Modul $\divR{S}{R}$ der Kähler-Differenziale von $S$ über $R$ und die dazugehörige universelle $R$-lineare Ableitung $\divf{S}$ bis auf eine eindeutige Isomorphie eindeutig bestimmt.
\end{lemma}
\begin{proof}
Seien $\functionfront{\divf{S}}{S}{\divR{S}{R}} und \functionfront{\divf{S}'}{S}{\divR{S}{R}'}$ beide eine universelle $R$-lineare Ableitung.\\
Durch die universelle Eigenschaft der universellen Ableitung erhalten wir eindeutig bestimmte Funktionen
$\functionfront{\varphi}{\divR{S}{R}}{\divR{S}{R}'}$ und $\functionfront{\varphi'}{\divR{S}{R}'}{\divR{S}{R}}$, für welche die folgenden Diagramme kommutieren:
\begin{center}
\begin{tikzcd}
S \arrow[r, "\divf{S}"] \arrow[rd, "\divf{S}'"'] & \divR{S}{R} \arrow[d, "\exists !\varphi", dashed] &  & S \arrow[r, "\divf{S}'"] \arrow[rd, "\divf{S}"'] & \divR{S}{R}' \arrow[d, "\exists ! \varphi'", dashed] \\
                                   & \divR{S}{R}'                                      &  &                                    & \divR{S}{R}                                     
\end{tikzcd}
\end{center}
Wende nun die Universelle Eingenschaft von $\divf{S}$ auf $\divf{S}$ selbst an und erhalte $id_{\divR{S}{R}} = \varphi' \circ \varphi$. 
\begin{center}
\begin{tikzcd}
S \arrow[r, "\divf{S}"] \arrow[rd, "\divf{S}"'] & \divR{S}{R} \arrow[d, "\exists ! id_{\divR{S}{R}} = \varphi' \circ \varphi", dashed] \\
                                   & \divR{S}{R}                                                              
\end{tikzcd}
\end{center}
Analog erhalte auch $id_{\divR{S}{R}'} = \varphi \circ \varphi'$. Damit existiert genau ein Isomorphismus $\functionfront{\varphi' \circ \varphi}{\divR{R}{S}}{\divR{R}{S}'}$ mit $\divf{S}' = \divf{S} \circ (\varphi' \circ \varphi)$.
\end{proof}


\ \\
\textcolor{blue}{\textbf{Propositon 11 delta}}
\begin{lemma}\label{Propositon 11 delta} \textit{[Lemma 16.11 \ModulsOfDifferenzials]} \\
Seien $S$, $S'$ zwei $R$-Algebren. Sei weiter $\functionfront{f}{S}{S'}$ ein $R$-Algebrenhomomorphismus und $\functionfront{\delta}{S}{S'}$ ein Homomorphismus abelscher Gruppen mit $\delta(S)^2 = 0$. Dann gilt:
\begin{center}
$f$ + $\delta$ ist ein $R$-Algebrenhomomorphismus\\
$\Leftrightarrow$\\
$\delta$ ist eine $R$-linear und $\forall s_1,s_2 \in S :\, \delta(s_1 \cdot s_2) = f(s_1)\delta(s_2) + f(s_2)\delta(s_1)$.
\end{center}
\end{lemma}
\begin{proof} \ \\
\begin{itemize}
\item[\underline{\glqq $\Rightarrow$ \grqq :}] Da $f$ und $f$ + $\delta$ $R$-linear sind, ist auch $\delta = (f + \delta) - f$ $R$-linear.\\
Seien nun $s_1,s_2 \in S$ beliebig, somit gilt:
\begin{gather*}
(f + \delta)(s_1 \cdot s_2) = (f + \delta)(s_1) \cdot (f + \delta)(s_2)\\
\Rightarrow f(s_1 \cdot s_2) + \delta(s_1 \cdot s_2) = f(s_1)f(s_2) + f(s_1)\delta(s_2) + f(s_2)\delta(s_1) + \delta(s_1)\delta(s_2)\\
\Rightarrow \delta(s_1 \cdot s_2) = f(s_1)\delta(s_2) + f(s_2)\delta(s_1) + \delta(s_1)\delta(s_2) \textit{ mit } \delta(s_1)\delta(s_2) \in \delta(S)^2 = 0 \\
\Rightarrow \delta(s_1 \cdot s_2) = f(s_1)\delta(s_2) + f(s_2)\delta(s_1) + \delta(s_1)\delta(s_2)
\end{gather*}
\item[\underline{\glqq $\Leftarrow$ \grqq :}]
Da $f$ und $\delta$ beide $R$-lineare Homomorphismen abelscher Gruppen sind, trifft die auch für $f + \delta$ zu.\\
Wähle nun also $s_1,s_2 \in S$ beliebig, somit gilt:
\begin{gather*}
(f + \delta)(s_1) \cdot (f + \delta)(s_2) \\
= f(s_1)f(s_2) + f(s_1)\delta(s_2) + f(s_2)\delta(s_1) + \delta(s_1)\delta(s_2)\\
= f(s_1 \cdot s_2) + \delta(s_1 \cdot s_2) \\
= (f + \delta)(s_1 \cdot s_2)
\end{gather*}
Damit haben wir gezeigt, dass $f + \delta$ ein $R$-Algebrenhomomorphismus ist.
\end{itemize}
\end{proof}


\ \\
\textcolor{blue}{\textbf{Kontruktion Kaehler-Differential}}
\begin{theorem} (Konstruktion des Kähler-Differentials\comment{\label{Kontruktion Kaehler-Differential}}\textit{[Theorem 16.21 \ModulsOfDifferenzials]} \\
Sei $S$ ein $R$-Algebra.\\
Betrachte die Multiplikationsabbildung $\function{\mu}{\Tensor{S}{R}{S}}{S}{\tensor{s_1}{R}{s_2}}{s_1 \cdot s_2}$ mit $I \defeq \kernel{\mu}$.\\
Definiere durch $S \oplus (\Tensor{S}{R}{S}) \longrightarrow \Tensor{S}{R}{S} ,(s,\tensor{s_1}{R}{s_2}) \longmapsto \tensor{ss_1}{R}{s_2}$ eine $S$-Modulstruktur auf $\Tensor{S}{R}{S}$.
\begin{center}
Dann ist durch $\function{e}{S}{I/I^2}{s}{[\tensor{s}{R}{1} - \tensor{1}{R}{s}]}$ die universelle $R$-lineare Ableitung auf $S$ definiert.
\end{center}
\end{theorem}
\begin{proof}
Zeige zunächst, dass $e$ eine Abbildung ist. Betrachte dazu:
\begin{gather*}
\function{f_1}{S}{\Tensor{S}{R}{S}}{s}{\tensor{s}{R}{1}} , \function{f_2}{S}{\Tensor{S}{R}{S}}{s}{\tensor{1}{R}{s}} \\
\textit{Damit ist die Wirkung von $S$ auf $\Tensor{S}{R}{S}$ durch }\\ S \oplus (\Tensor{S}{R}{S}) \longrightarrow \Tensor{S}{R}{S} ,(s,\tensor{s_1}{R}{s_2}) \longmapsto f_1(s)(\tensor{s_1}{R}{s_2}) \textit{ gegeben.}
\end{gather*}
Setze nun in der Notation von \cref{Propositon 11 delta} $f = f_1$ und $\delta = e$.\\
Damit ist $f + \delta = f_1 + \delta = f_2$ ein $R$-Algebrenhomomorphismus und es folgt aus \cref{Propositon 11 delta} und unserer Definition der Wirkung von $S$ auf $\Tensor{S}{R}{S}$, dass $e$ eine $R$-lineare Ableitung ist.\\
Durch die Universelle Eigenschaft von $\divf{S}$ existiert also genau ein $R$-Algebrenhomomorphismus $\functionfront{e'}{\divR{S}}{I/I^2}$ mit $e = \divf{S} \circ e'$.\\
Betrachte nun folgende Umkehrabbildung $\phi$ zu $e'$:
\begin{gather*}
\function{\phi}{I/I^2}{\divR{S}{R}}{[\tensor{s_1}{R}{s_2}]}{s_1\divf{S}(s_2)}
\end{gather*}
Um zu prüfen, dass $\phi$ die Umkehrabbildung von $e$ ist, wähle $s,s_1,s_2 \in S$ beliebig, somit gilt:
\begin{gather*}
(\phi \circ e')(\divf{S}(s)) = (\phi \circ e)(s) = \phi([\tensor{s}{R}{1} - \tensor{1}{R}{s}]) = s\divf{S}(1) + 1\divf{S}(b) = \divf{S}(b) \\
(e' \circ \phi)([\tensor{s_1}{R}{s_2}]) = e'(s_1\divf{s_2}) = s_1 e(s_2) = [s_1\tensor{1}{R}{s_2} - s_1\tensor{s_2}{R}{1}] =
[\tensor{s_1}{R}{s_2} - \tensor{s_1s_2}{R}{1}] = [\tensor{s_1}{R}{s_2}]
\end{gather*}
\end{proof}


\ \\
\textcolor{blue}{\textbf{Differenzial des Produktes von Algebren} \textit{[Proposition 16.10 \ModulsOfDifferenzials]}}
\begin{prop}\label{Differenzial des Produktes von Algebren}
Seien $S_1, \dots , S_n$ R-Algebren. Sei dazu $S \defeq \prod_{i \in \lbrace 1, \dots , n \rbrace} S_i$ deren direktes Produkt.
Dann gilt:
\begin{gather*}
\divR{S}{R} = \prod_{i \in \lbrace 1, \dots , n \rbrace} \divR{S_i}{R}
\end{gather*}
\end{prop}
\begin{proof}
Sei für $i \in \lbrace 1, \dots ,n \rbrace$ jeweils $e_i \in S$ die Einbettung es Einselement's von $S_i$ in $S$, somit ist $\functionfront{p_i}{e_iS}{S_i}$ ein Isomorphismus.\\
Nutze weiter, dass $e_i$ jeweils ein indempotentes Element von $({e_i}^2 = e_i)$ von $S$ ist:
\begin{gather*}
\text{Nach \cref{Differenzial indempotenter Elemente} gilt } \divf{S}(e_i) = 0 \\
\Rightarrow \forall s \in s : \divf{S}(e_is)= \divf{S}({e_i}^2s) = e_i\divf{S}(e_is) + e_is\divf{S}(e_i) = e_i\divf{S}(e_is) \\
\end{gather*}
Mit diesem Wissen können wir einen Isomorphismus $\functionfront{\Phi}{\divR{S}{R}}{\prod_{i \in \lbrace 1, \dots , n \rbrace} \divR{S_i}{R}}$ definieren:
\begin{center}
\begin{tikzcd}
\divR{S}{R} \arrow[d] & \divf{S}(s) = \sum_{i \in \lbrace 1, \dots , n \rbrace} \divf{S}(e_is) \arrow[d, maps to] \\
\prod_{i \in \lbrace 1, \dots , n \rbrace} e_i\divf{S}(e_iS) \arrow[d] & \left( e_1\divf{S}(e_1s), \dots , e_n\divf{S}(e_ns) \right) \arrow[d,, maps to] \\
\prod_{i \in \lbrace 1, \dots , n \rbrace} \divR{S_i}{R}                & \left( (\divf{S_1} \circ p_1) (s), \dots , (\divf{S_n} \circ p_n) (s) \right)                         
\end{tikzcd}
\end{center}
Da der Differenzialraum $\divR{S}{R}$ bis auf eine eindeutige Isomophie eindeutig ist \textit{(\cref{Eindeutigkeit des Kaehler-Differentials})}, definiere diesen ab jetzt als $\prod_{i \in \lbrace 1, \dots , n \rbrace} \divR{S_i}{R}$.
\end{proof}

\ \\
\textcolor{blue}{\textbf{Cotangent Sequenz}}
\begin{prop}\label{Cotangent Sequenz} \textbf{(Relativ Cotangent Sequenz)} \textit{[vgl. Proposition 16.2 \ModulsOfDifferenzials]}\\
Seien $\functionfront{\alpha}{R}{S}$ und $\functionfront{\beta}{S}{T}$ zwei Ringhomomorphismen. Dann existiert folgende exakte Sequenz:
\begin{center}
\begin{tikzcd}
\Tensor{T}{S}{\divR{S}{R}} \arrow[rrrr, "\tensor{t}{S}{\divf{S}(s)} \mapsto t(\divf{T_R} \circ \beta)(s)"] &  &  &  & \divR{T}{R} \arrow[rrr, "\divf{T_R}(t) \mapsto \divf{T_S}(t)"] &  &  & \divR{T}{S} \arrow[r]  & 0
\end{tikzcd}
\end{center}
Im Besonderen gilt für die Differenzialräume von $T$ über $R$ und $S$:
\begin{gather*}
\divR{T}{S} \simeq \divR{T}{R}/T \langle (\divf{T_R} \circ \beta)(S) \rangle
\end{gather*}
\end{prop}
\begin{proof}
Durch $st \defeq \beta (S) \cdot t$ und $rt \defeq (\beta \circ \alpha) (r) \cdot t$ können wir $T$ als $S$- bzw. $R$-Algebra betrachten.\\
Zeige zunächst, dass $\function{g}{\divR{T}{R}}{\divR{T}{S}}{\divf{T_R}(t)}{\divf{T_S}(t)}$ surjektiv ist:
\begin{itemize}
\item[]
$\divf{T_S}$ ist $R$ - Linear, da $R$ durch $(\beta \circ \alpha)$ auf $T$ wirkt, es lässt sich also die universelle Eigenschaft von $\divf{T_R}$ auf $\divf{T_S}$ anwenden:
\begin{center}
\begin{tikzcd}
T \arrow[r, "\divf{T_R}"] \arrow[rd, "\divf{T_S}"'] & \divR{T}{R} \arrow[d, "\exists ! g", dashed] \\
                                  & \divR{T}{S}                                  
\end{tikzcd}
\end{center}
Dies zeigt, dass $\function{g}{\divR{T}{R}}{\divR{T}{S}}{\divf{T_R}(t)}{\divf{T_S}(t)}$ surjektiv ist.
\end{itemize}
Zeige nun, dass $\divR{T}{S} \simeq \divR{T}{R}/T \langle (\divf{T_R} \circ \beta)(S) \rangle$ gilt:
\begin{itemize}
\item[] Definiere zunächst folgende $T$-lineare Ableitung:
\begin{gather*}
\function{e}{T}{\divR{T}{R} / T \langle (\divf{T_R} \circ \beta)(S) \rangle }{t}{[\divf{T_R}(t)]_{T \langle (\divf{T_R} \circ \beta)(S) \rangle}}
\end{gather*}
Wir sehen, dass $e$ auch $S$-linear ist:
\begin{gather*}
\text{Seien dazu } s \in S \text{ und } t \in T \text{ beliebig, somit gilt:}\\
e(st) = [\divf{T_R}(st)]_{T \langle (\divf{T_R} \circ \beta)(S) \rangle}\\
=[\beta(s) \divf{T_R}(t)]_{T \langle (\divf{T_R} \circ \beta)(S) \rangle} + [t \divf{T}(\beta(s))]_{T \langle (\divf{T} \circ \beta)(S) \rangle}\\
= [\beta(s) \divf{T}(t)]_{T \langle (\divf{T_R} \circ \beta)(S) \rangle} + 0 = se(t)
\end{gather*}
Dies bedeutet, dass wir die universelle Eigenschaft von $\divf{T_S}$ anwenden können:
\begin{center}
\begin{tikzcd}
T \arrow[r, "\divf{T_S}"] \arrow[rd, "e"'] & \divR{T}{S} \arrow[d, "\exists ! \varphi", dashed] \\
                                  & \divR{T}{R}/T \divR{S}{R}                                  
\end{tikzcd}
\end{center}
Dadurch erhalten wir $\functionfront{\varphi}{\divR{T}{S}}{\divR{T}{R}/T \divR{S}{R}}$.\\
Für die Umkehrfunktion $\phi$ nutze $\function{g}{\divR{T}{R}}{\divR{T}{S}}{\divf{T_R}(t)}{\divf{T_S}(t)}$ von Beginn des Beweises:
\begin{gather*}
\text{Für alle } s \in S \text{ gilt } \divf{T_S}(s) = 0.\\
\text{ Somit gilt } T \langle (\divf{T_R} \circ \beta)(S) \rangle \subseteq \kernel{g}.\\
\text{Also ist die Umkehrfunktion $\phi$ von $\varphi$ wohldefiniert: }\\
 \function{\phi}{\divR{T}{R}/T \langle (\divf{T_R} \circ \beta)(S) \rangle}{\divR{T}{S}}{[\divf{T_R}(t)]_{T \langle (\divf{T_R} \circ \beta)(S) \rangle}}{\divf{T_S}(t)}.
\end{gather*}
Damit gilt $\divR{T}{S} \simeq \divR{T}{R}/T \langle (\divf{T_R} \circ \beta)(S) \rangle$.\\
Auf unsere Sequenz bezogen bedeutet dies:
\begin{gather*}
\text{Es gilt } \immage{\divR{T}{R} \rightarrow \divR{T}{S}} \simeq \divR{T}{R} / \immage{\Tensor{T}{S}{\divR{S}{R}} \rightarrow \divR{T}{R}}.\\
\text{Somit gilt auch } \immage{ \Tensor{T}{S}{\divR{S}{R}} \rightarrow \divR{T}{R} } = \kernel{ \divR{T}{R} \rightarrow \divR{T}{S} }. 
\end{gather*}
Damit haben wir gezeigt, dass die \textbf{Relative Cotangent Sequenz} exakt ist.
\end{itemize}
\end{proof}

\ \\
\textcolor{blue}{\textbf{Konormale Sequenz} \textit{[vlg. Proposition 16.3 \ModulsOfDifferenzials]}}
\begin{satz} \label{Konormale Sequenz}
\raggedright
Sei $\functionfront{\pi}{S}{T}$ ein R-Algebrenephimorphismus mit Kern($\pi$):= I \\
Dann ist folgende Sequenz rechtsexakt: \\
\begin{center}
\begin{tikzcd}
I/I^2 \arrow[r, "f"] & \Tensor{T}{S}{\divR{S}{R}} \arrow[r, "g"] & \divR{T}{R} \arrow[r] & 0
\end{tikzcd}
\end{center}
\begin{spacing}{1.5}
mit: $\function{f}{I/I^2}{{\Tensor{T}{S}{\divR{S}{R}}}}{[a]_{I^2}}{\tensor{1}{S}{\divf{S}(a)}}$\\
\textsc{\leftskip2.3em} $\function{g}{\Tensor{T}{S}{\divR{S}{R}}}{\divR{T}{R}}{\tensor{b}{S}{\divf{S}(c)}}{b \cdot (\divf{S} \circ \pi)(c)}$
\end{spacing}
\end{satz}
\begin{proof} \ \\
\underline{$f$ ist wohldefiniert:} Seien $a,b\in I^2$. Zeige $f(a \cdot b)=0$ :
$$ f(a \cdot b) =
\tensor{1}{S}{( \divf{S} \circ \pi )(a \cdot b)} =
\tensor{1}{S}{\pi(a) \cdot (\divf{S} \circ \pi )(b) + \pi(b) \cdot ( \divf{S} \circ \pi )(a)} = 0$$
\underline{$D\pi$ ist surjektiv:}
\begin{center}
\begin{tikzcd}
\divR{S}{R} \arrow[r, " D\pi "]                & \divR{T}{R}                \\
S \arrow[r, " \pi "] \arrow[u, " \divf{S} "] & T \arrow[u, " \divf{T} "]
\end{tikzcd}
\end{center}
Da $\divR{S}{R}$ und $\divR{T}{S}$ jeweils von $\divf{S}$ und $\divf{T}$ erzeugt werden, vererbt sich die Surjektivität von $\pi$ auf $D\pi$. Somit ist auch $\Tensor{1}{S}{D\pi}$ surjektiv.\\
\underline{$\immage{f}=\kernel{g}$:}\\ Dies folgt direkt aus  der Isomorphie $(\Tensor{T}{S}{\divR{S}{R}})/Im(f) \simeq \divR{T}{R}$:
\begin{align*}
(\Tensor{T}{S}{\divR{S}{R}})/Im(f) \\
= (\Tensor{T}{S}{\divR{S}{R}})/(\Tensor{T}{S}{\divf{S}(I)}) \\
= \Tensor{T}{S}{(\divR{S}{R}/\divf{S}(I))} \\ 
= \Tensor{T}{S}{(\divf{S}(S)/ \divf{S}(I))} \\
\simeq \Tensor{T}{S}{\divf{S}(S/I)} \\
\simeq \Tensor{T}{S}{\divf{T}(T)} \\
\end{align*}
\end{proof}


\ \\
\textcolor{blue}{\textbf{Differenzial ist Ableitung} \textit{[Eigene Überlegung (Wichtig für Körpererweiterungen)]}}
\begin{bsp}\label{Differenzial ist Ableitung}
Sei $k$ ein Körper, somit entspricht $\function{\divf{k[x]}}{k[x]}{\divR{k[x]}{k}}{f}{f'\divf{k[x]}(x)}$ der analytischen Ableitung.\\
Teste dies an $f(x)=ax^2 + bx +c$:
\begin{gather*}
d(f(x)) = a \cdot d(x^2) + b \cdot d(x) = (2ax + b) d(x) = f'(x) d(x) 
\end{gather*}
\comment{lässt sich induktiv für Monome leicht zeigen}
\end{bsp}
\end{document}