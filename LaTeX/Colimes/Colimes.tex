\documentclass[10pt,a4paper]{report}
\usepackage[utf8]{inputenc}
\usepackage{amsmath}
\usepackage{amsthm}
\usepackage{amsfonts}
\usepackage{amssymb}
\usepackage{tikz-cd}
\usepackage{calc}
\usepackage{setspace}
\usepackage[german]{babel}
\usetikzlibrary{babel}
\usepackage{cleveref}

\newcommand{\comment}[1]{}
\renewcommand{\baselinestretch}{1.1}


\newcounter{Aussage}[chapter]

\newtheorem{satz}[Aussage]{Satz}
\newtheorem{theorem}[Aussage]{Theorem}
\newtheorem{prop}[Aussage]{Proposition}
\newtheorem{korrolar}[Aussage]{Korrolar}
\newtheorem{lemma}[Aussage]{Lemma}
\newtheorem{bem}[Aussage]{Bemerkung}
\newtheorem{definition}[Aussage]{Definition}

\newcommand{\functionfront}[3]{\nolinebreak{#1:#2 \longrightarrow #3}}
\newcommand{\functionback}[3]{\nolinebreak{#1:#2 \longmapsto #3}}
\newcommand{\function}[5]{\nolinebreak{#1:#2 \longrightarrow #3 \, , \, #4 \longmapsto #5}}
\newcommand{\divR}[2]{\Omega_{#1/#2}}
\newcommand{\Tensor}[3]{#1 \otimes_{#2} #3}
\newcommand{\tensor}[3]{#1 \otimes #3}
\newcommand{\lok}[2]{#1 [#2^{-1}]}
\newcommand{\loke}[3]{(#1,#2)_{mod\sim_{#3}}}

\newcommand{\colimes}[0]{\lim\limits_{ \longrightarrow }}
\newcommand{\infunctionfront}[3]{\nolinebreak{#1:#2 \hookrightarrow #3}}
\newcommand*{\defeq}{\mathrel{\vcenter{\baselineskip0.5ex \lineskiplimit0pt
                     \hbox{\scriptsize.}\hbox{\scriptsize.}}}%
                     =}
\newcommand*{\defshow}{\stackrel{!}{=}}
\newcommand{\kernel}[1]{kern(#1)}
\newcommand{\immage}[1]{im(#1)}

\begin{document}
\begin{korrolar}\comment{\label{Lokalisierung von Moduln als Kolimes}}
Sei M ein S-Modul, wobei eine R-Algebra ist. Sei weiter $U \subseteq S$ multiplikativ abgeschlossen. Dann gilt:
\begin{gather*}
\lok{M}{U} = \colimes \mathcal{C}
\end{gather*}
Wobei $\mathcal{C}$ aus den Objekten $\lbrace \Tensor{\lok{S}{U}}{\lok{S}{t}}{\lok{M}{t}} \vert t \in U \rbrace$ und folgenden Morphismen besteht:
\begin{gather*}
\tensor{\lok{S}{U}}{\lok{S}{t}}{\lok{M}{t}} \longrightarrow
\tensor{\lok{S}{U}}{\lok{S}{(tt')}}{\lok{M}{(tt')}} ,\\
\tensor{\loke{s}{u}{U}}{\lok{S}{t}}{\loke{m}{t^n}{t}} \longmapsto
\tensor{\loke{s}{u}{U}}{\lok{S}{t}}{\loke{t'^nm}{t^nt'^n}{t}} 
\end{gather*}
\end{korrolar}
\textit{Auch wenn sich \cref{Lokalisierung von Algebren als Kolimes} sich hier nicht direkt anwenden lässt, so können wir doch im Beweis gleich vorgehen.}
\begin{proof}
Schließe zunächst den trivialen Fall $0 \in U$ aus.\\
Sei $\functionfront{\psi}{\mathcal{C}}{A}$ der Colimes von $\mathcal{C}$. Zeige $\lok{S}{U} \simeq A$, definiere dazu folgenden Morphismus \comment{\label{das phi ein Mophismus ist überlasse ich dem Leser}}:
\begin{gather*}
\functionfront{\psi}{\mathcal{C}}{\lok{M}{U}} \\
\comment{
\function{\psi_{\Tensor{\lok{S}{U}}{\lok{S}{t}}{\lok{M}{t}}}}{\Tensor{\lok{S}{U}}{\lok{S}{t}}{\lok{M}{t}}}{\lok{M}{U}}{\tensor{\loke{s}{u}{U}}{\lok{S}{t}}{\loke{m}{t^n}{t}}}{\loke{sm}{ut^n}{U}} \\
}
\function{\psi_{t}}{\Tensor{\lok{S}{U}}{\lok{S}{t}}{\lok{M}{t}}}{\lok{M}{U}}{\tensor{\loke{s}{u}{U}}{\lok{S}{t}}{\loke{m}{t^n}{t}}}{\loke{sm}{ut^n}{U}}
\end{gather*}
Um die Wohldefiniertheit von $\phi$ zu zeigen seien $s,s' \in S ; \, t,u,u' \in U ; \, n,n' \in \mathbb{N}$ und $m,m' \in M$ beliebig. Somit gilt:
\begin{gather*}
\textit{Sei }\tensor{\loke{s}{u}{U}}{\lok{S}{t}}{\loke{m}{t^n}{t}} = \tensor{\loke{s'}{u'}{U}}{\lok{S}{t}}{\loke{m'}{t^{n'}}{t}} \\
\Rightarrow \tensor{\loke{s}{ut^n}{U}}{\lok{S}{t}}{\loke{m}{1}{t}} = \tensor{\loke{s'}{u't^{n'}}{U}}{\lok{S}{t}}{\loke{m'}{1}{t}} \\
\textit{da M ein S-Modul ist und für beliebige $v,v' \in S$ gilt } (\loke{v}{1}{t} = \loke{v'}{1}{t} \Rightarrow v = v') { ,folgt:}\\
\comment{\label{Es ist nich ganz klar, dass das so schon folgt}}
\exists k,k' \in S \textit{ mit } km = k'm' \textit{ und } \loke{s}{ut^nk}{U} = \loke{s'}{u't^{n'}k'}{U} \\
\Rightarrow \loke{sm}{ut^n}{U} = \loke{skm}{ut^nk}{U} = \loke{s'k'm'}{u't^{n'}k'}{U} =\loke{s'm'}{u't^{n'}}{U} 
\end{gather*}
Durch die Universelle Eigenschaft des Kolimes erhalten wir den eindeutigen Homomorphismus $\functionfront{\varphi}{A}{\lok{M}{U}}$.
\begin{center}
\begin{tikzcd}
  & \mathcal{C} \arrow[rd, "\psi"] \arrow[ld, "\psi'"'] &                                            \\
\lok{M}{U} &                                                     & A \arrow[ll, "\exists ! \varphi"', dashed]
\end{tikzcd}
\end{center}
Für $\functionfront{\phi}{\lok{M}{U}}{A}$ benötigen wir kleinere Vorüberlegungen.\\
Zunächst können wir jedes Element $\loke{m}{u} \in \lok{M}{U}$ als $\nolinebreak{\psi(\tensor{\loke{1}{u}{U}}{\lok{M}{t}}{\loke{m}{1}{}}})$schreiben.
Wobei mit $\psi$ gemeint ist, dass wir ein beliebiges $t \in U$ wählen und dann $\psi_t$ betrachten. Diese Verallgemeinerung ist möglich, da für beliebige $t,t',u \in U$ und $m \in M$ gilt:
\begin{gather*}
\psi_t({\tensor{\loke{1}{u}{U}}{\lok{M}{t}}{\loke{m}{1}{t}}}) =
\loke{m}{u}{U} = 
\psi_{t'}({\tensor{\loke{1}{u}{U}}{\lok{M}{t'}}{\loke{m}{1}{t'}}})
\end{gather*}
Definiere nun mit diesem Wissen folgenden Homomorphismus:
\begin{gather*}
\function{\phi}{\lok{M}{U}}{A}{\psi(\tensor{\loke{1}{u}{U}}{\lok{\loke{m}{1}{}}{M}}{t})}{\psi'(\tensor{\loke{1}{u}{U}}{\lok{\loke{m}{1}{}}{M}}{t})}
\end{gather*}
$\phi \circ \varphi = id_A$ ergibt sich direkt aus der Universellen Eigenschaft des Kolimes.\\
Für $\varphi \circ \phi \defshow id_{\lok{M}{U}}$ wähle $\loke{m}{u}{U} \in \lok{M}{U}$ beliebig, für dieses gilt:
\begin{gather*}
(\varphi \circ \phi) (\psi'(\tensor{\loke{1}{u}{U}}{\lok{M}{t}}{\loke{m}{1}{}})) \\
 =\varphi(\psi(\tensor{\loke{1}{u}{U}}{\lok{M}{t}}{\loke{m}{1}{}})) \\
  =\psi'(\tensor{\loke{1}{u}{U}}{\lok{M}{t}}{\loke{m}{1}{}})
\end{gather*}
Damit haben wir $A \simeq \lok{M}{U}$ gezeigt, definiere also ab sofort $\lok{M}{U}$ als den eindeutigen Kolimes von $\mathcal{C}$.
\end{proof}
\end{document}
