\documentclass[10pt,a4paper]{report}
\usepackage[utf8]{inputenc}
\usepackage{amsmath}
\usepackage{amsthm}
\usepackage{amsfonts}
\usepackage{amssymb}
\usepackage{color}
\usepackage{tikz-cd}
\usepackage{calc}
\usepackage{setspace}
\usepackage[german]{babel}
\usetikzlibrary{babel}
\usepackage{cleveref}

\newcommand{\comment}[1]{}
\renewcommand{\baselinestretch}{1.1}

\newcommand{\ModulsOfDifferenzials}{David Eisenbud 1994}
\newcommand{\Algebra}{Christian Karpfinger, Kurt Meyberg 2009}

\newcounter{Aussage}[chapter]

\newtheorem{satz}[Aussage]{Satz}
\newtheorem{theorem}[Aussage]{Theorem}
\newtheorem{prop}[Aussage]{Proposition}
\newtheorem{korrolar}[Aussage]{Korrolar}
\newtheorem{lemma}[Aussage]{Lemma}
\newtheorem{bem}[Aussage]{Bemerkung}
\newtheorem{definition}[Aussage]{Definition}
\newtheorem{bsp}[Aussage]{Beispiel}

\newcommand{\functionfront}[3]{\nolinebreak{#1:#2 \longrightarrow #3}}
\newcommand{\functionback}[3]{\nolinebreak{#1:#2 \longmapsto #3}}
\newcommand{\function}[5]{\nolinebreak{#1:#2 \longrightarrow #3 \, , \, #4 \longmapsto #5}}
\newcommand{\infunctionfront}[3]{\nolinebreak{#1:#2 \hookrightarrow #3}}
\newcommand{\divR}[2]{\Omega_{#1/#2}}
\newcommand{\divf}[1]{d_{#1}}
\comment{\newcommand{\divf}[2][]{d_{#1}}}
\newcommand{\Tensor}[3]{#1 \otimes_{#2} #3}
\newcommand{\tensor}[3]{#1 \otimes #3}
\newcommand{\lok}[2]{#1 [#2^{-1}]}
\newcommand{\loke}[3]{(\frac{#1}{#2})_{_{#3}}}
\comment{\newcommand{\loke}[3]{(#1,#2)_{mod\sim_{#3}}}}

\newcommand{\colimes}[0]{\lim\limits_{ \longrightarrow }}
\newcommand*{\defeq}{\mathrel{\vcenter{\baselineskip0.5ex \lineskiplimit0pt
                     \hbox{\scriptsize.}\hbox{\scriptsize.}}}%
                     =}
\newcommand*{\defeqr}{= \mathrel{\vcenter{\baselineskip0.5ex \lineskiplimit0pt
                     \hbox{\scriptsize.}\hbox{\scriptsize.}}}}

\newcommand*{\defshow}{\stackrel{!}{=}}
\newcommand{\kernel}[1]{kern(#1)}
\newcommand{\immage}[1]{im(#1)}
\newcommand{\Verz}[1]{\langle #1 \rangle}


\begin{document}
\chapter{Körpererweiterungen}
\section{Einführung in die Körpererweiterungen}
\textcolor{blue}{\textbf{Definition Transzenddenzbasis} \textit{[vlg. Anhang A1 \ModulsOfDifferenzials]}}
\begin{def}\label{Definition Transzenddenzbasis}
Sei $L/k$ eine Körpererweiterung. Dann definieren wir:
\begin{itemize}
\item[•] Eine endliche Teilmengen $\lbrace l_1, \dots ,l_n \rbrace \subseteq L$ heißt \underline{algebraisch unabhängig} über $k$, falls gilt:
\begin{gather*}
\forall P(x_1, \dots , x_n) \in k[x_1,\dots,x_n] : \, P(l_1, \dots , l_n) \neq 0
\end{gather*}
\item[•] Eine Teilmenge $B \subseteq L$ heißt \underline{transzendent} über $k$, falls jede ihrer endlichen Teilmengen $\lbrace b_1, \dots , b_n \rbrace \subseteq B$ algebraisch unabhängig über $k$ ist.
\item[•] Eine Teilmenge $B \subseteq L$ ist eine \underline{Transzendenzbasis} von $L/k$, falls sie transzendent über $k$ und die Körpererweiterung $L/k(B)$ algebraisch ist.
\item[•] Falls eine Transzendenzbasis von $B$ von $L/k$ existiert, sodass $k(B) = L$ gilt, so ist $L/k$ eine \underline{pur transzendente Körpererweiterung}.
\end{itemize}
\end{def}


\ \\
\textcolor{blue}{\textbf{pur transzendente Erweiterung} \textit{[Eigene Überlegung]}}
\begin{bem}\label{pur transzendente Erweiterung}
Sei $L/k$ eine pur transzendente Körpererweiterung mit Transzendenzbasis $B$. Dann gilt:
\begin{gather*}
L \simeq k(\lbrace x_i \rbrace_{i \in B})
\end{gather*}
Insbesondere ist $\lbrace x_i \rbrace_{i \in B}$ eine Transzendenzbasis der Körpererweiterung der rationalen Funktionen $k(\lbrace x_i \rbrace_{i \in B})$ über $k$.
\end{bem}


\ \\
\textcolor{blue}{\textbf{Transzendenzbasis ist maximale transzendente Menge} \textit{[Lemma 22.1 \Algebra]}}
\begin{lemma}\label{Transzendenzbasis ist maximale transzendente Menge}
Sei $L / k$ ein Körpererweiterung und $B \subseteq L$ eine über $k$ transzendente Teilmenge. Dann gilt:\\
B ist genau dann eine Transzendenzbasis von $L/k$, wenn $B$ bezüglich der Inklusion ein maximales Element der Menge aller über $k$ transzendenten Elemente aus $L$ ist.
\end{lemma}
\begin{proof} \ \\
\begin{itemize}
\item[\underline{\glqq $\Rightarrow$:\grqq}] Sei B eine Transzendenzbasis über $k$. Zeige, dass für ein beliebiges Element $a \in L \setminus B$ die Menge $B \cup \lbrace a \rbrace \subseteq L$ nicht transzendent über $k$ ist:
\begin{gather*}
\text{Da die Körpererweiterung $L / k(B)$ algebraisch ist existiert } 0 \neq P(x) \in k(B)[x] \text{ mit } P(a) = 0.
\end{gather*}
Aus der Definition von $k(B)$ geht hervor, dass $\lbrace b_1, \dots b_n \rbrace \subseteq B$ existiert, mit $P(x) \in k(\lbrace b_1, \dots b_n \rbrace)[x]$.\\ 
Wir können ohne weitere Einschränkung annehmen, dass $P(x) \in k[\lbrace b_1, \dots ,b_n \rbrace][x]$ gilt, denn falls dies nicht der Fall sein sollte, wähle $m \in \mathbb{N}$ groß genug, sodass $\left( P(x) \cdot \left( \prod_i^n b_i \right)^m \right) \in k[\lbrace b_1, \dots ,b_n \rbrace]$ gilt.
\begin{gather*}
\text{Wähle nun } P'(x_1, \dots , x_n , x) \in  k[x_1, \dots , x_n , x] \text{ mit } P'(b_1, \dots , b_n, x) = P(x).\\
\text{Dies erfüllt } P'(b_1, \dots , b_n, a) = 0.
\end{gather*}
Folglich ist $B \cup \lbrace b_1,\dots,b_n, a \rbrace$ algebraisch abhängig und insbesondere $B \cup \lbrace a \rbrace$ nicht transzendent über k.
\item[\underline{\glqq $\Leftarrow$:\grqq}] Sei $B$ bezüglich der Inklusion ein maximales Element der Menge aller über $k$ transzendenten Elemente aus $L$. Zeige für ein beliebiges Element $a \in L \setminus k(B)$, dass dieses algebraisch über $k(B)$ ist:\\
\ \\
Nach Voraussetzung existiert eine endliche Teilmenge $\lbrace b_1, \dots, b_n, a \rbrace \subseteq B \cup \lbrace a \rbrace$, welche algebraisch abhängig über $k$ ist.
\begin{gather*}
\text{Also existiert } P(x_1, \dots, x_{n+1}) \in k[x_1,\dots ,x_{n+1}] \text{ mit } P(b_1, \dots , b_n , a) = 0. \\
\Rightarrow \text{Für } P'(x) \defeq P(b_1, \dots , b_n , x) \in k(B)[x] \text{ gilt } P'(a) = 0
\end{gather*}
Es existiert also ein Polynom $P'(x) \defeq P(b_1, \dots , b_n , x) \in k(B)[x]$ mit $P'(a) = 0$ gefunden. Somit ist $a$ algebraisch über $k(B)$.
\end{itemize}
\end{proof}


\ \\
\textcolor{blue}{\textbf{} \textit{[\Algebra}]}
\begin{korrolar}\label{Existenz von Transzendenzbasen}
Jede Körpererweiterung $L \subseteq k$ besitzt eine Transzendenzbasis $B \subseteq L$.
\end{korrolar}
\begin{proof}
Verwende hierzu das Lemma von Zorn:\\
\cref{Transzendenzbasis ist maximale transzendente Menge} besagt, dass die Transzendenzbasen von $L/k$ gerade maximales Elemente der Menge aller über $k$ transzendenten Elemente aus $L$ sind.\\
\textcolor{red}{Das Lemma von Zorn besagt, dass jede partiell geordenete Menge, in der jede Kette eine obere Schranke besitzt ist ein Maximales Element besitzt \textit{[vlg. Kapitel A2.3 \Algebra]}.}\\
Sei also \lbraceB_1
\end{proof}


\ \\
\textcolor{blue}{\textbf{Transzendenzbasen sind immer gleich lang} \textit{[Theorem A1.1 \ModulsOfDifferenzials]}}
\begin{prop}\comment{\label{Transzendenzbasen sind immer gleich lang}}
Sei $L \supset k$ eine Körpererweiterung. Seinen weiter $A$, $B$ zwei Transzendenzbasen von $L$ über $k$. Dann gilt:
\begin{gather*}
\vert A \vert = \vert B \vert
\end{gather*}
Wir nennen $\vert B \vert$ den \underline{Transzendenzgrad} von $L$ über $k$.
\end{prop}
\begin{proof}
Im Fall von $\vert A \vert = \vert B \vert = \infty$ sind wir schon fertig, sei also ohne Einschränkung $A = \lbrace a_1, \dots , a_m \rbrace$ und $B = \lbrace b_1, \dots , b_n \rbrace$ mit $min(m,n) = n < \infty$.\\
Wir wollen zunächst in $n$ Schritten die Elemente aus $B$ durch Elemente aus $A$ ersetzten und damit zeigen, dass $\lbrace a_1, \dots , a_n \rbrace$ eine Transzendenzbasis von $L$ über $k$ ist:\\
\ \\
Für den $i$-ten Schritt definiere $A_i \defeq \lbrace a_1,\dots,a_{i-1} \rbrace \subseteq A$, $B_i \defeq \lbrace b_i,\dots,b_n \rbrace \subseteq B$ und gehe davon aus, dass $A_i \cup B_i$ eine Transzendenzbasis ist:\\
Nach \cref{Transzendenzbasis ist maximale transzendente Menge} ist $\lbrace a_i \rbrace \cup A_{i} \cup B_{i} = A_{i+1} \cup B_{i}$ nicht transzendent und somit algebraisch abhängig.
\begin{gather*}
\text{Also existiert } P \in k[x,x_1,\dots,x_n] \text{ mit } P(a_i,a_1,\dots,a_{i-1},b_i,\dots,b_n) = 0. \\
\text{Definiere } P'(x) \defeq P(a_i,a_1,\dots,a_{i-1},x,b_{i+1},\dots,b_n) \in k(A_{i+1} \cup B_{i+1})[x].\\
\text{Dieses erfüllt } P'(b_i) = 0.
\end{gather*}
Da $A_i \subseteq A$ algebraisch unabhängig ist, gilt $P(a_1,\dots,a_{i-1},x_i,\dots,x_n) \neq 0$. Nummeriere also gegebenenfalls $B$ vor der Bildung von $P'(x)$ so um, dass auch $P'(x) \neq 0$ gilt.\\
Die Existenz eines solchen $P'(x)$ zeigt uns, dass die Körpererweiterungen \\$L \subset k(A_{i+1} \cup B_i) = k(A_{i+1} \cup B_{i+1})(\lbrace b_i \rbrace) \subset k(A_{i+1} \cup B_{i+1})$ algebraisch sind und legt nahe, dass $A_{i+1} \cup B_{i+1}$ wieder eine Transzendenzbasis ist.\\
Um dies zu zeigen nehme zunächst an $A_{i+1} \cup B_{i+1}$ wäre algebraisch abhängig.
\begin{gather*}
\text{Also existiert } Q \in k[x_1, \dots ,x_n] \text{ mit } Q(a_1, \dots ,a_i,b_{i+1},\dots,b_n) = 0. \\
\text{Definiere } Q'(x) \defeq Q(a_1,\dots,a_{i-1},x,b_{i+1},b_n) \in k(a_1,\dots,a_{i-1},b_{i+1},b_n)[x]. \\
\text{Dieses erfüllt } Q'(a_i) = 0.
\end{gather*}
Da $(A_{i+1} \cup B_{i+1}) \setminus \lbrace a_i \rbrace \subseteq A_i \cup B_i$ algebraisch unabhängig ist gilt $Q'(x) \neq 0$.\\
Die Existenz eines solchen $Q'(x)$ zeigt uns, dass die Körpererweiterung \\$L \subset k(A_{i+1} \cup B_{i+1}) \subset k((A_{i+1} \cup B_{i+1})\setminus\lbrace a_i \rbrace) = k((A_i \cup B_i)\setminus\lbrace b_i \rbrace)$ algebraisch ist. Damit ist $(A_i\cup B_i)\setminus\lbrace b_i \rbrace$ eine Transzendenzbasis, was nach \cref{Transzendenzbasis ist maximale transzendente Menge} im Widerspruch dazu steht, dass $A_i \cup B_i$ eine Transzendenzbasis ist.\\
Folglich ist $A_{i+1} \cup B_{i+1}$ transzendent und somit eine Transzendenzbasis von $L$ über $k$.\ \\
\ \\
Dieses Verfahren zeigt uns, dass $\lbrace a_1, \dots , a_n \rbrace \subseteq A$ eine Transzendenbasis von $L$ über $k$ ist. Nach \cref{Transzendenzbasis ist maximale transzendente Menge} muss somit $A = \lbrace a_1, \dots , a_n \rbrace$ und $m = n$ gelten.
\end{proof}


\ \\
\begin{korrolar}\label{Transzendent ist pur transzendent plus algebraisch 1}
Für jede Körpererweiterung $L/k$ existiert ein Zwischenkörper $K \subseteq L$, sodass $K/k$ eine pur transzendente und $L/K$ eine algebraische Körpererweiterung ist.
\end{korrolar}
\begin{proof}
Nach \cref{Existenz von Transzendenzbasen} existiert eine Transzendenzbasis $B$ von $L/k$. Nach \cref{Definition Transzendenzbasis} ist somit $k(B)/k$ pur Transzendent und $L/k(B)$ algebraisch.\\
Wähle also $K \defeq k(B)$
\end{proof}


\ \\
\begin{bsp}\comment{\label{Unterschiedliche Transzendenzbasen bsp}}
Sei dazu $L = k(y)$ der Körper der rationalen Funktionen über $k$. Betrachte zwei unterschiedliche Transzendenzbasen von $L/k$:
\begin{itemize}
\item[\textbf{1.}] $B = \lbrace y \rbrace$ ist eine Transzendenzbasis von $L/k$ mit $\deg(L/k(B)) = 1$.
\item[\textbf{2.}] Für $n \in \mathbb{N}$ ist $B'= \lbrace y^n\rbrace$ eine Transzendenzbasis von $L/k$ mit $\deg(L/k(B)) = n$.
\begin{gather*}
f(x) = x^n - y^n \in k(y^n)[x] \text{ ist Minnimalpolynom von $x$ über $k(y^n)$.} \\
\Rightarrow k(y)/k(y^n) \text{ ist eine algebraische Körpererweiterung vom Grad $n$}
\end{gather*}
\end{itemize}
Dies zeigt, dass die Form des Körpers $k(B)$ und insbesondere der Grad der Körpererweiterung $L/k(B)$ sehr von der Wahl der Transzendenzbasis B abhängt.
\end{bsp}

\ \\
\textcolor{red}{
\begin{itemize}
\item[\underline{\textbf{Erinnerung:}}] Eine Algebraische Körpererweiterung $L \supset k$ heißt \underline{seperabel}, falls für alle $\alpha \in L$ das Minimalpolynom $f(x) \in k[x]$ von $\alpha$ über $L[x]$ in Linearfaktoren zerfällt.
\end{itemize}
}
\begin{definition}\label{Definition Seperabel}
Sei $L \supset k$ eine Körpererweiterung. Dann definieren wir:
\begin{itemize}
\item $L$ ist \underline{seperabel generiert} über $k$, falls eine Transzendenzbasis $B$ von $L$ über $k$ existiert, sodass $L/k(B)$ eine seperable Körpererweiterung ist.
\item $k$ ist \underline{seperabel} über $k$, falls jeder über $k$ endlich genierte Teilkörper von $L$ über $k$ seperabel generiert ist.
\end{itemize}
\end{definition}


\ \\
\begin{definition}
Sei $k$ ein Körper mit charakteristik p und sei weiter $L/k$ eine Körpererweiterung. Dann definieren wir:
\begin{itemize}
\item Eine endliche Teilmenge $B \subseteq L$ heißt p-Basis von $L$ über $k$, falls $W \defeq \lbrace \prod_{b \in B} b^i \vert i < p \rbrace$ eine Vektorraumbasis von K über $k * K^p$ bildet.
\end{itemize}
\end{definition}

\section{Differential von Körpererweiterungen}

\textcolor{blue}{\textbf{Definition der Differenzialbasis} \textit{[vlg. Chapter 16.5 \ModulsOfDifferenzials]}}
\begin{definition}\comment{\label{Definition der Differenzialbasis}}
Sei $L \supset k$ eine Körpererweiterung. Dann nennen wir eine Teilmenge $\lbrace b_i \rbrace_{i \in \Lambda} \subseteq L$ eine \underline{Differenzialbasis} von $L$ über $k$, falls $\lbrace \divf{K}(b_i)\rbrace_{i \in \Lambda}$ eine Vektorraumbasis von $\divR{L}{R}$ über $L$ ist.
\end{definition}


\ \\
\textcolor{blue}{\textbf{Differential von rationalen Funktionen 1} \textit{[vlg. Chapter 16.5 \ModulsOfDifferenzials]}}
\begin{bsp}\label{Differential von rationalen Funktionen 1}
Sei $k$ ein Körper und $L = k(\lbrace x_i \rbrace_{i \in \lbrace 1,\dots,n \rbrace})$ der Körper der rationalen Funktionen in $n$ Varablen über $k$.\\
Dann gilt:
\begin{gather*}
\divR{L}{k} \simeq L \Verz{\divf{k[x_1,\dots x_n]}(x_i)}
\end{gather*}
Insbesondere ist $\lbrace x_i \rbrace_{i \in \lbrace 1,\dots,n \rbrace}$ eine Differenzialbasis von $\divR{L}{k}$.
\end{bsp}
\begin{proof}
Betrachte $L = \lok{k[x_1,\dots,x_n]}{k[x_1,\dots,x_n]}$ als Lokalisierung um \cref{Differenzial der Lokalisierung} anwenden zu können. Anschließend forme noch $\divR{k[x_1,\dots,x_n]}{k}$ mithilfe von \cref{Differenzial von Polynomalgebren 1} isomorph um:
\begin{gather*}
\divR{L}{k} \simeq \tensor{L}{k[x_1,\dots,x_n]}{\divR{k[x_1,\dots,x_n]}{k}} \\
\simeq \tensor{L}{k[x_1,\dots,x_n]}{\oplus_{i \in \lbrace 1,\dots,n \rbrace} k[x_1,\dots,x_n] \Verz{\divf{k[x_1,\dots x_n]}(x_i)}} \\
\simeq L \Verz{\divf{k[x_1,\dots x_n]}(x_i)}
\end{gather*}
Damit ist $\lbrace \divf{L}(x_i) \rbrace_{i \in \lbrace 1,\dots,n \rbrace}$ eine Vektorraumbasis von $\divR{L}{k}$.
\end{proof}


\ \\
\textcolor{blue}{\textbf{Differential von rationalen Funktionen 2} \textit{[Aufgabe 16.6 \ModulsOfDifferenzials]}}
\begin{korrolar}\label{Differential von rationalen Funktionen 2}
Sei $k$ ein Körper und $L \supset k$ eine Körpererweiterung und $T = L(\lbrace x_i \rbrace_{i \in \lbrace 1,\dots,n \rbrace})$ der Körper der rationalen Funktionen in $n$ Varablen über $L$. Dann gilt:
\begin{gather*}
\divR{T}{k} \simeq (\Tensor{T}{L}{\divR{L}{R}}) \oplus \bigoplus_{i \in \lbrace 1,\dots,n \rbrace} T \Verz{\divf{T}(x_i)}
\end{gather*}
\end{korrolar}
\begin{proof}
Betrachten $T$ als Lokalisierung von $L[x_1,\dots,x_n]$ und gehen dann analog zu \cref{Differential von rationalen Funktionen 1} vor:
\begin{gather*}
\divR{T}{k} \simeq \Tensor{T}{L[x_1,\dots,x_n]}{\divR{L[x_1,\dots,x_n]}{k}} \textit{ (\cref{Differenzial der Lokalisierung})} \\
\divR{L[x_1,\dots,x_n]}{R} \simeq (\Tensor{L[x_1,\dots,x_n]}{L}{\divR{L}{R}}) \oplus_{i \in \lbrace 1,...,n \rbrace} L[x_1,\dots,x_n] \Verz{\divf{L[x_1,\dots,x_n]}(x_i)} \textit{ (\cref{Differenzial von Polynomalgebren 2})} \\
\Rightarrow \divR{T}{k} \simeq (\Tensor{T}{L}{\divR{L}{R}}) \oplus_{i \in \lbrace 1,\dots,n \rbrace} T \Verz{\divf{T}(x_i)}
\end{gather*}
\end{proof}


\ \\
\textcolor{blue}{\textbf{Cotangent Sequenz von Koerpern 1} \textit{[Aufgabe 16.6 \ModulsOfDifferenzials]}}
\begin{bem}\label{Cotangent Sequenz von Koerpern 1}
Sei $L \supset k$ eine Körpererweiterung und $T = L(x_1, \dots ,x_n)$ der Körper der rationalen Funktionen in $n$ Variablen über $L$. Dann ist die COTANGENT SEQUENZ \textit{(\cref{Cotangent Sequenz})} von $k \hookrightarrow L \hookrightarrow T$ eine kurze Exakte Sequenz:
\begin{center}
\begin{tikzcd}
0 \arrow[r] & \Tensor{T}{L}{\divR{L}{k}} \arrow[r] & \divR{T}{k} \arrow[r] & \divR{T}{L} \arrow[r] & 0
\end{tikzcd}
\end{center}
Im Genauen ist $\function{\varphi}{\Tensor{T}{L}{\divR{L}{k}}}{\divR{T}{k}}{\tensor{t}{L}{\divf{L}(l)}}{t \cdot \divf{T}(l)}$ injektiv.
\end{bem}
\begin{proof}
Die Injektivität von $\varphi$ folgt direkt aus der isomorphen Darstellung von $\divR{T}{k}$, die wir uns in \cref{Differential von rationalen Funktionen 2} erarbeitet haben.
\begin{gather*}
\divR{T}{k} \simeq (\Tensor{T}{L}{\divR{L}{R}}) \oplus \bigoplus_{i \in \lbrace 1,\dots,n \rbrace} T \Verz{\divf{T}(x_i)}
\end{gather*}
Um sicher zu gehen definiere $\varphi' \simeq \varphi$ und durchlaufe die in \cref{Differential von rationalen Funktionen 2} genutzten Isomorphismen noch einmal Schritt für Schritt:
\begin{center}
$\functionfront{\varphi'}{\Tensor{T}{L}{\divR{L}{k}}}{\Tensor{T}{L}{\divR{L}{R}} \oplus \bigoplus_{i \in \lbrace 1,\dots,n \rbrace} T \Verz{\divf{T}(x_i)}}$\\
\ \\
\begin{tikzcd}
\Tensor{T}{L}{\divR{L}{k}} \arrow[d, hook]                 &  &  & \tensor{t}{L}{\divf{L}(l)} \arrow[d, maps to] \\
\divR{T}{k} \arrow[d, "\cref{Differenzial der Lokalisierung}", two heads, hook] &  &  & t\divf{T}(l) \arrow[d, maps to] \\
\Tensor{T}{S}{\divR{L[x_1,\dots ,x_n]}{k}} \arrow[d, "\cref{Differenzial von Polynomalgebren 2}", two heads, hook] &  &  & \tensor{t}{S}{\divf{S}(l)} \arrow[d, maps to] \\
\Tensor{T}{S}{((\Tensor{S}{L}{\divR{L}{k}}) \oplus \bigoplus_{i \in \lbrace 1,\dots,n \rbrace} S \Verz{\divf{S}(x_i)}}) \arrow[d, "", two heads, hook] &  &  & \tensor{t}{S}{(\divf{L}(l),0)} \arrow[d, maps to] \\
(\Tensor{T}{L}{\divR{L}{R}}) \oplus \bigoplus_{i \in \lbrace 1,\dots,n \rbrace} T \Verz{\divf{T}(x_i)}                                 &  &  & (\tensor{t}{L}{\divf{L}(l)},0)                   
\end{tikzcd} 
\end{center}
Damit ist $\varphi$ eine injektive Einbettung von $\Tensor{T}{L}{\divR{L}{k}}$ in $\divR{T}{k}$.
\end{proof}


\ \\
\textcolor{blue}{\textbf{Aufbaulemma Koerperdifferenzial} \textit{[vlg. Lemma 16.15 \ModulsOfDifferenzials]}}
\begin{lemma}\label{Aufbaulemma Koerperdifferenzial}
Sei $L \subset T$ eine seperable und algebraische Körpererweiterung und $R \longrightarrow L$ ein Ringhomomorphismus. Dann gilt:
\begin{gather*}
\divR{T}{R} = \Tensor{T}{L}{\divR{L}{R}}
\end{gather*}
Insbesondere ist in diesem Fall die COTANGENT SEQUENZ \textit{(\cref{Cotangent Sequenz})} von $R \rightarrow L \hookrightarrow T$ eine kurze Exakte Sequenz:
\begin{center}
\begin{tikzcd}
0 \arrow[r] & \Tensor{T}{L}{\divR{L}{R}} \arrow[r] & \divR{T}{R} \arrow[r] & \divR{T}{L} \arrow[r] & 0
\end{tikzcd}
\end{center}
\end{lemma}
\begin{proof}
Wähle $\alpha \in T$ mit $L[\alpha] = T$. Sei weiter f(x) das Minimalpolynom von $\alpha$. Betrachte dazu die conormale Sequenz von  $\functionfront{\pi}{L[x]}{L[x]/(f) \simeq T}$ \textit{(\cref{Konormale Sequenz})}:
\begin{center}
\begin{tikzcd}
(f)/(f^2) \arrow[r, "\tensor{1}{L[x]}{\divf{L[x]}}"] & \Tensor{T}{L[x]}{\divR{L[x]}{R}} \arrow[r, "D\pi"] & \divR{T}{R} \arrow[r] & 0
\end{tikzcd}
\end{center}
Wende nun Proposition 16.6 auf $\divR{L[x]}{R}$ an und tensoriere mit $T$, somit gilt:
\begin{gather*}
\Tensor{T}{L[x]}{\divR{L[x]}{R}} \simeq \Tensor{T}{L}{\divR{L}{R}} \oplus T\langle \divf{L[x]}(x) \rangle
\end{gather*}
Zusammen mit der conormalen Sequenz bedeutet dies:
\begin{gather*}
\divR{T}{R} \simeq (\Tensor{T}{L}{\divR{L}{R}} \oplus T \Verz{\divf{L[x]}(x)})/(\divf{L[x]}(f))
\end{gather*}
Wenn wir $\functionfront{\divf{L[x]}}{(f)}{\Tensor{T}{L}{\divR{L}{R}} \oplus T \Verz{\divf{L[x](x)}}}$ wie in \cref{Differenzial ist Ableitung} betrachten , sehen wir:
\begin{gather*}
\divf{L[x]}((f)) = J \oplus (f'(\alpha)\divf{L[x]}) = J \oplus T \Verz{\divf{S[x]}(x)}\\
\text{,wobei $J \subseteq \Tensor{T}{L}{\divR{L}{R}}$ ein Ideal ist.}
\end{gather*}
Für die letzte Gleichheit nutze, dass $T \supset L$ seperabel und somit $f'(\alpha) \neq 0$ ist und nach obiger Wahl $T = L[\alpha]$ gilt.\\
Damit erhalten wir nun:
\begin{gather*}
\divR{T}{R} \simeq (\Tensor{T}{L}{\divR{L}{R}})/J \\
\Rightarrow \Tensor{T}{L}{\divR{L}{R}} \hookrightarrow \divR{T}{R} \textit{ ist surjektiv.}
\end{gather*}
Somit muss J = 0 gelten und es folgt $\Tensor{T}{L}{\divR{L}{R}} \simeq \divR{T}{R}$.\\
Damit haben wir insbesondere auch gezeigt, dass $\Tensor{T}{L}{\divR{L}{R}} \rightarrow \divR{T}{R}$ injektiv und somit die COTANGENT SEQUENZ von $R \rightarrow L \hookrightarrow T$ eine kurze exakte Sequenz ist.
\end{proof}


\ \\
\textcolor{blue}{\textbf{Transzendenzbasis ist Differenzialbasis} \textit{[vlg. Theorem 16.4 \ModulsOfDifferenzials]}}
\begin{theorem}\comment{\label{Transzendenzbasis ist Differenzialbasis}}
Sei $T \supset k$ eine seperabel generierte Körpererweiterung und $B = \lbrace b_i \rbrace_{i \in \Lambda} \subseteq T$. Dann ist $B$ genau dann eine Differenzialbasis von $T$ über $k$, falls eine der folgedenen Bedingungen erfüllt ist:
\begin{itemize}
\item[\textbf{1.}] char(k) = 0 und $B$ ist eine Transzendenzbasis von $T$ über $k$.
\item[\textbf{2.}] char(k) = p und $B$ ist eine p-Basis von $T$ über $k$.
\end{itemize}
\end{theorem}
\begin{proof}
\ \\
\begin{itemize}
\item[\underline{\textbf{1.}\glqq$\Leftarrow$\grqq:}] Sei $B$ eine Transzendenzbasis von $T$ über $k$.\\
Damit ist die Körpererweiterung $L \defeq k(B) \supset k$ algebraisch und seperabel. \comment{ mit \label{*Transzendenzbasisdef}} Mit \cref{Aufbaulemma Koerperdifferenzial} folgt:
\begin{gather*}
\divR{T}{k} = \Tensor{T}{L}{\divR{L}{k}}
\end{gather*}
Betrachte $L = \lok{k[B]}{k[B] \setminus 0}$ als Lokalisierung und wende \cref{Differenzial der Lokalisierung} auf $\divR{L}{k}$ an, somit gilt:
\begin{gather*}
\divR{L}{k} = \Tensor{L}{k[B]}{\divR{k[B]}{k}}
\end{gather*}
In \cref{Differenzial von Polynomalgebren 1} \comment{\label{*Differenzial von Polynomalgebren brauche ich für unendliche Mengen Lambda}} haben wir gesehen, dass $\divR{k[B]}{k}$ ein freis Modul über $k[B]$ mit $\lbrace b_i \rbrace_{i \in \Lambda}$ als Basis ist. Dies liefert uns letztendlich die gewünschte Darstellung
\begin{gather*}
\divR{T}{k} = \bigoplus_{\lbrace i \in \Lambda \rbrace} T \Verz{\divf{T}(b_i)}.
\end{gather*}
\item[\underline{\textbf{1.}\glqq$\Rightarrow$\grqq:}]Sei $\divf{T}(B)$ eine Vektorraumbasis von $\divR{T}{k}$.\\
Zeige zunächst, dass T algebraisch über $L \defeq k(B)$ ist:
\begin{gather*}
\text{Die COTANGENT SEQUENZ \textit{(\cref{Cotangent Sequenz})} von $k \hookrightarrow L \hookrightarrow T$ besagt }\\
\divR{T}{L} \simeq \divR{T}{k}/T \Verz{\divf{T}(S)} \text{ und nach Vorraussetzung gilt } \divR{T}{k} = T \Verz{\divf{T}(B)}.
\\
\Rightarrow \divR{T}{L} \simeq \divR{T}{k}/T\Verz{\divf{T}(L)} = \divR{T}{k}/T\Verz{\divf{T}(B)}=
\divR{T}{k}/\divR{T}{k} = 0
\end{gather*}
Da, wie wir in \glqq$\Leftarrow_{1.}$\grqq gezeigt haben, jede Transzendenzbasis $B'$ von $T$ über $L$ auch eine Differenzialbasis von $\divR{T}{L} = 0$ ist, gilt für diese $B' = \emptyset$. Somit ist $T$ schon algebraisch über $L$.\\
\ \\
Zeige noch, dass $B$ auch algebraisch unabhängig über $L$ ist:\\
Sei dazu $\Gamma$ eine minimale Teilmenge von $\Lambda$, für welche $T$ noch algebraisch über $k(\lbrace b_i \rbrace_{i \in \Gamma})$ ist. Für diese ist $\lbrace b_i \rbrace_{i \in \Gamma}$ algebraisch unabhängig über K.\\
Damit ist nach \glqq$\Leftarrow_{1.}$\grqq $\lbrace b_i \rbrace_{i \in \Gamma}$ ebenfalls eine Differenzialbasis von $T$ über $k$. Also muss schon $\Gamma = \Lambda$ gegolten haben und $B$ ist eine Transzendenzbasis von $T$ über $k$.
\item[\underline{\textbf{2.}\glqq$\Leftarrow$\grqq:}] Sei B eine p-Basis von T über k.\\
Somit wird nach DEFINITION-PROPOSITION \comment{\label{*p-Basis ist minnimaler Erzeuger von T als Algebra}} $T$ von $B$ als Algebra über $(k * T^p)$ und $\divR{T}{(k * T^p)}$ von $\divf{T}(B)$ als Vektorraum über $T$ 
\textit{(PROPOSITION)} \comment{\label{*Differenzial vererbt Erzeugendensystem}} erzeugt. Zeige also $\divR{T}{k} \simeq \divR{T}{(T^p * k)}$:\\
Die Cotangent Sequenz \textit{(\cref{Cotangent Sequenz})} von $K \hookrightarrow (k * T^p) \hookrightarrow T$ besagt:
\begin{gather*}
\divR{T}{(T^p * k)} \simeq \divR{T}{k}/\divf{T}(T^p * k)
\end{gather*}
\begin{gather*}
\text{Für beliege } t^p \in T^p \text{ gilt } \divf{T}(t^p) = pt^{p-1}\divf{T}(t) = 0 \text{,  da }char(T) = p.\\
\Rightarrow \divf{T}(T^p * k) = \divf{T}(k(T^p)) = 0
\end{gather*}
Damit ist $\functionfront{\divf{T}}{T}{\divR{T}{k}}$ auch $(T^p *k)$-linear und es gilt $\divR{T}{k} \simeq \divR{T}{(T^p * k)}$.
\item[\underline{\textbf{2.}\glqq$\Rightarrow$\grqq:}] Sei $\divf{T}(B)$ eine Vektorraumbasis von $\divR{T}{k}$.\\
Zeige zunächst, dass $T$ von $B$ als Algebra über $k$ erzeugt wird:
\begin{gather*}
\text{Die COTANGENT SEQUENZ (\cref{Cotangent Sequenz}) von $k \hookrightarrow L \defeq k(B) \hookrightarrow T$ besagt }\\
\divR{T}{L} \simeq \divR{T}{k}/T \Verz{\divf{T}(L)} \text{ und nach Vorraussetzung gilt } \divR{T}{k} = T \langle \divf{T}(B) \rangle. \\
\Rightarrow \divR{T}{L} \simeq \divR{T}{k}/T\Verz{\divf{T}(L)} = \divR{T}{k}/T\Verz{\divf{T}(B)}=
\divR{T}{k}/\divR{T}{k} = 0
\end{gather*}
Da, wie wir in \glqq$\Leftarrow_{2.}$\grqq gezeigt haben, jede p-Basis $B'$ von $T$ über $L$ auch eine Differenzialbasis von $\divR{T}{L} = 0$ ist, gilt für diese $B' = \emptyset$. Somit wird $T$ schon von $B$ als Algebra über $k$ erzeugt.\\
\ \\
Zeige noch, dass $B$ auch minimal als Erzeugendensystem von $T$ als Algebra über $k$ ist:\\
Sei dazu $\Gamma$ die minimale Teilmenge von $\Lambda$, für welche $T$ noch von $\lbrace b_i \rbrace_{i \in \Gamma}$ als Algebra über $k$ erzeugt wird. Dann ist $\lbrace b_i \rbrace_{i \in \Gamma}$ eine p-Basis von $T$ über $k$. Somit ist nach \glqq$\Leftarrow_{2.}$\grqq $\lbrace b_i \rbrace_{i \in \Gamma}$ ebenfalls eine Differenzialbasis von $T$ über $k$. Es muss also schon $\Gamma = \Lambda$ gegolten haben und $B$ ist eine p-Basis von $T$ über $k$. \comment{\label{*p-Basis ist minnimaler Erzeuger von T als Algebra}}
\end{itemize}
\end{proof}
\end{document}
