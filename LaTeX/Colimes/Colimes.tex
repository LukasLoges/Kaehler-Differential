\documentclass[10pt,a4paper]{report}
\usepackage[utf8]{inputenc}
\usepackage{amsmath}
\usepackage{amsthm}
\usepackage{amsfonts}
\usepackage{amssymb}
\usepackage{color}
\usepackage{tikz-cd}
\usepackage{calc}
\usepackage{setspace}
\usepackage[german]{babel}
\usetikzlibrary{babel}
\usepackage{cleveref}

\newcommand{\comment}[1]{}
\renewcommand{\baselinestretch}{1.1}

\newcommand{\ModulsOfDifferenzials}{David Eisenbud 1994}

\newcounter{Aussage}[chapter]

\newtheorem{satz}[Aussage]{Satz}
\newtheorem{theorem}[Aussage]{Theorem}
\newtheorem{prop}[Aussage]{Proposition}
\newtheorem{korrolar}[Aussage]{Korrolar}
\newtheorem{lemma}[Aussage]{Lemma}
\newtheorem{bem}[Aussage]{Bemerkung}
\newtheorem{definition}[Aussage]{Definition}
\newtheorem{bsp}[Aussage]{Beispiel}

\newcommand{\functionfront}[3]{\nolinebreak{#1:#2 \longrightarrow #3}}
\newcommand{\functionback}[3]{\nolinebreak{#1:#2 \longmapsto #3}}
\newcommand{\function}[5]{\nolinebreak{#1:#2 \longrightarrow #3 \, , \, #4 \longmapsto #5}}
\newcommand{\infunctionfront}[3]{\nolinebreak{#1:#2 \hookrightarrow #3}}
\newcommand{\divR}[2]{\Omega_{#1/#2}}
\newcommand{\divf}[1]{d_{#1}}
\comment{\newcommand{\divf}[2][]{d_{#1}}}
\newcommand{\Tensor}[3]{#1 \otimes_{#2} #3}
\newcommand{\tensor}[3]{#1 \otimes #3}
\newcommand{\lok}[2]{#1 [#2^{-1}]}
\newcommand{\loke}[3]{(\frac{#1}{#2})_{_{#3}}}
\comment{\newcommand{\loke}[3]{(#1,#2)_{mod\sim_{#3}}}}

\newcommand{\colimes}[0]{\lim\limits_{ \longrightarrow }}
\newcommand*{\defeq}{\mathrel{\vcenter{\baselineskip0.5ex \lineskiplimit0pt
                     \hbox{\scriptsize.}\hbox{\scriptsize.}}}%
                     =}
\newcommand*{\defeqr}{= \mathrel{\vcenter{\baselineskip0.5ex \lineskiplimit0pt
                     \hbox{\scriptsize.}\hbox{\scriptsize.}}}}

\newcommand*{\defshow}{\stackrel{!}{=}}
\newcommand{\kernel}[1]{kern(#1)}
\newcommand{\immage}[1]{im(#1)}
\newcommand{\Verz}[1]{\langle #1 \rangle}


\begin{document}
\begin{satz}\label{Differenzial von Polynomalgebren 1}
Differenzial von Polynomalgebren 1
\end{satz}


\ \\
\textcolor{blue}{\textbf{Differenzial indempotenter Elemente} \textit{[Aufgabe 16.1 \ModulsOfDifferenzials]}}
\begin{lemma}\label{Differenzial indempotenter Elemente}
Sei S eine R-Algebra und $\functionfront{d}{S}{M}$ ein beliebige Ableitung von S in ein $S$-Modul $M$. Sei weiter $a \in S$ ein indempotentes Element \textcolor{red}{($a^2 = a$)}.\\
\begin{center}
Dann gilt $d(a) = 0$. 
\end{center}
\end{lemma}
\begin{proof}
Nutze hierfür allein die Leibnizregel \textit{(DEFINITION)}: \comment{\label{*Definition Leibnizregel}}  
\begin{gather*}
\text{Schritt 1: } \divf{S}(a) = \divf{S}(a^2) = a\divf{S}(a) + a\divf{S}(a) \\
\text{Schritt 2: } a\divf{S}(a) = a\divf{S}(a^2) = a^2\divf{S}(a) + a^2\divf{S}(a) = a\divf{S}(a) + a\divf{S}(a)\\
\Rightarrow \divf{S}(a) = a\divf{S}(a) = 0
\end{gather*}
\end{proof}


\ \\
\textcolor{blue}{\textbf{Differenzial des Produktes von Algebren} \textit{[Proposition 16.10 \ModulsOfDifferenzials]}}
\begin{prop}\label{Differenzial des Produktes von Algebren}
Seien $S_1, \dots , S_n$ R-Algebren. Sei dazu $S \defeq \prod_{i \in \lbrace 1, \dots , n \rbrace} S_i$ die direkte Summe.
Dann gilt:
\begin{gather*}
\divR{S}{R} = \prod_{i \in \lbrace 1, \dots , n \rbrace} \divR{S_i}{R}
\end{gather*}
\end{prop}
\begin{proof}
Sei für $i \in \lbrace 1, \dots ,n \rbrace$ jeweils $e_i \in S$ die Einbettung es Einselement's von $S_i$ in $S$, somit ist $\functionfront{p_i}{e_iS}{S_i}$ ein Isomorphismus.\\
Nutze weiter, dass $e_i$ ein indempotentes Element \textcolor{red}{$({e_i}^2 = e_i)$} von $S$ ist:
\begin{gather*}
\text{Nach \cref{Differenzial indempotenter Elemente} gilt } \divf{S}(e_i) = 0 \\
\Rightarrow \forall s \in s : \divf{S}(e_is)= \divf{S}({e_i}^2s) = e_i\divf{S}(e_is) + e_is\divf{S}(e_i) = e_i\divf{S}(e_is) \\
\end{gather*}
Mit diesem Wissen können wir einen Isomorphismus $\functionfront{\Phi}{\divR{S}{R}}{\prod_{i \in \lbrace 1, \dots , n \rbrace} \divR{S_i}{R}}$ definieren:
\begin{center}
\begin{tikzcd}
\Phi: \, \divR{S}{R} \arrow[r]          & \prod_{i \in \lbrace 1, \dots , n \rbrace} e_i\divf{S}(e_iS) \arrow[r]          & \prod_{i \in \lbrace 1, \dots , n \rbrace} \divR{S_i}{R} \\
\divf{S}(s) = \sum_{i \in \lbrace 1, \dots , n \rbrace} \divf{S}(e_is) \arrow[r, maps to] & \left( e_1\divf{S}(e_1s), \dots , e_n\divf{S}(e_ns) \right) \arrow[r, maps to] & \left( (\divf{S_1} \circ p_1) (s), \dots , (\divf{S_n} \circ p_n) (s) \right)
\end{tikzcd}
\end{center}
Da der Differenzialraum $\divR{S}{R}$ bis auf eine eindeutige Isomophie eindeutig ist \textit{(PROPOSITION)},\comment{\label{*Differenzial ist eindeutig}} definiere diesen ab jetzt als $\prod_{i \in \lbrace 1, \dots , n \rbrace} \divR{S_i}{R}$.
\end{proof}


\ \\
\textcolor{blue}{\textbf{Differenzial algebraischer Algebren ist Null} \textit{[Aufgabe 16.11 \ModulsOfDifferenzials]}}
\begin{bsp}\comment{\label{Differenzial algebraischer Algebren ist Null}}
Sei $K$ ein Körper mit $char(K) = 0$ \comment{Dies habe ich im gesamten Beweis nicht verwendet.} und $T$ eine noethersche K-Algebra. Dann gilt:
\begin{gather*}
\divR{T}{K} = 0 \\
\Leftrightarrow \\
T = \prod_{i \in \lbrace 1, \dots, n \rbrace} K(\alpha_i) \textit{ ist ein endliches Produkt algebraischer Körpererweiterungen.} 
\end{gather*}
\end{bsp}
\begin{proof}
\ \\
\begin{itemize}
\item[\underline{{\glqq $\Rightarrow$ \grqq :}}]
Da $T$ noethersch ist, ist $T$ als Algebra über $K$ endlich erzeugt und es gilt:
\begin{gather*}
\comment{
T = T'/I \\
\text{Mit: } T' \defeq \left( \prod_{i \in \lbrace 1,\dots,n \rbrace} K[\alpha_i] \right) \text{ und } I \defeq \left( \prod_{i \in \lbrace 1,\dots,n \rbrace} I_i \right) \subseteq T' \text{ ist ein Ideal.} \\
\text{Also gilt } T = \prod_{i \in \lbrace 1,\dots,n \rbrace} K[\alpha_i]/I_i
}
T = \prod_{i \in \lbrace 1,\dots,n \rbrace} K[\alpha_i]/I_i \\
\text{Wobei } I_i \subseteq K[\alpha_i] \text{ ein Ideal ist.} (\forall i \in \lbrace 1, \dots ,n \rbrace)
\end{gather*}
Zur Vereinfachung definiere $T' \defeq \prod_{i \in \lbrace 1,\dots,n \rbrace} K[\alpha_i]$. Betrachte nun den Differentialraum von T genauer:
\begin{gather*}
\divR{T}{K} = \divf{T'} \left( \prod_{i \in \lbrace 1,\dots,n \rbrace} K[\alpha_i]/I_i \right)\\
= \prod_{i \in \lbrace 1,\dots,n \rbrace} \divf{K[\alpha]}\left( K[\alpha_i]/I_i \right) \textit{ (\cref{Differenzial des Produktes von Algebren})}
\end{gather*}
Betrachte also jeweils für $i \in \lbrace 1,\dots,n \rbrace$ die $K$-Algebra $K[\alpha_i]/I_i$.\\ 
Sei $I_i \neq K[\alpha_i]$, da andernfalls $K[\alpha_i]/I_i = 0$ und somit $\alpha_i$ kein Erzeuger vor T wäre. \\
Unterscheide nun zwischen den zwei möglichen Fällen \underline{$I_i = 0$} und \underline{$I_i \neq 0$}:
\begin{itemize}
\item[\underline{\textbf{1.} $I_i = 0$:}] Da $\divR{T}{K} = 0$ gilt, muss  $K[\alpha_i] = 0$ gelten.\\
$\mathbb{A}$nnhame: $\alpha_i$ ist transzendent über K.
\begin{gather*}
\text{Dies bedeutet } K[\alpha_i] \simeq K[x] \\
\Rightarrow \divR{K[\alpha_i]}{K} \simeq K[x]\langle \divf{K[x]}(x) \rangle \neq 0 \textit{ (\cref{Differenzial von Polynomalgebren 1})}
\end{gather*}
Dies steht allerdings im Widerspruch zu $K[\alpha_i] = 0$. Folglich war unsere $\mathbb{A}$nnahme falsch und $\alpha_i$ ist algebraisch über K.\\
Folglich ist $K[\alpha_i] = K(\alpha_i)$ \comment{Nutze hier \label{*fuer a algebraisch gilt K[a] = K(a)}} eine algebraische Körpererweiterung.
\item[\underline{\textbf{2.} $I_i \neq 0$:}]
Zunächst sehen wir, dass $\alpha_i$ transzendent sein muss, da sonst $K[\alpha_i] = K(\alpha_i)$ ein Körper wäre und somit $I_i = K(\alpha_i)$ gelten würde.\\
Also ist $\alpha_i$ transzendent und es gilt:
\begin{gather*}
K[\alpha_i] \simeq K[x] \text{ und } I \simeq (f(x)) \text{ mit } f(x) \in K[x] \\
\Rightarrow K[\alpha_i] \simeq K[\beta_1, \dots \beta_n] = K(\beta_1, \dots \beta_n) \text{, wobei $\beta_1, \dots \beta_n$ die Nullstellen von f sind.}
\end{gather*}
Somit haben wir gezeigt, dass auch in diesem Fall $K[\alpha_i]/I_i$ eine Algebraische Körpererweiterung ist. 
\end{itemize}
\item[\underline{{\glqq $\Leftarrow$ \grqq :}}]
\cref{Differenzial des Produktes von Algebren} besagt, dass das direkte Produkt unter Bildung des Differenzials erhalten bleibt, also gilt in diesem Fall:
\begin{gather*}
\divR{T}{K} = \prod_{i \in \lbrace 1, \dots, n \rbrace} \divR{K(\alpha_i)}{K}
\end{gather*} 
Nach Voraussetzung sind alle Körpererweiterungen $K{\alpha_i} \supset K$ algebraisch. Wir haben schon in BSP gesehen, dass somit deren Differentiale gleich 0 sind. Folglich ist auch das direkte Produkt der einzelnen Differenziale und somit $\divR{T}{K}$ gleich 0.
\end{itemize}
\end{proof}
\end{document}
