\documentclass[10pt,a4paper]{report}
\usepackage[utf8]{inputenc}
\usepackage{amsmath}
\usepackage{amsthm}
\usepackage{amsfonts}
\usepackage{amssymb}
\usepackage{tikz-cd}
\usepackage{calc}
\usepackage{setspace}
\usepackage[german]{babel}
\usetikzlibrary{babel}
\usepackage{cleveref}

\newcommand{\comment}[1]{}
\renewcommand{\baselinestretch}{1.1}

\newcommand{\ModulsOfDifferenzials}{Kommutativ Algebra with a view Torwards Algebraic Geometrie [David Eisenbud 1994]}

\newcounter{Aussage}[chapter]

\newtheorem{satz}[Aussage]{Satz}
\newtheorem{theorem}[Aussage]{Theorem}
\newtheorem{prop}[Aussage]{Proposition}
\newtheorem{korrolar}[Aussage]{Korrolar}
\newtheorem{lemma}[Aussage]{Lemma}
\newtheorem{bem}[Aussage]{Bemerkung}
\newtheorem{definition}[Aussage]{Definition}
\newtheorem{bsp}[Aussage]{Beispiel}

\newcommand{\functionfront}[3]{\nolinebreak{#1:#2 \longrightarrow #3}}
\newcommand{\functionback}[3]{\nolinebreak{#1:#2 \longmapsto #3}}
\newcommand{\function}[5]{\nolinebreak{#1:#2 \longrightarrow #3 \, , \, #4 \longmapsto #5}}
\newcommand{\infunctionfront}[3]{\nolinebreak{#1:#2 \hookrightarrow #3}}
\newcommand{\divR}[2]{\Omega_{#1/#2}}
\newcommand{\divf}[1]{d_{#1}}
\newcommand{\Tensor}[3]{#1 \otimes_{#2} #3}
\newcommand{\tensor}[3]{#1 \otimes #3}
\newcommand{\lok}[2]{#1 [#2^{-1}]}
\newcommand{\loke}[3]{(\frac{#1}{#2})_{_{#3}}}
\comment{\newcommand{\loke}[3]{(#1,#2)_{mod\sim_{#3}}}}

\newcommand{\colimes}[0]{\lim\limits_{ \longrightarrow }}
\newcommand*{\defeq}{\mathrel{\vcenter{\baselineskip0.5ex \lineskiplimit0pt
                     \hbox{\scriptsize.}\hbox{\scriptsize.}}}%
                     =}
\newcommand*{\defeqr}{= \mathrel{\vcenter{\baselineskip0.5ex \lineskiplimit0pt
                     \hbox{\scriptsize.}\hbox{\scriptsize.}}}}

\newcommand*{\defshow}{\stackrel{!}{=}}
\newcommand{\kernel}[1]{kern(#1)}
\newcommand{\immage}[1]{im(#1)}

\begin{document}
\begin{bsp}
Sei $k$ ein Körper und $K = k(\lbrace x_i \rbrace_{i \in \lbrace 1,\dots,n \rbrace})$ der Körper der rationalen Funktionen in $n$ Varablen über $k$.\\
Dann ist $\lbrace x_i \rbrace_{i \in \lbrace 1,\dots,n \rbrace}$ eine Differenzialbasis von $\divR{K}{k}$.
\end{bsp}
\begin{proof}
Sehe $K = \lok{k[x_1,\dots,x_n]}{k[x_1,\dots,x_n]}$ als Lokalisierung. Somit gilt nach LOKALISIERUNG und POLYNOMRING:
\begin{gather*}
\divR{K}{k} \simeq \tensor{K}{k[x_1,\dots,x_n]}{\divR{k[x_1,\dots,x_n]}{k}} \simeq \tensor{K}{k[x_1,\dots,x_n]}{\oplus_{i \in \lbrace 1,\dots,n \rbrace} k[x_1,\dots,x_n]\langle \divf{k[x_1,\dots x_n]}(x_i) \rangle} \simeq K\langle \divf{k[x_1,\dots x_n]}(x_i) \rangle
\end{gather*}
Somit ist $\lbrace x_i \rbrace_{i \in \lbrace 1,\dots,n \rbrace}$ ein Erzeugenden-System von $\divR{K}{k}$.
\end{proof}
\begin{lemma}
Sei $R \longrightarrow S \subset T$ ein Ringhomomorphismus und $S \subset T$ eine seperable und algebraische Körpererweiterung. Dann gilt:
\begin{gather*}
\divR{T}{R} = \tensor{}
\end{gather*}
\end{lemma}
\end{document}