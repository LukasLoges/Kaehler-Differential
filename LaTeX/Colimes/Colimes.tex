\documentclass[10pt,a4paper]{report}
\usepackage[utf8]{inputenc}
\usepackage{amsmath}
\usepackage{amsthm}
\usepackage{amsfonts}
\usepackage{amssymb}
\usepackage{color}
\usepackage{tikz-cd}
\usepackage{calc}
\usepackage{setspace}
\usepackage[german]{babel}
\usetikzlibrary{babel}
\usepackage{cleveref}

\newcommand{\comment}[1]{}
\renewcommand{\baselinestretch}{1.1}

\newcommand{\ModulsOfDifferenzials}{David Eisenbud 1994}

\newcounter{Aussage}[chapter]

\newtheorem{satz}[Aussage]{Satz}
\newtheorem{theorem}[Aussage]{Theorem}
\newtheorem{prop}[Aussage]{Proposition}
\newtheorem{korrolar}[Aussage]{Korrolar}
\newtheorem{lemma}[Aussage]{Lemma}
\newtheorem{bem}[Aussage]{Bemerkung}
\newtheorem{definition}[Aussage]{Definition}
\newtheorem{bsp}[Aussage]{Beispiel}

\newcommand{\functionfront}[3]{\nolinebreak{#1:#2 \longrightarrow #3}}
\newcommand{\functionback}[3]{\nolinebreak{#1:#2 \longmapsto #3}}
\newcommand{\function}[5]{\nolinebreak{#1:#2 \longrightarrow #3 \, , \, #4 \longmapsto #5}}
\newcommand{\infunctionfront}[3]{\nolinebreak{#1:#2 \hookrightarrow #3}}
\newcommand{\divR}[2]{\Omega_{#1/#2}}
\newcommand{\divf}[1]{d_{#1}}
\comment{\newcommand{\divf}[2][]{d_{#1}}}
\newcommand{\Tensor}[3]{#1 \otimes_{#2} #3}
\newcommand{\tensor}[3]{#1 \otimes #3}
\newcommand{\lok}[2]{#1 [#2^{-1}]}
\newcommand{\loke}[3]{(\frac{#1}{#2})_{_{#3}}}
\comment{\newcommand{\loke}[3]{(#1,#2)_{mod\sim_{#3}}}}

\newcommand{\colimes}[0]{\lim\limits_{ \longrightarrow }}
\newcommand*{\defeq}{\mathrel{\vcenter{\baselineskip0.5ex \lineskiplimit0pt
                     \hbox{\scriptsize.}\hbox{\scriptsize.}}}%
                     =}
\newcommand*{\defeqr}{= \mathrel{\vcenter{\baselineskip0.5ex \lineskiplimit0pt
                     \hbox{\scriptsize.}\hbox{\scriptsize.}}}}

\newcommand*{\defshow}{\stackrel{!}{=}}
\newcommand{\kernel}[1]{kern(#1)}
\newcommand{\immage}[1]{im(#1)}
\newcommand{\Verz}[1]{\langle #1 \rangle}


\begin{document}
\begin{satz}\label{Differenzial des Kolimes von R-Algebren}
Differenzial des Kolimes von R-Algebren
\end{satz}

\begin{satz}\label{Differential von rationalen Funktionen 1}
Differential von rationalen Funktionen 1
\end{satz}


\ \\
\textcolor{blue}{\textbf{seperabel generierte Koerpererweiterung mit DifR(T)(R) ist 0} \textit{[Aufgabe 16.10 \ModulsOfDifferenzials (steht im Bezug zu Korrolar 16.17)]}}
\begin{bsp}\comment{\label{unendliche, seperabel generierte Koerpererweiterung mit DifR(T)(R) ist 0}}
Sei $k$ ein Körper mit $char(k) = p > 0$ und sei weiter $K(x)$ der Raum der Rationalen Funktionen über k.
\begin{gather*}
\textit{Definiere: } L \defeq k(x^{1/{p^{\infty}}}) = \colimes \lbrace k(x^{1/{p^{n}}}) \vert n \in \mathbb{N} \rbrace
\end{gather*}
Dann gilt : $\divR{L}{k} = 0$\\
\textcolor{red}{Prüfe noch, ob $L \supset k$ eine seperabel generierte Körpererweiterung \comment{\label{*Def seperabel generierte Körpererweiterung}} ist.}
\end{bsp}
\begin{proof}\comment{Skizzenhaft}
Es gilt:
\begin{gather*}
\divf{L}(x^{1/{p^{n}}})
= \divf{L} \left( \prod_{i \in \lbrace 1,\dots,p \rbrace} x^{1/{p^{n+1}}}\right)
=  p \cdot \left( \prod_{i \in \lbrace 1,\dots,p-1 \rbrace} x^{1/{p^{n+1}}} \right) \cdot \divf{L}(x^{1/{p^{n+1}}})
= 0
\end{gather*}
Nute noch \cref{Differenzial des Kolimes von R-Algebren} und \cref{Differential von rationalen Funktionen 1}, um zu folgern, dass $\divR{L}{k}$ von \\
$\lbrace \divf{L}(x^{1/{p^{n}}}) \vert n \in \mathbb{N} \rbrace$ erzeugt wird.
\end{proof}
\end{document}
