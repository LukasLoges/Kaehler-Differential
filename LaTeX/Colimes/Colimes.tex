\documentclass[10pt,a4paper]{report}
\usepackage[utf8]{inputenc}
\usepackage{amsmath}
\usepackage{amsthm}
\usepackage{amsfonts}
\usepackage{amssymb}
\usepackage{tikz-cd}
\usepackage{calc}
\usepackage{setspace}
\usepackage[german]{babel}
\usetikzlibrary{babel}
\usepackage{cleveref}

\newcommand{\comment}[1]{}
\renewcommand{\baselinestretch}{1.1}

\newcommand{\ModulsOfDifferenzials}{Kommutativ Algebra with a view Torwards Algebraic Geometrie [David Eisenbud 1994]}

\newcounter{Aussage}[chapter]

\newtheorem{satz}[Aussage]{Satz}
\newtheorem{theorem}[Aussage]{Theorem}
\newtheorem{prop}[Aussage]{Proposition}
\newtheorem{korrolar}[Aussage]{Korrolar}
\newtheorem{lemma}[Aussage]{Lemma}
\newtheorem{bem}[Aussage]{Bemerkung}
\newtheorem{definition}[Aussage]{Definition}

\newcommand{\functionfront}[3]{\nolinebreak{#1:#2 \longrightarrow #3}}
\newcommand{\functionback}[3]{\nolinebreak{#1:#2 \longmapsto #3}}
\newcommand{\function}[5]{\nolinebreak{#1:#2 \longrightarrow #3 \, , \, #4 \longmapsto #5}}
\newcommand{\infunctionfront}[3]{\nolinebreak{#1:#2 \hookrightarrow #3}}
\newcommand{\divR}[2]{\Omega_{#1/#2}}
\newcommand{\divf}[1]{d_{#1}}
\newcommand{\Tensor}[3]{#1 \otimes_{#2} #3}
\newcommand{\tensor}[3]{#1 \otimes #3}
\newcommand{\lok}[2]{#1 [#2^{-1}]}
\newcommand{\loke}[3]{(\frac{#1}{#2})_{_{#3}}}
\comment{\newcommand{\loke}[3]{(#1,#2)_{mod\sim_{#3}}}}

\newcommand{\colimes}[0]{\lim\limits_{ \longrightarrow }}
\newcommand*{\defeq}{\mathrel{\vcenter{\baselineskip0.5ex \lineskiplimit0pt
                     \hbox{\scriptsize.}\hbox{\scriptsize.}}}%
                     =}
\newcommand*{\defeqr}{= \mathrel{\vcenter{\baselineskip0.5ex \lineskiplimit0pt
                     \hbox{\scriptsize.}\hbox{\scriptsize.}}}}

\newcommand*{\defshow}{\stackrel{!}{=}}
\newcommand{\kernel}[1]{kern(#1)}
\newcommand{\immage}[1]{im(#1)}

\begin{document}
\begin{bem}\label{Darstellung der Polynomalgebra als Tensorprodukt}
Die Polynomalgebra $R[x_1,...,x_d]$ über R lässt sich wie folgt als Tensorprodukt darstellen:
\begin{gather*}
R[x_1,...,x_n] = \bigotimes_{i \in \lbrace 1,...,n \rbrace} R[x_i]
\end{gather*}
Genauer gilt für zwei Polynomalgebren $A = R[x_1,...,x_{n_A}], \, B = R[y_1,...,y_{n_B}]$ über R:
\begin{gather*}
\Tensor{A}{R}{B}  = R[x_1,...,x_{n_A},y_1,...,y_{n_B}]
\end{gather*}
\end{bem}
Skizziere den Beweis.
\begin{proof}
Zeige, dass für $\function{g}{A \oplus B}{R[x_1,...,x_{n_A},y_1,...,y_{n_B}]}{(a,b)}{a \cdot b}$ die Universelle Eigenschaft des Tensorproduktes gilt:
\begin{center}
\begin{tikzcd}
A \oplus B \arrow[r, "g"] \arrow[rd, "f"] & {R[x_1,...,x_{n_A},y_1,...,y_{n_B}]} \arrow[d, "\exists ! \varphi", dashed] \\
                                          & M                                                                  
\end{tikzcd}
\end{center}
Es ist leicht nachzurechnen, dass es sich bei $\varphi$ um folgende Funktion handeln muss:
\begin{gather*}
\function{\varphi}{R[x_1,...,x_{n_A},y_1,...,y_{n_B}]}{M}{(x_i \cdot y_j)}{f(x_i,1) \cdot f(1,y_i)}
\end{gather*}
\end{proof}

\begin{korrolar}\comment{\label{Differenzial von Polynomalgebren}}
Sei $S = R[x_1,...,x_n]$ eine Polynomalgebra über R. Dann gilt:
\begin{gather*}
\divR{S}{R} = \bigoplus_{i \in \lbrace 1,...,n \rbrace} S \langle \divf{S}(x_i) \rangle 
\end{gather*}
Wobei $S\langle \divf{S}(x_i)\rangle$ das von $\divf{S}(x_i)$ erzeugt Modul über S ist.
\end{korrolar}
\begin{proof}
Wie in \cref{Darstellung der Polynomalgebra als Tensorprodukt} gezeigt, können wir $S$ als $\bigotimes_{i \in \lbrace 1,...,n \rbrace} R[x_i]$ schreiben. In \textbf{Kählerdifferenzial des Kolimes von R-Algebren} \comment{\cref{Kählerdifferenzial des Kolimes von R-Algebren}} haben wir gezeigt, wie das Differenzial eines solchen Tensorproduktes aussieht:
\begin{gather*}
\divR{S}{R} = \bigoplus_{i \in \lbrace 1,...,n \rbrace} (\Tensor{S}{R[x_i]}{\divR{R[x_i]}{R}})
\end{gather*}
Da $R[x_i]$ die aus dem Element $x_i$ erzeugte Algebra über $R$ ist, folgt \textit{[vlg. BEMERKUNG ZU ENDLICH ERZEUGTEN ALGEBREN]}: 
\begin{gather*}
\divR{S}{R} = \bigoplus_{i \in \lbrace 1,...,n \rbrace} (\Tensor{S}{R[x_i]}{R[x_i]\langle \divf{S[x_i]}(x_i) \rangle})
\simeq \bigoplus_{i \in \lbrace 1,...,n \rbrace} S \langle \divf{S}(x_i) \rangle 
\end{gather*}
Für die letzte Isomorphie nutze, dass wegen $R[x_i] \subseteq S$ zum Einen $\divf{R[x_i]}$ als Einschränkung von $\divf{S}$ gesehen werden kann und zum Anderen $\Tensor{S}{R[x_i]}{R[x_i]} \simeq S$ gilt.
\end{proof}
\end{document}