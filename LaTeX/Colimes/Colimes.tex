\documentclass[10pt,a4paper]{report}
\usepackage[utf8]{inputenc}
\usepackage{amsmath}
\usepackage{amsthm}
\usepackage{amsfonts}
\usepackage{amssymb}
\usepackage{color}
\usepackage{tikz-cd}
\usepackage{calc}
\usepackage{setspace}
\usepackage[german]{babel}
\usetikzlibrary{babel}
\usepackage{cleveref}

\newcommand{\comment}[1]{}
\renewcommand{\baselinestretch}{1.1}

\newcommand{\ModulsOfDifferenzials}{David Eisenbud 1994}
\newcommand{\Algebra}{Christian Karpfinger, Kurt Meyberg 2009}

\newcounter{Aussage}[chapter]

\newtheorem{satz}[Aussage]{Satz}
\newtheorem{theorem}[Aussage]{Theorem}
\newtheorem{prop}[Aussage]{Proposition}
\newtheorem{korrolar}[Aussage]{Korrolar}
\newtheorem{lemma}[Aussage]{Lemma}
\newtheorem{bem}[Aussage]{Bemerkung}
\newtheorem{definition}[Aussage]{Definition}
\newtheorem{bsp}[Aussage]{Beispiel}

\newcommand{\functionfront}[3]{\nolinebreak{#1:#2 \longrightarrow #3}}
\newcommand{\functionback}[3]{\nolinebreak{#1:#2 \longmapsto #3}}
\newcommand{\function}[5]{\nolinebreak{#1:#2 \longrightarrow #3 \, , \, #4 \longmapsto #5}}
\newcommand{\infunctionfront}[3]{\nolinebreak{#1:#2 \hookrightarrow #3}}
\newcommand{\divR}[2]{\Omega_{#1/#2}}
\newcommand{\divf}[1]{d_{#1}}
\comment{\newcommand{\divf}[2][]{d_{#1}}}
\newcommand{\Tensor}[3]{#1 \otimes_{#2} #3}
\newcommand{\tensor}[3]{#1 \otimes #3}
\newcommand{\lok}[2]{#1 [#2^{-1}]}
\newcommand{\loke}[3]{(\frac{#1}{#2})_{_{#3}}}
\comment{\newcommand{\loke}[3]{(#1,#2)_{mod\sim_{#3}}}}

\newcommand{\colimes}[0]{\lim\limits_{ \longrightarrow }}
\newcommand*{\defeq}{\mathrel{\vcenter{\baselineskip0.5ex \lineskiplimit0pt
                     \hbox{\scriptsize.}\hbox{\scriptsize.}}}%
                     =}
\newcommand*{\defeqr}{= \mathrel{\vcenter{\baselineskip0.5ex \lineskiplimit0pt
                     \hbox{\scriptsize.}\hbox{\scriptsize.}}}}

\newcommand*{\defshow}{\stackrel{!}{=}}
\newcommand{\kernel}[1]{kern(#1)}
\newcommand{\immage}[1]{im(#1)}
\newcommand{\Verz}[1]{\langle #1 \rangle}


\begin{document}
\textcolor{blue}{Definition Leibnizregel}
\begin{definition}\label{Definition Leibnizregel} \textit{[Kapitel 16 \ModulsOfDifferenzials]}\\
Sei S ein Ring und M ein S-Modul
\begin{itemize}
\item[]Ein Homomoprphismus abelscher Gruppen $\functionfront{d}{S}{M}$ ist eine \underline{Ableitung}, falls gilt:
\begin{gather*}
\forall s_1,s_2 \in S :\, d(s_1 \cdot s_2) = s_1d(s_2) + s_2d(s_1) \textbf{ (Leibnitzregel)}
\end{gather*}
\item[]Sei S eine R-Algebra, dann nennen wir eine \underline{Ableitung} $\functionfront{d}{S}{M}$ \underline{$R$-linear}, falls sie zusätzlich ein R-Modulhomomorphismus ist, also falls gilt:
\begin{gather*}
\forall r_1,r_2 \in R \, \forall s_1,s_2 \in S : \, d(r_1 s_1 + r_2 s_2) = r_1 d(s_1) + r_2 d(s_2)
\end{gather*}
\end{itemize}
\end{definition}


\ \\
\textcolor{blue}{\textbf{Differenzial indempotenter Elemente}}
\begin{lemma}\label{Differenzial indempotenter Elemente} \textit{[Aufgabe 16.1 \ModulsOfDifferenzials]} \\
Sei S ein Ring, $M$ ein $S$-Modul und $\functionfront{d}{S}{M}$ eine Ableitung. Sei weiter $a \in S$ ein indempotentes Element \textcolor{red}{($a^2 = a$)}.
\begin{center}
Dann gilt $d(a) = 0$. 
\end{center}
Insbesondere gilt somit auch $d(1) = 0$.
\end{lemma}
\begin{proof}
Nutze hierfür allein die Leibnizregel \textit{(cref{Definition Leibnizregel})}
\begin{gather*}
\text{Schritt 1: } \divf{S}(a) = \divf{S}(a^2) = a\divf{S}(a) + a\divf{S}(a) \\
\text{Schritt 2: } a\divf{S}(a) = a\divf{S}(a^2) = a^2\divf{S}(a) + a^2\divf{S}(a) = a\divf{S}(a) + a\divf{S}(a)\\
\Rightarrow \divf{S}(a) = a\divf{S}(a) = 0
\end{gather*}
\end{proof}


\ \\
\begin{definition}
Sei $S$ eine $R$-Algebra.\\
Das $S$-Modul $\divR{S}{R}$ der Kähler-Differenziale von $S$ über $R$ und die dazugehörige universelle $R$-lineare Ableitung$\functionfront{\divf{S}}{S}{\divR{S}{R}}$ sind durch die folgende Universelle Eigenschaft definiert
\end{definition}


\ \\
\textcolor{blue}{\textbf{Propositon 11 delta}}
\begin{lemma}\label{Propositon 11 delta} \textit{[Lemma 16.11 \ModulsOfDifferenzials]} \\
Seien $S$, $S'$ zwei $R$-Algebren. Sei weiter $\functionfront{f}{S}{S'}$ ein $R$-Algebrenhomomorphismus und $\functionfront{\delta}{S}{S'}$ ein Homomorphismus abelscher Gruppen mit $\delta(S)^2 = 0$. Dann gilt:
\begin{center}
$f$ + $\delta$ ist ein $R$-Algebrenhomomorphismus\\
$\Leftrightarrow$\\
$\delta$ ist eine $R$-linear und es gilt $\forall s_1,s_2 \in S :\, \delta(s_1 \cdot s_2) = f(s_1)\delta(s_2) + f(s_2)\delta(s_1)$.
\end{center}
\end{lemma}
\begin{proof} \ \\
\begin{itemize}
\item[\underline{\glqq $\Rightarrow$ \grqq :}] Da $f$ und $f$ + $\delta$ $R$-linear sind, ist auch $\delta = (f + \delta) - f$ $R$-linear.\\
Seien nun $s_1,s_2 \in S$ beliebig, somit gilt:
\begin{gather*}
(f + \delta)(s_1 \cdot s_2) = (f + \delta)(s_1) \cdot (f + \delta)(s_2)\\
\Rightarrow f(s_1 \cdot s_2) + \delta(s_1 \cdot s_2) = f(s_1)f(s_2) + f(s_1)\delta(s_2) + f(s_2)\delta(s_1) + \delta(s_1)\delta(s_2)\\
\Rightarrow \delta(s_1 \cdot s_2) = f(s_1)\delta(s_2) + f(s_2)\delta(s_1) + \delta(s_1)\delta(s_2) \textit{ mit } \delta(s_1)\delta(s_2) \in \delta(S)^2 = 0 \\
\Rightarrow \delta(s_1 \cdot s_2) = f(s_1)\delta(s_2) + f(s_2)\delta(s_1) + \delta(s_1)\delta(s_2)
\end{gather*}
\item[\underline{\glqq $\Leftarrow$ \grqq :}]
Da $f$ und $\delta$ beide $R$-lineare Homomorphismen abelscher Gruppen sind, trifft die auch für $f + \delta$ zu.\\
Wähle nun also $s_1,s_2 \in S$ beliebig, somit gilt:
\begin{gather*}
(f + \delta)(s_1) \cdot (f + \delta)(s_2) \\
= f(s_1)f(s_2) + f(s_1)\delta(s_2) + f(s_2)\delta(s_1) + \delta(s_1)\delta(s_2)\\
= f(s_1 \cdot s_2) + \delta(s_1 \cdot s_2)
= (f + \delta)(s_1 \cdot s_2)
\end{gather*}
Damit haben wir gezeigt, dass $f + \delta$ ein $R$-Algebrenhomomorphismus ist.
\end{itemize}
\end{proof}

\end{document}