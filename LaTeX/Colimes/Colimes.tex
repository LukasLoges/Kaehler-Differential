\documentclass[10pt,a4paper]{report}
\usepackage[utf8]{inputenc}
\usepackage{amsmath}
\usepackage{amsthm}
\usepackage{amsfonts}
\usepackage{amssymb}
\usepackage{tikz-cd}
\usepackage{calc}
\usepackage{setspace}
\usepackage[german]{babel}
\usetikzlibrary{babel}
\usepackage{cleveref}

\newcommand{\comment}[1]{}
\renewcommand{\baselinestretch}{1.1}


\newcounter{Aussage}[chapter]

\newtheorem{satz}[Aussage]{Satz}
\newtheorem{theorem}[Aussage]{Theorem}
\newtheorem{prop}[Aussage]{Proposition}
\newtheorem{korrolar}[Aussage]{Korrolar}
\newtheorem{lemma}[Aussage]{Lemma}
\newtheorem{bem}[Aussage]{Bemerkung}
\newtheorem{definition}[Aussage]{Definition}

\newcommand{\functionfront}[3]{\nolinebreak{#1:#2 \longrightarrow #3}}
\newcommand{\functionback}[3]{\nolinebreak{#1:#2 \longmapsto #3}}
\newcommand{\function}[5]{\nolinebreak{#1:#2 \longrightarrow #3 \, , \, #4 \longmapsto #5}}
\newcommand{\divR}[2]{\Omega_{#1/#2}}
\newcommand{\Tensor}[3]{#1 \otimes_{#2} #3}
\newcommand{\tensor}[3]{#1 \otimes #3}

\newcommand{\colimes}[0]{\lim\limits_{ \longrightarrow }}
\newcommand{\infunctionfront}[3]{\nolinebreak{#1:#2 \hookrightarrow #3}}
\newcommand*{\defeq}{\mathrel{\vcenter{\baselineskip0.5ex \lineskiplimit0pt
                     \hbox{\scriptsize.}\hbox{\scriptsize.}}}%
                     =}
\newcommand*{\defshow}{\stackrel{!}{=}}
\newcommand{\kernel}[1]{kern(#1)}
\newcommand{\immage}[1]{im(#1)}

\begin{document}
\comment{Beide Beweise sind sehr kurz gefasst}
\begin{prop}
\ \\
\begin{itemize}
\item[\textbf{1.}]
Sei $T = \otimes_{i \in \Lambda} S_i$ das Coprodukt der R-Algebren $S_i$.\\
Dann gilt:
\begin{gather*}
\divR{T}{R} \simeq \bigoplus_{i\in \Lambda} ( \Tensor{T}{S_i}{\divR{S_i}{R}} )
\end{gather*}
\item[\textbf{2.}]
Seien $S_1,S_2$ R-Algebren und $\functionfront{\varphi,\varphi'}{S_1}{S_2}$ R-Algebra-Homomorphismen. Sei weiter $\functionfront{q}{S_2}{T}$ der Differenzenkokern von $\varphi$,$\varphi '$.
Dann ist folgende Sequenz rechtsexakt:\\
\begin{center}
\begin{tikzcd}
\Tensor{T}{S_1}{\divR{S_1}{R}} \arrow[r, "f"] & \Tensor{T}{S_2}{\divR{S_2}{R}} \arrow[r, "g"] & \divR{T}{R} \arrow[r] & 0
\end{tikzcd}
\begin{gather*}
\textit{mit: } \function{f}{\tensor{T}{S_1}{\divR{S_1}{R}}}{\Tensor{T}{S_2}{\divR{S_2}{R}}}{\tensor{t}{S_2}{d_{S_1}(x_1)}}{\tensor{t}{S_2}{d_{S_2}(\varphi(x_1) - \varphi(x_2))}}\\
\function{g}{\Tensor{T}{S_2}{\divR{S_2}{R}}}{\divR{T}{R}}{\tensor{t}{S_2}{d_{S_2}(x_2)}}{(d_{S_2}\circ q)(x_2)}
\end{gather*}
\end{center}
\end{itemize}
\end{prop}
\begin{proof}
Für \textit{\textbf{1.}} Finde durch die Universelle Eigenschaft des Kählerdifferenzials Isomorphismen $ \divR{T}{R} \longleftrightarrow \bigoplus_{i \in \Lambda} ( \Tensor{T}{S_i}{\divR{S_i}{R}} )$.\\
Definiere das Differenzial $\function{e}{T}{\sum_{i \in \Lambda} \Tensor{T}{S_i}{\divR{S_i}{R}}}{(\tensor{s_i}{R}{...})}{(\tensor{1}{S_i}{d_{S_1},...)}}$ und erhalte dadurch
\begin{center}
\begin{tikzcd}
T \arrow[rd, "e"'] \arrow[r, "d_T"] & \divR{T}{R} \arrow[d, "\exists ! \varphi", dashed] \\
                                    & \sum_{i\in \Lambda} \Tensor{T}{S_i}{\divR{S_i}{R}}                                       
\end{tikzcd}
$\functionfront{\varphi}{\divR{T}{R}}{\bigoplus_{i\in \Lambda} ( \Tensor{T}{S_i}{\divR{S_i}{R}} )}$.
\end{center}
Definiere nun das Differenzial $k: S_i \hookrightarrow T \longrightarrow \divR{T}{R}$ und erhalte dadurch
\begin{center}
\begin{tikzcd}
S_i \arrow[rd, "k"'] \arrow[r, "d_{S_i}"] & \divR{S_i}{R} \arrow[d, "\exists ! k'", dashed] \arrow[r, "a"] & \Tensor{T}{S_i}{\divR{S_i}{R}} \arrow[ld, "\phi_i"] \\
                                          & \divR{T}{R}                                                    &                     
\end{tikzcd}
$\functionfront{\phi_i}{\bigoplus_{i\in \Lambda} ( \Tensor{T}{S_i}{\divR{S_i}{R}} )}{\divR{T}{R}}$\\
\begin{gather*}
\function{\phi}{\sum_{i\in \Lambda} ( \Tensor{T}{S_i}{\divR{S_i}{R}})}{\divR{T}{R}}{(\tensor{t_1}{S_1}{d_{S_i}s_1},...)}{\prod_{i\in \Lambda} t_i \cdot \phi_i(d_{S_i}(s_i))}. 
\end{gather*}
\end{center}
Damit haben wir zwei zueinander inverse Funktionen $\varphi ,\phi$ gefunden.\\
$\Rightarrow \divR{T}{R} \simeq \bigoplus_{i\in \Lambda} ( \Tensor{T}{S_i}{\divR{S_i}{R}} )$\\
\ \\
Für \textit{\textbf{2.}} Wende \cref{prop16.3} auf den Differenzenkokern $\functionfront{q}{S_2}{S_2/Q}$ an und erhalte dadurch eine exakte Sequenz, welche ähnlich zu der gesuchten ist:
\begin{center}
\begin{tikzcd}
Q/Q^2 \arrow[r, "f'"] & \tensor{T}{S_2}{\divR{S_2}{R}} \arrow[r, "g"] & \divR{T}{R} \arrow[r] & 0
\end{tikzcd}
\end{center}
$mit: \function{f'}{Q/Q^2}{{\Tensor{T}{S_2}{\divR{S}{R}}}}{[s_2]_{Q^2}}{\tensor{1}{S_2}{d_{S_2}(s_2)}}$\\
Somit gilt $\immage{f} = \Tensor{T}{S_2}{d_{S_2}(Q)} = \immage{f'}$.\\
$\Rightarrow$ die gesuchte Sequenz ist exakt.
\end{proof}


\comment
{
\begin{prop}
Seien $S_1,S_2$ R-Algebra`s und $\functionfront{\varphi,\varphi'}{S_1}{S_2}$ R-Algebra-Homomorphismen. Sei weiter $\functionfront{q}{S_2}{T}$ der Differenzenkokern von $\varphi$,$\varphi '$.
Dann ist folgende Sequenz rechtsexakt:\\
\begin{center}
\begin{tikzcd}
\Tensor{T}{S_1}{\divR{S_1}{R}} \arrow[r, "F"] & \Tensor{T}{S_2}{\divR{S_2}{R}} \arrow[r, "G"] & \divR{T}{R} \arrow[r] & 0
\end{tikzcd}
\end{center}
\begin{spacing}{1.5}
mit: $\function{F}{\Tensor{T}{S_1}{\divR{S_1}{R}}}{\Tensor{T}{S_2}{\divR{S_2}{R}}}{\tensor{t}{S_2}{d_{S_1}(x_1)}}{\tensor{t}{S_2}{d_{S_2}(\varphi(x_1) - \varphi(x_2))}}$\\
$\function{G}{\Tensor{T}{S_2}{\divR{S_2}{R}}}{\divR{T}{R}}{\tensor{t}{S_2}{d_{S_2}(x_2)}}{(d_{S_2}\circ q)(x_2)}$
\end{spacing}
\end{prop}
Beweis mit PROP 16.3
\begin{proof}
[Falls $d$ surjektiv ist:] nach PROP 16.3 ist folgende Sequenz rechtsexakt:
\begin{center}
\begin{tikzcd}
Q/Q^2 \arrow[r, "f'"] & \tensor{T}{S_2}{\divR{S_2}{R}} \arrow[r, "g"] & \divR{T}{R} \arrow[r] & 0
\end{tikzcd}
\end{center}
mit: $\function{f'}{Q/Q^2}{{\Tensor{T}{S_2}{\divR{S}{R}}}}{[s_2]_{Q^2}}{\tensor{1}{S_2}{d_{S_2}(s_2)}}$\\
Zeige also noch$immage{f} = Immage{f'}$:
\end{proof}
}
\end{document}
