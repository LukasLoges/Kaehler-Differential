\documentclass[10pt,a4paper]{report}
\usepackage[utf8]{inputenc}
\usepackage{amsmath}
\usepackage{amsthm}
\usepackage{amsfonts}
\usepackage{amssymb}
\usepackage{color}
\usepackage{tikz-cd}
\usepackage{calc}
\usepackage{setspace}
\usepackage[german]{babel}
\usetikzlibrary{babel}
\usepackage{cleveref}

\newcommand{\comment}[1]{}
\renewcommand{\baselinestretch}{1.1}

\newcommand{\ModulsOfDifferenzials}{David Eisenbud 1994}

\newcounter{Aussage}[chapter]

\newtheorem{satz}[Aussage]{Satz}
\newtheorem{theorem}[Aussage]{Theorem}
\newtheorem{prop}[Aussage]{Proposition}
\newtheorem{korrolar}[Aussage]{Korrolar}
\newtheorem{lemma}[Aussage]{Lemma}
\newtheorem{bem}[Aussage]{Bemerkung}
\newtheorem{definition}[Aussage]{Definition}
\newtheorem{bsp}[Aussage]{Beispiel}

\newcommand{\functionfront}[3]{\nolinebreak{#1:#2 \longrightarrow #3}}
\newcommand{\functionback}[3]{\nolinebreak{#1:#2 \longmapsto #3}}
\newcommand{\function}[5]{\nolinebreak{#1:#2 \longrightarrow #3 \, , \, #4 \longmapsto #5}}
\newcommand{\infunctionfront}[3]{\nolinebreak{#1:#2 \hookrightarrow #3}}
\newcommand{\divR}[2]{\Omega_{#1/#2}}
\newcommand{\divf}[1]{d_{#1}}
\comment{\newcommand{\divf}[2][]{d_{#1}}}
\newcommand{\Tensor}[3]{#1 \otimes_{#2} #3}
\newcommand{\tensor}[3]{#1 \otimes #3}
\newcommand{\lok}[2]{#1 [#2^{-1}]}
\newcommand{\loke}[3]{(\frac{#1}{#2})_{_{#3}}}
\comment{\newcommand{\loke}[3]{(#1,#2)_{mod\sim_{#3}}}}

\newcommand{\colimes}[0]{\lim\limits_{ \longrightarrow }}
\newcommand*{\defeq}{\mathrel{\vcenter{\baselineskip0.5ex \lineskiplimit0pt
                     \hbox{\scriptsize.}\hbox{\scriptsize.}}}%
                     =}
\newcommand*{\defeqr}{= \mathrel{\vcenter{\baselineskip0.5ex \lineskiplimit0pt
                     \hbox{\scriptsize.}\hbox{\scriptsize.}}}}

\newcommand*{\defshow}{\stackrel{!}{=}}
\newcommand{\kernel}[1]{kern(#1)}
\newcommand{\immage}[1]{im(#1)}
\newcommand{\Verz}[1]{\langle #1 \rangle}


\begin{document}
\begin{satz}\label{Cotangent Sequenz}
Cotangent Sequenz
\end{satz}

\begin{satz}\label{Konormale Sequenz}
Konormale Sequenz
\end{satz}

\begin{satz}\label{Differenzial von Polynomalgebren 2}
Differenzial von Polynomalgebren 2
\end{satz}

\begin{satz}\label{Differenzial der Lokalisierung}
Differenzial der Lokalisierung
\end{satz}

\begin{satz}\label{Differential von rationalen Funktionen 1}
Differential von rationalen Funktionen 1
\end{satz}


\ \\
\textcolor{blue}{\textbf{Cotangent Sequenz von Körpern 3} \textit{[Aufgabe 16.6 b) \ModulsOfDifferenzials]}}
\begin{bsp}\comment{\label{Cotangent Sequenz von Körpern 3}}
Seien $T \supset L \supset k$ endliche Körpererweiterungen. Betrachte die COTANGENT SEQUENZ \textit{(\cref{Cotangent Sequenz})} von $k \hookrightarrow L \hookrightarrow T$:
\begin{center}
\begin{tikzcd}
\Tensor{T}{L}{\divR{L}{k}} \arrow[r, "\varphi"] & \divR{T}{k} \arrow[r] & \divR{T}{L} \arrow[r] & 0
\end{tikzcd}
\ \\
\end{center}
Sei weiter die Körpererweiterung $T \supset L$ algebraisch und pur inseperabel.
\begin{gather*}
\textcolor{green}{ \textit{Eine Körpererweiterung heißt pur inseperabel, falls gilt:} }\\
\textcolor{green}{ char(L) = p > 0 \textit{ und } \forall t \in T \exists l \in  L \exists n \in \mathbb{N} : t^{p^n} = l }
\end{gather*}
Existiere weiter ein $\alpha \in T$ mit $L(\alpha) = T$ und $Mipo(\alpha) = f(x) = x^p - a$.\\
Dann gilt:
\begin{center}
$\varphi$ ist injektiv $\Leftrightarrow$ $\divf{L}(a) = 0$
\end{center}
\end{bsp}


\begin{proof}
\ \\
Betrachte die Konormale Sequenz \textit{(\cref{Konormale Sequenz})} von $\function{\pi}{L[x]}{L{x}/(f(x)) \simeq T}{P(x)}{[P(x)]}$ \textbf{(1)}. Forme diese leicht um \textbf{(2)}, sodass wir sie mit der COTANGENT SEQUENZ von $k \hookrightarrow L \hookrightarrow T$ \textbf{(3)} vergleichen können:
\begin{center}
\begin{tikzcd}
(f(x))/(f(x)^2) \arrow[r, "\tensor{1}{L[x]}{\divf{L[x]}}"] & \Tensor{T}{L[x]}{\divR{L[x]}{k}} \arrow[r, "D\pi"]       & \divR{T}{k} \arrow[r] & 0           &   & (1) \\
T\Verz{\divf{L[x]}(f(x))} \arrow[r, hook]              & \Tensor{T}{L}{\divR{L}{k}} \oplus T\Verz{\divf{L[x]}(x)} \arrow[r, "\widetilde{D\pi}"]      & \divR{T}{k} \arrow[r] & 0           &   & (2) \\
                               & \Tensor{T}{L}{\divR{L}{k}} \arrow[r, "\varphi"] & \divR{T}{k} \arrow[r] & \divR{T}{L} \arrow[r] & 0 & (3)
\end{tikzcd}
\end{center}
Zeige zunächst, das \textbf{(2)} auch wirkliche exakt ist:
\begin{gather*}
(\tensor{1}{L[x]}{\divf{L[x]}})(f(x))
 =  \Tensor{T}{L[x]}{L[x]\Verz{\divf{L[x]}(f(x))}}
  \simeq T \Verz{\divf{L[x]}(f(x))} \\
\Rightarrow \text{ Ersetze } \functionfront{\tensor{1}{L[x]}{\divf{L[x]}}}{(f(x))/(f(x)^2)}{\Tensor{T}{L[x]}{\divR{L[x]}{k}}}\\ \text{ durch }  T \Verz{\divf{L[x]}(f(x))} \hookrightarrow \Tensor{T}{L[x]}{\divR{L[x]}{k}}. \\
\ \\
\text{Weiter gilt nach \cref{Differenzial von Polynomalgebren 2} } \divR{L[x]}{k} \simeq \Tensor{L[x]}{L}{\divR{L}{k}} \oplus L[x]\Verz{\divf{L[x]}(x)},\\
\text{also auch } \Tensor{T}{L[x]}{\divR{L[x]}{k}} \simeq \Tensor{T}{L}{\divR{L}{k}} \oplus T\Verz{\divf{L[x]}(x)}.
\end{gather*}
\begin{itemize}
\item[\underline{\glqq $\Rightarrow$ \grqq:}]
Wenn wir nun unsere zwei exakten Sequenzen betrachten sehen wir, dass $\varphi$ eine Einschränkung von $D\pi$ auf einen kleineren Definitionsbereich ist. Zeige also, dass $D\pi$ injektiv ist:
\begin{gather*}
\text{Nach Vorraussetung gilt } \divf{T}(a) = 0 \\
\Rightarrow d_{T}(f) = d_{T}(x^p) - d_{T}(a) =  px^{p-1}d_{T}(x) - d_{T}(a) = 0 - 0 \\
\Rightarrow T\Verz{\divf{L[x]}(f(x))} = 0\\
\text{Bezogen auf unser exakte Sequenz \textbf{(2)} bedeutet dies, dass $D \pi$ injektiv ist.}
\end{gather*}
\item[\underline{\glqq $\Leftarrow$ \grqq:}]
\begin{gather*}
\text{In $T$ gilt f(x) = 0} \\
\Rightarrow 0 = \divf{T}(f(x)) = \divf{T}(x^p) - \divf{T}(a) = \divf{T}(a) \\
\Rightarrow \varphi(\tensor{1}{L}{d_{L}(a)}) = \divf{T}(a) = 0 
\end{gather*}
Da $\varphi$  nach Voraussetzung injektiv ist, gilt $\tensor{1}{L[x]}{\divf{L[x]}(a)} = 0$ und somit auch $\divf{L}(a) = 0$.
\end{itemize}
\end{proof}
\end{document}
