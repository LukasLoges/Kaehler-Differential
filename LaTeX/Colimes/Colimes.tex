\documentclass[10pt,a4paper]{report}
\usepackage[utf8]{inputenc}
\usepackage{amsmath}
\usepackage{amsthm}
\usepackage{amsfonts}
\usepackage{amssymb}
\usepackage{tikz-cd}
\usepackage{calc}
\usepackage{setspace}
\usepackage[german]{babel}
\usetikzlibrary{babel}
\usepackage{cleveref}

\newcommand{\comment}[1]{}
\renewcommand{\baselinestretch}{1.1}


\newcounter{Aussage}[chapter]

\newtheorem{satz}[Aussage]{Satz}
\newtheorem{theorem}[Aussage]{Theorem}
\newtheorem{prop}[Aussage]{Proposition}
\newtheorem{korrolar}[Aussage]{Korrolar}
\newtheorem{lemma}[Aussage]{Lemma}
\newtheorem{bem}[Aussage]{Bemerkung}
\newtheorem{definition}[Aussage]{Definition}

\newcommand{\functionfront}[3]{\nolinebreak{#1:#2 \longrightarrow #3}}
\newcommand{\functionback}[3]{\nolinebreak{#1:#2 \longmapsto #3}}
\newcommand{\function}[5]{\nolinebreak{#1:#2 \longrightarrow #3 \, , \, #4 \longmapsto #5}}
\newcommand{\divR}[2]{\Omega_{#1/#2}}
\newcommand{\tensor}[3]{#1 \otimes_{#2} #3}

\newcommand{\colimes}[0]{\lim\limits_{ \longrightarrow }}
\newcommand{\infunctionfront}[3]{\nolinebreak{#1:#2 \hookrightarrow #3}}
\newcommand*{\defeq}{\mathrel{\vcenter{\baselineskip0.5ex \lineskiplimit0pt
                     \hbox{\scriptsize.}\hbox{\scriptsize.}}}%
                     =}
\newcommand*{\defshow}{\stackrel{!}{=}}
\newcommand{\kernel}[1]{kern(#1)}
\newcommand{\immage}[1]{im(#1)}

\begin{document}
\begin{prop} \label{R-Algebra-Coprodukt,Differenzenkokern}
in der Kategorie der R-Algeben existieren Coprodukte und Differenzenkokerne, wobei:
\begin{itemize}
\item[\textbf{1.}] Das Coprodukt $\colimes (\infunctionfront{\mathcal{F}}{\lbrace B_i \rbrace_{i \in \Lambda}}{R-Agebren}$ einer endlichen Familie von $R-Algebren$ entspricht deren Tesorprodukt $\bigotimes_{i \in \Lambda} B_i$. 
\item[\textbf{2.}] Seien $\functionfront{f,g}{C_1}{C_2}$ R-Algebra-Homomorphismen, setze $Q \defeq \lbrace f(x) - g(x)\mid x \in C_2 \rbrace$.\\
Dann ist $\function{q}{C_2}{C_2/Q}{y}{[y]}$ der Differenzenkokern von $f,g$.
\end{itemize}
\end{prop}

\begin{proof}
Zu \textit{\textbf{1.}}:\ \\
Nutze die universellen Eigenschaften des Tensorproduktes und des Kähler-Differenzials.\\
Es sind also der Morphismus $\functionfront{\psi}{(\infunctionfront{\mathcal{F}}{\lbrace B_i \rbrace}{(R-Algebren)})}{\colimes \mathcal{F}}$ und die bilineare Abbildung $\functionfront{g}{\oplus_i B_i}{\otimes_i B_i}$ gegeben.\\
Konstruieren den Morphismus $\functionfront{\psi'}{\mathcal{F}}{\otimes_i B_i}$ durch $\function{\psi'_i}{B_i}{\otimes_i B_i}{b_i}{g(1,..,1,b_i,1,..,1)}$ für $i \in \lambda$. Konstruiere außerdem die bilineare Abbildung $\function{f}{\oplus_i B_i}{\colimes \mathcal{F}}{b}{\prod_i \psi_i b_i}$.\\
\ \\
Durch die universellen Eigenschaften erhalten wir die R-Algebra-Homomorphismen $\functionfront{\varphi}{\colimes \mathcal{F}}{\bigotimes _i B_i}$ und $\functionfront{\phi}{\bigotimes _i B_i}{\colimes \mathcal{F}}$.
\begin{center}
\begin{tikzcd}
  & \infunctionfront{\mathcal{F}}{\lbrace B_i \rbrace}{(R-Algebren)} \arrow[rd, "\psi"] \arrow[ld, "\psi'"'] &                                            &  &   & \oplus_i B_i \arrow[rd, "g"] \arrow[ld, "f"'] &                                         \\
\otimes_i B_i &                                           & \colimes \mathcal{F} \arrow[ll, "\exists ! \varphi"', dashed] &  & \colimes \mathcal{F} &                                    & \otimes_i B_i \arrow[ll, "\exists ! \phi"', dashed]
\end{tikzcd}
\end{center}
Weiter ergeben sich auch durch die universellen Eigenschaften\\ $id_{\colimes \mathcal{F}} = \phi \circ \varphi \textit{ und } id_{\bigotimes_i B_i} = \varphi \circ \phi$.
\begin{center}
\begin{tikzcd}
         & \mathcal{F} \arrow[rd, "\psi"] \arrow[ld, "\psi"'] &                                                                              &  &                  & \bigoplus_i B_i \arrow[rd, "g"] \arrow[ld, "g"'] &                                                                                     \\
\colimes \mathcal{F} &                                                    & \colimes \mathcal{F} \arrow[ll, "\exists ! id_{\colimes \mathcal{F}} = \phi \circ \varphi ", dashed] &  & \bigotimes_i B_i &                                                  & \bigotimes_i B_i \arrow[ll, "\exists ! id_{\bigotimes_i B_i} = \varphi \circ \phi "', dashed]
\end{tikzcd}
\end{center}
\ \\
\ \\
\ \\
Zu \textit{\textbf{2.}}:\
$$q \circ f = q \circ g \textit{ gilt, da } \kernel{q} = Q = \lbrace f(x) - g(x)\mid x \in C_2 \rbrace$$

Sei nun eine Funktion $q' \in Hom_{\mathcal{A}}(C_2,T')$ mit $q' \circ f = q' \circ$ gegeben.
\begin{spacing}{1.5}
$q' \circ (f - g) = 0 \Rightarrow Q\textit{ ist Untermodul von }Q' \defeq \kernel{q'}.$\\
$\textit{Nach HOMOMORPHIESATZ [kommutative Algebra 2.10] gilt } \nolinebreak{C_2/Q' \simeq (C_2/Q)/(Q'/Q))}.$\\
$\Rightarrow \textit{für } \function{q'}{C_2}{(C_2/Q)/(Q'/Q))}{y}{[y]'} \textit{ ist eine isomorphw Darstellung von } \functionfront{q'}{C_2}{T'}$\\
$\Rightarrow \exists ! \function{\varphi}{C_2/Q}{(C_2/Q)/(Q'/Q)}{[y]}{[y]'}\textit{ mit }(\varphi \circ q) = q'.$
\end{spacing}
\end{proof}
\end{document}
