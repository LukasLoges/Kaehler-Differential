\documentclass[10pt,a4paper]{report}
\usepackage[utf8]{inputenc}
\usepackage{amsmath}
\usepackage{amsthm}
\usepackage{amsfonts}
\usepackage{amssymb}
\usepackage{color}
\usepackage{tikz-cd}
\usepackage{calc}
\usepackage{setspace}
\usepackage[german]{babel}
\usetikzlibrary{babel}
\usepackage{cleveref}

\newcommand{\comment}[1]{}
\renewcommand{\baselinestretch}{1.1}

\newcommand{\ModulsOfDifferenzials}{David Eisenbud 1994}
\newcommand{\Algebra}{Christian Karpfinger, Kurt Meyberg 2009}

\newcounter{Aussage}[chapter]

\newtheorem{satz}[Aussage]{Satz}
\newtheorem{theorem}[Aussage]{Theorem}
\newtheorem{prop}[Aussage]{Proposition}
\newtheorem{korrolar}[Aussage]{Korrolar}
\newtheorem{lemma}[Aussage]{Lemma}
\newtheorem{bem}[Aussage]{Bemerkung}
\newtheorem{definition}[Aussage]{Definition}
\newtheorem{bsp}[Aussage]{Beispiel}

\newcommand{\functionfront}[3]{\nolinebreak{#1:#2 \longrightarrow #3}}
\newcommand{\functionback}[3]{\nolinebreak{#1:#2 \longmapsto #3}}
\newcommand{\function}[5]{\nolinebreak{#1:#2 \longrightarrow #3 \, , \, #4 \longmapsto #5}}
\newcommand{\infunctionfront}[3]{\nolinebreak{#1:#2 \hookrightarrow #3}}
\newcommand{\divR}[2]{\Omega_{#1/#2}}
\newcommand{\divf}[1]{d_{#1}}
\comment{\newcommand{\divf}[2][]{d_{#1}}}
\newcommand{\Tensor}[3]{#1 \otimes_{#2} #3}
\newcommand{\tensor}[3]{#1 \otimes #3}
\newcommand{\lok}[2]{#1 [#2^{-1}]}
\newcommand{\loke}[3]{(\frac{#1}{#2})_{_{#3}}}
\comment{\newcommand{\loke}[3]{(#1,#2)_{mod\sim_{#3}}}}

\newcommand{\colimes}[0]{\lim\limits_{ \longrightarrow }}
\newcommand*{\defeq}{\mathrel{\vcenter{\baselineskip0.5ex \lineskiplimit0pt
                     \hbox{\scriptsize.}\hbox{\scriptsize.}}}%
                     =}
\newcommand*{\defeqr}{= \mathrel{\vcenter{\baselineskip0.5ex \lineskiplimit0pt
                     \hbox{\scriptsize.}\hbox{\scriptsize.}}}}

\newcommand*{\defshow}{\stackrel{!}{=}}
\newcommand{\kernel}[1]{kern(#1)}
\newcommand{\immage}[1]{im(#1)}
\newcommand{\Verz}[1]{\langle #1 \rangle}


\begin{document}
\begin{definition}\label{Definition Leibnizregel}
Definition Leibnizregel
\end{definition}

\begin{lemma}\label{Summe von Derivationen}
Summe von Derivationen
\end{lemma}

\begin{bem}\label{Derivation ist Ableitung}
Derivation ist Ableitung
\end{bem}

\begin{bem}\label{Unendliche Indexmengen}
Unendliche Indexmengen
\end{bem}

\begin{bem}\label{Darstellung der Polynomalgebra als Tensorprodukt}
Darstellung der Polynomalgebra als Tensorprodukt
\end{bem}

\begin{prop}\label{R-Algebra-Kolimiten}
R-Algebra-Kolimiten
\end{prop}

\begin{satz}\label{Konormale Sequenz}
Konormale Sequenz
\end{satz}

\begin{bem}\label{NeuDifferenzenkokerndef}
NeuDifferenzenkokerndef
\end{bem}
\chapter{Kolimes}
\section{Ableiten von Polynomen}
\begin{korrolar}\label{NNDerivation ist Ableitung}\textit{[Eigene Überlegung]}\\
Für Differentialraum des Plynomrings $R[x]$ gilt:
\begin{gather*}
\divR{R[x]}{R} = R[x]\langle \div{R[x]}(x) \rangle
\end{gather*}
Wobei $R[x] \langle \divf{R[x]}(x)\rangle$ das von $\divf{R[x]}(x)$ erzeugt Modul über $R[x]$ ist.\\
Genauer gesagt entspricht die universellen Derivation des Polynomrings $R[x]$ der formalen Ableitung von Polynomfunktionen, wie wir sie aus der Analysis kennen. Für $P(x) \in R[x]$ gilt also:
\begin{gather*}
\divf{R[x]}(P(x)) = P'(x)\divf{R[x]}(x)
\end{gather*}
\end{korrolar}


\ \\
\begin{prop}\label{NNDifferenzial des Produktes von Algebren}
Seien $S_1, \dots , S_n$ R-Algebren. Sei dazu $S \defeq \prod_{i \in \lbrace 1, \dots , n \rbrace} S_i$ deren direktes Produkt.
Dann gilt:
\begin{gather*}
\divR{S}{R} = \prod_{i \in \lbrace 1, \dots , n \rbrace} \divR{S_i}{R}
\end{gather*}
\end{prop}


\ \\
\textcolor{blue}{\textbf{Darstellung der Polynomalgebra als Tensorprodukt}}
\begin{bem}\label{Darstellung der Polynomalgebra als Tensorprodukt}\textit{[Eigene Überlegung]}\\
Die Polynomalgebra $R[\lbrace x_i \rbrace_{i \in \Lambda}]$ über R lässt sich wie folgt als Tensorprodukt darstellen:
\begin{gather*}
R[\lbrace x_i \rbrace_{i \in  \Lambda}] \simeq \bigotimes_{i \in \Lambda} R[x_i]
\end{gather*}
\end{bem}
\begin{proof}
Im Falle einer endlichen Indexmenge $\Lambda$ wollen wir induktiv vorgehen. Seien für den Induktionsschritt $n,m \in \mathbb{N}$ und $\nolinebreak{S_x \defeq R[x_1, \dots x_n]}$, $\nolinebreak{S_y \defeq R[y_1, \dots , y_m]}$ zwei Polynomalgebren über R, zeige:
\begin{gather*}
S_{xy} \defeq R[x_1, \dots , x_n, y_1 , \dots , y_m] \simeq \Tensor{S_x}{R}{S_y}
\end{gather*}
Dazu betrachten wir folgende bilineare Funktion:
\begin{gather*}
\function{g'}{S_x \oplus S_y}{S}{(P,Q)}{P \cdot Q}
\end{gather*}
Erhalte nun eine Funktion $\functionfront{\varphi}{\Tensor{S_x}{R}{S_y}}{S_{xy}}$ aus der universellen Eigenschaft des Tensorproduktes:
\begin{center}
\begin{tikzcd}
S_x \oplus S_y \arrow[r, "g"] \arrow[rd, "g'"'] & \Tensor{S_x}{R}{S_y} \arrow[d, "\exists ! \varphi", dashed] \\
                                                & S_{xy}                                       
\end{tikzcd}
\ \\
$\function{\varphi}{\Tensor{S_x}{R}{S_y}}{S_{xy}}{\tensor{P}{R}{Q}}{P \cdot Q}$
\end{center}
Der Homomorphismus $\varphi$ ist surjektiv und bildet die Erzeuger $\lbrace \tensor{x_i}{R}{1} \rbrace \cup \lbrace \tensor{1}{R}{y_j} \rbrace$ von $\Tensor{S_x}{R}{S_y}$ eindeutig auf die Erzeuger $\lbrace x_i \rbrace \cup \lbrace y_j \rbrace$ von $S_{xy}$ ab. Folglich ist $\varphi$ ein Isomorphismus.\\
Indunktiv erhalten wir daraus für den Fall $\vert \Lambda \vert < \infty$ folgenden Isomorphismus:
\begin{gather*}
\function{\Phi}{\bigotimes_{i \in \Lambda} R[x_i]}{R[\lbrace x_i \rbrace_{i \in \Lambda} ]}{(P_1(x_1), \dots P_n(x_n))}{\prod_{i=1}^{n} P_i(x_i)}
\end{gather*}
Dies ist auch im Fall $\Lambda = \infty$ ein Isomorphismus, da wir auch in diesem Fall nur Tensorprodukte endlich vieler Polynome bzw. Polynome in endlich vielen Variablen betrachten \textit{(siehe \cref{Unendliche Indexmengen})}.\\
Bedenke zuletzt noch, dass das Tensorprodukt $\bigotimes_{i \in \Lambda} R[x_i]$ bis auf eine Eindeutige Isomorphie eindeutig bestimmt ist.
\end{proof}


\ \\
\textcolor{blue}{\textbf{Differenzial des Koproduktes}}
\begin{prop} \label{Differenzial des Koproduktes} \textit{[vlg. Korolar 16.5 \ModulsOfDifferenzials]}\\
Seien $\lbrace S_i \rbrace_{i \in \Lambda}$ R-Algebren und $T = \bigotimes_{i \in \Lambda} S_i$ deren Koprodukt.\\
Dann gilt:
\begin{gather*}
\divR{T}{R} = \bigoplus_{i\in \Lambda} ( \Tensor{T}{S_i}{\divR{S_i}{R}} )\\
\text{mit: }\function{\divf{T}}{R}{\divR{T}{R}}{\left(\otimes_{i = 1}^n s_i\right)}{\left(\tensor{\left(\otimes_{i = 2}^n s_i\right)}{S_1}{\divf{R[x_1]}(s_1)},\dots,\tensor{\left( \otimes_{i = 1}^{n - 1} s_i \right)}{S_n}{\divf{R[x_n]}}(s_n)\right)}
\end{gather*}
\end{prop}
\begin{proof} Zeige, dass $\bigoplus_{i\in \Lambda} ( \Tensor{T}{S_i}{\divR{S_i}{R}}) \simeq \divR{T}{R}$ gilt.\\
Für $i \in \Lambda$ lässt sich  $T$ als $\Tensor{\left(\bigotimes_{j \in \Lambda \setminus \lbrace i \rbrace} S_i \right)}{R}{S_i}$ betrachten, nutze dies um folgende $R$-lineare Derivationen zu definieren:
\begin{gather*}
\comment{
\function{e_i}{T}{\bigoplus_{i \in \Lambda} \Tensor{T}{S_i}{\divR{S_i}{R}}}{\tensor{(\otimes_{j \neq i} s_j)}{R}{s_i}}{(0,\dots,0,\tensor{(\otimes_{j \neq i} s_j)}{R}{d_{S_i}(s_i)},0,\dots,0)} \\
\function{e}{T}{\bigoplus_{i \in \Lambda} \Tensor{T}{S_i}{\divR{S_i}{R}}}{\tensor{(\otimes_{j \neq i} s_j)}{R}{s_i}}{\sum_{i = 1}^n e_i(\tensor{(\otimes_{j \neq i} s_j)}{R}{s_i})}
}
\function{e_i}{T}{\bigoplus_{i \in \Lambda} \Tensor{T}{S_i}{\divR{S_i}{R}}}{\tensor{(\otimes_{j \neq i} s_j)}{R}{s_i}}{(0,\dots,0,\tensor{(\otimes_{j \neq i} s_j)}{R}{d_{S_i}(s_i)},0,\dots,0)} \\
\function{e}{T}{\bigoplus_{i \in \Lambda} \Tensor{T}{S_i}{\divR{S_i}{R}}}{t}{\sum_{i = 1}^n e_i(t)}
\end{gather*}
Da $\divf{S_i}$ eine Derivation ist, ist $e_i$ und somit nach \cref{Summe von Derivationen} und \cref{Unendliche Indexmengen} auch $e$ eine Derivation. Mithilfe der universellen Eigenschaft von $\divf{T}$ erhalten wir einen eindeutigen Homomorphismus $\varphi$ mit $\varphi \circ \divf{T} = e$:
\comment{
\begin{center}
\begin{tikzcd}
T \arrow[rd, "e"'] \arrow[r, "\divf{T}"] & \divR{T}{R} \arrow[d, "\exists ! \varphi", dashed] \\
                                    & \bigoplus_{i\in \Lambda} \Tensor{T}{S_i}{\divR{S_i}{R}}                                       
\end{tikzcd}
\end{center}
}
\begin{gather*}
\function{\varphi}{\divR{T}{R}}{\bigoplus_{i\in \Lambda}(\Tensor{T}{S_i}{\divR{S_i}{R}})}{\divf{T}(s_1 \otimes \dots \otimes s_n)}{\sum_{i = 1}^n e_i(\tensor{(\otimes_{j \neq i} s_j)}{R}{s_i})}\\
\functionback{\varphi}{\divf{T}(1 \otimes s_i \otimes 1)}{(0,\tensor{1}{S_i}{d_{S_i}(s_i)},0)}
\end{gather*}
Suche nun eine Umkehrfunktion $\phi$ zu $\varphi$. Definiere dazu für $i \in \Lambda$ folgendes R-lineares Differential:
\begin{gather*}
\function{h_i}{S_i}{\divR{T}{R}}{s_i}{\divf{T}(\tensor{(\otimes_{j \neq i} 1)}{R}{s_i})}
\end{gather*}
Mithilfe der universellen Eigenschaft von $\divf{S_i}$ erhalten wir dadurch einen eindeutigen Homomorphismus $h_i'$ mit $h_i' \circ \divf{T} = h_i$. Nutze diesen um einen weiteren Homomorphismus zu definieren:
\begin{gather*}
\function{\phi_i}{\Tensor{T}{S_i}{\divR{S_i}{R}}}{\divR{T}{R}}{\tensor{t}{S_i}{\divf{S_i}(s_i)}}{t \cdot (h' \circ \divf{S_i})(s_i) = t \cdot h_i(s_i)}
\end{gather*}
Damit erhalten wir folgenden kommutatives Diagramm:
\begin{center}
\begin{tikzcd}
S_i \arrow[rd, "h_i"'] \arrow[r, "\divf{S_i}"] & \divR{S_i}{R} \arrow[d, "\exists ! k'", dashed] \arrow[r, , hook] & \Tensor{T}{S_i}{\divR{S_i}{R}} \arrow[ld, "\phi_i"] \\
                                          & \divR{T}{R}                                                    &                     
\end{tikzcd}
\end{center}
Zuletzt bilden wir die Summe $\phi \defeq \sum_{i \in \Lambda} \phi_i$ und erhalten damit eine Umkehrfunktion von $\varphi$:
\begin{gather*}
\function{\phi}{\bigoplus_{i\in \Lambda} ( \Tensor{T}{S_i}{\divR{S_i}{R}})}{\divR{T}{R}}{(\tensor{t_i}{S_1}{\divf{S_1}(s_1)},\dots,\tensor{t_n}{S_n}{\divf{S_n}(s_n)})}{\sum_{i = 1}^n t_i \cdot h_i(s_i)}\\
\functionback{\phi}{(0,\tensor{1}{S_i}{d_{S_i}(s_i)},0)}{\divf{T}(1 \otimes s_i \otimes 1)}
\end{gather*}
Somit gilt $\bigoplus_{i\in \Lambda} ( \Tensor{T}{S_i}{\divR{S_i}{R}}) \simeq \divR{T}{R}$.\\
Definiere also ab jetzt $\bigoplus_{i\in \Lambda} (\Tensor{T}{S_i}{\divR{S_i}{R}})$ als des Differentialraum von $T$ über $R$. Damit gilt $\divf{T} = e$.\\
\end{proof}


\ \\
\textcolor{blue}{\textbf{Mehrdimmensionales Algebraisches Differentieren}}
\begin{bem}\label{Mehrdimmensionales Algebraisches Differentieren} \textit{[Eigene Bemerkung]}\\
Sei $R(\lbrace x_i \rbrace_{i \in \Lambda})$ ein Polynomring über R. Bezeichne mit $\delta_{j}$ die formale Ableitung in Richtung $x_j$, wie wir sie aus der Analysis für Polynomfunktionen über $\mathbb{R}^n$ kennen:
\begin{gather*}
\functionfront{\delta_{j}}{R(\lbrace x_i \rbrace_{i \in \Lambda})}{R(\lbrace x_i \rbrace_{i \in \Lambda})}\\
{\sum_k \left( a_k \cdot x_j^{n_{j,k}}\prod_{i \neq j} x_i^{n_{i,k}}\right)}
\longmapsto
{\sum_{k,n_{j,k} > 0} \left( a_k \cdot n_{j,k} \cdot x_j^{n_{j,k} - 1}\prod_{i \neq j} x_i^{n_{i,k}}\right)}
\end{gather*}
Betrachte den Differentialraum von $R[\lbrace x_i \rbrace_{i \in \Lambda}]$  über $R[\lbrace x_i \rbrace_{i \in \Lambda \setminus\lbrace j \rbrace}]$:
\begin{gather*}
\functionfront{\divf{j}}{R[\lbrace x_i \rbrace_{i \in \Lambda}]}{\divR{R[\lbrace x_i \rbrace_{i \in \Lambda}]}{R[\lbrace x_i \rbrace_{i \in \Lambda \setminus\lbrace j \rbrace}]}}
\end{gather*}
Nach \cref{Derivation ist Ableitung} entspricht $\divf{j}$ der formalen Ableitung $\delta_{j}$. Für $P_j(x_j),P(x_1,\cdots,x_n) \in R{\lbrace x_i \rbrace}_{i \in \Lambda}$ gilt also:
\begin{gather}
\delta_{j}(P_j(x_j)) = P'(x_j)\\
\divf{j}(P(x_1,\cdots,x_n)) = \delta_{j}(P(x_1, \cdots ,x_n))\divf{j}(x_j)
\end{gather}
\end{bem}


\ \\
\textcolor{blue}{\textbf{Differenzial von Polynomalgebren 1} \textit{[vlg. Proposition 16.1 \ModulsOfDifferenzials]}}
\begin{korrolar}\label{Differenzial von Polynomalgebren 1}
Sei $S = R[x_1,...,x_n]$ eine Polynomalgebra über R. Dann gilt:
\begin{gather*}
\divR{S}{R} = \bigoplus_{i = 1}^n S \langle \divf{S}(x_i) \rangle 
\end{gather*}
Für die universelle Derivation $\divf{S}$ gilt hierbei mit der Notation von \cref{Mehrdimmensionales Algebraisches Differentieren}:
\begin{gather*}
\function{\divf{S}}{S}{\divR{S}{R}}{P(x_1,\cdots,x_n)}{(\delta_{1}(P)\divf{S}(x_1), \cdots , \delta_{n}(P)\divf{S}(x_n))}
\end{gather*}
\end{korrolar}
\begin{proof}
Wie in \cref{Darstellung der Polynomalgebra als Tensorprodukt} gezeigt, ist $S$ isomorph zu $S' \defeq \bigotimes_{i = 1}^n R[x_i]$. In \cref{Differenzial des Koproduktes} haben wir gezeigt, wie das Differenzial eines solchen Tensorproduktes aussieht:
\begin{gather*}
\divR{S'}{R} = \bigoplus_{i \in \Lambda} (\Tensor{S'}{R[x_i]}{\divR{R[x_i]}{R}})
\end{gather*}
Mithilfe von \cref{Derivation ist Ableitung} können wir ${\divR{R[x_i]}{R}}$ für $i \in \Lambda$ weiter umformen:
\begin{gather*}
\divR{S'}{R} = \bigoplus_{i = 1}^n (\Tensor{S'}{R[x_i]}{R[x_i]\langle \divf{R[x_i]}(x_i) \rangle})
\simeq \bigoplus_{i = 1}^n S' \langle \divf{R[x_i]}(x_i) \rangle 
\end{gather*}
Nutze nun $S' \simeq S$ und betrachte $\divf{R[x_i]}$ als Einschränkung von $\divf{S}$. Dadurch erhalten wir die gewünschte Darstellung von $\divR{S}{R}$.\\
Definiere ab nun also $\divR{S}{R} = \bigoplus_{i = 1}^n S \langle \divf{S}(x_i) \rangle$.\\
Um zu zeigen, dass hierbei die universelle Derivation die gewünschte Form annimmt gehe zunächst die bisher genutzten Derivationen und Isomorphismen durch:
\begin{center}
$\functionfront{\divf{S}}{S}{\divR{S}{R}}$
\begin{tikzcd}
S \arrow[d]  &  & \prod_{i = 1}^n P_i(x_i) \arrow[d, maps to] \\
S' \arrow[d] &  & \otimes_{i = 1}^n P_i(x_i)\arrow[d, maps to]                        \\
\bigoplus_{i = 1}^n (\Tensor{S'}{R[x_i]}{R[x_i]\langle \divf{R[x_i]}(x_i) \rangle}) \arrow[d]  &  & (\dots , (\tensor{\otimes_{k \neq i} P_k(x_k)}{R[x_i]}{\divf{R{x_i}}(P(x_i))}) , \dots) \arrow[d, maps to]                        \\
\bigoplus_{i \in \Lambda} S \langle \divf{S}(x_i) \rangle            &  & (\dots , \left( \prod_{k \neq i} P_k(x_k) \right) P'(x_i) \divf{S}(x_i) , \dots)
\end{tikzcd}
\end{center}
Betrachte nun \cref{Mehrdimmensionales Algebraisches Differentieren}. Dabei stellen wir fest, dass wir für $j \in \Lambda$ von $\divf{S}(x_j) = \divf{j}(x_j)$ ausgehen können, da $\lbrace \divf{S}(x_i) \rbrace_{i \in \Lambda}$ linear unabhängig ist.\\
Rechne also für $\prod_{i = 1}^n P_i(x_i) \in R[\lbrace x_i \rbrace_{i \in \Lambda}]$ und $j \in \lbrace 1, \cdots , n \rbrace$ nach, ob unsere gewünschte Darstellung von $\divf{S}$ zutrifft:
\begin{gather*}
\delta_{j}\left(\prod_{i = 1}^n P(x_i)\right)\divf{S}(x_j) = \divf{j}\left(\prod_{i = 1}^n P(x_i)\right)
\textit{ (\cref{Mehrdimmensionales Algebraisches Differentieren})}\\
= P_j(x_j)\divf{j} \left( \prod_{i \neq j} P_i(x_i) \right) + \left( \prod_{i \neq j} P_i(x_i) \right) \divf{j}(P_j(x_j))
\textit{ (Leibnizregel)} \\
= 0 + \left( \prod_{i \neq j} P_i(x_i) \right) \delta_j(P(x_j))\divf{j}(x_j)
=  \left( \prod_{i \neq j} P_i(x_i) \right) P'(x_j)\divf{j}(x_j)
\end{gather*}
\end{proof}


\ \\
\textcolor{blue}{\textbf{Differenzial von Polynomalgebren 2} \textit{[vgl. Korrolar 16.6 \ModulsOfDifferenzials]}}
\begin{korrolar}\label{Differenzial von Polynomalgebren 2}
Sei S eine R-Algebra und $T \defeq S[x_1,...,x_n]$ eine Polynomalgebra über S. Dann gilt:
\begin{gather*}
\divR{T}{R} \simeq (\Tensor{T}{S}{\divR{S}{R}}) \oplus \bigoplus_{i = 1}^n T \Verz{\divf{T}(x_i)}
\end{gather*}
\end{korrolar}
\begin{proof}
Betrachte $T$ als Tensorprodukt über R-Algebren und wende anschließend \cref{Differenzial des Koproduktes} an:
\begin{gather*}
T \simeq \Tensor{S}{R}{R[x_1,...,x_n]} \\
\Rightarrow \divR{T}{R} \simeq (\Tensor{T}{S}{\divR{S}{R}}) \oplus (\Tensor{T}{R[x_1,...,x_n]}{\divR{R[x_1,...,x_n]}{R}})
\end{gather*}
Zuletzt wende den soeben gezeigten \cref{Differenzial von Polynomalgebren 1} an und nutze schließlich $R[x_1,...,x_n] \subseteq T$ um das Tensorprodukt zu vereinfachen:
\begin{gather*}
\Tensor{T}{R[x_1,...,x_n]}{\divR{R[x_1,...,x_n]}{R}}\\
\simeq \Tensor{T}{R[x_1,...,x_n]}{\bigoplus_{i = 1}^n R[x_1,...,x_n]\langle \divf{R[x_i]}(x_i) \rangle } \\
\simeq \bigoplus_{i = 1}^n T \langle \divf{R}(x_i) \rangle
\end{gather*}
Damit haben wir Isomorphie gezeigt. Definiere also $(\Tensor{T}{S}{\divR{S}{R}}) \oplus \bigoplus_{i = 1}^n T \Verz{\divf{T}(x_i)}$ als den Differentialraum von $T$ über $R$.\\
Abschließend wollen wir noch $\divf{T}$ betrachten, sei dazu $s\prod_{i =1}^k x_i^{n_i} \in T$ ein beliebiges Monom:
\begin{gather*}
d_T\left(s\prod_{i =1}^k x_i^{n_i}\right) \\
= \left(\tensor{\prod_{i =1}^k x_i^{n_i}}{S}{\divf{S}(s)},
 \tensor{s}{R[\lbrace x_i \rbrace_{i \in \Lambda}]}{\divf{R[\lbrace x_i \rbrace_{i \in \Lambda}]}}\left(\prod_{i =1}^k x_i^{n_i}\right)\right)\\
=\left(\tensor{\prod_{i =1}^k x_i^{n_i}}{S}{\divf{S}(s)},
s \cdot \delta_1 \left( \prod_{i =1}^k x_i^{n_i} \right) \divf{R[x_1](x_1)} , \dots , s \cdot \delta_n \left( \prod_{i =1}^k x_i^{n_i} \right) \divf{R[x_n](x_n)} \right)
\end{gather*}
\end{proof}


\ \\
\begin{prop}\label{Differenzial des Produktes von Algebren}
Seien $S_1, \dots , S_n$ R-Algebren. Sei dazu $S \defeq \prod_{i \in \lbrace 1, \dots , n \rbrace} S_i$ deren direktes Produkt.
Dann gilt:
\begin{gather*}
\divR{S}{R} = \prod_{i \in \lbrace 1, \dots , n \rbrace} \divR{S_i}{R}
\end{gather*}
Wobei die universelle Derivation folgende Form hat:
\begin{gather*}
\function{\divf{S}}{S}{\prod_{i \in \lbrace 1, \dots , n \rbrace} \divR{S_i}{R}}{s}{((\divf{S_1} \circ p_1)(s) , \dots , (\divf{S_1} \circ p_1)(s) )}
\end{gather*}
\end{prop}


\ \\
\begin{korrolar}\label{Jakobimatrizen}\textit{[Eigene Überlegung]}\\
Sei $S = R[x_1 \cdots , x_m]$ der Polynomring in $m$ Variablen über $R$ und $\nolinebreak{S^n = \bigoplus_{k = 1}^n S}$ der $n$-fache Produktraum von $S$.\\
Somit entspricht mit der Notation von \cref{Mehrdimmensionales Algebraisches Differentieren} der Differentialraum von $S^n$ über $R$ den Jakobimatrizen, wie wir sie aus Analysis kennen:
\begin{gather*}
\divR{S}{R} = \bigoplus_{i = 1}^n \left( \bigoplus_{j = 1}^m S\langle \divf{S}(x_i) \rangle \right)\\
\text{mit: } \function{\divf{S^n}}{S^n}{\divR{S^n}{R}}{P}{(\delta_j(P_i))_{i \in \lbrace 1 , \dots , n \rbrace , j \in \lbrace 1 , \dots , m \rbrace}}
\end{gather*}
Wobei wir $J_{(P_1, \dots ,P_n)} \defeq (\delta_j(P_i))_{i \in \lbrace 1 , \dots , n \rbrace , j \in \lbrace 1 , \dots , m \rbrace}$ die Jakobimatrix von $P$ nennen.
\end{korrolar}
\begin{proof}
Zunächst erinnern wir uns daran, dass bei Algebren und Moduln die endlichen Summen den endlichen Produkten entsprechen.
In \cref{Differenzial des Produktes von Algebren} haben wir den Differentialraum endlicher Produkte beschrieben:
\begin{gather*}
\divR{S^n}{R} = \bigoplus_{i=1}^n \divR{S}{R}
\end{gather*}
In \cref{Differenzial von Polynomalgebren 1} haben wir gesehen, dass $\divR{S}{R}$ dem gewünschten Produktraum entspricht:
\begin{gather*}
\divR{S}{R} = \bigoplus_{j = 1}^m S\langle \divf{S}(x_i) \rangle
\end{gather*}
Betrachte also noch genauer, wie die universelle Ableitung in diesem beiden Fällen beschrieben wird. Für ein beliebiges $P = (P_1, \dots , P_n) \in S^n$ gilt:
\begin{gather*}
\divf{S^n}
\begin{pmatrix}
P_1\\
\vdots\\
P_n
\end{pmatrix}
=
\begin{pmatrix}
\divf{S}(P_1)\\
\vdots\\
\divf{S}(P_n)
\end{pmatrix}
= 
\begin{pmatrix}
\delta_1(P_1) \divf{S}(x_1) & \dots & \delta_n(P_1) \divf{S}(x_n)\\
\vdots & \ddots & \vdots \\
\delta_1(P_n) \divf{S}(x_1) & \dots & \delta_n(P_1) \divf{S}(x_n)
\end{pmatrix}\
\end{gather*}
Es gilt also $\divf{S^n}(P) = (\delta_j(P_i)\divf{S}(x_j))_{i \in \lbrace 1 , \dots , n \rbrace , j \in \lbrace 1 , \dots , m \rbrace}$, was genau der Bildung der Jakobimatrix entspricht.
\end{proof}


\ \\
\newpage
\begin{korrolar}\label{Derivtion mittels Jakobimatrizen} \textit{[Kapitel 16.1 \ModulsOfDifferenzials]}\\
Sei  $S = R[x_1, \dots ,x_m]$ ein Polynomring über $R$ und $I = (P_1,\dots, P_n) \subseteq S$ ein Ideal. Betrachte $T = S/I$. Dann gilt:
\begin{gather*}
\divR{T}{R} = \left( \bigoplus_{i = 1}^m S \langle \divf{S}(x_i) \rangle \right) / \left\lbrace (t_1, \cdots, t_n) J_{(P_1,\dots, P_n)} \, \vert (t_1 , \dots t_n) \in T^n \right\rbrace\\
\text{mit: } 
\function{\divf{T}}{T}{\divR{T}{R}}{\left[Q(x_1,\cdots, x_m)\right]_T}{\left[\delta_1(Q)\divf{S}(x_1), \cdots , \delta_m(Q)\divf{S}(x_m)
\right]_{J_{(P_1 , \dots , P_n)}}}
\end{gather*}
\end{korrolar}
\begin{proof}
Betrachte zunächst die Conormale Sequenz \textit{(\cref{Konormale Sequenz})} von $\function{\pi}{S}{T}{s}{[s]_T}$:
\begin{center}
\begin{tikzcd}
I/I^2 \arrow[r, "\tensor{1}{S}{d_{S}}"] & \Tensor{T}{S}{\divR{S}{R}} \arrow[r, "D\pi"] & \divR{T}{R} \arrow[r] & 0
\end{tikzcd}
\ \\
Nach dieser gilt $\divR{T}{R} \simeq (\Tensor{T}{S}{\divR{S}{R}}) / (\tensor{1}{S}{d_{S}})(I)$.
\end{center}
Nach \cref{Differenzial von Polynomalgebren 1} und da $T = S/I$ ein Faktorring von $S$ ist, ist die folgende Funktion $\Phi$ eine Isomorphie:
\begin{gather*}
\Phi: 
\Tensor{T}{S}{\divR{S}{R}} 
\longrightarrow \Tensor{T}{S}{\bigoplus_{i = 1}^m S \langle \divf{S}(x_i) \rangle} 
\longrightarrow \bigoplus_{i = 1}^m T \langle \divf{S}(x_i) \rangle\\
\functionback{\Phi}{\tensor{[s]_T}{}{\divf{S}(Q)}}{([s \cdot \delta_{1}(P)]_T\divf{S}(x_1), \dots , [s \cdot \delta_{m}(P)]_T\divf{S}(x_m))}
\end{gather*}
Betrachte nun also noch $(\Phi \circ (\tensor{1}{S}{d_{S}}))(I)$ näher. Sei dazu $P = \sum_{i = 1}^n s_iP_i \in I$ beliebig, somit gilt:
\begin{gather*}
\text{Wir können eine solche Summe als } P = \sum_{i = 1}^n s_iP_i = (s_1, \cdots s_n)
\begin{pmatrix}
P_1\\
\vdots\\
P_n
\end{pmatrix}
\text{ schreiben. Also:} \\
(\Phi \circ (\tensor{1}{S}{d_{S}}))(P)
= \sum_{i = 1}^n \divf{S}(s_iP_i)\\
= \sum_{i = 1}^n [s_i]_T \cdot \divf{S}(P_i) + [P_i]_T \cdot \divf{S}(s_i) \textit{\hspace{1,5em}(Leibnizregel)}\\
= \sum_{i = 1}^n [s_i]_T \cdot \divf{S}(P_i) + 0 \text{\hspace{2em}($[P_i]_T = 0$, \textit{da} $T = S/I$)} \\
= ([s_1]_T , \cdots , [s_n]_T)
\begin{pmatrix}
\divf{S}(P_1)\\
\vdots\\
\divf{S}(P_n)
\end{pmatrix}\\
=
([s_1]_T , \cdots , [s_n]_T)
\begin{pmatrix}
\delta_1(P_1) \divf{S}(x_1) & \dots & \delta_n(P_1) \divf{S}(x_n)\\
\vdots & \ddots & \vdots \\
\delta_1(P_n) \divf{S}(x_1) & \dots & \delta_n(P_1) \divf{S}(x_n)
\end{pmatrix}
\end{gather*}
Mit der Notation aus \cref{Jakobimatrizen} gilt somit:
\begin{gather*}
(\Phi \circ (\tensor{1}{S}{\divf{S}}))(I) = \left\lbrace (t_1, \cdots, t_n) J_{(P_1,\dots, P_n)} \, \vert (t_1 , \dots t_n) \in T^n \right\rbrace
\end{gather*}
Damit haben wir folgende Isomorphie gezeigt:
\begin{gather*}
\divR{T}{R} 
\simeq (\Tensor{T}{S}{\divR{S}{R}}) / (\tensor{1}{S}{d_{S}})(I) 
\simeq \left( \bigoplus_{i =1}^m S \langle \divf{S}(x_i) \rangle \right) / \left\lbrace (t) J_{(P_1,\dots, P_n)} \, \vert t \in T^n \right\rbrace
\end{gather*}
Da der Differentialraum von $T$ über $R$ bis auf eine eindeutige Isomorphie eindeutig ist, definiere diesen ab nun über diese Isomorphie. Anhand von $\Phi$ sehen wir, dass somit auch $\divf{T}$ die geforderte Form annimmt.
\end{proof}


\ \\

\begin{bsp}\comment{\label{Bsp Moduloraeume}} \textit{[Eigene Überlegungen]}\\
Kürze $\divf{S}(s)$ durch $dx$ ab.
\begin{itemize}
\item Betrachte dem Ring $S = \mathbb{Z}[x]/(3x) = \lbrace \nolinebreak{\sum_{i = 0}^n a_i xi \vert \, a_0 \in \mathbb{Z} \, \wedge \, a_i \in \mathbb{Z}_3 \text{ für } i \geq 1 \rbrace}$ als $\mathbb{Z}$-Algebra. Dann gilt:
\begin{gather*}
\divR{S}{\mathbb{Z}} = \left( S\divf{S}d(x) \right)/\lbrace s \cdot 3dx \vert s \in S \rbrace = S/(3\mathbb{Z}) dx = \mathbb{Z}_3[x] dx
\end{gather*}

\item Für $n \in \mathbb{N}$ mit $n \geq 2$ ist $S = \mathbb{Z}[x]/(nx)$ eine $\mathbb{Z}$-Algebra. Hierbei gilt:
\begin{gather*}
\divR{S}{\mathbb{Z}} = S/(n) dx = \mathbb{Z}_n[x] dx
\end{gather*}

\item Betrachte für $n \in \mathbb{N}$ mit $n \geq 2$ die $\mathbb{Z}$-Algebra $S = \mathbb{Z}[x,y]/(nxy)$. Für diese gilt:
\begin{gather*}
\divR{S}{\mathbb{Z}}
= (S dx \oplus S dy) / (Sny \cdot dx \oplus Snx \cdot dy)\\
= S/(ny)dx \oplus S/(nx)dy\\
= \mathbb{Z}/(ny)dx \oplus \mathbb{Z}/(nx) dy\\\
\end{gather*}

\item Betrachte für $n,m \in \mathbb{N}$ mit $n,m \geq 2$ die $\mathbb{Z}$-Algebra $S = \mathbb{Z}[x,y]/(nx,ny)$. Für diese gilt:
\begin{gather*}
\divR{S}{\mathbb{Z}}
= (S dx)/\lbrace (s_1,s_2)
\begin{pmatrix}
ndx\\
mdx
\end{pmatrix}
 \cdot dx \vert (s_1,s_2) \in S^2 \rbrace \\
= \mathbb{Z}/(n\mathbb{Z} + m\mathbb{Z})[x]dx 
= \mathbb{Z}_{ggT(n,m)}[x]dx
\end{gather*}

\item Betrachte für $n,m \in \mathbb{N}$ mit $n,m \geq 2$ die $\mathbb{Z}$-Algebra $S = \nolinebreak{\mathbb{Z}[x,y,z]/(nx + ny, mx + mz, mz^2)}$
\end{itemize}
\end{bsp}

\end{document}