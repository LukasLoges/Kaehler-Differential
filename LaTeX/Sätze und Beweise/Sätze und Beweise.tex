\documentclass[10pt,a4paper]{report}
\usepackage[utf8]{inputenc}
\usepackage{amsmath}
\usepackage{amsthm}
\usepackage{amsfonts}
\usepackage{amssymb}
\usepackage{tikz-cd}
\usepackage{calc}
\usepackage{setspace}
\usepackage[german]{babel}
\usetikzlibrary{babel}
\usepackage{cleveref}

\newcommand{\comment}[1]{}
\renewcommand{\baselinestretch}{1.1}

\newtheorem{satz}{Satz}
\newtheorem{theorem}[satz]{Theorem}
\newtheorem{prop}[satz]{Proposition}
\newtheorem{korrolar}[satz]{Korrolar}
\newtheorem{lemma}[satz]{Lemma}
\newtheorem{bem}[satz]{Bemerkung}
\newtheorem{definition}[satz]{Definition}

\newcommand{\functionfront}[3]{\nolinebreak{#1:#2 \longrightarrow #3}}
\newcommand{\functionback}[3]{\nolinebreak{#1:#2 \longmapsto #3}}
\newcommand{\function}[5]{\nolinebreak{#1:#2 \longrightarrow #3 \, , \, #4 \longmapsto #5}}
\newcommand{\divR}[2]{\Omega_{#1/#2}}
\newcommand{\tensor}[3]{#1 \otimes_#2 #3}

\newcommand{\colimes}[0]{\lim\limits_{ \longrightarrow }}
\newcommand{\infunctionfront}[3]{\nolinebreak{#1:#2 \hookrightarrow #3}}
\newcommand*{\defeq}{\mathrel{\vcenter{\baselineskip0.5ex \lineskiplimit0pt
                     \hbox{\scriptsize.}\hbox{\scriptsize.}}}%
                     =}
\newcommand*{\defshow}{\stackrel{!}{=}}
\newcommand{\kernel}[1]{KERN(#1)}
\newcommand{\immage}[1]{BILD(#1)}

\begin{document}
Proposition 16.3 aus Moduls of Differenzials
\begin{satz}
\raggedright
Sei $\functionfront{\pi}{S}{T}$ ein R-Algebrenephimorphismus mit Kern($\pi$):= I \\
Dann ist folgende Sequenz rechtsexakt: \\
\begin{center}
\begin{tikzcd}
I/I^2 \arrow[r, "f"] & \tensor{T}{S}{\divR{S}{R}} \arrow[r, "g"] & \divR{T}{R} \arrow[r] & 0
\end{tikzcd}
\end{center}
\begin{spacing}{1.5}
mit: $\function{f}{I/I^2}{{\tensor{T}{S}{\divR{S}{R}}}}{[a]_{I^2}}{\tensor{1}{S}{d_S(a)}}$\\
\leftskip2.3em $\function{g}{\tensor{T}{S}{\divR{S}{R}}}{\divR{T}{R}}{\tensor{b}{S}{d_S(c)}}{b \cdot (d \circ \pi)(c)}$
\end{spacing}
\end{satz}

\begin{proof} \ \\
\underline{$f$ ist wohldefiniert:} Seien $a,b\in I^2$. Zeige $f(a \cdot b)=0$ :
$$ f(a \cdot b) =
\tensor{1}{S}{( d_S \circ \pi )(a \cdot b)} =
\tensor{1}{S}{\pi(a) \cdot (d_S \circ \pi )(b) + \pi(b) \cdot ( d_S \circ \pi )(a)} =0$$
\underline{$D\pi$ ist surjektiv:}

\begin{center}
\begin{tikzcd}
\divR{S}{R} \arrow[r, " D\pi "]                & \divR{T}{R}                \\
S \arrow[r, " \pi "] \arrow[u, " d_S "] & T \arrow[u, " d_T "]
\end{tikzcd}
\end{center}
Da $\divR{S}{R}$ und $\divR{T}{S}$ jeweils von $d_S$ und $d_T$ erzeugt werden, vererbt sich die Surjektivität von $\pi$ auf $D\pi$. Somit ist auch $\tensor{1}{S}{D\pi}$ surjektiv.\\
\underline{$immage{f}=\kernel{g}$:}\\ Dies folgt direkt aus folgender Isomorphie: $(\tensor{T}{S}{\divR{S}{R}})/Im(f) \simeq \divR{T}{R} = \immage{f}$.
$$(\tensor{T}{S}{\divR{S}{R}})/Im(f) = (\tensor{T}{S}{\divR{S}{R}})/(\tensor{T}{S}{d_S(I)}) = \tensor{T}{S}{(\divR{S}{R}/d_S(I))} \simeq \tensor{T}{S}{d_S(S/I)} \simeq \tensor{T}{S}{d_T(T)}$$

\end{proof}
\end{document}
