\documentclass[10pt,a4paper]{report}
\usepackage[utf8]{inputenc}
\usepackage{amsmath}
\usepackage{amsthm}
\usepackage{amsfonts}
\usepackage{amssymb}
\usepackage{tikz-cd}
\usepackage{calc}
\usepackage{setspace}
\usepackage[german]{babel}
\usetikzlibrary{babel}
\usepackage{cleveref}

\newcommand{\comment}[1]{}
\renewcommand{\baselinestretch}{1.1}


\newcounter{Aussage}[chapter]

\newtheorem{satz}[Aussage]{Satz}
\newtheorem{theorem}[Aussage]{Theorem}
\newtheorem{prop}[Aussage]{Proposition}
\newtheorem{korrolar}[Aussage]{Korrolar}
\newtheorem{lemma}[Aussage]{Lemma}
\newtheorem{bem}[Aussage]{Bemerkung}
\newtheorem{definition}[Aussage]{Definition}

\newcommand{\functionfront}[3]{\nolinebreak{#1:#2 \longrightarrow #3}}
\newcommand{\functionback}[3]{\nolinebreak{#1:#2 \longmapsto #3}}
\newcommand{\function}[5]{\nolinebreak{#1:#2 \longrightarrow #3 \, , \, #4 \longmapsto #5}}
\newcommand{\divR}[2]{\Omega_{#1/#2}}
\newcommand{\Tensor}[3]{#1 \otimes_{#2} #3}
\newcommand{\tensor}[3]{#1 \otimes #3}
\newcommand{\lok}[2]{#1 [#2^{-1}]}
\newcommand{\loke}[3]{(#1,#2)_{mod\sim_{#3}}}

\newcommand{\colimes}[0]{\lim\limits_{ \longrightarrow }}
\newcommand{\infunctionfront}[3]{\nolinebreak{#1:#2 \hookrightarrow #3}}
\newcommand*{\defeq}{\mathrel{\vcenter{\baselineskip0.5ex \lineskiplimit0pt
                     \hbox{\scriptsize.}\hbox{\scriptsize.}}}%
                     =}
\newcommand*{\defshow}{\stackrel{!}{=}}
\newcommand{\kernel}[1]{kern(#1)}
\newcommand{\immage}[1]{im(#1)}

\begin{document}
\chapter{grundlegende Sätze}
Proposition 16.3 aus Moduls of Differenzials
\begin{satz} \label{prop16.3}
\raggedright
Sei $\functionfront{\pi}{S}{T}$ ein R-Algebrenephimorphismus mit Kern($\pi$):= I \\
Dann ist folgende Sequenz rechtsexakt: \\
\begin{center}
\begin{tikzcd}
I/I^2 \arrow[r, "f"] & \Tensor{T}{S}{\divR{S}{R}} \arrow[r, "g"] & \divR{T}{R} \arrow[r] & 0
\end{tikzcd}
\end{center}
\begin{spacing}{1.5}
mit: $\function{f}{I/I^2}{{\Tensor{T}{S}{\divR{S}{R}}}}{[a]_{I^2}}{\tensor{1}{S}{d_S(a)}}$\\
\textsc{\leftskip2.3em} $\function{g}{\Tensor{T}{S}{\divR{S}{R}}}{\divR{T}{R}}{\tensor{b}{S}{d_S(c)}}{b \cdot (d \circ \pi)(c)}$
\end{spacing}
\end{satz}

\begin{proof} \ \\
\underline{$f$ ist wohldefiniert:} Seien $a,b\in I^2$. Zeige $f(a \cdot b)=0$ :
$$ f(a \cdot b) =
\tensor{1}{S}{( d_S \circ \pi )(a \cdot b)} =
\tensor{1}{S}{\pi(a) \cdot (d_S \circ \pi )(b) + \pi(b) \cdot ( d_S \circ \pi )(a)} =0$$
\underline{$D\pi$ ist surjektiv:}

\begin{center}
\begin{tikzcd}
\divR{S}{R} \arrow[r, " D\pi "]                & \divR{T}{R}                \\
S \arrow[r, " \pi "] \arrow[u, " d_S "] & T \arrow[u, " d_T "]
\end{tikzcd}
\end{center}
Da $\divR{S}{R}$ und $\divR{T}{S}$ jeweils von $d_S$ und $d_T$ erzeugt werden, vererbt sich die Surjektivität von $\pi$ auf $D\pi$. Somit ist auch $\Tensor{1}{S}{D\pi}$ surjektiv.\\
\underline{$\immage{f}=\kernel{g}$:}\\ Dies folgt direkt aus folgender Isomorphie: $(\Tensor{T}{S}{\divR{S}{R}})/Im(f) \simeq \divR{T}{R} = \immage{f}$.
$$(\Tensor{T}{S}{\divR{S}{R}})/Im(f) = (\Tensor{T}{S}{\divR{S}{R}})/(\Tensor{T}{S}{d_S(I)}) = \Tensor{T}{S}{(\divR{S}{R}/d_S(I))} \simeq \Tensor{T}{S}{d_S(S/I)} \simeq \Tensor{T}{S}{d_T(T)}$$
\end{proof}

\chapter{Colimes}
\comment
{}
\begin{definition}
Sei $\mathcal{A}$ eine Kathegorie und $C \in \mathcal{A}$ ein Objekt
\begin{itemize}
\item Ein \underline{Diagramm} über $\mathcal{A}$ ist eine Kathegorie $\mathcal{B}$ zusammen mit einem Funktor $\functionfront{\mathcal{F}}{\mathcal{B}}{\mathcal{A}}$.
\item Ein \underline{Morphismus} $\functionfront{\psi}{\mathcal{F}}{C}$ ist eine Menge von Funktionen 
$\nolinebreak{\lbrace \psi_B \in Hom(F(B),C) \vert B \in \mathcal{B} \rbrace}$, wobei für alle $B_1,B_2 \in \mathcal{B}$ und $\varphi \in Hom(B_1,B_2)$ folgendes Diagramm kommutiert:
\begin{center}
\begin{tikzcd}
\mathcal{F}(B_1) \arrow[rrd, "\psi_{B_1}"] \arrow[dd, "\mathcal{F}(\varphi )"] &  &   \\
                                   &  & C \\
\mathcal{F}(B_2) \arrow[rru, "\psi_{B_2}"]                 &  &  
\end{tikzcd}
\end{center}
\item Der \underline{Colimes} $\colimes \mathcal{F}$ eines Diagramms $\functionfront{\mathcal{F}}{\mathcal{B}}{\mathcal{A}}$ ist ein Objekt $A \in \mathcal{A}$ zusammen mit einem Morphismus $\functionfront{\psi}{\mathcal{F}}{A}$ und folgender universellen Eigenschaft:
\begin{center}
\comment{$\forall Morphismen \functionfront{\psi '}{\mathcal{F}}{\mathcal{A}'}\exists ! \varphi \in Hom_{\mathcal{A}}(A,A') \forall B \in \mathcal{B}: \varphi \circ \psi_B = \psi'_B $}

für alle Morphismen $\functionfront{\psi '}{\mathcal{F}}{A'}$ existiert genau eine Funktion $\varphi \in Hom(A,A')$, sodass folgendes Diagramm kommutiert:
\begin{tikzcd}
  & \mathcal{F} \arrow[rd, "\psi"] \arrow[ld, "\psi '"'] &                            \\
A' &                                    & A \arrow[ll, "\exists ! \varphi "', dashed]
\end{tikzcd}
\end{center}

\end{itemize}
\end{definition}
Bei dem Colimes handelt es sich um ein Objekt aus $\mathcal{A}$, das die Eigenschaften besitzt, welche alle Objekte in \nolinebreak{$\lbrace \mathcal{F} (B) \vert B \in \mathcal{B} \rbrace$} gemein haben und einem dazugehörigen Morphismus, welcher die Eigenschaften der Funktionen aus \nolinebreak{$ \lbrace \mathcal{F}(f) \vert f \in \cup \lbrace\ Hom_{\mathcal{B}}(B,B') \vert B \in \mathcal{B} \rbrace \rbrace$} erhält. Der Colimes kann also unformal als ein Art Schnitt von $\mathcal{F}(\mathcal{B}) \subseteq \mathcal{A}$ gesehen werden.\\
Meistens handelt es sich bei einem Diagramm um eine Unterkathegorie $\mathcal{B} \subseteq \mathcal{A}$ zusammen mit dem Inklusionsfunktor $\infunctionfront{\mathcal{F}}{\mathcal{B}}{\mathcal{A}}$. In diesem Fall wird im folgendem zur Vereinfachung von dem Diagramm $\mathcal{B}$ gesprochen.\\
Bevor der Cokern weiter charakterisiert wird, zeigen wir zunächst, dass er durch die obige Definition eindeutig bestimmt ist.
\begin{lemma}
Seien $\mathcal{B},\mathcal{A}$ zwei Kategorien und $\functionfront{\mathcal{F}}{\mathcal{B}}{\mathcal{A}}$ ein Funktor, so git:\\ 
Im Falle der Existenz sind $\colimes \mathcal{F}$ und der dazugehörige Morphismus $\functionfront{\psi}{\mathcal{F}}{A}$ bis auf Isomorphie eindeutig bestimmt.
\end{lemma}
\begin{proof}
Seien $A_1 \in \mathcal{A}, (\functionfront{\psi_1}{\mathcal{F}}{A_1}) $ und $A_ \in \mathcal{A} , (\functionfront{\psi_2}{\mathcal{F}}{A_2}) $ beide gleich $\colimes \mathcal{F}$:\\
Durch die universelle Eigenschaft des Colimes erhalte die eindeutig bestimmten Funktionen $\varphi_1 \in Hom_{\mathcal{A}}(A_1,A_2)$ und $\varphi_2 \in Hom_{\mathcal{A}}(A_2,A_1)$, für die folgende Diagramme kommutieren:

\comment{$\functionfront{\varphi_1}{\mathcal{A}_1}{\mathcal{A}_2}$ und $\functionfront{\varphi_2}{\mathcal{A}_2}{\mathcal{A}_1}$}
\begin{center}
\begin{tikzcd}
  & \mathcal{F} \arrow[rd, "\psi_1"] \arrow[ld, "\psi_2"'] &                            &  &   & \mathcal{F} \arrow[rd, "\psi_2"] \arrow[ld, "\psi_1"'] &                            \\
A_2 &                                    & A_1 \arrow[ll, "\exists ! \varphi_1"', dashed] &  & A_1 &                                    & A_2 \arrow[ll, "\exists ! \varphi_2"', dashed]
\end{tikzcd}
\end{center}
\begin{flushleft}
Wende nun die Universelle Eigenschaft von $\psi_1$ auf $\psi_1$ selbst an und erhalte $id_{A_1} = \varphi_2 \circ \varphi_1$. Analog erhalte auch $id_{A_2} = \varphi_1 \circ \varphi_2$.
\end{flushleft}
\begin{center}
\begin{tikzcd}
  & \mathcal{F} \arrow[rd, "\psi_1"] \arrow[ld, "\psi_1"'] &                            \\
A_1 &                                    & A_1 \arrow[ll, "\exists ! id_{A_1} = \varphi_2 \circ \varphi_1"', dashed]
\end{tikzcd}
\end{center}
\end{proof}

Im folgenden beschäftigen wir uns mit dem besonderen Fall des $\colimes \functionfront{\mathcal{F}}{\mathcal{B}}{\mathcal{A}}$, bei welchem $\mathcal{B}$ eine Unterkategorie von $\mathcal{A}$ ist. Dazu untersuchen wir bei einer gegebenen Kategorie $\mathcal{A}$ das Coprodukt einer Menge von Objekten $A_i \in \mathcal{A}$, sowie den Differenzkokern zweier Morphismen $f,g \in Hom_{\mathcal{A}}(C_1,C_2)$.

\begin{definition} \label{altDifferenzkoerndef}
Sei $\mathcal{A}$ eine Kathegorie.
\begin{itemize}
\item Das Coprodukt von $ \lbrace B_i \rbrace \subseteq \mathcal{A}$ wird durch $\prod_i \lbrace B_i \rbrace := \colimes(\infunctionfront{\mathcal{F}}{\mathcal{B}}{\mathcal{A}})$ definiert, 
wobei $\mathcal{B}$ $\lbrace B_i \rbrace$ als Objekte und die Identitätsabbildungen $\functionfront{id_{B_i}}{B_i}{B_i}$ als Morphismen enthält.
\item Der Differenzkokern (oder auch Koequilizer) von $f,g \in Hom_{\mathcal{A}}(C_1,C_2)$ wird durch $\colimes(\infunctionfront{\mathcal{F}}{\mathcal{C}}{\mathcal{A}})$ definiert,
wobei $\mathcal{C}$ $\lbrace C_1,C_2 \rbrace$ als Objekte und $ \lbrace f,g \rbrace \defeq Hom_{\mathcal{C}}(C_1,C_2)$ als Morphismen enthält.
\end{itemize}
\end{definition}

In der Einführung des Differenzkokern`s in \cref{altDifferenzkoerndef} ist deutliche zu sehen, inwiefern dieser ein Colimes ist. Um mit dem Differenzkokern zu arbeiten wird er allerdings meist anders Eingeführt. Daher betrachten auch wir ab nun eine andere, aber äquivalente Definition des Differenzkokern`s.

\begin{lemma}\label{Differenzenkokerndef} Sei $\mathcal{A}$ eine Kathegorie mit $C_1,C_2 \in Hom_{\mathcal{A}}(C_1,C_2)$, so sind folgende Formulierungen äquivalent zur Definition des Differenzenkokern $Z \defeq \colimes \functionfront{\mathcal{F}}{\mathcal{C}}{\mathcal{A}}$
\begin{itemize}
\item[1.] Es existiert ein Morphismus $\functionfront{\psi}{\mathcal{F}}{Z}$, mit der Eigenschaft, dass für alle Morphismen $\functionfront{\psi '}{\mathcal{F}}{Z '}$ genau ein $\varphi \in Hom_{\mathcal{A}}(Z,Z')$ mit $\varphi \circ \psi = \psi '$ existiert.
\item[2.] Es existiert ein $q \in Hom_{\mathcal{A}}(C_2,Z)$ mit $q \circ f = q \circ g$ und der Eigenschaft, dass für alle Morphismen $q' \in Hom_{\mathcal{A}}(C_2,Z)$ mit $q' \circ f = q' \circ g$ genau ein $\varphi \in Hom_{\mathcal{A}}(Z,Z')$ mit $\varphi \circ q = q'$ existiert.
\begin{center}
\begin{tikzcd}
C_1 \arrow[r, "{f,g}"] \arrow[r] & C_2 \arrow[r, "q"] \arrow[rd, "q'"] & Z \arrow[d, "\exists !\varphi", dashed] \\
                                 &                                     & Z'                                     
\end{tikzcd}
\end{center}
\end{itemize}
\end{lemma}
\begin{proof}
\ \\
\begin{itemize}
\item \underline{1 $\Rightarrow$ 2:}
\begin{itemize}
\item[] Da $\functionfront{\psi}{\mathcal{F}}{T}$ ein Morphismus ist, gilt für $\lbrace f,g \rbrace = Hom_{\mathcal{C}}(C_1,C_2)$:\\ $\psi_{C_1} = \psi_{C_2} \circ f = \psi_{C_1} \circ \psi_{C_2}$, setze also 
 $q  \defeq \psi_{C_2}$.
\item[] Sei nun $q' \in Hom_{\mathcal{A}}(C_2,T)$ mit der Eigenschaft $q' \circ f = q' \circ g$ gegeben:\\
 Definiere den Morphismus $\functionfront{\psi '}{\mathcal{F}}{T}$ als $\lbrace \psi_1 = q' \circ f , \psi_2 = q' \rbrace$,  somit folgt direkt aus der Universellen Eigenschaft von $\psi$, dass genau ein $\varphi \in Hom_{A}(C_2,T)$ existiert, mit $ \varphi \circ q = q '$.
\end{itemize}
\item \underline{2 $\Rightarrow$ 1:}
\begin{itemize}
\item[] Definiere $\functionfront{\psi }{\mathcal{F}}{T}$ als $\lbrace \psi_1 = q \circ f , \psi_2 = q \rbrace$.
Durch die Eigenschaft von $q$ gilt $\psi_{C_1} = \psi_{C_2} \circ f = \psi_{C_2} \circ g$.
\item[] Sei nun $\functionfront{\psi '}{\mathcal{F}}{\mathcal{A}}$ ein beliebiger Morphismus.\\
Definiere $d' \defeq \psi '$, somit existiert durch die Eigenschaft von $d$ genau ein $\varphi \in Hom_{\mathcal{A}}(C_2,T)$ mit $\varphi \circ q = q'$. \\
$\Rightarrow \varphi \circ \psi_2 = \psi '_2$ 
und $\varphi \circ \psi_1 = \varphi \circ \psi_2 \circ f = \varphi \circ \psi '_2 \circ f = \varphi \circ \psi '_1$
\end{itemize}
\end{itemize}
\end{proof}

Wenn im weiteren Verlauf von dem Differenzenkokern zweier Homomorphismen $\functionfront{f,g}{C_1}{C_2}$ gesprchen wird, meinen wir damit den Homomorphismus $\functionfront{q}{C_2}{T}$ aus \cref{Differenzenkokerndef}.

\begin{bem}
Seien $f,g \in Hom_{\mathcal{A}}(S_1,S_2)$ R-Algebra-Homomorphismen, so können wir für den Differenzenkokern $\functionfront{q}{S_2}{T}$ für ein beliebiges $S_1$-Modul das Tensorprodukt $\Tensor{T}{C_1}{M}$ definieren.
\begin{gather*}
\textit{für } s_1 \in S_1 \textit{ und } \tensor{t}{S_1}{m}) \in \Tensor{T}{C_1}{M} \textit{ gilt: }\\
s_1 \cdot (\tensor{t}{S_1}{m}) = \tensor{((q \circ f)(s_1)) \cdot t}{S_1}{m} = \tensor{((q \circ g)) \cdot (s_1)t}{S_1}{m}
\end{gather*}
\end{bem}

Für einen kommutativen Ring $R$ definieren wir $\mathcal{M}$ als die Kategorie der $R-Module$
\begin{lemma}\label{R-Modul-Colimiten}
In $\mathcal{M}$ existieren Coprodukte und Differenzenkokerne, wobei:
\begin{itemize}
\item[\textbf{1.}] das Coprodukt $\prod_i M_i = \bigoplus_i M_i$ entspricht der direkten Summe
\item[\textbf{2.}] der Differenzenkokern zweier Homomorphismen $\functionfront{f,g}{M_1}{M_2}$ entspricht dem Kokern $M/\immage{f-g}$ der Differenzenabbildung.
\end{itemize}
\end{lemma}
\begin{proof}
\begin{itemize}

für \textit{\textbf{1.}} Sei $\functionfront{\phi}{\lbrace M_i \rbrace}{\mathcal{M}}$ ein beliebiger Morphismus. Zeige: \\
\begin{center}
\begin{tikzcd}
  & \infunctionfront{\mathcal{F}}{\lbrace M_i \rbrace}{\mathcal{M}} \arrow[rd, "\psi_i"] \arrow[ld, "\psi_i"'] &                                            \\
M' &                                              & \bigoplus_i M_i \arrow[ll, "\exists ! \varphi"', dashed]
\end{tikzcd}\\
\end{center}
Für ein beliebiges $i$ existiert genau ein $\function{\varphi_i}{M_i \bigoplus 0}{M'}{(0,...,0,m_i,0,...,0}{\psi_i '(m_i)}$
 mit $\psi_i ' = \psi_i \circ \varphi_i$\\
$\Rightarrow  \exists ! \function{\varphi}{\bigoplus_i M_i}{M'}{(m_1,...,m_n)}{\sum_i \psi_i(m_i)}$\\
\ \\
\textit{\textbf{2.}} ist Analog zu \cref{R-Algebra-Colimiten}
\end{itemize}
\end{proof}

\begin{prop} \label{R-Algebra-Colimiten}
in der Kategorie der R-Algebren existieren Coprodukte und Differenzenkokerne, wobei:
\begin{itemize}
\item[\textbf{1.}] Das Coprodukt einer endlichen Familie von $R-Algebren$ $\lbrace B_i \rbrace_{i \in \Lambda}$ entspricht deren Tesorprodukt $\bigotimes_{i \in \Lambda} B_i$. 
\item[\textbf{2.}] Der Differenzenkokern zweier R-Algebra-Homomorphismen $\functionfront{f,g}{S_1}{Ss_2}$ einspricht dem Homomorphismus $\function{q}{S_2}{C_2/Q}{y}{[y]}$, wobei $Q \defeq \lbrace f(x) - g(x)\mid x \in C_2 \rbrace$ das Bild der Differenz von $f$ und $g$ ist.
\end{itemize}
\end{prop}

\begin{proof}
Zu \textit{\textbf{1.}}:\\
Sei $\infunctionfront{\mathcal{F}}{\lbrace B_i \rbrace}{(R-Algebren)}$ der Inklusionsfunktor. Nutze die universellen Eigenschaften des Tensorproduktes und des Kähler-Differenzials um einen Isomorphismus zwischen $\colimes \mathcal{F}$ und $\bigotimes_{i \in \Lambda} B_i$ zu finden.\\ Es sind der Morphismus $\functionfront{\psi}{\mathcal{F}}{\colimes \mathcal{F}}$ und die bilineare Abbildung $\functionfront{g}{\oplus_i B_i}{\otimes_i B_i}$ gegeben.\\
Konstruiere den Morphismus $\functionfront{\psi'}{\mathcal{F}}{\otimes_i B_i}$ durch $\function{\psi'_i}{B_i}{\otimes_i B_i}{b_i}{g(1,..,1,b_i,1,..,1)}$ für $i \in \lambda$ und die bilineare Abbildung $\function{f}{\oplus_i B_i}{\colimes \mathcal{F}}{b}{\prod_i \psi_i b_i}$.\\
\ \\
Somit liefern uns die universellen Eigenschaften folgende zwei R-Algebra-Homomorphismen:
\begin{gather*}
\functionfront{\varphi}{\colimes \mathcal{F}}{\bigotimes_i B_i} \\
\functionfront{\phi}{\bigotimes_i B_i}{\colimes \mathcal{F}}.
\end{gather*}
\begin{center}
\begin{tikzcd}
  & \mathcal{F} \arrow[rd, "\psi"] \arrow[ld, "\psi'"'] &                                            &  &   & \oplus_i B_i \arrow[rd, "g"] \arrow[ld, "f"'] &                                         \\
\otimes_i B_i &                                           & \colimes \mathcal{F} \arrow[ll, "\exists ! \varphi"', dashed] &  & \colimes \mathcal{F} &                                    & \otimes_i B_i \arrow[ll, "\exists ! \phi"', dashed]
\end{tikzcd}
\end{center}
Die Eindeutigkeit der universellen Eigenschaften liefert uns, das $\varphi$ und $\phi$ zueinander Inverse sind und somit haben wir unsere gesuchten Isomorphismen zwischen $\colimes \mathcal{F}$ und $\bigotimes_i B_i$ gefunden.
\begin{center}
\begin{tikzcd}
         & \mathcal{F} \arrow[rd, "\psi"] \arrow[ld, "\psi"'] &                                                                              &  &                  & \bigoplus_i B_i \arrow[rd, "g"] \arrow[ld, "g"'] &                                                                                     \\
\colimes \mathcal{F} &                                                    & \colimes \mathcal{F} \arrow[ll, "\exists ! id_{\colimes \mathcal{F}} = \phi \circ \varphi ", dashed] &  & \bigotimes_i B_i &                                                  & \bigotimes_i B_i \arrow[ll, "\exists ! id_{\bigotimes_i B_i} = \varphi \circ \phi "', dashed]
\end{tikzcd}
\end{center}
\ \\
Zu \textit{\textbf{2.}}:\
$$q \circ f = q \circ g \textit{ gilt, da } \kernel{q} = Q = \lbrace f(x) - g(x)\mid x \in C_2 \rbrace$$

Sei nun eine Funktion $q' \in Hom_{\mathcal{A}}(C_2,T')$ mit $q' \circ f = q' \circ$ gegeben.
\begin{gather*}
q' \circ (f - g) = 0 \Rightarrow Q\textit{ ist Untermodul von }Q' \defeq \kernel{q'}.\\
\textit{Nach HOMOMORPHIESATZ [kommutative Algebra 2.10] gilt somit } \nolinebreak{C_2/Q' \simeq (C_2/Q)/(Q'/Q))}.\\
\Rightarrow \function{q'}{C_2}{(C_2/Q)/(Q'/Q))}{y}{[y]'} \textit{ ist eine isomorphe Darstellung von } \functionfront{q'}{C_2}{T'}\\
\Rightarrow \exists ! \function{\varphi}{C_2/Q}{(C_2/Q)/(Q'/Q)}{[y]}{[y]'}\textit{ mit }(\varphi \circ q) = q'.
\end{gather*}
\end{proof}

Verbinde nun den Colimes mit dem Kählerdifferenziel:\\
Differenzialekokerne bleiben unter der Bildung von Differenzialen erhalten:
Korrolar 16.7 aus Moduls of Differenzials:
\comment{Beide Beweise sind sehr kurz gefasst}
\begin{prop} \comment{\label{Kählerdifferenzial des Colimes}}
\ \\
\begin{itemize}
\item[\textbf{1.}]
Sei $T = \otimes_{i \in \Lambda} S_i$ das Coprodukt der R-Algebren $S_i$.\\
Dann gilt:
\begin{gather*}
\divR{T}{R} \simeq \bigoplus_{i\in \Lambda} ( \Tensor{T}{S_i}{\divR{S_i}{R}} )
\end{gather*}
\item[\textbf{2.}]
Seien $S_1,S_2$ R-Algebren und $\functionfront{\varphi,\varphi'}{S_1}{S_2}$ R-Algebra-Homomorphismen. Sei weiter $\functionfront{q}{S_2}{T}$ der Differenzenkokern von $\varphi$,$\varphi '$.
Dann ist folgende Sequenz rechtsexakt:\\
\begin{center}
\begin{tikzcd}
\Tensor{T}{S_1}{\divR{S_1}{R}} \arrow[r, "f"] & \Tensor{T}{S_2}{\divR{S_2}{R}} \arrow[r, "g"] & \divR{T}{R} \arrow[r] & 0
\end{tikzcd}
\begin{gather*}
\textit{mit: } \function{f}{\tensor{T}{S_1}{\divR{S_1}{R}}}{\Tensor{T}{S_2}{\divR{S_2}{R}}}{\tensor{t}{S_2}{d_{S_1}(x_1)}}{\tensor{t}{S_2}{d_{S_2}(\varphi(x_1) - \varphi(x_2))}}\\
\function{g}{\Tensor{T}{S_2}{\divR{S_2}{R}}}{\divR{T}{R}}{\tensor{t}{S_2}{d_{S_2}(x_2)}}{(d_{S_2}\circ q)(x_2)}
\end{gather*}
\end{center}
\end{itemize}
\end{prop}
\begin{proof}\ \\
Für \textit{\textbf{1.}} finde durch die Universelle Eigenschaft des Kählerdifferenzials Isomorphismen $ \divR{T}{R} \longleftrightarrow \bigoplus_{i \in \Lambda} ( \Tensor{T}{S_i}{\divR{S_i}{R}} )$.\\
Definiere das Differenzial $\function{e}{T}{\sum_{i \in \Lambda} \Tensor{T}{S_i}{\divR{S_i}{R}}}{(\tensor{s_i}{R}{...})}{(\tensor{1}{S_i}{d_{S_1},...)}}$ und erhalte dadurch
\begin{center}
\begin{tikzcd}
T \arrow[rd, "e"'] \arrow[r, "d_T"] & \divR{T}{R} \arrow[d, "\exists ! \varphi", dashed] \\
                                    & \sum_{i\in \Lambda} \Tensor{T}{S_i}{\divR{S_i}{R}}                                       
\end{tikzcd}
$\functionfront{\varphi}{\divR{T}{R}}{\bigoplus_{i\in \Lambda} ( \Tensor{T}{S_i}{\divR{S_i}{R}} )}$.
\end{center}
Definiere nun das Differenzial $k: S_i \hookrightarrow T \longrightarrow \divR{T}{R}$ und erhalte dadurch
\begin{center}
\begin{tikzcd}
S_i \arrow[rd, "k"'] \arrow[r, "d_{S_i}"] & \divR{S_i}{R} \arrow[d, "\exists ! k'", dashed] \arrow[r, "a"] & \Tensor{T}{S_i}{\divR{S_i}{R}} \arrow[ld, "\phi_i"] \\
                                          & \divR{T}{R}                                                    &                     
\end{tikzcd}
$\functionfront{\phi_i}{\bigoplus_{i\in \Lambda} ( \Tensor{T}{S_i}{\divR{S_i}{R}} )}{\divR{T}{R}}$\\
\begin{gather*}
\function{\phi}{\sum_{i\in \Lambda} ( \Tensor{T}{S_i}{\divR{S_i}{R}})}{\divR{T}{R}}{(\tensor{t_1}{S_1}{d_{S_i}s_1},...)}{\prod_{i\in \Lambda} t_i \cdot \phi_i(d_{S_i}(s_i))}. 
\end{gather*}
\end{center}
Damit haben wir zwei zueinander inverse Funktionen $\varphi ,\phi$ gefunden.\\
$\Rightarrow \divR{T}{R} \simeq \bigoplus_{i\in \Lambda} ( \Tensor{T}{S_i}{\divR{S_i}{R}} )$\\
\ \\
Für \textit{\textbf{2.}} Wende \cref{prop16.3} auf den Differenzenkokern $\functionfront{q}{S_2}{S_2/Q}$ an und erhalte dadurch eine exakte Sequenz, welche ähnlich zu der gesuchten ist:
\begin{center}
\begin{tikzcd}
Q/Q^2 \arrow[r, "f'"] & \tensor{T}{S_2}{\divR{S_2}{R}} \arrow[r, "g"] & \divR{T}{R} \arrow[r] & 0
\end{tikzcd}
\end{center}
mit $\function{f'}{Q/Q^2}{{\Tensor{T}{S_2}{\divR{S}{R}}}}{[s_2]_{Q^2}}{\tensor{1}{S_2}{d_{S_2}(s_2)}}$.\\
Somit gilt $\immage{f} = \Tensor{T}{S_2}{d_{S_2}(Q)} = \immage{f'}$.\\
$\Rightarrow$ die gesuchte Sequenz ist exakt.
\end{proof}

\chapter{Aufgaben}
Exersize A6.7
\begin{lemma}\comment{\label{Lokalisierung als Colimes}}
Seit $S$ eine $R-Algebra$ und $U \subseteq S$ multiplikativ abgeschlossen.
Dann gilt:
\begin{gather*}
 S[U^{-1}] = \colimes (\infunctionfront{\mathcal{F}}{\mathcal{B}}{(R-Algebren)})
\end{gather*}
Wobei $\mathcal{B}$ aus den Objekten $\lbrace \lok{S}{t} \vert t \in U \rbrace$ und den Morphismen\\
$\lok{S}{t} \longrightarrow \lok{S}{tt'}, \loke{s}{t^n}{t} \longmapsto \loke{st'^n}{t^nt'^n}{(tt')} \,
\forall t,t' \in U$ besteht.\\
\end{lemma}
\begin{proof}
Sei $\functionfront{\psi}{\mathcal{F}}{A}$ der Colimes von $\mathcal{F}$. Zeige $\lok{S}{U} \simeq A$, definiere dazu:
\begin{gather*}
\functionfront{\psi'}{\mathcal{F}}{\lok{S}{U}}\\
\function{\psi'_{\lok{S}{t}}}{\lok{S}{t}}{\lok{S}{t}}{\loke{s}{t^n}{t}}{\loke{s}{t^n}{U}}
\end{gather*}
$\psi'$ ist ein Morphismus, da für beliebige $t,t' \in U$ und $s \in S$ gilt:
$$\loke{s}{t^n}{U} = \loke{st'^n}{t^nt'^n}{U}$$
Durch die Universelle Eigenschaft des Colimes, erhalten wir den Homomorphismus $\functionfront{\varphi}{A}{\lok{S}{U}}$.
\begin{center}
\begin{tikzcd}
            & \mathcal{F} \arrow[rd, "\psi"] \arrow[ld, "\psi'"'] &                                            \\
{S[U^{-1}]} &                                                     & A \arrow[ll, "\exists ! \varphi"', dashed]
\end{tikzcd}
\end{center}
Für $\functionfront{\phi}{S[U^{-1}]}{A}$ benötigen wir kleinere Vorüberlegungen.\\
Zunächst können wir jedes Element $(s,u)_{mod\sim_{U}} \in \lok{S}{U}$ als $\psi_{\lok{S}{t}}(\loke{s}{t}{t})$ schreiben.\\
\comment{\label{wobei u = t}}
Weiter gilt für alle $s_1,s_2 \in S , \, t_1,t_2 \in U$: 
\begin{align*}
\psi'_{\lok{S}{t}}(\loke{s_1}{t_1}{t}) = \psi'_{\lok{S}{t}}(\loke{s_2}{t_2}{t})\\
\Rightarrow  \exists u \in U: (s_1t_1 - s_2t_2) \cdot u = 0\\
\Rightarrow  \loke{s_1u}{t_1u}{tu} = \loke{s_2u}{t_2u}{tu}\\
\Rightarrow  \psi_{\lok{S}{t}}(\loke{s_1}{t_1}{t}) = \psi_{\lok{S}{t}}(\loke{s_2}{t_2}{t})
\end{align*}
Mit diesem Wissen können wir den R-Algebra-Homomorphismus $\functionfront{\phi}{\lok{S}{U}}{A}$ definieren:
\begin{gather*}
\function{\phi}{\lok{S}{U}}{A}{\psi'_{\lok{S}{t}}(\loke{s}{t}{t})}{\psi_{\lok{S}{t}}(\loke{s}{t}{t})}
\end{gather*}
$\phi \circ \varphi = id_A$ ergibt sich direkt aus der Universellen Eigenschaft des Colimes:
\begin{center}
\begin{tikzcd}
  & \mathcal{F} \arrow[rd, "\psi"] \arrow[ld, "\psi"'] &                                                              \\
A &                                                    & A \arrow[ll, "\exists ! id_A = \phi \circ \varphi"', dashed]
\end{tikzcd}
\end{center}
Für $\varphi \circ \phi = id_{\lok{S}{U}}$ wähle beliebige $s \in S , t \in U$, für diese gilt:
\begin{gather*}
(\varphi \circ \phi)(\psi(\loke{s}{t}{t})) =
 \varphi (\psi'(\loke{s}{t}{t}) =
  \psi(\loke{s}{t}{t})
\end{gather*}
Damit haben wir gezeigt, dass $\varphi,\phi$ Isomorphismen sind und somit $A \simeq \lok{S}{U}$ gilt.\\
Da der Colimes bis auf Isomorphie eindeutig ist, definiere ab sofort $\lok{S}{U}$ als den eindeutigen Colimes von
 $\functionfront{\mathcal{F}}{\mathcal{B}}{(R-Algebren)}$.
 \end{proof}
\end{document}
