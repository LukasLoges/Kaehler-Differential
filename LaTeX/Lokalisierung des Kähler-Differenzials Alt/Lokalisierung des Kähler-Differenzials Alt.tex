\documentclass[10pt,a4paper]{report}
\usepackage[utf8]{inputenc}
\usepackage{amsmath}
\usepackage{amsthm}
\usepackage{amsfonts}
\usepackage{amssymb}
\usepackage{tikz-cd}
\usepackage{calc}
\usepackage{setspace}
\usepackage[german]{babel}
\usetikzlibrary{babel}
\usepackage{cleveref}

\newcommand{\comment}[1]{}
\renewcommand{\baselinestretch}{1.1}


\newcounter{Aussage}[chapter]

\newtheorem{satz}[Aussage]{Satz}
\newtheorem{theorem}[Aussage]{Theorem}
\newtheorem{prop}[Aussage]{Proposition}
\newtheorem{korrolar}[Aussage]{Korrolar}
\newtheorem{lemma}[Aussage]{Lemma}
\newtheorem{bem}[Aussage]{Bemerkung}
\newtheorem{definition}[Aussage]{Definition}

\newcommand{\functionfront}[3]{\nolinebreak{#1:#2 \longrightarrow #3}}
\newcommand{\functionback}[3]{\nolinebreak{#1:#2 \longmapsto #3}}
\newcommand{\function}[5]{\nolinebreak{#1:#2 \longrightarrow #3 \, , \, #4 \longmapsto #5}}
\newcommand{\divR}[2]{\Omega_{#1/#2}}
\newcommand{\Tensor}[3]{#1 \otimes_{#2} #3}
\newcommand{\tensor}[3]{#1 \otimes #3}
\newcommand{\lok}[2]{#1 [#2^{-1}]}
\newcommand{\loke}[3]{(\frac{#1}{#2})_{_#3}}
\comment{\newcommand{\loke}[3]{(#1,#2)_{mod\sim_{#3}}}}

\newcommand{\colimes}[0]{\lim\limits_{ \longrightarrow }}
\newcommand{\infunctionfront}[3]{\nolinebreak{#1:#2 \hookrightarrow #3}}
\newcommand*{\defeq}{\mathrel{\vcenter{\baselineskip0.5ex \lineskiplimit0pt
                     \hbox{\scriptsize.}\hbox{\scriptsize.}}}%
                     =}
                     
                     
\newcommand*{\defeqr}{= \mathrel{\vcenter{\baselineskip0.5ex \lineskiplimit0pt
                     \hbox{\scriptsize.}\hbox{\scriptsize.}}}}


\newcommand*{\defshow}{\stackrel{!}{=}}
\newcommand{\kernel}[1]{kern(#1)}
\newcommand{\immage}[1]{im(#1)}

\begin{document}
\begin{bem}\label{Vereinfachung des Kolimes}
Seien $\mathcal{B} \nsubseteq \mathcal{A}$ zwei Kategorien und $\functionfront{\mathcal{F}}{\mathcal{B}}{\mathcal{A}}$ ein Funktor.\\
Dann gilt im Falle der Existenz $\colimes \mathcal{F} = \colimes \mathcal{F}(\mathcal{A})$ 
\end{bem}


\begin{theorem}\comment{\label{Lokalisierung des Kähler-Differenzials}}
Sei $S$ eine $R-Algebra$ und $U \subseteq S$ multiplikativ abgeschlossen.
Dann gilt:
\begin{gather*}
\divR{\lok{S}{U}}{R} \simeq \Tensor{\lok{S}{U}}{S}{\divR{S}{R}} \text{, Wobei:}\\
 d_{\lok{S}{U}}(\loke{1}{u}{U}) \longmapsto -\tensor{\loke{1}{u^2}{U}}{S}{d_S(u)}
\end{gather*}
\end{theorem}
\begin{proof}
Wir wollen THEOREM16.8 \comment{\label{THEOREM16.8}} auf $\mathcal{B} = \lbrace \lok{S}{t} \vert t \in U \rbrace$ aus \cref{Lokalisierung von Algebren als Kolimes} anwenden.\\
Zeige also zunächsten den einfacheren Fall $\divR{\lok{S}{t}}{R} \simeq \Tensor{\lok{S}{t}}{S}{\divR{S}{R}}$ für ein beliebiges $t \in U$:
\begin{itemize}
\item[]
Verwende die Existenz des Isomorphismus \comment{$\functionfront{\alpha}{\lok{S}{t}}{S[x]/(tx -1)}$ und} $\functionfront{\beta}{S[x]/(tx -1)}{\lok{S}{t}}$. Weiter gilt nach PROPOSITION16.6 \comment{\label{Korrolar16.6}} $\divR{S[x]}{R} \simeq \Tensor{S[x]}{S}{\divR{S}{R}} \oplus S[x]d_{S[x]}(x)$. Somit folgt:
\begin{gather*}
\divR{\lok{S}{t}}{R} \\
 \simeq (\Tensor{S[x]}{S}{\divR{S}{R}} \oplus S[x] dx) / ((tx - 1) \cdot d_{S[x]}(tx - 1)) \\
  \simeq (\Tensor{S[x]/(tx -1)}{S}{\divR{S}{R}} \oplus (S[x]/(tx - 1)) dx) / (td_{S[x]}(x) + xd_{S[x]}(t)) \\
   \simeq (\Tensor{\lok{S}{t}}{S}{\divR{S}{R}} \oplus \lok{S}{t} d_{S[x]}(x) / (td_{S[x]}(x) + xd_{S[x]}(t)) \defeqr M
\end{gather*}
Zeige, dass sich jedes Element aus $M$ eindeutig durch ein Element aus $\Tensor{\lok{S}{t}}{S}{\divR{S}{R}} \oplus 0 $ darstellen lässt.\\
Sei dazu $[(\loke{s}{t^n}{t}d_{S}(s'') , \loke{s'}{t^{n'}}{t}d_{S[x]}(x) )]$ ein beliebiger Erzeuger von $M$. Somit gilt:
\begin{gather*}
\beta(x) = \loke{1}{t}{t} \textit{ und } sd_{S}(x) + xd_{S}(t) = 0 \\
 \Rightarrow d_{S}(x) = -\loke{1}{t^2}{t} \\
  \Rightarrow [(\loke{s}{t^n}{t}d_{S}(s'') , \loke{s'}{t^{n'}}{t}d_{S[x]}(x) )] = [( \loke{s}{t^n}{t}d_{S}(s'') - \loke{s'}{t^{n' +2}}{t} d_{s}(t), 0 )]
\end{gather*}
Damit ist $\Tensor{\lok{S}{t}}{S}{\divR{S}{R}}$ ein Repräsentantensystem von $M$ und es existiert ein Isomorphismus
\begin{gather*}
\function{\gamma_t}{\divR{\lok{S}{t}}{R}}{\Tensor{\lok{S}{t}}{S}{\divR{S}{R}}}{d_{\lok{S}{t}}(\loke{1}{t}{t})}{\tensor{\loke{-1}{t^2}{t}}{S}{d_S(t)}}
\end{gather*}


\comment{
Damit ist $\Tensor{\lok{S}{t}}{S}{\divR{S}{R}}$ ein Repräsentantensystem von $M$ und es gilt:
\begin{gather*} 
\divR{\lok{S}{t}}{R} \comment{\simeq \Tensor{S[x]/(tx -1)}{S}{\divR{S}{R}}} \simeq \Tensor{\lok{S}{t}}{S}{\divR{S}{R}}
\end{gather*}
Sei $\functionfront{\gamma_t}{\divR{\lok{S}{t}}{R}}{\Tensor{\lok{S}{t}}{S}{\divR{S}{R}}}$ ein Isomorphismus (für beliebige $t \in U$).
}
\end{itemize}


Zeige nun den Allgemeinen Fall $\divR{\lok{S}{U}}{R} \simeq \Tensor{\lok{S}{U}}{S}{\divR{S}{R}}$:

Wähle $\mathcal{B} = \lbrace \lok{S}{t} \vert t \in U \rbrace$ wie in \cref{Lokalisierung von Algebren als Kolimes}, sodass $\colimes \mathcal{B} = \lok{S}{U}$ gilt.\\
Mit THEOREM16.8 \comment{\label{THEOREM16.8}} folgt somit:
\begin{gather*}
\divR{\lok{S}{U}}{R}  = \colimes{\mathcal{F}} \text{ mit:}\\
\function{\mathcal{F}}{\mathcal{B}}{(\lok{S}{U} - Module)}{\lok{S}{t}}{\tensor{\lok{S}{U}}{\lok{S}{t}}{\divR{\lok{S}{t}}{R}}}\\
( \functionfront{\varphi}{\lok{S}{t}}{\lok{S}{tt'}} )\\ \longmapsto 
( \functionfront{\tensor{1}{\lok{S}{t}}{D\varphi}}{ \Tensor{\lok{S}{U}}{\lok{S}{t}}{\divR{\lok{S}{t}}{R}}}{ \Tensor{\lok{S}{U}}{\lok{S}{t}}{( \Tensor{\lok{S}{t}}{\lok{S}{t}}{\divR{\lok{S}{tt'}}{R}})}} )
\end{gather*}
Zur Vereinfachung der Morphismen in $\mathcal{F}(\mathcal{B})$ definiere folgenden Isomorphismus:
\begin{gather*}
\functionfront{f}{ \Tensor{\lok{S}{U}}{\lok{S}{t}}{( \Tensor{\lok{S}{t}}{\lok{S}{t}}{\divR{\lok{S}{tt'}}{R}})}}{\Tensor{\lok{S}{U}}{\lok{S}{tt'}}{\divR{\lok{S}{tt'}}{R}}}\\
\tensor{\frac{s}{u}}{\lok{S}{t}}{( \tensor{\frac{s'}{t}}{\lok{S}{t}}{d_{\lok{S}{tt'}}(x)} )}
= \tensor{\frac{s}{u}}{\lok{S}{t}}{( \tensor{1}{\lok{S}{t}}{\varphi(\frac{s'}{t})d_{\lok{S}{tt'}}(x)} )}
\longmapsto \tensor{\frac{s}{u}}{\lok{S}{tt'}}{\varphi(\frac{s'}{t})d_{\lok{S}{tt'}}(x)}
\end{gather*}
Zusammen mit $\functionfront{\gamma_t}{\divR{\lok{S}{t}}{R}}{\Tensor{\lok{S}{U}}{S}{\divR{S}{R}}}$ ergibt sich folgendes kommutatives Diagramm:
\begin{center}
\begin{tikzcd}
\lok{S}{t} \arrow[rr, "\varphi"] \arrow[dd, "\mathcal{F}"]                 &  & \lok{S}{tt'} \arrow[d, "\mathcal{F}"]  \\
                                                              &  & \Tensor{\lok{S}{U}}{\lok{S}{t}}{(\Tensor{\lok{S}{t}}{\lok{S}{t}}{\divR{\lok{S}{tt'}}{R}})} \arrow[d, "f"]                 \\
\Tensor{\lok{S}{U}}{\lok{S}{t}}{\divR{\lok{S}{t}}{R}} \arrow[d, "\gamma_t"] \arrow[rru, "\tensor{1}{S}{D\varphi}"] &  & \Tensor{\lok{S}{U}}{\lok{S}{tt'}}{\divR{\lok{S}{tt'}}{R}} \arrow[d, "\gamma_{tt'}"] \\
\Tensor{\lok{S}{U}}{\lok{S}{t}}{(\Tensor{\lok{S}{t}}{S}{\divR{S}{R}})} \arrow[rr, "\phi"]                                          &  & \Tensor{\lok{S}{U}}{\lok{S}{tt'}}{(\Tensor{\lok{S}{tt'}}{S}{\divR{S}{R}})}     \\
\ \\
\frac{s}{t} \arrow[rr, "\varphi", maps to] \arrow[d, "d_{\lok{S}{t}}", maps to]                                                         &  & \frac{s}{tt'} \arrow[d, "d_{\lok{S}{tt'}}", maps to]                                                    \\
\frac{1}{t}d_{\lok{S}{t}}(s) + sd_{\lok{S}{t}}(\frac{1}{t}) \arrow[rr, "f \circ (\tensor{1}{S}{D\varphi})", maps to] \arrow[d, "\gamma_t", maps to] &  & \frac{t'}{tt'}d_{\lok{S}{tt'}}(s) + sd_{\lok{S}{tt'}}(\frac{t'}{tt'}) \arrow[d, "\gamma_{tt'}", maps to] \\
\frac{1}{t}d_{S}(s) - \frac{s}{t^2}d_{S}(t) \arrow[rr, "\phi", maps to]                                                                &  & \frac{t'}{tt'}d_{S}(s) - \frac{st'}{(tt')^2}d_{S}(tt') \textbf{(3)}                                                
\end{tikzcd}
\end{center}
Erhalte daraus mithilfe von \cref{Vereinfachung des Kolimes}:
\begin{gather*}
\divR{\lok{S}{U}}{R}  = \colimes{\mathcal{C}} \text{, wobei:} \\
\mathcal{C} = \lbrace \Tensor{\lok{S}{U}}{\lok{S}{t}}{( \Tensor{\lok{S}{t}}{S}{\divR{S}{R}} )} \vert t \in U \rbrace \textit{ mit den Morphismen }\\
\functionfront{\phi}{\Tensor{\lok{S}{U}}{\lok{S}{t}}{( \Tensor{\lok{S}{t}}{S}{\divR{S}{R}} )}}{\Tensor{\lok{S}{U}}{\lok{S}{tt'}}{( \Tensor{\lok{S}{tt'}}{S}{\divR{S}{R}} )}} \\
\tensor{\loke{s}{u}{U}}{\lok{S}{t}}{( \tensor{\loke{1}{t^n}}{S}{d_{S}(x)} )} \longmapsto \tensor{\loke{s}{u}{U}}{\lok{S}{tt'}}{( \tensor{\loke{t'^n}{(tt')^n}}{S}{d_{S}(x)} )}
\end{gather*}
Somit entspricht $\mathcal{C}$ dem Fall aus \cref{Lokalisierung von Moduln als Kolimes} und es gilt $\colimes \mathcal{C} = \lok{\divR{S}{R}}{U}$.
\end{proof}
\end{document}
\comment{\frac{1}{t}d_{S}(s) + \frac{s}{t}d_{S}(t) \longmapsto \frac{t'}{tt'}d_{S}(s) + \frac{st'}{tt'}d_{S}(tt')}